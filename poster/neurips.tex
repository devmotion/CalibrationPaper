% arara: lualatex: { shell: true }
% arara: biber
% arara: lualatex: { shell: true }
% arara: lualatex: { shell: true }
\documentclass{article}

\usepackage{luatex85}

% layout
\usepackage[a0paper,portrait]{geometry}

% Math packages
\usepackage{mathtools,amssymb}

% Use Fira fonts
\RequirePackage[factor=0]{microtype} % No protrusion
\usepackage[no-math]{fontspec}
\defaultfontfeatures{Ligatures=TeX}
\setsansfont[
  BoldFont={Fira Sans SemiBold},
  ItalicFont={Fira Sans BookItalic},
  BoldItalicFont={Fira Sans SemiBold Italic}
]{Fira Sans Book}
\setmonofont[BoldFont={Fira Mono Medium}]{Fira Mono}

% Use sans serif fonts
\renewcommand{\familydefault}{\sfdefault}

% Add sans serif math fonts
\usepackage[scaled]{newtxsf}
\usepackage{bm}

% language support
\usepackage{polyglossia}
\setdefaultlanguage{english}
\usepackage{csquotes}

% better looking tables
\usepackage{booktabs}

% colors
\usepackage[RGB]{xcolor}
\usepackage{UUcolorPantone}

\newcommand{\hl}[1]{\begingroup\bfseries\boldmath\color{uured}#1\endgroup}

% graphics
\usepackage{graphics}

% captions
\usepackage{caption,subcaption}
\captionsetup{font=scriptsize}

% fancy lists
\usepackage{enumitem}
\setlist[itemize,1]{label={\color{uured}$\blacktriangleright$}}

% hyperlinks
\usepackage{hyperref}

% boxes
\usepackage[poster,xparse]{tcolorbox}

% poster settings
\tcbposterset{
  coverage = {spread,top=0pt,bottom=0pt},
  boxes = {enhanced,
    % Colors
    coltext=black,coltitle=black,colback=white,colbacktitle=uulightgrey,
    % Fonts
    fonttitle=\bfseries\Large\scshape,fontupper=\large,fontlower=\large,
    % Margins
    boxsep=0pt,left=\tcbpostercolspacing,right=\tcbpostercolspacing,
    top=\tcbposterrowspacing,toptitle=\tcbposterrowspacing,
    bottom=\tcbposterrowspacing,bottomtitle=\tcbposterrowspacing,
    % Subtitles
    subtitle style={opacityframe=0,opacityback=0.5,fontupper=\large\bfseries},
  }
}

% plots
\usepackage{pgfplots,pgfplotstable}
\pgfplotsset{compat=1.16}

% general settings for plots
\pgfplotsset{grid style=dashed}
\pgfplotsset{enlargelimits=auto}

\usepgfplotslibrary{groupplots,colorbrewer}
\usetikzlibrary{plotmarks,calc}

% plotting options
\pgfplotsset{table/search path={data/}}
\pgfplotsset{max space between ticks=150}
\pgfplotsset{every axis/.append style={axis background style={fill=gray!10}},tick label style={font={\footnotesize}}, label style={font={\small}}}
\pgfplotsset{every axis plot/.append style={thick}}
\pgfplotsset{every axis legend/.append style={font=\small, fill=none}}

% bibliography
\usepackage[backend=biber,sortcites,doi=false,isbn=false,url=false,giveninits=true,maxbibnames=10,style=alphabetic]{biblatex}
\addbibresource{references.bib}
\AtBeginBibliography{\scriptsize} % small font size
\AtEveryBibitem{%
\ifentrytype{inproceedings}{
    \clearfield{review}%
    \clearfield{editor}%
    \clearfield{series}%%
}{}
}

% automatic references
\usepackage{cleveref}

% some abbreviations
\newcommand{\Prob}{\mathbb{P}}
\newcommand*{\E}{\mathbb{E}}
\DeclareMathOperator*{\argmax}{arg\,max}
\DeclareMathOperator{\ECE}{ECE}
\DeclareMathOperator{\biasedskce}{SKCE_b}
\DeclareMathOperator{\unbiasedskce}{SKCE_{uq}}
\DeclareMathOperator{\linearskce}{SKCE_{ul}}
\DeclareMathOperator{\asympunbiasedskce}{aSKCE_{uq}}
\DeclareMathOperator{\asymplinearskce}{aSKCE_{ul}}
\DeclareMathOperator{\measure}{CE}
\DeclareMathOperator{\kernelmeasure}{KCE}
\DeclareMathOperator{\squaredkernelmeasure}{SKCE}
\DeclareMathOperator{\Expect}{\mathbb{E}}
\DeclareMathOperator{\Dir}{Dir}
\DeclareMathOperator{\Categorical}{Cat}

% Define number of columns and rows
\tcbposterset{poster={columns=2,rows=8}}

% Define split box
\DeclareTColorBox{splitbox}{ O{} }{standard jigsaw,sharp corners,
  sidebyside,sidebyside align=center,opacityfill=0,#1}

% metadata
\title{Calibration tests in multi-class classification:\\A unifying framework}
\author{David Widmann$^*$ Fredrik Lindsten$^\dagger$ Dave Zachariah$^*$}
\def\institute{$^*$Department of Information Technology, Uppsala University\\ $^\ddagger$Division of Statistics and Machine Learning, Linköping University}
\date{}
\def\contact{david.widmann@it.uu.se fredrik.lindsten@liu.se dave.zachariah@it.uu.se}

% Default settings
\pagestyle{empty}

\begin{document}
\begin{tcbposter}[fontsize=26pt]
  \makeatletter
  \posterbox[blankest,width=\paperwidth,interior engine=path,interior style={color=white},%
  top=\tcbposterrowspacing,bottom=\tcbposterrowspacing,left=0.15\paperwidth,right=0.15\paperwidth,%
  height=0.15\paperwidth,height plus=0.15\paperwidth,halign=flush center,valign=center, %
  underlay={%
    \node[right,inner sep=0pt,outer sep=0pt,text width=0.15\paperwidth,align=center] at (frame.west) {\includegraphics[width=0.12\paperwidth]{figures/logos/UU.pdf}};%
    \node[left,inner sep=0pt,outer sep=0pt,text width=0.15\paperwidth,align=center] at (frame.east) {\includegraphics[width=0.12\paperwidth]{figures/logos/LiU.pdf}};%
  }]{name=title,column=1,below=top,xshift=-\tcbpostercolspacing}{%
    \vspace*{2ex}%
    {\bfseries\fontsize{4.5ex}{5.2ex}\selectfont{\@title}}\\[2ex]%
    {\LARGE{\@author}}\\[1ex]%
    {\large{\institute}}%
  }%
  \makeatother

  % footline
  \makeatletter
  \posterbox[blankest,width=\paperwidth,%
  top=5pt,bottom=5pt,left=\tcbpostercolspacing,right=\tcbpostercolspacing,%
  valign=center,fontupper=\ttfamily\small,interior engine=path, interior style={color=uumidgrey}%
  ]{name=footline,column=1,above=bottom,xshift=-\tcbpostercolspacing}{%
    \@date\hfill\contact%
  }%
  \makeatother

  % references
  \posterbox[adjusted title=References]{name=references,column=1,above=footline,span=2}{
    \printbibliography[heading=none]
  }

  \posterbox[adjusted title=Summary of our work]{name=summary,column=1,below=title}{
    \begin{itemize}
    \item We propose a \hl{unifying framework} for quantifying calibration error of probabilistic classifiers that encompasses several existing error measures.
    \item We introduce a new \hl{kernel calibration error} (KCE), for which we derive \hl{unbiased and consistent estimators}.
    \item We show how the transfer of calibration error estimates to \hl{probabilities of false rejection} makes them interpretable and statistically commensurable.
    \item We provide \hl{Julia packages} for calibration estimation.
    \end{itemize}
  }

  \posterbox[adjusted title=General setup]{name=setup,column=1,below=summary}{
    \begin{itemize}
    \item Let $X$ be random inputs (features) of $m$ classes $1,\ldots,m$, denoted by $Y$.
    \item Consider a \hl{probabilistic model} $g$ that predicts a probability distribution of classes $g(X) \in \Delta^{m}$ for input $X$, where $\Delta^m \coloneqq \{z \in \mathbb{R}^m_{\geq 0} : \|z\|_1 = 1\}$ denotes the $(m-1)$-dimensional probability simplex.
    \item Ideally $g$ predicts $g_y(X) = \Prob[Y = y \,|\,X]$ for all classes $y$.
    \end{itemize}
  }

  \posterbox[adjusted title={Calibration}]{name=calibration,column=1,below=setup}{
    Although in practice the model will never be ideal but at most close to it, we can strive to satisfy other desirable statistical properties such as \hl{calibration}.
    \begin{tcolorbox}[colback=blondsvag, halign=center]
      Informally, in the long run every prediction should match the relative frequencies of the observed classes.
    \end{tcolorbox}
    \vspace{-\topsep}
    \begin{itemize}
    \item Mathematically, a model \(g\) is calibrated if
      \begin{equation*}
        g_y(X) = \Prob[Y = y \,|\, g(X)] \quad \text{almost surely for all classes } y,
      \end{equation*}
      or equivalently if
      \begin{equation}\label{eq:calibration}
        g(X) = r(g(X)) \coloneqq (\mathbb{P}[Y = 1 \,|\, g(X)], \ldots, \mathbb{P}[Y = m \,|\, g(X)]) \quad \text{almost surely}.
      \end{equation}

    \item There are many calibrated models \parencite{vaicenavicius19_evaluat}, so there is hope to find one.
      \begin{tcolorbox}[colback=blondsvag]
        Consider three equally likely classes with triangular distributions of inputs:
        \begin{center}
          \pgfplotsset{width=0.25\textwidth,height=0.1\textwidth}
          \begin{tikzpicture}
\begin{axis}[xlabel={$x$}, ylabel={$p_{X|Y}(x \,|\, y)$}, no marks, legend style={at={{(0.5, 1.1)}}, anchor={south}, legend columns={-1}}, %legend pos={outer north east},%
enlarge x limits={false}, axis on top, ymin={0}]
    \addplot+[fill, fill opacity={0.2}, thick]
        table[row sep={\\}]
        {
            \\
            -3.0  0.0  \\
            -2.99  0.0024999999999999467  \\
            -2.98  0.0050000000000000044  \\
            -2.97  0.007499999999999951  \\
            -2.96  0.010000000000000009  \\
            -2.95  0.012499999999999956  \\
            -2.94  0.015000000000000013  \\
            -2.93  0.01749999999999996  \\
            -2.92  0.020000000000000018  \\
            -2.91  0.022499999999999964  \\
            -2.9  0.025000000000000022  \\
            -2.89  0.02749999999999997  \\
            -2.88  0.030000000000000027  \\
            -2.87  0.03249999999999997  \\
            -2.86  0.03500000000000003  \\
            -2.85  0.03749999999999998  \\
            -2.84  0.040000000000000036  \\
            -2.83  0.04249999999999998  \\
            -2.82  0.04500000000000004  \\
            -2.81  0.04749999999999999  \\
            -2.8  0.050000000000000044  \\
            -2.79  0.05249999999999999  \\
            -2.78  0.05500000000000005  \\
            -2.77  0.057499999999999996  \\
            -2.76  0.06000000000000005  \\
            -2.75  0.0625  \\
            -2.74  0.06499999999999995  \\
            -2.73  0.0675  \\
            -2.72  0.06999999999999995  \\
            -2.71  0.07250000000000001  \\
            -2.7  0.07499999999999996  \\
            -2.69  0.07750000000000001  \\
            -2.68  0.07999999999999996  \\
            -2.67  0.08250000000000002  \\
            -2.66  0.08499999999999996  \\
            -2.65  0.08750000000000002  \\
            -2.64  0.08999999999999997  \\
            -2.63  0.09250000000000003  \\
            -2.62  0.09499999999999997  \\
            -2.61  0.09750000000000003  \\
            -2.6  0.09999999999999998  \\
            -2.59  0.10250000000000004  \\
            -2.58  0.10499999999999998  \\
            -2.57  0.10750000000000004  \\
            -2.56  0.10999999999999999  \\
            -2.55  0.11250000000000004  \\
            -2.54  0.11499999999999999  \\
            -2.53  0.11750000000000005  \\
            -2.52  0.12  \\
            -2.51  0.12250000000000005  \\
            -2.5  0.125  \\
            -2.49  0.12749999999999995  \\
            -2.48  0.13  \\
            -2.47  0.13249999999999995  \\
            -2.46  0.135  \\
            -2.45  0.13749999999999996  \\
            -2.44  0.14  \\
            -2.43  0.14249999999999996  \\
            -2.42  0.14500000000000002  \\
            -2.41  0.14749999999999996  \\
            -2.4  0.15000000000000002  \\
            -2.39  0.15249999999999997  \\
            -2.38  0.15500000000000003  \\
            -2.37  0.15749999999999997  \\
            -2.36  0.16000000000000003  \\
            -2.35  0.16249999999999998  \\
            -2.34  0.16500000000000004  \\
            -2.33  0.16749999999999998  \\
            -2.32  0.17000000000000004  \\
            -2.31  0.1725  \\
            -2.3  0.17500000000000004  \\
            -2.29  0.1775  \\
            -2.28  0.18000000000000005  \\
            -2.27  0.1825  \\
            -2.26  0.18500000000000005  \\
            -2.25  0.1875  \\
            -2.24  0.18999999999999995  \\
            -2.23  0.1925  \\
            -2.22  0.19499999999999995  \\
            -2.21  0.1975  \\
            -2.2  0.19999999999999996  \\
            -2.19  0.2025  \\
            -2.18  0.20499999999999996  \\
            -2.17  0.20750000000000002  \\
            -2.16  0.20999999999999996  \\
            -2.15  0.21250000000000002  \\
            -2.14  0.21499999999999997  \\
            -2.13  0.21750000000000003  \\
            -2.12  0.21999999999999997  \\
            -2.11  0.22250000000000003  \\
            -2.1  0.22499999999999998  \\
            -2.09  0.22750000000000004  \\
            -2.08  0.22999999999999998  \\
            -2.07  0.23250000000000004  \\
            -2.06  0.235  \\
            -2.05  0.23750000000000004  \\
            -2.04  0.24  \\
            -2.03  0.24250000000000005  \\
            -2.02  0.245  \\
            -2.01  0.24750000000000005  \\
            -2.0  0.25  \\
            -1.99  0.2525  \\
            -1.98  0.255  \\
            -1.97  0.2575  \\
            -1.96  0.26  \\
            -1.95  0.2625  \\
            -1.94  0.265  \\
            -1.93  0.2675  \\
            -1.92  0.27  \\
            -1.91  0.2725  \\
            -1.9  0.275  \\
            -1.89  0.2775  \\
            -1.88  0.28  \\
            -1.87  0.2825  \\
            -1.86  0.285  \\
            -1.85  0.2875  \\
            -1.84  0.29  \\
            -1.83  0.2925  \\
            -1.82  0.295  \\
            -1.81  0.2975  \\
            -1.8  0.3  \\
            -1.79  0.3025  \\
            -1.78  0.305  \\
            -1.77  0.3075  \\
            -1.76  0.31  \\
            -1.75  0.3125  \\
            -1.74  0.315  \\
            -1.73  0.3175  \\
            -1.72  0.32  \\
            -1.71  0.3225  \\
            -1.7  0.325  \\
            -1.69  0.3275  \\
            -1.68  0.33  \\
            -1.67  0.3325  \\
            -1.66  0.335  \\
            -1.65  0.3375  \\
            -1.64  0.34  \\
            -1.63  0.3425  \\
            -1.62  0.345  \\
            -1.61  0.3475  \\
            -1.6  0.35  \\
            -1.59  0.3525  \\
            -1.58  0.355  \\
            -1.57  0.3575  \\
            -1.56  0.36  \\
            -1.55  0.3625  \\
            -1.54  0.365  \\
            -1.53  0.3675  \\
            -1.52  0.37  \\
            -1.51  0.3725  \\
            -1.5  0.375  \\
            -1.49  0.3775  \\
            -1.48  0.38  \\
            -1.47  0.3825  \\
            -1.46  0.385  \\
            -1.45  0.3875  \\
            -1.44  0.39  \\
            -1.43  0.3925  \\
            -1.42  0.395  \\
            -1.41  0.3975  \\
            -1.4  0.4  \\
            -1.39  0.4025  \\
            -1.38  0.405  \\
            -1.37  0.4075  \\
            -1.36  0.41  \\
            -1.35  0.4125  \\
            -1.34  0.415  \\
            -1.33  0.4175  \\
            -1.32  0.42  \\
            -1.31  0.4225  \\
            -1.3  0.425  \\
            -1.29  0.4275  \\
            -1.28  0.43  \\
            -1.27  0.4325  \\
            -1.26  0.435  \\
            -1.25  0.4375  \\
            -1.24  0.44  \\
            -1.23  0.4425  \\
            -1.22  0.445  \\
            -1.21  0.4475  \\
            -1.2  0.45  \\
            -1.19  0.4525  \\
            -1.18  0.455  \\
            -1.17  0.4575  \\
            -1.16  0.46  \\
            -1.15  0.4625  \\
            -1.14  0.465  \\
            -1.13  0.4675  \\
            -1.12  0.47  \\
            -1.11  0.4725  \\
            -1.1  0.475  \\
            -1.09  0.4775  \\
            -1.08  0.48  \\
            -1.07  0.4825  \\
            -1.06  0.485  \\
            -1.05  0.4875  \\
            -1.04  0.49  \\
            -1.03  0.4925  \\
            -1.02  0.495  \\
            -1.01  0.4975  \\
            -1.0  0.5  \\
            -0.99  0.4975  \\
            -0.98  0.495  \\
            -0.97  0.4925  \\
            -0.96  0.49  \\
            -0.95  0.4875  \\
            -0.94  0.485  \\
            -0.93  0.48250000000000004  \\
            -0.92  0.48  \\
            -0.91  0.47750000000000004  \\
            -0.9  0.475  \\
            -0.89  0.47250000000000003  \\
            -0.88  0.47  \\
            -0.87  0.4675  \\
            -0.86  0.46499999999999997  \\
            -0.85  0.4625  \\
            -0.84  0.45999999999999996  \\
            -0.83  0.4575  \\
            -0.82  0.45499999999999996  \\
            -0.81  0.4525  \\
            -0.8  0.45  \\
            -0.79  0.4475  \\
            -0.78  0.445  \\
            -0.77  0.4425  \\
            -0.76  0.44  \\
            -0.75  0.4375  \\
            -0.74  0.435  \\
            -0.73  0.4325  \\
            -0.72  0.43  \\
            -0.71  0.4275  \\
            -0.7  0.425  \\
            -0.69  0.4225  \\
            -0.68  0.42000000000000004  \\
            -0.67  0.4175  \\
            -0.66  0.41500000000000004  \\
            -0.65  0.4125  \\
            -0.64  0.41000000000000003  \\
            -0.63  0.4075  \\
            -0.62  0.405  \\
            -0.61  0.40249999999999997  \\
            -0.6  0.4  \\
            -0.59  0.39749999999999996  \\
            -0.58  0.395  \\
            -0.57  0.39249999999999996  \\
            -0.56  0.39  \\
            -0.55  0.3875  \\
            -0.54  0.385  \\
            -0.53  0.3825  \\
            -0.52  0.38  \\
            -0.51  0.3775  \\
            -0.5  0.375  \\
            -0.49  0.3725  \\
            -0.48  0.37  \\
            -0.47  0.3675  \\
            -0.46  0.365  \\
            -0.45  0.3625  \\
            -0.44  0.36  \\
            -0.43  0.3575  \\
            -0.42  0.355  \\
            -0.41  0.3525  \\
            -0.4  0.35  \\
            -0.39  0.34750000000000003  \\
            -0.38  0.345  \\
            -0.37  0.3425  \\
            -0.36  0.33999999999999997  \\
            -0.35  0.3375  \\
            -0.34  0.335  \\
            -0.33  0.3325  \\
            -0.32  0.33  \\
            -0.31  0.3275  \\
            -0.3  0.325  \\
            -0.29  0.3225  \\
            -0.28  0.32  \\
            -0.27  0.3175  \\
            -0.26  0.315  \\
            -0.25  0.3125  \\
            -0.24  0.31  \\
            -0.23  0.3075  \\
            -0.22  0.305  \\
            -0.21  0.3025  \\
            -0.2  0.3  \\
            -0.19  0.2975  \\
            -0.18  0.295  \\
            -0.17  0.2925  \\
            -0.16  0.29  \\
            -0.15  0.2875  \\
            -0.14  0.28500000000000003  \\
            -0.13  0.2825  \\
            -0.12  0.28  \\
            -0.11  0.2775  \\
            -0.1  0.275  \\
            -0.09  0.2725  \\
            -0.08  0.27  \\
            -0.07  0.2675  \\
            -0.06  0.265  \\
            -0.05  0.2625  \\
            -0.04  0.26  \\
            -0.03  0.2575  \\
            -0.02  0.255  \\
            -0.01  0.2525  \\
            0.0  0.25  \\
            0.01  0.2475  \\
            0.02  0.245  \\
            0.03  0.2425  \\
            0.04  0.24  \\
            0.05  0.2375  \\
            0.06  0.235  \\
            0.07  0.23249999999999998  \\
            0.08  0.23  \\
            0.09  0.2275  \\
            0.1  0.225  \\
            0.11  0.2225  \\
            0.12  0.22  \\
            0.13  0.2175  \\
            0.14  0.215  \\
            0.15  0.2125  \\
            0.16  0.21  \\
            0.17  0.2075  \\
            0.18  0.20500000000000002  \\
            0.19  0.2025  \\
            0.2  0.2  \\
            0.21  0.1975  \\
            0.22  0.195  \\
            0.23  0.1925  \\
            0.24  0.19  \\
            0.25  0.1875  \\
            0.26  0.185  \\
            0.27  0.1825  \\
            0.28  0.18  \\
            0.29  0.1775  \\
            0.3  0.175  \\
            0.31  0.1725  \\
            0.32  0.16999999999999998  \\
            0.33  0.16749999999999998  \\
            0.34  0.16499999999999998  \\
            0.35  0.1625  \\
            0.36  0.16  \\
            0.37  0.1575  \\
            0.38  0.155  \\
            0.39  0.1525  \\
            0.4  0.15  \\
            0.41  0.14750000000000002  \\
            0.42  0.14500000000000002  \\
            0.43  0.14250000000000002  \\
            0.44  0.14  \\
            0.45  0.1375  \\
            0.46  0.135  \\
            0.47  0.1325  \\
            0.48  0.13  \\
            0.49  0.1275  \\
            0.5  0.125  \\
            0.51  0.1225  \\
            0.52  0.12  \\
            0.53  0.1175  \\
            0.54  0.11499999999999999  \\
            0.55  0.11249999999999999  \\
            0.56  0.10999999999999999  \\
            0.57  0.10750000000000001  \\
            0.58  0.10500000000000001  \\
            0.59  0.10250000000000001  \\
            0.6  0.1  \\
            0.61  0.0975  \\
            0.62  0.095  \\
            0.63  0.0925  \\
            0.64  0.09  \\
            0.65  0.0875  \\
            0.66  0.08499999999999999  \\
            0.67  0.08249999999999999  \\
            0.68  0.07999999999999999  \\
            0.69  0.07750000000000001  \\
            0.7  0.07500000000000001  \\
            0.71  0.07250000000000001  \\
            0.72  0.07  \\
            0.73  0.0675  \\
            0.74  0.065  \\
            0.75  0.0625  \\
            0.76  0.06  \\
            0.77  0.057499999999999996  \\
            0.78  0.05499999999999999  \\
            0.79  0.05249999999999999  \\
            0.8  0.04999999999999999  \\
            0.81  0.04749999999999999  \\
            0.82  0.04500000000000001  \\
            0.83  0.04250000000000001  \\
            0.84  0.04000000000000001  \\
            0.85  0.037500000000000006  \\
            0.86  0.035  \\
            0.87  0.0325  \\
            0.88  0.03  \\
            0.89  0.027499999999999997  \\
            0.9  0.024999999999999994  \\
            0.91  0.022499999999999992  \\
            0.92  0.01999999999999999  \\
            0.93  0.017499999999999988  \\
            0.94  0.015000000000000013  \\
            0.95  0.012500000000000011  \\
            0.96  0.010000000000000009  \\
            0.97  0.007500000000000007  \\
            0.98  0.0050000000000000044  \\
            0.99  0.0025000000000000022  \\
            1.0  0.0  \\
            1.01  0.0  \\
            1.02  0.0  \\
            1.03  0.0  \\
            1.04  0.0  \\
            1.05  0.0  \\
            1.06  0.0  \\
            1.07  0.0  \\
            1.08  0.0  \\
            1.09  0.0  \\
            1.1  0.0  \\
            1.11  0.0  \\
            1.12  0.0  \\
            1.13  0.0  \\
            1.14  0.0  \\
            1.15  0.0  \\
            1.16  0.0  \\
            1.17  0.0  \\
            1.18  0.0  \\
            1.19  0.0  \\
            1.2  0.0  \\
            1.21  0.0  \\
            1.22  0.0  \\
            1.23  0.0  \\
            1.24  0.0  \\
            1.25  0.0  \\
            1.26  0.0  \\
            1.27  0.0  \\
            1.28  0.0  \\
            1.29  0.0  \\
            1.3  0.0  \\
            1.31  0.0  \\
            1.32  0.0  \\
            1.33  0.0  \\
            1.34  0.0  \\
            1.35  0.0  \\
            1.36  0.0  \\
            1.37  0.0  \\
            1.38  0.0  \\
            1.39  0.0  \\
            1.4  0.0  \\
            1.41  0.0  \\
            1.42  0.0  \\
            1.43  0.0  \\
            1.44  0.0  \\
            1.45  0.0  \\
            1.46  0.0  \\
            1.47  0.0  \\
            1.48  0.0  \\
            1.49  0.0  \\
            1.5  0.0  \\
            1.51  0.0  \\
            1.52  0.0  \\
            1.53  0.0  \\
            1.54  0.0  \\
            1.55  0.0  \\
            1.56  0.0  \\
            1.57  0.0  \\
            1.58  0.0  \\
            1.59  0.0  \\
            1.6  0.0  \\
            1.61  0.0  \\
            1.62  0.0  \\
            1.63  0.0  \\
            1.64  0.0  \\
            1.65  0.0  \\
            1.66  0.0  \\
            1.67  0.0  \\
            1.68  0.0  \\
            1.69  0.0  \\
            1.7  0.0  \\
            1.71  0.0  \\
            1.72  0.0  \\
            1.73  0.0  \\
            1.74  0.0  \\
            1.75  0.0  \\
            1.76  0.0  \\
            1.77  0.0  \\
            1.78  0.0  \\
            1.79  0.0  \\
            1.8  0.0  \\
            1.81  0.0  \\
            1.82  0.0  \\
            1.83  0.0  \\
            1.84  0.0  \\
            1.85  0.0  \\
            1.86  0.0  \\
            1.87  0.0  \\
            1.88  0.0  \\
            1.89  0.0  \\
            1.9  0.0  \\
            1.91  0.0  \\
            1.92  0.0  \\
            1.93  0.0  \\
            1.94  0.0  \\
            1.95  0.0  \\
            1.96  0.0  \\
            1.97  0.0  \\
            1.98  0.0  \\
            1.99  0.0  \\
            2.0  0.0  \\
            2.01  0.0  \\
            2.02  0.0  \\
            2.03  0.0  \\
            2.04  0.0  \\
            2.05  0.0  \\
            2.06  0.0  \\
            2.07  0.0  \\
            2.08  0.0  \\
            2.09  0.0  \\
            2.1  0.0  \\
            2.11  0.0  \\
            2.12  0.0  \\
            2.13  0.0  \\
            2.14  0.0  \\
            2.15  0.0  \\
            2.16  0.0  \\
            2.17  0.0  \\
            2.18  0.0  \\
            2.19  0.0  \\
            2.2  0.0  \\
            2.21  0.0  \\
            2.22  0.0  \\
            2.23  0.0  \\
            2.24  0.0  \\
            2.25  0.0  \\
            2.26  0.0  \\
            2.27  0.0  \\
            2.28  0.0  \\
            2.29  0.0  \\
            2.3  0.0  \\
            2.31  0.0  \\
            2.32  0.0  \\
            2.33  0.0  \\
            2.34  0.0  \\
            2.35  0.0  \\
            2.36  0.0  \\
            2.37  0.0  \\
            2.38  0.0  \\
            2.39  0.0  \\
            2.4  0.0  \\
            2.41  0.0  \\
            2.42  0.0  \\
            2.43  0.0  \\
            2.44  0.0  \\
            2.45  0.0  \\
            2.46  0.0  \\
            2.47  0.0  \\
            2.48  0.0  \\
            2.49  0.0  \\
            2.5  0.0  \\
            2.51  0.0  \\
            2.52  0.0  \\
            2.53  0.0  \\
            2.54  0.0  \\
            2.55  0.0  \\
            2.56  0.0  \\
            2.57  0.0  \\
            2.58  0.0  \\
            2.59  0.0  \\
            2.6  0.0  \\
            2.61  0.0  \\
            2.62  0.0  \\
            2.63  0.0  \\
            2.64  0.0  \\
            2.65  0.0  \\
            2.66  0.0  \\
            2.67  0.0  \\
            2.68  0.0  \\
            2.69  0.0  \\
            2.7  0.0  \\
            2.71  0.0  \\
            2.72  0.0  \\
            2.73  0.0  \\
            2.74  0.0  \\
            2.75  0.0  \\
            2.76  0.0  \\
            2.77  0.0  \\
            2.78  0.0  \\
            2.79  0.0  \\
            2.8  0.0  \\
            2.81  0.0  \\
            2.82  0.0  \\
            2.83  0.0  \\
            2.84  0.0  \\
            2.85  0.0  \\
            2.86  0.0  \\
            2.87  0.0  \\
            2.88  0.0  \\
            2.89  0.0  \\
            2.9  0.0  \\
            2.91  0.0  \\
            2.92  0.0  \\
            2.93  0.0  \\
            2.94  0.0  \\
            2.95  0.0  \\
            2.96  0.0  \\
            2.97  0.0  \\
            2.98  0.0  \\
            2.99  0.0  \\
            3.0  0.0  \\
        }
        \closedcycle
        ;
    \addlegendentry {$y = 1$}
    \addplot+[fill, fill opacity={0.2}, thick]
        table[row sep={\\}]
        {
            \\
            -3.0  0.0  \\
            -2.99  0.0  \\
            -2.98  0.0  \\
            -2.97  0.0  \\
            -2.96  0.0  \\
            -2.95  0.0  \\
            -2.94  0.0  \\
            -2.93  0.0  \\
            -2.92  0.0  \\
            -2.91  0.0  \\
            -2.9  0.0  \\
            -2.89  0.0  \\
            -2.88  0.0  \\
            -2.87  0.0  \\
            -2.86  0.0  \\
            -2.85  0.0  \\
            -2.84  0.0  \\
            -2.83  0.0  \\
            -2.82  0.0  \\
            -2.81  0.0  \\
            -2.8  0.0  \\
            -2.79  0.0  \\
            -2.78  0.0  \\
            -2.77  0.0  \\
            -2.76  0.0  \\
            -2.75  0.0  \\
            -2.74  0.0  \\
            -2.73  0.0  \\
            -2.72  0.0  \\
            -2.71  0.0  \\
            -2.7  0.0  \\
            -2.69  0.0  \\
            -2.68  0.0  \\
            -2.67  0.0  \\
            -2.66  0.0  \\
            -2.65  0.0  \\
            -2.64  0.0  \\
            -2.63  0.0  \\
            -2.62  0.0  \\
            -2.61  0.0  \\
            -2.6  0.0  \\
            -2.59  0.0  \\
            -2.58  0.0  \\
            -2.57  0.0  \\
            -2.56  0.0  \\
            -2.55  0.0  \\
            -2.54  0.0  \\
            -2.53  0.0  \\
            -2.52  0.0  \\
            -2.51  0.0  \\
            -2.5  0.0  \\
            -2.49  0.0  \\
            -2.48  0.0  \\
            -2.47  0.0  \\
            -2.46  0.0  \\
            -2.45  0.0  \\
            -2.44  0.0  \\
            -2.43  0.0  \\
            -2.42  0.0  \\
            -2.41  0.0  \\
            -2.4  0.0  \\
            -2.39  0.0  \\
            -2.38  0.0  \\
            -2.37  0.0  \\
            -2.36  0.0  \\
            -2.35  0.0  \\
            -2.34  0.0  \\
            -2.33  0.0  \\
            -2.32  0.0  \\
            -2.31  0.0  \\
            -2.3  0.0  \\
            -2.29  0.0  \\
            -2.28  0.0  \\
            -2.27  0.0  \\
            -2.26  0.0  \\
            -2.25  0.0  \\
            -2.24  0.0  \\
            -2.23  0.0  \\
            -2.22  0.0  \\
            -2.21  0.0  \\
            -2.2  0.0  \\
            -2.19  0.0  \\
            -2.18  0.0  \\
            -2.17  0.0  \\
            -2.16  0.0  \\
            -2.15  0.0  \\
            -2.14  0.0  \\
            -2.13  0.0  \\
            -2.12  0.0  \\
            -2.11  0.0  \\
            -2.1  0.0  \\
            -2.09  0.0  \\
            -2.08  0.0  \\
            -2.07  0.0  \\
            -2.06  0.0  \\
            -2.05  0.0  \\
            -2.04  0.0  \\
            -2.03  0.0  \\
            -2.02  0.0  \\
            -2.01  0.0  \\
            -2.0  0.0  \\
            -1.99  0.0025000000000000022  \\
            -1.98  0.0050000000000000044  \\
            -1.97  0.007500000000000007  \\
            -1.96  0.010000000000000009  \\
            -1.95  0.012500000000000011  \\
            -1.94  0.015000000000000013  \\
            -1.93  0.017500000000000016  \\
            -1.92  0.020000000000000018  \\
            -1.91  0.02250000000000002  \\
            -1.9  0.025000000000000022  \\
            -1.89  0.027500000000000024  \\
            -1.88  0.030000000000000027  \\
            -1.87  0.03249999999999997  \\
            -1.86  0.034999999999999976  \\
            -1.85  0.03749999999999998  \\
            -1.84  0.03999999999999998  \\
            -1.83  0.04249999999999998  \\
            -1.82  0.044999999999999984  \\
            -1.81  0.04749999999999999  \\
            -1.8  0.04999999999999999  \\
            -1.79  0.05249999999999999  \\
            -1.78  0.05499999999999999  \\
            -1.77  0.057499999999999996  \\
            -1.76  0.06  \\
            -1.75  0.0625  \\
            -1.74  0.065  \\
            -1.73  0.0675  \\
            -1.72  0.07  \\
            -1.71  0.07250000000000001  \\
            -1.7  0.07500000000000001  \\
            -1.69  0.07750000000000001  \\
            -1.68  0.08000000000000002  \\
            -1.67  0.08250000000000002  \\
            -1.66  0.08500000000000002  \\
            -1.65  0.08750000000000002  \\
            -1.64  0.09000000000000002  \\
            -1.63  0.09250000000000003  \\
            -1.62  0.09499999999999997  \\
            -1.61  0.09749999999999998  \\
            -1.6  0.09999999999999998  \\
            -1.59  0.10249999999999998  \\
            -1.58  0.10499999999999998  \\
            -1.57  0.10749999999999998  \\
            -1.56  0.10999999999999999  \\
            -1.55  0.11249999999999999  \\
            -1.54  0.11499999999999999  \\
            -1.53  0.1175  \\
            -1.52  0.12  \\
            -1.51  0.1225  \\
            -1.5  0.125  \\
            -1.49  0.1275  \\
            -1.48  0.13  \\
            -1.47  0.1325  \\
            -1.46  0.135  \\
            -1.45  0.1375  \\
            -1.44  0.14  \\
            -1.43  0.14250000000000002  \\
            -1.42  0.14500000000000002  \\
            -1.41  0.14750000000000002  \\
            -1.4  0.15000000000000002  \\
            -1.39  0.15250000000000002  \\
            -1.38  0.15500000000000003  \\
            -1.37  0.15749999999999997  \\
            -1.36  0.15999999999999998  \\
            -1.35  0.16249999999999998  \\
            -1.34  0.16499999999999998  \\
            -1.33  0.16749999999999998  \\
            -1.32  0.16999999999999998  \\
            -1.31  0.1725  \\
            -1.3  0.175  \\
            -1.29  0.1775  \\
            -1.28  0.18  \\
            -1.27  0.1825  \\
            -1.26  0.185  \\
            -1.25  0.1875  \\
            -1.24  0.19  \\
            -1.23  0.1925  \\
            -1.22  0.195  \\
            -1.21  0.1975  \\
            -1.2  0.2  \\
            -1.19  0.2025  \\
            -1.18  0.20500000000000002  \\
            -1.17  0.20750000000000002  \\
            -1.16  0.21000000000000002  \\
            -1.15  0.21250000000000002  \\
            -1.14  0.21500000000000002  \\
            -1.13  0.21750000000000003  \\
            -1.12  0.21999999999999997  \\
            -1.11  0.22249999999999998  \\
            -1.1  0.22499999999999998  \\
            -1.09  0.22749999999999998  \\
            -1.08  0.22999999999999998  \\
            -1.07  0.23249999999999998  \\
            -1.06  0.235  \\
            -1.05  0.2375  \\
            -1.04  0.24  \\
            -1.03  0.2425  \\
            -1.02  0.245  \\
            -1.01  0.2475  \\
            -1.0  0.25  \\
            -0.99  0.2525  \\
            -0.98  0.255  \\
            -0.97  0.2575  \\
            -0.96  0.26  \\
            -0.95  0.2625  \\
            -0.94  0.265  \\
            -0.93  0.26749999999999996  \\
            -0.92  0.27  \\
            -0.91  0.27249999999999996  \\
            -0.9  0.275  \\
            -0.89  0.27749999999999997  \\
            -0.88  0.28  \\
            -0.87  0.2825  \\
            -0.86  0.28500000000000003  \\
            -0.85  0.2875  \\
            -0.84  0.29000000000000004  \\
            -0.83  0.2925  \\
            -0.82  0.29500000000000004  \\
            -0.81  0.2975  \\
            -0.8  0.3  \\
            -0.79  0.3025  \\
            -0.78  0.305  \\
            -0.77  0.3075  \\
            -0.76  0.31  \\
            -0.75  0.3125  \\
            -0.74  0.315  \\
            -0.73  0.3175  \\
            -0.72  0.32  \\
            -0.71  0.3225  \\
            -0.7  0.325  \\
            -0.69  0.3275  \\
            -0.68  0.32999999999999996  \\
            -0.67  0.3325  \\
            -0.66  0.33499999999999996  \\
            -0.65  0.3375  \\
            -0.64  0.33999999999999997  \\
            -0.63  0.3425  \\
            -0.62  0.345  \\
            -0.61  0.34750000000000003  \\
            -0.6  0.35  \\
            -0.59  0.35250000000000004  \\
            -0.58  0.355  \\
            -0.57  0.35750000000000004  \\
            -0.56  0.36  \\
            -0.55  0.3625  \\
            -0.54  0.365  \\
            -0.53  0.3675  \\
            -0.52  0.37  \\
            -0.51  0.3725  \\
            -0.5  0.375  \\
            -0.49  0.3775  \\
            -0.48  0.38  \\
            -0.47  0.3825  \\
            -0.46  0.385  \\
            -0.45  0.3875  \\
            -0.44  0.39  \\
            -0.43  0.3925  \\
            -0.42  0.395  \\
            -0.41  0.3975  \\
            -0.4  0.4  \\
            -0.39  0.40249999999999997  \\
            -0.38  0.405  \\
            -0.37  0.4075  \\
            -0.36  0.41000000000000003  \\
            -0.35  0.4125  \\
            -0.34  0.415  \\
            -0.33  0.4175  \\
            -0.32  0.42  \\
            -0.31  0.4225  \\
            -0.3  0.425  \\
            -0.29  0.4275  \\
            -0.28  0.43  \\
            -0.27  0.4325  \\
            -0.26  0.435  \\
            -0.25  0.4375  \\
            -0.24  0.44  \\
            -0.23  0.4425  \\
            -0.22  0.445  \\
            -0.21  0.4475  \\
            -0.2  0.45  \\
            -0.19  0.4525  \\
            -0.18  0.455  \\
            -0.17  0.4575  \\
            -0.16  0.46  \\
            -0.15  0.4625  \\
            -0.14  0.46499999999999997  \\
            -0.13  0.4675  \\
            -0.12  0.47  \\
            -0.11  0.4725  \\
            -0.1  0.475  \\
            -0.09  0.4775  \\
            -0.08  0.48  \\
            -0.07  0.4825  \\
            -0.06  0.485  \\
            -0.05  0.4875  \\
            -0.04  0.49  \\
            -0.03  0.4925  \\
            -0.02  0.495  \\
            -0.01  0.4975  \\
            0.0  0.5  \\
            0.01  0.4975  \\
            0.02  0.495  \\
            0.03  0.4925  \\
            0.04  0.49  \\
            0.05  0.4875  \\
            0.06  0.485  \\
            0.07  0.4825  \\
            0.08  0.48  \\
            0.09  0.4775  \\
            0.1  0.475  \\
            0.11  0.4725  \\
            0.12  0.47  \\
            0.13  0.4675  \\
            0.14  0.46499999999999997  \\
            0.15  0.4625  \\
            0.16  0.46  \\
            0.17  0.4575  \\
            0.18  0.455  \\
            0.19  0.4525  \\
            0.2  0.45  \\
            0.21  0.4475  \\
            0.22  0.445  \\
            0.23  0.4425  \\
            0.24  0.44  \\
            0.25  0.4375  \\
            0.26  0.435  \\
            0.27  0.4325  \\
            0.28  0.43  \\
            0.29  0.4275  \\
            0.3  0.425  \\
            0.31  0.4225  \\
            0.32  0.42  \\
            0.33  0.4175  \\
            0.34  0.415  \\
            0.35  0.4125  \\
            0.36  0.41000000000000003  \\
            0.37  0.4075  \\
            0.38  0.405  \\
            0.39  0.40249999999999997  \\
            0.4  0.4  \\
            0.41  0.3975  \\
            0.42  0.395  \\
            0.43  0.3925  \\
            0.44  0.39  \\
            0.45  0.3875  \\
            0.46  0.385  \\
            0.47  0.3825  \\
            0.48  0.38  \\
            0.49  0.3775  \\
            0.5  0.375  \\
            0.51  0.3725  \\
            0.52  0.37  \\
            0.53  0.3675  \\
            0.54  0.365  \\
            0.55  0.3625  \\
            0.56  0.36  \\
            0.57  0.35750000000000004  \\
            0.58  0.355  \\
            0.59  0.35250000000000004  \\
            0.6  0.35  \\
            0.61  0.34750000000000003  \\
            0.62  0.345  \\
            0.63  0.3425  \\
            0.64  0.33999999999999997  \\
            0.65  0.3375  \\
            0.66  0.33499999999999996  \\
            0.67  0.3325  \\
            0.68  0.32999999999999996  \\
            0.69  0.3275  \\
            0.7  0.325  \\
            0.71  0.3225  \\
            0.72  0.32  \\
            0.73  0.3175  \\
            0.74  0.315  \\
            0.75  0.3125  \\
            0.76  0.31  \\
            0.77  0.3075  \\
            0.78  0.305  \\
            0.79  0.3025  \\
            0.8  0.3  \\
            0.81  0.2975  \\
            0.82  0.29500000000000004  \\
            0.83  0.2925  \\
            0.84  0.29000000000000004  \\
            0.85  0.2875  \\
            0.86  0.28500000000000003  \\
            0.87  0.2825  \\
            0.88  0.28  \\
            0.89  0.27749999999999997  \\
            0.9  0.275  \\
            0.91  0.27249999999999996  \\
            0.92  0.27  \\
            0.93  0.26749999999999996  \\
            0.94  0.265  \\
            0.95  0.2625  \\
            0.96  0.26  \\
            0.97  0.2575  \\
            0.98  0.255  \\
            0.99  0.2525  \\
            1.0  0.25  \\
            1.01  0.2475  \\
            1.02  0.245  \\
            1.03  0.2425  \\
            1.04  0.24  \\
            1.05  0.2375  \\
            1.06  0.235  \\
            1.07  0.23249999999999998  \\
            1.08  0.22999999999999998  \\
            1.09  0.22749999999999998  \\
            1.1  0.22499999999999998  \\
            1.11  0.22249999999999998  \\
            1.12  0.21999999999999997  \\
            1.13  0.21750000000000003  \\
            1.14  0.21500000000000002  \\
            1.15  0.21250000000000002  \\
            1.16  0.21000000000000002  \\
            1.17  0.20750000000000002  \\
            1.18  0.20500000000000002  \\
            1.19  0.2025  \\
            1.2  0.2  \\
            1.21  0.1975  \\
            1.22  0.195  \\
            1.23  0.1925  \\
            1.24  0.19  \\
            1.25  0.1875  \\
            1.26  0.185  \\
            1.27  0.1825  \\
            1.28  0.18  \\
            1.29  0.1775  \\
            1.3  0.175  \\
            1.31  0.1725  \\
            1.32  0.16999999999999998  \\
            1.33  0.16749999999999998  \\
            1.34  0.16499999999999998  \\
            1.35  0.16249999999999998  \\
            1.36  0.15999999999999998  \\
            1.37  0.15749999999999997  \\
            1.38  0.15500000000000003  \\
            1.39  0.15250000000000002  \\
            1.4  0.15000000000000002  \\
            1.41  0.14750000000000002  \\
            1.42  0.14500000000000002  \\
            1.43  0.14250000000000002  \\
            1.44  0.14  \\
            1.45  0.1375  \\
            1.46  0.135  \\
            1.47  0.1325  \\
            1.48  0.13  \\
            1.49  0.1275  \\
            1.5  0.125  \\
            1.51  0.1225  \\
            1.52  0.12  \\
            1.53  0.1175  \\
            1.54  0.11499999999999999  \\
            1.55  0.11249999999999999  \\
            1.56  0.10999999999999999  \\
            1.57  0.10749999999999998  \\
            1.58  0.10499999999999998  \\
            1.59  0.10249999999999998  \\
            1.6  0.09999999999999998  \\
            1.61  0.09749999999999998  \\
            1.62  0.09499999999999997  \\
            1.63  0.09250000000000003  \\
            1.64  0.09000000000000002  \\
            1.65  0.08750000000000002  \\
            1.66  0.08500000000000002  \\
            1.67  0.08250000000000002  \\
            1.68  0.08000000000000002  \\
            1.69  0.07750000000000001  \\
            1.7  0.07500000000000001  \\
            1.71  0.07250000000000001  \\
            1.72  0.07  \\
            1.73  0.0675  \\
            1.74  0.065  \\
            1.75  0.0625  \\
            1.76  0.06  \\
            1.77  0.057499999999999996  \\
            1.78  0.05499999999999999  \\
            1.79  0.05249999999999999  \\
            1.8  0.04999999999999999  \\
            1.81  0.04749999999999999  \\
            1.82  0.044999999999999984  \\
            1.83  0.04249999999999998  \\
            1.84  0.03999999999999998  \\
            1.85  0.03749999999999998  \\
            1.86  0.034999999999999976  \\
            1.87  0.03249999999999997  \\
            1.88  0.030000000000000027  \\
            1.89  0.027500000000000024  \\
            1.9  0.025000000000000022  \\
            1.91  0.02250000000000002  \\
            1.92  0.020000000000000018  \\
            1.93  0.017500000000000016  \\
            1.94  0.015000000000000013  \\
            1.95  0.012500000000000011  \\
            1.96  0.010000000000000009  \\
            1.97  0.007500000000000007  \\
            1.98  0.0050000000000000044  \\
            1.99  0.0025000000000000022  \\
            2.0  0.0  \\
            2.01  0.0  \\
            2.02  0.0  \\
            2.03  0.0  \\
            2.04  0.0  \\
            2.05  0.0  \\
            2.06  0.0  \\
            2.07  0.0  \\
            2.08  0.0  \\
            2.09  0.0  \\
            2.1  0.0  \\
            2.11  0.0  \\
            2.12  0.0  \\
            2.13  0.0  \\
            2.14  0.0  \\
            2.15  0.0  \\
            2.16  0.0  \\
            2.17  0.0  \\
            2.18  0.0  \\
            2.19  0.0  \\
            2.2  0.0  \\
            2.21  0.0  \\
            2.22  0.0  \\
            2.23  0.0  \\
            2.24  0.0  \\
            2.25  0.0  \\
            2.26  0.0  \\
            2.27  0.0  \\
            2.28  0.0  \\
            2.29  0.0  \\
            2.3  0.0  \\
            2.31  0.0  \\
            2.32  0.0  \\
            2.33  0.0  \\
            2.34  0.0  \\
            2.35  0.0  \\
            2.36  0.0  \\
            2.37  0.0  \\
            2.38  0.0  \\
            2.39  0.0  \\
            2.4  0.0  \\
            2.41  0.0  \\
            2.42  0.0  \\
            2.43  0.0  \\
            2.44  0.0  \\
            2.45  0.0  \\
            2.46  0.0  \\
            2.47  0.0  \\
            2.48  0.0  \\
            2.49  0.0  \\
            2.5  0.0  \\
            2.51  0.0  \\
            2.52  0.0  \\
            2.53  0.0  \\
            2.54  0.0  \\
            2.55  0.0  \\
            2.56  0.0  \\
            2.57  0.0  \\
            2.58  0.0  \\
            2.59  0.0  \\
            2.6  0.0  \\
            2.61  0.0  \\
            2.62  0.0  \\
            2.63  0.0  \\
            2.64  0.0  \\
            2.65  0.0  \\
            2.66  0.0  \\
            2.67  0.0  \\
            2.68  0.0  \\
            2.69  0.0  \\
            2.7  0.0  \\
            2.71  0.0  \\
            2.72  0.0  \\
            2.73  0.0  \\
            2.74  0.0  \\
            2.75  0.0  \\
            2.76  0.0  \\
            2.77  0.0  \\
            2.78  0.0  \\
            2.79  0.0  \\
            2.8  0.0  \\
            2.81  0.0  \\
            2.82  0.0  \\
            2.83  0.0  \\
            2.84  0.0  \\
            2.85  0.0  \\
            2.86  0.0  \\
            2.87  0.0  \\
            2.88  0.0  \\
            2.89  0.0  \\
            2.9  0.0  \\
            2.91  0.0  \\
            2.92  0.0  \\
            2.93  0.0  \\
            2.94  0.0  \\
            2.95  0.0  \\
            2.96  0.0  \\
            2.97  0.0  \\
            2.98  0.0  \\
            2.99  0.0  \\
            3.0  0.0  \\
        }
        \closedcycle
        ;
    \addlegendentry {$y = 2$}
    \addplot+[fill, fill opacity={0.2}, thick]
        table[row sep={\\}]
        {
            \\
            -3.0  0.0  \\
            -2.99  0.0  \\
            -2.98  0.0  \\
            -2.97  0.0  \\
            -2.96  0.0  \\
            -2.95  0.0  \\
            -2.94  0.0  \\
            -2.93  0.0  \\
            -2.92  0.0  \\
            -2.91  0.0  \\
            -2.9  0.0  \\
            -2.89  0.0  \\
            -2.88  0.0  \\
            -2.87  0.0  \\
            -2.86  0.0  \\
            -2.85  0.0  \\
            -2.84  0.0  \\
            -2.83  0.0  \\
            -2.82  0.0  \\
            -2.81  0.0  \\
            -2.8  0.0  \\
            -2.79  0.0  \\
            -2.78  0.0  \\
            -2.77  0.0  \\
            -2.76  0.0  \\
            -2.75  0.0  \\
            -2.74  0.0  \\
            -2.73  0.0  \\
            -2.72  0.0  \\
            -2.71  0.0  \\
            -2.7  0.0  \\
            -2.69  0.0  \\
            -2.68  0.0  \\
            -2.67  0.0  \\
            -2.66  0.0  \\
            -2.65  0.0  \\
            -2.64  0.0  \\
            -2.63  0.0  \\
            -2.62  0.0  \\
            -2.61  0.0  \\
            -2.6  0.0  \\
            -2.59  0.0  \\
            -2.58  0.0  \\
            -2.57  0.0  \\
            -2.56  0.0  \\
            -2.55  0.0  \\
            -2.54  0.0  \\
            -2.53  0.0  \\
            -2.52  0.0  \\
            -2.51  0.0  \\
            -2.5  0.0  \\
            -2.49  0.0  \\
            -2.48  0.0  \\
            -2.47  0.0  \\
            -2.46  0.0  \\
            -2.45  0.0  \\
            -2.44  0.0  \\
            -2.43  0.0  \\
            -2.42  0.0  \\
            -2.41  0.0  \\
            -2.4  0.0  \\
            -2.39  0.0  \\
            -2.38  0.0  \\
            -2.37  0.0  \\
            -2.36  0.0  \\
            -2.35  0.0  \\
            -2.34  0.0  \\
            -2.33  0.0  \\
            -2.32  0.0  \\
            -2.31  0.0  \\
            -2.3  0.0  \\
            -2.29  0.0  \\
            -2.28  0.0  \\
            -2.27  0.0  \\
            -2.26  0.0  \\
            -2.25  0.0  \\
            -2.24  0.0  \\
            -2.23  0.0  \\
            -2.22  0.0  \\
            -2.21  0.0  \\
            -2.2  0.0  \\
            -2.19  0.0  \\
            -2.18  0.0  \\
            -2.17  0.0  \\
            -2.16  0.0  \\
            -2.15  0.0  \\
            -2.14  0.0  \\
            -2.13  0.0  \\
            -2.12  0.0  \\
            -2.11  0.0  \\
            -2.1  0.0  \\
            -2.09  0.0  \\
            -2.08  0.0  \\
            -2.07  0.0  \\
            -2.06  0.0  \\
            -2.05  0.0  \\
            -2.04  0.0  \\
            -2.03  0.0  \\
            -2.02  0.0  \\
            -2.01  0.0  \\
            -2.0  0.0  \\
            -1.99  0.0  \\
            -1.98  0.0  \\
            -1.97  0.0  \\
            -1.96  0.0  \\
            -1.95  0.0  \\
            -1.94  0.0  \\
            -1.93  0.0  \\
            -1.92  0.0  \\
            -1.91  0.0  \\
            -1.9  0.0  \\
            -1.89  0.0  \\
            -1.88  0.0  \\
            -1.87  0.0  \\
            -1.86  0.0  \\
            -1.85  0.0  \\
            -1.84  0.0  \\
            -1.83  0.0  \\
            -1.82  0.0  \\
            -1.81  0.0  \\
            -1.8  0.0  \\
            -1.79  0.0  \\
            -1.78  0.0  \\
            -1.77  0.0  \\
            -1.76  0.0  \\
            -1.75  0.0  \\
            -1.74  0.0  \\
            -1.73  0.0  \\
            -1.72  0.0  \\
            -1.71  0.0  \\
            -1.7  0.0  \\
            -1.69  0.0  \\
            -1.68  0.0  \\
            -1.67  0.0  \\
            -1.66  0.0  \\
            -1.65  0.0  \\
            -1.64  0.0  \\
            -1.63  0.0  \\
            -1.62  0.0  \\
            -1.61  0.0  \\
            -1.6  0.0  \\
            -1.59  0.0  \\
            -1.58  0.0  \\
            -1.57  0.0  \\
            -1.56  0.0  \\
            -1.55  0.0  \\
            -1.54  0.0  \\
            -1.53  0.0  \\
            -1.52  0.0  \\
            -1.51  0.0  \\
            -1.5  0.0  \\
            -1.49  0.0  \\
            -1.48  0.0  \\
            -1.47  0.0  \\
            -1.46  0.0  \\
            -1.45  0.0  \\
            -1.44  0.0  \\
            -1.43  0.0  \\
            -1.42  0.0  \\
            -1.41  0.0  \\
            -1.4  0.0  \\
            -1.39  0.0  \\
            -1.38  0.0  \\
            -1.37  0.0  \\
            -1.36  0.0  \\
            -1.35  0.0  \\
            -1.34  0.0  \\
            -1.33  0.0  \\
            -1.32  0.0  \\
            -1.31  0.0  \\
            -1.3  0.0  \\
            -1.29  0.0  \\
            -1.28  0.0  \\
            -1.27  0.0  \\
            -1.26  0.0  \\
            -1.25  0.0  \\
            -1.24  0.0  \\
            -1.23  0.0  \\
            -1.22  0.0  \\
            -1.21  0.0  \\
            -1.2  0.0  \\
            -1.19  0.0  \\
            -1.18  0.0  \\
            -1.17  0.0  \\
            -1.16  0.0  \\
            -1.15  0.0  \\
            -1.14  0.0  \\
            -1.13  0.0  \\
            -1.12  0.0  \\
            -1.11  0.0  \\
            -1.1  0.0  \\
            -1.09  0.0  \\
            -1.08  0.0  \\
            -1.07  0.0  \\
            -1.06  0.0  \\
            -1.05  0.0  \\
            -1.04  0.0  \\
            -1.03  0.0  \\
            -1.02  0.0  \\
            -1.01  0.0  \\
            -1.0  0.0  \\
            -0.99  0.0025000000000000022  \\
            -0.98  0.0050000000000000044  \\
            -0.97  0.007500000000000007  \\
            -0.96  0.010000000000000009  \\
            -0.95  0.012500000000000011  \\
            -0.94  0.015000000000000013  \\
            -0.93  0.017499999999999988  \\
            -0.92  0.01999999999999999  \\
            -0.91  0.022499999999999992  \\
            -0.9  0.024999999999999994  \\
            -0.89  0.027499999999999997  \\
            -0.88  0.03  \\
            -0.87  0.0325  \\
            -0.86  0.035  \\
            -0.85  0.037500000000000006  \\
            -0.84  0.04000000000000001  \\
            -0.83  0.04250000000000001  \\
            -0.82  0.04500000000000001  \\
            -0.81  0.04749999999999999  \\
            -0.8  0.04999999999999999  \\
            -0.79  0.05249999999999999  \\
            -0.78  0.05499999999999999  \\
            -0.77  0.057499999999999996  \\
            -0.76  0.06  \\
            -0.75  0.0625  \\
            -0.74  0.065  \\
            -0.73  0.0675  \\
            -0.72  0.07  \\
            -0.71  0.07250000000000001  \\
            -0.7  0.07500000000000001  \\
            -0.69  0.07750000000000001  \\
            -0.68  0.07999999999999999  \\
            -0.67  0.08249999999999999  \\
            -0.66  0.08499999999999999  \\
            -0.65  0.0875  \\
            -0.64  0.09  \\
            -0.63  0.0925  \\
            -0.62  0.095  \\
            -0.61  0.0975  \\
            -0.6  0.1  \\
            -0.59  0.10250000000000001  \\
            -0.58  0.10500000000000001  \\
            -0.57  0.10750000000000001  \\
            -0.56  0.10999999999999999  \\
            -0.55  0.11249999999999999  \\
            -0.54  0.11499999999999999  \\
            -0.53  0.1175  \\
            -0.52  0.12  \\
            -0.51  0.1225  \\
            -0.5  0.125  \\
            -0.49  0.1275  \\
            -0.48  0.13  \\
            -0.47  0.1325  \\
            -0.46  0.135  \\
            -0.45  0.1375  \\
            -0.44  0.14  \\
            -0.43  0.14250000000000002  \\
            -0.42  0.14500000000000002  \\
            -0.41  0.14750000000000002  \\
            -0.4  0.15  \\
            -0.39  0.1525  \\
            -0.38  0.155  \\
            -0.37  0.1575  \\
            -0.36  0.16  \\
            -0.35  0.1625  \\
            -0.34  0.16499999999999998  \\
            -0.33  0.16749999999999998  \\
            -0.32  0.16999999999999998  \\
            -0.31  0.1725  \\
            -0.3  0.175  \\
            -0.29  0.1775  \\
            -0.28  0.18  \\
            -0.27  0.1825  \\
            -0.26  0.185  \\
            -0.25  0.1875  \\
            -0.24  0.19  \\
            -0.23  0.1925  \\
            -0.22  0.195  \\
            -0.21  0.1975  \\
            -0.2  0.2  \\
            -0.19  0.2025  \\
            -0.18  0.20500000000000002  \\
            -0.17  0.2075  \\
            -0.16  0.21  \\
            -0.15  0.2125  \\
            -0.14  0.215  \\
            -0.13  0.2175  \\
            -0.12  0.22  \\
            -0.11  0.2225  \\
            -0.1  0.225  \\
            -0.09  0.2275  \\
            -0.08  0.23  \\
            -0.07  0.23249999999999998  \\
            -0.06  0.235  \\
            -0.05  0.2375  \\
            -0.04  0.24  \\
            -0.03  0.2425  \\
            -0.02  0.245  \\
            -0.01  0.2475  \\
            0.0  0.25  \\
            0.01  0.2525  \\
            0.02  0.255  \\
            0.03  0.2575  \\
            0.04  0.26  \\
            0.05  0.2625  \\
            0.06  0.265  \\
            0.07  0.2675  \\
            0.08  0.27  \\
            0.09  0.2725  \\
            0.1  0.275  \\
            0.11  0.2775  \\
            0.12  0.28  \\
            0.13  0.2825  \\
            0.14  0.28500000000000003  \\
            0.15  0.2875  \\
            0.16  0.29  \\
            0.17  0.2925  \\
            0.18  0.295  \\
            0.19  0.2975  \\
            0.2  0.3  \\
            0.21  0.3025  \\
            0.22  0.305  \\
            0.23  0.3075  \\
            0.24  0.31  \\
            0.25  0.3125  \\
            0.26  0.315  \\
            0.27  0.3175  \\
            0.28  0.32  \\
            0.29  0.3225  \\
            0.3  0.325  \\
            0.31  0.3275  \\
            0.32  0.33  \\
            0.33  0.3325  \\
            0.34  0.335  \\
            0.35  0.3375  \\
            0.36  0.33999999999999997  \\
            0.37  0.3425  \\
            0.38  0.345  \\
            0.39  0.34750000000000003  \\
            0.4  0.35  \\
            0.41  0.3525  \\
            0.42  0.355  \\
            0.43  0.3575  \\
            0.44  0.36  \\
            0.45  0.3625  \\
            0.46  0.365  \\
            0.47  0.3675  \\
            0.48  0.37  \\
            0.49  0.3725  \\
            0.5  0.375  \\
            0.51  0.3775  \\
            0.52  0.38  \\
            0.53  0.3825  \\
            0.54  0.385  \\
            0.55  0.3875  \\
            0.56  0.39  \\
            0.57  0.39249999999999996  \\
            0.58  0.395  \\
            0.59  0.39749999999999996  \\
            0.6  0.4  \\
            0.61  0.40249999999999997  \\
            0.62  0.405  \\
            0.63  0.4075  \\
            0.64  0.41000000000000003  \\
            0.65  0.4125  \\
            0.66  0.41500000000000004  \\
            0.67  0.4175  \\
            0.68  0.42000000000000004  \\
            0.69  0.4225  \\
            0.7  0.425  \\
            0.71  0.4275  \\
            0.72  0.43  \\
            0.73  0.4325  \\
            0.74  0.435  \\
            0.75  0.4375  \\
            0.76  0.44  \\
            0.77  0.4425  \\
            0.78  0.445  \\
            0.79  0.4475  \\
            0.8  0.45  \\
            0.81  0.4525  \\
            0.82  0.45499999999999996  \\
            0.83  0.4575  \\
            0.84  0.45999999999999996  \\
            0.85  0.4625  \\
            0.86  0.46499999999999997  \\
            0.87  0.4675  \\
            0.88  0.47  \\
            0.89  0.47250000000000003  \\
            0.9  0.475  \\
            0.91  0.47750000000000004  \\
            0.92  0.48  \\
            0.93  0.48250000000000004  \\
            0.94  0.485  \\
            0.95  0.4875  \\
            0.96  0.49  \\
            0.97  0.4925  \\
            0.98  0.495  \\
            0.99  0.4975  \\
            1.0  0.5  \\
            1.01  0.4975  \\
            1.02  0.495  \\
            1.03  0.4925  \\
            1.04  0.49  \\
            1.05  0.4875  \\
            1.06  0.485  \\
            1.07  0.4825  \\
            1.08  0.48  \\
            1.09  0.4775  \\
            1.1  0.475  \\
            1.11  0.4725  \\
            1.12  0.47  \\
            1.13  0.4675  \\
            1.14  0.465  \\
            1.15  0.4625  \\
            1.16  0.46  \\
            1.17  0.4575  \\
            1.18  0.455  \\
            1.19  0.4525  \\
            1.2  0.45  \\
            1.21  0.4475  \\
            1.22  0.445  \\
            1.23  0.4425  \\
            1.24  0.44  \\
            1.25  0.4375  \\
            1.26  0.435  \\
            1.27  0.4325  \\
            1.28  0.43  \\
            1.29  0.4275  \\
            1.3  0.425  \\
            1.31  0.4225  \\
            1.32  0.42  \\
            1.33  0.4175  \\
            1.34  0.415  \\
            1.35  0.4125  \\
            1.36  0.41  \\
            1.37  0.4075  \\
            1.38  0.405  \\
            1.39  0.4025  \\
            1.4  0.4  \\
            1.41  0.3975  \\
            1.42  0.395  \\
            1.43  0.3925  \\
            1.44  0.39  \\
            1.45  0.3875  \\
            1.46  0.385  \\
            1.47  0.3825  \\
            1.48  0.38  \\
            1.49  0.3775  \\
            1.5  0.375  \\
            1.51  0.3725  \\
            1.52  0.37  \\
            1.53  0.3675  \\
            1.54  0.365  \\
            1.55  0.3625  \\
            1.56  0.36  \\
            1.57  0.3575  \\
            1.58  0.355  \\
            1.59  0.3525  \\
            1.6  0.35  \\
            1.61  0.3475  \\
            1.62  0.345  \\
            1.63  0.3425  \\
            1.64  0.34  \\
            1.65  0.3375  \\
            1.66  0.335  \\
            1.67  0.3325  \\
            1.68  0.33  \\
            1.69  0.3275  \\
            1.7  0.325  \\
            1.71  0.3225  \\
            1.72  0.32  \\
            1.73  0.3175  \\
            1.74  0.315  \\
            1.75  0.3125  \\
            1.76  0.31  \\
            1.77  0.3075  \\
            1.78  0.305  \\
            1.79  0.3025  \\
            1.8  0.3  \\
            1.81  0.2975  \\
            1.82  0.295  \\
            1.83  0.2925  \\
            1.84  0.29  \\
            1.85  0.2875  \\
            1.86  0.285  \\
            1.87  0.2825  \\
            1.88  0.28  \\
            1.89  0.2775  \\
            1.9  0.275  \\
            1.91  0.2725  \\
            1.92  0.27  \\
            1.93  0.2675  \\
            1.94  0.265  \\
            1.95  0.2625  \\
            1.96  0.26  \\
            1.97  0.2575  \\
            1.98  0.255  \\
            1.99  0.2525  \\
            2.0  0.25  \\
            2.01  0.24750000000000005  \\
            2.02  0.245  \\
            2.03  0.24250000000000005  \\
            2.04  0.24  \\
            2.05  0.23750000000000004  \\
            2.06  0.235  \\
            2.07  0.23250000000000004  \\
            2.08  0.22999999999999998  \\
            2.09  0.22750000000000004  \\
            2.1  0.22499999999999998  \\
            2.11  0.22250000000000003  \\
            2.12  0.21999999999999997  \\
            2.13  0.21750000000000003  \\
            2.14  0.21499999999999997  \\
            2.15  0.21250000000000002  \\
            2.16  0.20999999999999996  \\
            2.17  0.20750000000000002  \\
            2.18  0.20499999999999996  \\
            2.19  0.2025  \\
            2.2  0.19999999999999996  \\
            2.21  0.1975  \\
            2.22  0.19499999999999995  \\
            2.23  0.1925  \\
            2.24  0.18999999999999995  \\
            2.25  0.1875  \\
            2.26  0.18500000000000005  \\
            2.27  0.1825  \\
            2.28  0.18000000000000005  \\
            2.29  0.1775  \\
            2.3  0.17500000000000004  \\
            2.31  0.1725  \\
            2.32  0.17000000000000004  \\
            2.33  0.16749999999999998  \\
            2.34  0.16500000000000004  \\
            2.35  0.16249999999999998  \\
            2.36  0.16000000000000003  \\
            2.37  0.15749999999999997  \\
            2.38  0.15500000000000003  \\
            2.39  0.15249999999999997  \\
            2.4  0.15000000000000002  \\
            2.41  0.14749999999999996  \\
            2.42  0.14500000000000002  \\
            2.43  0.14249999999999996  \\
            2.44  0.14  \\
            2.45  0.13749999999999996  \\
            2.46  0.135  \\
            2.47  0.13249999999999995  \\
            2.48  0.13  \\
            2.49  0.12749999999999995  \\
            2.5  0.125  \\
            2.51  0.12250000000000005  \\
            2.52  0.12  \\
            2.53  0.11750000000000005  \\
            2.54  0.11499999999999999  \\
            2.55  0.11250000000000004  \\
            2.56  0.10999999999999999  \\
            2.57  0.10750000000000004  \\
            2.58  0.10499999999999998  \\
            2.59  0.10250000000000004  \\
            2.6  0.09999999999999998  \\
            2.61  0.09750000000000003  \\
            2.62  0.09499999999999997  \\
            2.63  0.09250000000000003  \\
            2.64  0.08999999999999997  \\
            2.65  0.08750000000000002  \\
            2.66  0.08499999999999996  \\
            2.67  0.08250000000000002  \\
            2.68  0.07999999999999996  \\
            2.69  0.07750000000000001  \\
            2.7  0.07499999999999996  \\
            2.71  0.07250000000000001  \\
            2.72  0.06999999999999995  \\
            2.73  0.0675  \\
            2.74  0.06499999999999995  \\
            2.75  0.0625  \\
            2.76  0.06000000000000005  \\
            2.77  0.057499999999999996  \\
            2.78  0.05500000000000005  \\
            2.79  0.05249999999999999  \\
            2.8  0.050000000000000044  \\
            2.81  0.04749999999999999  \\
            2.82  0.04500000000000004  \\
            2.83  0.04249999999999998  \\
            2.84  0.040000000000000036  \\
            2.85  0.03749999999999998  \\
            2.86  0.03500000000000003  \\
            2.87  0.03249999999999997  \\
            2.88  0.030000000000000027  \\
            2.89  0.02749999999999997  \\
            2.9  0.025000000000000022  \\
            2.91  0.022499999999999964  \\
            2.92  0.020000000000000018  \\
            2.93  0.01749999999999996  \\
            2.94  0.015000000000000013  \\
            2.95  0.012499999999999956  \\
            2.96  0.010000000000000009  \\
            2.97  0.007499999999999951  \\
            2.98  0.0050000000000000044  \\
            2.99  0.0024999999999999467  \\
            3.0  0.0  \\
        }
        \closedcycle
        ;
    \addlegendentry {$y = 3$}
\end{axis}
\end{tikzpicture}

        \end{center}
        The following models are all calibrated but only the first one is ideal:
        \begin{center}
          \pgfplotsset{width=0.25\linewidth,height=0.1\textwidth,ytick={0,0.5,1}}
          \begin{tikzpicture}
\begin{groupplot}[group style={group size={4 by 1}, ylabels at={edge left}, yticklabels at={edge left}}, xlabel={$x$}, ylabel={$g_y(x)$}, ymin={0}, ymax={1}, enlarge x limits={false}, axis on top, no marks, title style={font={\small}}]
    \nextgroupplot[title={$g_y(X) = \mathbb{P}[Y = y \,|\, X]$}]
    \addplot+[fill, fill opacity={0.2}, thick]
        table[row sep={\\}]
        {
            \\
            -3.0  nan  \\
            -2.99  1.0  \\
            -2.98  1.0  \\
            -2.97  1.0  \\
            -2.96  1.0  \\
            -2.95  1.0  \\
            -2.94  1.0  \\
            -2.93  1.0  \\
            -2.92  1.0  \\
            -2.91  1.0  \\
            -2.9  1.0  \\
            -2.89  1.0  \\
            -2.88  1.0  \\
            -2.87  1.0  \\
            -2.86  1.0  \\
            -2.85  1.0  \\
            -2.84  1.0  \\
            -2.83  1.0  \\
            -2.82  1.0  \\
            -2.81  1.0  \\
            -2.8  1.0  \\
            -2.79  1.0  \\
            -2.78  1.0  \\
            -2.77  1.0  \\
            -2.76  1.0  \\
            -2.75  1.0  \\
            -2.74  1.0  \\
            -2.73  1.0  \\
            -2.72  1.0  \\
            -2.71  1.0  \\
            -2.7  1.0  \\
            -2.69  1.0  \\
            -2.68  1.0  \\
            -2.67  1.0  \\
            -2.66  1.0  \\
            -2.65  1.0  \\
            -2.64  1.0  \\
            -2.63  1.0  \\
            -2.62  1.0  \\
            -2.61  1.0  \\
            -2.6  1.0  \\
            -2.59  1.0  \\
            -2.58  1.0  \\
            -2.57  1.0  \\
            -2.56  1.0  \\
            -2.55  1.0  \\
            -2.54  1.0  \\
            -2.53  1.0  \\
            -2.52  1.0  \\
            -2.51  1.0  \\
            -2.5  1.0  \\
            -2.49  1.0  \\
            -2.48  1.0  \\
            -2.47  1.0  \\
            -2.46  1.0  \\
            -2.45  1.0  \\
            -2.44  1.0  \\
            -2.43  1.0  \\
            -2.42  1.0  \\
            -2.41  1.0  \\
            -2.4  1.0  \\
            -2.39  1.0  \\
            -2.38  1.0  \\
            -2.37  1.0  \\
            -2.36  1.0  \\
            -2.35  1.0  \\
            -2.34  1.0  \\
            -2.33  1.0  \\
            -2.32  1.0  \\
            -2.31  1.0  \\
            -2.3  1.0  \\
            -2.29  1.0  \\
            -2.28  1.0  \\
            -2.27  1.0  \\
            -2.26  1.0  \\
            -2.25  1.0  \\
            -2.24  1.0  \\
            -2.23  1.0  \\
            -2.22  1.0  \\
            -2.21  1.0  \\
            -2.2  1.0  \\
            -2.19  1.0  \\
            -2.18  1.0  \\
            -2.17  1.0  \\
            -2.16  1.0  \\
            -2.15  1.0  \\
            -2.14  1.0  \\
            -2.13  1.0  \\
            -2.12  1.0  \\
            -2.11  1.0  \\
            -2.1  1.0  \\
            -2.09  1.0  \\
            -2.08  1.0  \\
            -2.07  1.0  \\
            -2.06  1.0  \\
            -2.05  1.0  \\
            -2.04  1.0  \\
            -2.03  1.0  \\
            -2.02  1.0  \\
            -2.01  1.0  \\
            -2.0  1.0  \\
            -1.99  0.9901960784313726  \\
            -1.98  0.9807692307692307  \\
            -1.97  0.9716981132075472  \\
            -1.96  0.9629629629629629  \\
            -1.95  0.9545454545454545  \\
            -1.94  0.9464285714285714  \\
            -1.93  0.9385964912280701  \\
            -1.92  0.9310344827586207  \\
            -1.91  0.923728813559322  \\
            -1.9  0.9166666666666666  \\
            -1.89  0.9098360655737704  \\
            -1.88  0.9032258064516129  \\
            -1.87  0.8968253968253969  \\
            -1.86  0.8906250000000001  \\
            -1.85  0.8846153846153847  \\
            -1.84  0.8787878787878788  \\
            -1.83  0.873134328358209  \\
            -1.82  0.8676470588235294  \\
            -1.81  0.8623188405797102  \\
            -1.8  0.8571428571428572  \\
            -1.79  0.852112676056338  \\
            -1.78  0.8472222222222222  \\
            -1.77  0.8424657534246576  \\
            -1.76  0.8378378378378378  \\
            -1.75  0.8333333333333334  \\
            -1.74  0.8289473684210527  \\
            -1.73  0.8246753246753247  \\
            -1.72  0.8205128205128205  \\
            -1.71  0.8164556962025317  \\
            -1.7  0.8125  \\
            -1.69  0.8086419753086419  \\
            -1.68  0.8048780487804877  \\
            -1.67  0.8012048192771084  \\
            -1.66  0.7976190476190476  \\
            -1.65  0.7941176470588235  \\
            -1.64  0.7906976744186046  \\
            -1.63  0.7873563218390804  \\
            -1.62  0.7840909090909092  \\
            -1.61  0.7808988764044944  \\
            -1.6  0.7777777777777778  \\
            -1.59  0.7747252747252747  \\
            -1.58  0.7717391304347826  \\
            -1.57  0.7688172043010753  \\
            -1.56  0.7659574468085106  \\
            -1.55  0.7631578947368421  \\
            -1.54  0.7604166666666666  \\
            -1.53  0.7577319587628866  \\
            -1.52  0.7551020408163265  \\
            -1.51  0.7525252525252525  \\
            -1.5  0.75  \\
            -1.49  0.7475247524752475  \\
            -1.48  0.7450980392156863  \\
            -1.47  0.7427184466019418  \\
            -1.46  0.7403846153846154  \\
            -1.45  0.7380952380952381  \\
            -1.44  0.7358490566037735  \\
            -1.43  0.7336448598130841  \\
            -1.42  0.7314814814814815  \\
            -1.41  0.7293577981651376  \\
            -1.4  0.7272727272727273  \\
            -1.39  0.7252252252252253  \\
            -1.38  0.7232142857142857  \\
            -1.37  0.7212389380530974  \\
            -1.36  0.7192982456140351  \\
            -1.35  0.717391304347826  \\
            -1.34  0.7155172413793104  \\
            -1.33  0.7136752136752137  \\
            -1.32  0.711864406779661  \\
            -1.31  0.7100840336134454  \\
            -1.3  0.7083333333333334  \\
            -1.29  0.7066115702479339  \\
            -1.28  0.7049180327868853  \\
            -1.27  0.7032520325203252  \\
            -1.26  0.7016129032258065  \\
            -1.25  0.7  \\
            -1.24  0.6984126984126984  \\
            -1.23  0.6968503937007874  \\
            -1.22  0.6953125  \\
            -1.21  0.6937984496124031  \\
            -1.2  0.6923076923076923  \\
            -1.19  0.6908396946564885  \\
            -1.18  0.6893939393939393  \\
            -1.17  0.6879699248120301  \\
            -1.16  0.6865671641791045  \\
            -1.15  0.6851851851851851  \\
            -1.14  0.6838235294117647  \\
            -1.13  0.6824817518248175  \\
            -1.12  0.6811594202898551  \\
            -1.11  0.6798561151079137  \\
            -1.1  0.6785714285714286  \\
            -1.09  0.6773049645390071  \\
            -1.08  0.676056338028169  \\
            -1.07  0.6748251748251748  \\
            -1.06  0.6736111111111112  \\
            -1.05  0.6724137931034483  \\
            -1.04  0.6712328767123288  \\
            -1.03  0.6700680272108843  \\
            -1.02  0.668918918918919  \\
            -1.01  0.6677852348993288  \\
            -1.0  0.6666666666666666  \\
            -0.99  0.6611295681063123  \\
            -0.98  0.6556291390728477  \\
            -0.97  0.6501650165016502  \\
            -0.96  0.6447368421052632  \\
            -0.95  0.639344262295082  \\
            -0.94  0.6339869281045751  \\
            -0.93  0.6286644951140066  \\
            -0.92  0.6233766233766234  \\
            -0.91  0.6181229773462784  \\
            -0.9  0.6129032258064515  \\
            -0.89  0.6077170418006431  \\
            -0.88  0.6025641025641025  \\
            -0.87  0.5974440894568691  \\
            -0.86  0.5923566878980892  \\
            -0.85  0.5873015873015873  \\
            -0.84  0.5822784810126581  \\
            -0.83  0.5772870662460569  \\
            -0.82  0.5723270440251571  \\
            -0.81  0.5673981191222571  \\
            -0.8  0.5625  \\
            -0.79  0.5576323987538941  \\
            -0.78  0.5527950310559007  \\
            -0.77  0.5479876160990712  \\
            -0.76  0.5432098765432098  \\
            -0.75  0.5384615384615384  \\
            -0.74  0.5337423312883436  \\
            -0.73  0.5290519877675841  \\
            -0.72  0.5243902439024389  \\
            -0.71  0.5197568389057751  \\
            -0.7  0.5151515151515151  \\
            -0.69  0.5105740181268882  \\
            -0.68  0.5060240963855422  \\
            -0.67  0.5015015015015015  \\
            -0.66  0.4970059880239522  \\
            -0.65  0.4925373134328358  \\
            -0.64  0.48809523809523814  \\
            -0.63  0.48367952522255186  \\
            -0.62  0.4792899408284024  \\
            -0.61  0.47492625368731556  \\
            -0.6  0.4705882352941177  \\
            -0.59  0.46627565982404684  \\
            -0.58  0.4619883040935673  \\
            -0.57  0.4577259475218658  \\
            -0.56  0.4534883720930233  \\
            -0.55  0.4492753623188406  \\
            -0.54  0.44508670520231214  \\
            -0.53  0.4409221902017291  \\
            -0.52  0.4367816091954023  \\
            -0.51  0.43266475644699137  \\
            -0.5  0.42857142857142855  \\
            -0.49  0.42450142450142453  \\
            -0.48  0.42045454545454547  \\
            -0.47  0.4164305949008498  \\
            -0.46  0.4124293785310734  \\
            -0.45  0.4084507042253521  \\
            -0.44  0.4044943820224719  \\
            -0.43  0.4005602240896358  \\
            -0.42  0.3966480446927374  \\
            -0.41  0.39275766016713093  \\
            -0.4  0.38888888888888884  \\
            -0.39  0.38504155124653744  \\
            -0.38  0.38121546961325964  \\
            -0.37  0.37741046831955927  \\
            -0.36  0.3736263736263736  \\
            -0.35  0.36986301369863017  \\
            -0.34  0.366120218579235  \\
            -0.33  0.36239782016348776  \\
            -0.32  0.3586956521739131  \\
            -0.31  0.3550135501355014  \\
            -0.3  0.35135135135135137  \\
            -0.29  0.3477088948787062  \\
            -0.28  0.34408602150537637  \\
            -0.27  0.34048257372654156  \\
            -0.26  0.3368983957219251  \\
            -0.25  0.3333333333333333  \\
            -0.24  0.3297872340425532  \\
            -0.23  0.32625994694960214  \\
            -0.22  0.32275132275132273  \\
            -0.21  0.31926121372031663  \\
            -0.2  0.3157894736842105  \\
            -0.19  0.3123359580052493  \\
            -0.18  0.30890052356020936  \\
            -0.17  0.3054830287206266  \\
            -0.16  0.3020833333333333  \\
            -0.15  0.2987012987012987  \\
            -0.14  0.2953367875647669  \\
            -0.13  0.2919896640826873  \\
            -0.12  0.288659793814433  \\
            -0.11  0.2853470437017995  \\
            -0.1  0.2820512820512821  \\
            -0.09  0.27877237851662406  \\
            -0.08  0.2755102040816327  \\
            -0.07  0.27226463104325704  \\
            -0.06  0.2690355329949239  \\
            -0.05  0.26582278481012656  \\
            -0.04  0.26262626262626265  \\
            -0.03  0.2594458438287154  \\
            -0.02  0.2562814070351759  \\
            -0.01  0.2531328320802005  \\
            0.0  0.25  \\
            0.01  0.24812030075187969  \\
            0.02  0.24623115577889446  \\
            0.03  0.2443324937027708  \\
            0.04  0.24242424242424243  \\
            0.05  0.24050632911392403  \\
            0.06  0.23857868020304568  \\
            0.07  0.2366412213740458  \\
            0.08  0.23469387755102042  \\
            0.09  0.23273657289002558  \\
            0.1  0.23076923076923078  \\
            0.11  0.22879177377892032  \\
            0.12  0.2268041237113402  \\
            0.13  0.22480620155038758  \\
            0.14  0.2227979274611399  \\
            0.15  0.22077922077922077  \\
            0.16  0.21875  \\
            0.17  0.216710182767624  \\
            0.18  0.21465968586387435  \\
            0.19  0.2125984251968504  \\
            0.2  0.2105263157894737  \\
            0.21  0.20844327176781002  \\
            0.22  0.20634920634920634  \\
            0.23  0.20424403183023873  \\
            0.24  0.2021276595744681  \\
            0.25  0.2  \\
            0.26  0.1978609625668449  \\
            0.27  0.19571045576407506  \\
            0.28  0.1935483870967742  \\
            0.29  0.19137466307277629  \\
            0.3  0.18918918918918917  \\
            0.31  0.18699186991869918  \\
            0.32  0.18478260869565216  \\
            0.33  0.18256130790190733  \\
            0.34  0.180327868852459  \\
            0.35  0.17808219178082194  \\
            0.36  0.1758241758241758  \\
            0.37  0.17355371900826447  \\
            0.38  0.1712707182320442  \\
            0.39  0.16897506925207756  \\
            0.4  0.16666666666666666  \\
            0.41  0.16434540389972147  \\
            0.42  0.16201117318435757  \\
            0.43  0.1596638655462185  \\
            0.44  0.15730337078651688  \\
            0.45  0.15492957746478875  \\
            0.46  0.15254237288135594  \\
            0.47  0.15014164305949007  \\
            0.48  0.14772727272727273  \\
            0.49  0.1452991452991453  \\
            0.5  0.14285714285714285  \\
            0.51  0.14040114613180515  \\
            0.52  0.13793103448275862  \\
            0.53  0.13544668587896252  \\
            0.54  0.1329479768786127  \\
            0.55  0.13043478260869562  \\
            0.56  0.12790697674418602  \\
            0.57  0.12536443148688048  \\
            0.58  0.12280701754385967  \\
            0.59  0.12023460410557185  \\
            0.6  0.11764705882352942  \\
            0.61  0.11504424778761062  \\
            0.62  0.11242603550295859  \\
            0.63  0.10979228486646883  \\
            0.64  0.10714285714285714  \\
            0.65  0.1044776119402985  \\
            0.66  0.10179640718562874  \\
            0.67  0.09909909909909909  \\
            0.68  0.09638554216867469  \\
            0.69  0.09365558912386708  \\
            0.7  0.09090909090909093  \\
            0.71  0.08814589665653497  \\
            0.72  0.08536585365853659  \\
            0.73  0.08256880733944955  \\
            0.74  0.07975460122699388  \\
            0.75  0.07692307692307693  \\
            0.76  0.07407407407407407  \\
            0.77  0.07120743034055727  \\
            0.78  0.06832298136645962  \\
            0.79  0.06542056074766354  \\
            0.8  0.062499999999999986  \\
            0.81  0.05956112852664575  \\
            0.82  0.056603773584905676  \\
            0.83  0.053627760252365944  \\
            0.84  0.05063291139240507  \\
            0.85  0.04761904761904763  \\
            0.86  0.0445859872611465  \\
            0.87  0.0415335463258786  \\
            0.88  0.03846153846153846  \\
            0.89  0.03536977491961415  \\
            0.9  0.032258064516129024  \\
            0.91  0.029126213592233  \\
            0.92  0.025974025974025962  \\
            0.93  0.022801302931596077  \\
            0.94  0.01960784313725492  \\
            0.95  0.016393442622950834  \\
            0.96  0.013157894736842117  \\
            0.97  0.00990099009900991  \\
            0.98  0.006622516556291397  \\
            0.99  0.0033222591362126277  \\
            1.0  0.0  \\
            1.01  0.0  \\
            1.02  0.0  \\
            1.03  0.0  \\
            1.04  0.0  \\
            1.05  0.0  \\
            1.06  0.0  \\
            1.07  0.0  \\
            1.08  0.0  \\
            1.09  0.0  \\
            1.1  0.0  \\
            1.11  0.0  \\
            1.12  0.0  \\
            1.13  0.0  \\
            1.14  0.0  \\
            1.15  0.0  \\
            1.16  0.0  \\
            1.17  0.0  \\
            1.18  0.0  \\
            1.19  0.0  \\
            1.2  0.0  \\
            1.21  0.0  \\
            1.22  0.0  \\
            1.23  0.0  \\
            1.24  0.0  \\
            1.25  0.0  \\
            1.26  0.0  \\
            1.27  0.0  \\
            1.28  0.0  \\
            1.29  0.0  \\
            1.3  0.0  \\
            1.31  0.0  \\
            1.32  0.0  \\
            1.33  0.0  \\
            1.34  0.0  \\
            1.35  0.0  \\
            1.36  0.0  \\
            1.37  0.0  \\
            1.38  0.0  \\
            1.39  0.0  \\
            1.4  0.0  \\
            1.41  0.0  \\
            1.42  0.0  \\
            1.43  0.0  \\
            1.44  0.0  \\
            1.45  0.0  \\
            1.46  0.0  \\
            1.47  0.0  \\
            1.48  0.0  \\
            1.49  0.0  \\
            1.5  0.0  \\
            1.51  0.0  \\
            1.52  0.0  \\
            1.53  0.0  \\
            1.54  0.0  \\
            1.55  0.0  \\
            1.56  0.0  \\
            1.57  0.0  \\
            1.58  0.0  \\
            1.59  0.0  \\
            1.6  0.0  \\
            1.61  0.0  \\
            1.62  0.0  \\
            1.63  0.0  \\
            1.64  0.0  \\
            1.65  0.0  \\
            1.66  0.0  \\
            1.67  0.0  \\
            1.68  0.0  \\
            1.69  0.0  \\
            1.7  0.0  \\
            1.71  0.0  \\
            1.72  0.0  \\
            1.73  0.0  \\
            1.74  0.0  \\
            1.75  0.0  \\
            1.76  0.0  \\
            1.77  0.0  \\
            1.78  0.0  \\
            1.79  0.0  \\
            1.8  0.0  \\
            1.81  0.0  \\
            1.82  0.0  \\
            1.83  0.0  \\
            1.84  0.0  \\
            1.85  0.0  \\
            1.86  0.0  \\
            1.87  0.0  \\
            1.88  0.0  \\
            1.89  0.0  \\
            1.9  0.0  \\
            1.91  0.0  \\
            1.92  0.0  \\
            1.93  0.0  \\
            1.94  0.0  \\
            1.95  0.0  \\
            1.96  0.0  \\
            1.97  0.0  \\
            1.98  0.0  \\
            1.99  0.0  \\
            2.0  0.0  \\
            2.01  0.0  \\
            2.02  0.0  \\
            2.03  0.0  \\
            2.04  0.0  \\
            2.05  0.0  \\
            2.06  0.0  \\
            2.07  0.0  \\
            2.08  0.0  \\
            2.09  0.0  \\
            2.1  0.0  \\
            2.11  0.0  \\
            2.12  0.0  \\
            2.13  0.0  \\
            2.14  0.0  \\
            2.15  0.0  \\
            2.16  0.0  \\
            2.17  0.0  \\
            2.18  0.0  \\
            2.19  0.0  \\
            2.2  0.0  \\
            2.21  0.0  \\
            2.22  0.0  \\
            2.23  0.0  \\
            2.24  0.0  \\
            2.25  0.0  \\
            2.26  0.0  \\
            2.27  0.0  \\
            2.28  0.0  \\
            2.29  0.0  \\
            2.3  0.0  \\
            2.31  0.0  \\
            2.32  0.0  \\
            2.33  0.0  \\
            2.34  0.0  \\
            2.35  0.0  \\
            2.36  0.0  \\
            2.37  0.0  \\
            2.38  0.0  \\
            2.39  0.0  \\
            2.4  0.0  \\
            2.41  0.0  \\
            2.42  0.0  \\
            2.43  0.0  \\
            2.44  0.0  \\
            2.45  0.0  \\
            2.46  0.0  \\
            2.47  0.0  \\
            2.48  0.0  \\
            2.49  0.0  \\
            2.5  0.0  \\
            2.51  0.0  \\
            2.52  0.0  \\
            2.53  0.0  \\
            2.54  0.0  \\
            2.55  0.0  \\
            2.56  0.0  \\
            2.57  0.0  \\
            2.58  0.0  \\
            2.59  0.0  \\
            2.6  0.0  \\
            2.61  0.0  \\
            2.62  0.0  \\
            2.63  0.0  \\
            2.64  0.0  \\
            2.65  0.0  \\
            2.66  0.0  \\
            2.67  0.0  \\
            2.68  0.0  \\
            2.69  0.0  \\
            2.7  0.0  \\
            2.71  0.0  \\
            2.72  0.0  \\
            2.73  0.0  \\
            2.74  0.0  \\
            2.75  0.0  \\
            2.76  0.0  \\
            2.77  0.0  \\
            2.78  0.0  \\
            2.79  0.0  \\
            2.8  0.0  \\
            2.81  0.0  \\
            2.82  0.0  \\
            2.83  0.0  \\
            2.84  0.0  \\
            2.85  0.0  \\
            2.86  0.0  \\
            2.87  0.0  \\
            2.88  0.0  \\
            2.89  0.0  \\
            2.9  0.0  \\
            2.91  0.0  \\
            2.92  0.0  \\
            2.93  0.0  \\
            2.94  0.0  \\
            2.95  0.0  \\
            2.96  0.0  \\
            2.97  0.0  \\
            2.98  0.0  \\
            2.99  0.0  \\
            3.0  nan  \\
        }
        \closedcycle
        ;
    \addplot+[fill, fill opacity={0.2}, thick]
        table[row sep={\\}]
        {
            \\
            -3.0  nan  \\
            -2.99  0.0  \\
            -2.98  0.0  \\
            -2.97  0.0  \\
            -2.96  0.0  \\
            -2.95  0.0  \\
            -2.94  0.0  \\
            -2.93  0.0  \\
            -2.92  0.0  \\
            -2.91  0.0  \\
            -2.9  0.0  \\
            -2.89  0.0  \\
            -2.88  0.0  \\
            -2.87  0.0  \\
            -2.86  0.0  \\
            -2.85  0.0  \\
            -2.84  0.0  \\
            -2.83  0.0  \\
            -2.82  0.0  \\
            -2.81  0.0  \\
            -2.8  0.0  \\
            -2.79  0.0  \\
            -2.78  0.0  \\
            -2.77  0.0  \\
            -2.76  0.0  \\
            -2.75  0.0  \\
            -2.74  0.0  \\
            -2.73  0.0  \\
            -2.72  0.0  \\
            -2.71  0.0  \\
            -2.7  0.0  \\
            -2.69  0.0  \\
            -2.68  0.0  \\
            -2.67  0.0  \\
            -2.66  0.0  \\
            -2.65  0.0  \\
            -2.64  0.0  \\
            -2.63  0.0  \\
            -2.62  0.0  \\
            -2.61  0.0  \\
            -2.6  0.0  \\
            -2.59  0.0  \\
            -2.58  0.0  \\
            -2.57  0.0  \\
            -2.56  0.0  \\
            -2.55  0.0  \\
            -2.54  0.0  \\
            -2.53  0.0  \\
            -2.52  0.0  \\
            -2.51  0.0  \\
            -2.5  0.0  \\
            -2.49  0.0  \\
            -2.48  0.0  \\
            -2.47  0.0  \\
            -2.46  0.0  \\
            -2.45  0.0  \\
            -2.44  0.0  \\
            -2.43  0.0  \\
            -2.42  0.0  \\
            -2.41  0.0  \\
            -2.4  0.0  \\
            -2.39  0.0  \\
            -2.38  0.0  \\
            -2.37  0.0  \\
            -2.36  0.0  \\
            -2.35  0.0  \\
            -2.34  0.0  \\
            -2.33  0.0  \\
            -2.32  0.0  \\
            -2.31  0.0  \\
            -2.3  0.0  \\
            -2.29  0.0  \\
            -2.28  0.0  \\
            -2.27  0.0  \\
            -2.26  0.0  \\
            -2.25  0.0  \\
            -2.24  0.0  \\
            -2.23  0.0  \\
            -2.22  0.0  \\
            -2.21  0.0  \\
            -2.2  0.0  \\
            -2.19  0.0  \\
            -2.18  0.0  \\
            -2.17  0.0  \\
            -2.16  0.0  \\
            -2.15  0.0  \\
            -2.14  0.0  \\
            -2.13  0.0  \\
            -2.12  0.0  \\
            -2.11  0.0  \\
            -2.1  0.0  \\
            -2.09  0.0  \\
            -2.08  0.0  \\
            -2.07  0.0  \\
            -2.06  0.0  \\
            -2.05  0.0  \\
            -2.04  0.0  \\
            -2.03  0.0  \\
            -2.02  0.0  \\
            -2.01  0.0  \\
            -2.0  0.0  \\
            -1.99  0.00980392156862746  \\
            -1.98  0.019230769230769246  \\
            -1.97  0.028301886792452855  \\
            -1.96  0.03703703703703707  \\
            -1.95  0.04545454545454549  \\
            -1.94  0.05357142857142862  \\
            -1.93  0.06140350877192987  \\
            -1.92  0.06896551724137936  \\
            -1.91  0.07627118644067803  \\
            -1.9  0.0833333333333334  \\
            -1.89  0.09016393442622958  \\
            -1.88  0.09677419354838716  \\
            -1.87  0.10317460317460311  \\
            -1.86  0.10937499999999994  \\
            -1.85  0.11538461538461534  \\
            -1.84  0.12121212121212116  \\
            -1.83  0.126865671641791  \\
            -1.82  0.13235294117647056  \\
            -1.81  0.13768115942028983  \\
            -1.8  0.14285714285714282  \\
            -1.79  0.14788732394366194  \\
            -1.78  0.15277777777777776  \\
            -1.77  0.15753424657534246  \\
            -1.76  0.16216216216216217  \\
            -1.75  0.16666666666666666  \\
            -1.74  0.17105263157894737  \\
            -1.73  0.17532467532467533  \\
            -1.72  0.1794871794871795  \\
            -1.71  0.18354430379746836  \\
            -1.7  0.18750000000000003  \\
            -1.69  0.19135802469135804  \\
            -1.68  0.19512195121951223  \\
            -1.67  0.1987951807228916  \\
            -1.66  0.2023809523809524  \\
            -1.65  0.2058823529411765  \\
            -1.64  0.2093023255813954  \\
            -1.63  0.21264367816091959  \\
            -1.62  0.21590909090909088  \\
            -1.61  0.2191011235955056  \\
            -1.6  0.22222222222222218  \\
            -1.59  0.22527472527472525  \\
            -1.58  0.22826086956521738  \\
            -1.57  0.23118279569892472  \\
            -1.56  0.23404255319148934  \\
            -1.55  0.23684210526315788  \\
            -1.54  0.23958333333333331  \\
            -1.53  0.2422680412371134  \\
            -1.52  0.24489795918367346  \\
            -1.51  0.2474747474747475  \\
            -1.5  0.25  \\
            -1.49  0.2524752475247525  \\
            -1.48  0.2549019607843137  \\
            -1.47  0.25728155339805825  \\
            -1.46  0.25961538461538464  \\
            -1.45  0.2619047619047619  \\
            -1.44  0.2641509433962264  \\
            -1.43  0.2663551401869159  \\
            -1.42  0.26851851851851855  \\
            -1.41  0.27064220183486243  \\
            -1.4  0.27272727272727276  \\
            -1.39  0.2747747747747748  \\
            -1.38  0.2767857142857143  \\
            -1.37  0.27876106194690264  \\
            -1.36  0.2807017543859649  \\
            -1.35  0.2826086956521739  \\
            -1.34  0.2844827586206896  \\
            -1.33  0.2863247863247863  \\
            -1.32  0.288135593220339  \\
            -1.31  0.2899159663865546  \\
            -1.3  0.2916666666666667  \\
            -1.29  0.2933884297520661  \\
            -1.28  0.29508196721311475  \\
            -1.27  0.2967479674796748  \\
            -1.26  0.29838709677419356  \\
            -1.25  0.3  \\
            -1.24  0.30158730158730157  \\
            -1.23  0.3031496062992126  \\
            -1.22  0.3046875  \\
            -1.21  0.3062015503875969  \\
            -1.2  0.3076923076923077  \\
            -1.19  0.30916030534351147  \\
            -1.18  0.3106060606060606  \\
            -1.17  0.31203007518796994  \\
            -1.16  0.31343283582089554  \\
            -1.15  0.3148148148148148  \\
            -1.14  0.3161764705882353  \\
            -1.13  0.3175182481751825  \\
            -1.12  0.3188405797101449  \\
            -1.11  0.32014388489208634  \\
            -1.1  0.3214285714285714  \\
            -1.09  0.3226950354609929  \\
            -1.08  0.323943661971831  \\
            -1.07  0.32517482517482516  \\
            -1.06  0.3263888888888889  \\
            -1.05  0.3275862068965517  \\
            -1.04  0.3287671232876712  \\
            -1.03  0.3299319727891156  \\
            -1.02  0.3310810810810811  \\
            -1.01  0.33221476510067116  \\
            -1.0  0.3333333333333333  \\
            -0.99  0.3355481727574751  \\
            -0.98  0.33774834437086093  \\
            -0.97  0.33993399339933994  \\
            -0.96  0.34210526315789475  \\
            -0.95  0.34426229508196726  \\
            -0.94  0.3464052287581699  \\
            -0.93  0.3485342019543974  \\
            -0.92  0.35064935064935066  \\
            -0.91  0.35275080906148865  \\
            -0.9  0.3548387096774194  \\
            -0.89  0.35691318327974275  \\
            -0.88  0.358974358974359  \\
            -0.87  0.36102236421725237  \\
            -0.86  0.36305732484076436  \\
            -0.85  0.36507936507936506  \\
            -0.84  0.3670886075949367  \\
            -0.83  0.36908517350157727  \\
            -0.82  0.37106918238993714  \\
            -0.81  0.3730407523510972  \\
            -0.8  0.37499999999999994  \\
            -0.79  0.37694704049844235  \\
            -0.78  0.37888198757763975  \\
            -0.77  0.38080495356037153  \\
            -0.76  0.38271604938271603  \\
            -0.75  0.38461538461538464  \\
            -0.74  0.3865030674846626  \\
            -0.73  0.38837920489296635  \\
            -0.72  0.39024390243902435  \\
            -0.71  0.39209726443769  \\
            -0.7  0.393939393939394  \\
            -0.69  0.39577039274924475  \\
            -0.68  0.3975903614457831  \\
            -0.67  0.3993993993993994  \\
            -0.66  0.40119760479041916  \\
            -0.65  0.40298507462686567  \\
            -0.64  0.40476190476190477  \\
            -0.63  0.4065281899109792  \\
            -0.62  0.40828402366863903  \\
            -0.61  0.41002949852507375  \\
            -0.6  0.4117647058823529  \\
            -0.59  0.41348973607038125  \\
            -0.58  0.4152046783625731  \\
            -0.57  0.41690962099125367  \\
            -0.56  0.41860465116279066  \\
            -0.55  0.42028985507246375  \\
            -0.54  0.42196531791907516  \\
            -0.53  0.4236311239193084  \\
            -0.52  0.42528735632183906  \\
            -0.51  0.4269340974212034  \\
            -0.5  0.42857142857142855  \\
            -0.49  0.43019943019943024  \\
            -0.48  0.4318181818181818  \\
            -0.47  0.43342776203966005  \\
            -0.46  0.4350282485875706  \\
            -0.45  0.4366197183098592  \\
            -0.44  0.43820224719101125  \\
            -0.43  0.43977591036414565  \\
            -0.42  0.44134078212290506  \\
            -0.41  0.44289693593314766  \\
            -0.4  0.4444444444444445  \\
            -0.39  0.445983379501385  \\
            -0.38  0.44751381215469616  \\
            -0.37  0.4490358126721763  \\
            -0.36  0.45054945054945056  \\
            -0.35  0.4520547945205479  \\
            -0.34  0.45355191256830596  \\
            -0.33  0.4550408719346049  \\
            -0.32  0.4565217391304348  \\
            -0.31  0.45799457994579945  \\
            -0.3  0.45945945945945943  \\
            -0.29  0.4609164420485175  \\
            -0.28  0.4623655913978495  \\
            -0.27  0.46380697050938335  \\
            -0.26  0.4652406417112299  \\
            -0.25  0.4666666666666667  \\
            -0.24  0.46808510638297873  \\
            -0.23  0.46949602122015915  \\
            -0.22  0.4708994708994709  \\
            -0.21  0.47229551451187335  \\
            -0.2  0.4736842105263158  \\
            -0.19  0.47506561679790027  \\
            -0.18  0.47643979057591623  \\
            -0.17  0.47780678851174935  \\
            -0.16  0.4791666666666667  \\
            -0.15  0.4805194805194805  \\
            -0.14  0.48186528497409326  \\
            -0.13  0.48320413436692505  \\
            -0.12  0.4845360824742268  \\
            -0.11  0.4858611825192802  \\
            -0.1  0.48717948717948717  \\
            -0.09  0.48849104859335035  \\
            -0.08  0.4897959183673469  \\
            -0.07  0.4910941475826972  \\
            -0.06  0.49238578680203043  \\
            -0.05  0.49367088607594933  \\
            -0.04  0.494949494949495  \\
            -0.03  0.4962216624685139  \\
            -0.02  0.49748743718592964  \\
            -0.01  0.49874686716791977  \\
            0.0  0.5  \\
            0.01  0.49874686716791977  \\
            0.02  0.49748743718592964  \\
            0.03  0.4962216624685139  \\
            0.04  0.494949494949495  \\
            0.05  0.49367088607594933  \\
            0.06  0.49238578680203043  \\
            0.07  0.4910941475826972  \\
            0.08  0.4897959183673469  \\
            0.09  0.48849104859335035  \\
            0.1  0.48717948717948717  \\
            0.11  0.48586118251928023  \\
            0.12  0.4845360824742268  \\
            0.13  0.48320413436692505  \\
            0.14  0.48186528497409326  \\
            0.15  0.4805194805194805  \\
            0.16  0.4791666666666667  \\
            0.17  0.47780678851174935  \\
            0.18  0.47643979057591623  \\
            0.19  0.47506561679790027  \\
            0.2  0.4736842105263158  \\
            0.21  0.47229551451187335  \\
            0.22  0.4708994708994709  \\
            0.23  0.46949602122015915  \\
            0.24  0.46808510638297873  \\
            0.25  0.4666666666666667  \\
            0.26  0.4652406417112299  \\
            0.27  0.46380697050938335  \\
            0.28  0.4623655913978495  \\
            0.29  0.4609164420485175  \\
            0.3  0.45945945945945943  \\
            0.31  0.45799457994579945  \\
            0.32  0.4565217391304348  \\
            0.33  0.4550408719346049  \\
            0.34  0.45355191256830596  \\
            0.35  0.4520547945205479  \\
            0.36  0.45054945054945056  \\
            0.37  0.4490358126721763  \\
            0.38  0.44751381215469616  \\
            0.39  0.445983379501385  \\
            0.4  0.4444444444444445  \\
            0.41  0.44289693593314766  \\
            0.42  0.44134078212290506  \\
            0.43  0.43977591036414565  \\
            0.44  0.43820224719101125  \\
            0.45  0.4366197183098592  \\
            0.46  0.4350282485875706  \\
            0.47  0.43342776203966005  \\
            0.48  0.4318181818181818  \\
            0.49  0.43019943019943024  \\
            0.5  0.42857142857142855  \\
            0.51  0.4269340974212034  \\
            0.52  0.42528735632183906  \\
            0.53  0.4236311239193084  \\
            0.54  0.42196531791907516  \\
            0.55  0.42028985507246375  \\
            0.56  0.41860465116279066  \\
            0.57  0.41690962099125367  \\
            0.58  0.4152046783625731  \\
            0.59  0.41348973607038125  \\
            0.6  0.4117647058823529  \\
            0.61  0.41002949852507375  \\
            0.62  0.40828402366863903  \\
            0.63  0.4065281899109792  \\
            0.64  0.40476190476190477  \\
            0.65  0.40298507462686567  \\
            0.66  0.40119760479041916  \\
            0.67  0.3993993993993994  \\
            0.68  0.3975903614457831  \\
            0.69  0.39577039274924475  \\
            0.7  0.393939393939394  \\
            0.71  0.39209726443769  \\
            0.72  0.39024390243902435  \\
            0.73  0.38837920489296635  \\
            0.74  0.3865030674846626  \\
            0.75  0.38461538461538464  \\
            0.76  0.38271604938271603  \\
            0.77  0.38080495356037153  \\
            0.78  0.37888198757763975  \\
            0.79  0.37694704049844235  \\
            0.8  0.37499999999999994  \\
            0.81  0.3730407523510972  \\
            0.82  0.37106918238993714  \\
            0.83  0.36908517350157727  \\
            0.84  0.3670886075949367  \\
            0.85  0.36507936507936506  \\
            0.86  0.36305732484076436  \\
            0.87  0.36102236421725237  \\
            0.88  0.358974358974359  \\
            0.89  0.35691318327974275  \\
            0.9  0.3548387096774194  \\
            0.91  0.35275080906148865  \\
            0.92  0.35064935064935066  \\
            0.93  0.3485342019543974  \\
            0.94  0.3464052287581699  \\
            0.95  0.34426229508196726  \\
            0.96  0.34210526315789475  \\
            0.97  0.33993399339933994  \\
            0.98  0.33774834437086093  \\
            0.99  0.3355481727574751  \\
            1.0  0.3333333333333333  \\
            1.01  0.33221476510067116  \\
            1.02  0.3310810810810811  \\
            1.03  0.3299319727891156  \\
            1.04  0.3287671232876712  \\
            1.05  0.3275862068965517  \\
            1.06  0.3263888888888889  \\
            1.07  0.32517482517482516  \\
            1.08  0.323943661971831  \\
            1.09  0.3226950354609929  \\
            1.1  0.3214285714285714  \\
            1.11  0.32014388489208634  \\
            1.12  0.3188405797101449  \\
            1.13  0.3175182481751825  \\
            1.14  0.3161764705882353  \\
            1.15  0.3148148148148148  \\
            1.16  0.31343283582089554  \\
            1.17  0.31203007518796994  \\
            1.18  0.3106060606060606  \\
            1.19  0.30916030534351147  \\
            1.2  0.3076923076923077  \\
            1.21  0.3062015503875969  \\
            1.22  0.3046875  \\
            1.23  0.3031496062992126  \\
            1.24  0.30158730158730157  \\
            1.25  0.3  \\
            1.26  0.29838709677419356  \\
            1.27  0.2967479674796748  \\
            1.28  0.29508196721311475  \\
            1.29  0.2933884297520661  \\
            1.3  0.2916666666666667  \\
            1.31  0.2899159663865546  \\
            1.32  0.288135593220339  \\
            1.33  0.2863247863247863  \\
            1.34  0.2844827586206896  \\
            1.35  0.2826086956521739  \\
            1.36  0.2807017543859649  \\
            1.37  0.27876106194690264  \\
            1.38  0.2767857142857143  \\
            1.39  0.2747747747747748  \\
            1.4  0.27272727272727276  \\
            1.41  0.27064220183486243  \\
            1.42  0.26851851851851855  \\
            1.43  0.2663551401869159  \\
            1.44  0.2641509433962264  \\
            1.45  0.2619047619047619  \\
            1.46  0.25961538461538464  \\
            1.47  0.25728155339805825  \\
            1.48  0.2549019607843137  \\
            1.49  0.2524752475247525  \\
            1.5  0.25  \\
            1.51  0.2474747474747475  \\
            1.52  0.24489795918367346  \\
            1.53  0.2422680412371134  \\
            1.54  0.23958333333333331  \\
            1.55  0.23684210526315788  \\
            1.56  0.23404255319148934  \\
            1.57  0.23118279569892472  \\
            1.58  0.22826086956521738  \\
            1.59  0.22527472527472525  \\
            1.6  0.22222222222222218  \\
            1.61  0.2191011235955056  \\
            1.62  0.21590909090909088  \\
            1.63  0.21264367816091959  \\
            1.64  0.2093023255813954  \\
            1.65  0.2058823529411765  \\
            1.66  0.2023809523809524  \\
            1.67  0.1987951807228916  \\
            1.68  0.19512195121951223  \\
            1.69  0.19135802469135804  \\
            1.7  0.18750000000000003  \\
            1.71  0.18354430379746836  \\
            1.72  0.1794871794871795  \\
            1.73  0.17532467532467533  \\
            1.74  0.17105263157894737  \\
            1.75  0.16666666666666666  \\
            1.76  0.16216216216216217  \\
            1.77  0.15753424657534246  \\
            1.78  0.15277777777777776  \\
            1.79  0.14788732394366194  \\
            1.8  0.14285714285714282  \\
            1.81  0.13768115942028983  \\
            1.82  0.13235294117647056  \\
            1.83  0.126865671641791  \\
            1.84  0.12121212121212116  \\
            1.85  0.11538461538461534  \\
            1.86  0.10937499999999994  \\
            1.87  0.10317460317460311  \\
            1.88  0.09677419354838716  \\
            1.89  0.09016393442622958  \\
            1.9  0.0833333333333334  \\
            1.91  0.07627118644067803  \\
            1.92  0.06896551724137936  \\
            1.93  0.06140350877192987  \\
            1.94  0.05357142857142862  \\
            1.95  0.04545454545454549  \\
            1.96  0.03703703703703707  \\
            1.97  0.028301886792452855  \\
            1.98  0.019230769230769246  \\
            1.99  0.00980392156862746  \\
            2.0  0.0  \\
            2.01  0.0  \\
            2.02  0.0  \\
            2.03  0.0  \\
            2.04  0.0  \\
            2.05  0.0  \\
            2.06  0.0  \\
            2.07  0.0  \\
            2.08  0.0  \\
            2.09  0.0  \\
            2.1  0.0  \\
            2.11  0.0  \\
            2.12  0.0  \\
            2.13  0.0  \\
            2.14  0.0  \\
            2.15  0.0  \\
            2.16  0.0  \\
            2.17  0.0  \\
            2.18  0.0  \\
            2.19  0.0  \\
            2.2  0.0  \\
            2.21  0.0  \\
            2.22  0.0  \\
            2.23  0.0  \\
            2.24  0.0  \\
            2.25  0.0  \\
            2.26  0.0  \\
            2.27  0.0  \\
            2.28  0.0  \\
            2.29  0.0  \\
            2.3  0.0  \\
            2.31  0.0  \\
            2.32  0.0  \\
            2.33  0.0  \\
            2.34  0.0  \\
            2.35  0.0  \\
            2.36  0.0  \\
            2.37  0.0  \\
            2.38  0.0  \\
            2.39  0.0  \\
            2.4  0.0  \\
            2.41  0.0  \\
            2.42  0.0  \\
            2.43  0.0  \\
            2.44  0.0  \\
            2.45  0.0  \\
            2.46  0.0  \\
            2.47  0.0  \\
            2.48  0.0  \\
            2.49  0.0  \\
            2.5  0.0  \\
            2.51  0.0  \\
            2.52  0.0  \\
            2.53  0.0  \\
            2.54  0.0  \\
            2.55  0.0  \\
            2.56  0.0  \\
            2.57  0.0  \\
            2.58  0.0  \\
            2.59  0.0  \\
            2.6  0.0  \\
            2.61  0.0  \\
            2.62  0.0  \\
            2.63  0.0  \\
            2.64  0.0  \\
            2.65  0.0  \\
            2.66  0.0  \\
            2.67  0.0  \\
            2.68  0.0  \\
            2.69  0.0  \\
            2.7  0.0  \\
            2.71  0.0  \\
            2.72  0.0  \\
            2.73  0.0  \\
            2.74  0.0  \\
            2.75  0.0  \\
            2.76  0.0  \\
            2.77  0.0  \\
            2.78  0.0  \\
            2.79  0.0  \\
            2.8  0.0  \\
            2.81  0.0  \\
            2.82  0.0  \\
            2.83  0.0  \\
            2.84  0.0  \\
            2.85  0.0  \\
            2.86  0.0  \\
            2.87  0.0  \\
            2.88  0.0  \\
            2.89  0.0  \\
            2.9  0.0  \\
            2.91  0.0  \\
            2.92  0.0  \\
            2.93  0.0  \\
            2.94  0.0  \\
            2.95  0.0  \\
            2.96  0.0  \\
            2.97  0.0  \\
            2.98  0.0  \\
            2.99  0.0  \\
            3.0  nan  \\
        }
        \closedcycle
        ;
    \addplot+[fill, fill opacity={0.2}, thick]
        table[row sep={\\}]
        {
            \\
            -3.0  nan  \\
            -2.99  0.0  \\
            -2.98  0.0  \\
            -2.97  0.0  \\
            -2.96  0.0  \\
            -2.95  0.0  \\
            -2.94  0.0  \\
            -2.93  0.0  \\
            -2.92  0.0  \\
            -2.91  0.0  \\
            -2.9  0.0  \\
            -2.89  0.0  \\
            -2.88  0.0  \\
            -2.87  0.0  \\
            -2.86  0.0  \\
            -2.85  0.0  \\
            -2.84  0.0  \\
            -2.83  0.0  \\
            -2.82  0.0  \\
            -2.81  0.0  \\
            -2.8  0.0  \\
            -2.79  0.0  \\
            -2.78  0.0  \\
            -2.77  0.0  \\
            -2.76  0.0  \\
            -2.75  0.0  \\
            -2.74  0.0  \\
            -2.73  0.0  \\
            -2.72  0.0  \\
            -2.71  0.0  \\
            -2.7  0.0  \\
            -2.69  0.0  \\
            -2.68  0.0  \\
            -2.67  0.0  \\
            -2.66  0.0  \\
            -2.65  0.0  \\
            -2.64  0.0  \\
            -2.63  0.0  \\
            -2.62  0.0  \\
            -2.61  0.0  \\
            -2.6  0.0  \\
            -2.59  0.0  \\
            -2.58  0.0  \\
            -2.57  0.0  \\
            -2.56  0.0  \\
            -2.55  0.0  \\
            -2.54  0.0  \\
            -2.53  0.0  \\
            -2.52  0.0  \\
            -2.51  0.0  \\
            -2.5  0.0  \\
            -2.49  0.0  \\
            -2.48  0.0  \\
            -2.47  0.0  \\
            -2.46  0.0  \\
            -2.45  0.0  \\
            -2.44  0.0  \\
            -2.43  0.0  \\
            -2.42  0.0  \\
            -2.41  0.0  \\
            -2.4  0.0  \\
            -2.39  0.0  \\
            -2.38  0.0  \\
            -2.37  0.0  \\
            -2.36  0.0  \\
            -2.35  0.0  \\
            -2.34  0.0  \\
            -2.33  0.0  \\
            -2.32  0.0  \\
            -2.31  0.0  \\
            -2.3  0.0  \\
            -2.29  0.0  \\
            -2.28  0.0  \\
            -2.27  0.0  \\
            -2.26  0.0  \\
            -2.25  0.0  \\
            -2.24  0.0  \\
            -2.23  0.0  \\
            -2.22  0.0  \\
            -2.21  0.0  \\
            -2.2  0.0  \\
            -2.19  0.0  \\
            -2.18  0.0  \\
            -2.17  0.0  \\
            -2.16  0.0  \\
            -2.15  0.0  \\
            -2.14  0.0  \\
            -2.13  0.0  \\
            -2.12  0.0  \\
            -2.11  0.0  \\
            -2.1  0.0  \\
            -2.09  0.0  \\
            -2.08  0.0  \\
            -2.07  0.0  \\
            -2.06  0.0  \\
            -2.05  0.0  \\
            -2.04  0.0  \\
            -2.03  0.0  \\
            -2.02  0.0  \\
            -2.01  0.0  \\
            -2.0  0.0  \\
            -1.99  0.0  \\
            -1.98  0.0  \\
            -1.97  0.0  \\
            -1.96  0.0  \\
            -1.95  0.0  \\
            -1.94  0.0  \\
            -1.93  0.0  \\
            -1.92  0.0  \\
            -1.91  0.0  \\
            -1.9  0.0  \\
            -1.89  0.0  \\
            -1.88  0.0  \\
            -1.87  0.0  \\
            -1.86  0.0  \\
            -1.85  0.0  \\
            -1.84  0.0  \\
            -1.83  0.0  \\
            -1.82  0.0  \\
            -1.81  0.0  \\
            -1.8  0.0  \\
            -1.79  0.0  \\
            -1.78  0.0  \\
            -1.77  0.0  \\
            -1.76  0.0  \\
            -1.75  0.0  \\
            -1.74  0.0  \\
            -1.73  0.0  \\
            -1.72  0.0  \\
            -1.71  0.0  \\
            -1.7  0.0  \\
            -1.69  0.0  \\
            -1.68  0.0  \\
            -1.67  0.0  \\
            -1.66  0.0  \\
            -1.65  0.0  \\
            -1.64  0.0  \\
            -1.63  0.0  \\
            -1.62  0.0  \\
            -1.61  0.0  \\
            -1.6  0.0  \\
            -1.59  0.0  \\
            -1.58  0.0  \\
            -1.57  0.0  \\
            -1.56  0.0  \\
            -1.55  0.0  \\
            -1.54  0.0  \\
            -1.53  0.0  \\
            -1.52  0.0  \\
            -1.51  0.0  \\
            -1.5  0.0  \\
            -1.49  0.0  \\
            -1.48  0.0  \\
            -1.47  0.0  \\
            -1.46  0.0  \\
            -1.45  0.0  \\
            -1.44  0.0  \\
            -1.43  0.0  \\
            -1.42  0.0  \\
            -1.41  0.0  \\
            -1.4  0.0  \\
            -1.39  0.0  \\
            -1.38  0.0  \\
            -1.37  0.0  \\
            -1.36  0.0  \\
            -1.35  0.0  \\
            -1.34  0.0  \\
            -1.33  0.0  \\
            -1.32  0.0  \\
            -1.31  0.0  \\
            -1.3  0.0  \\
            -1.29  0.0  \\
            -1.28  0.0  \\
            -1.27  0.0  \\
            -1.26  0.0  \\
            -1.25  0.0  \\
            -1.24  0.0  \\
            -1.23  0.0  \\
            -1.22  0.0  \\
            -1.21  0.0  \\
            -1.2  0.0  \\
            -1.19  0.0  \\
            -1.18  0.0  \\
            -1.17  0.0  \\
            -1.16  0.0  \\
            -1.15  0.0  \\
            -1.14  0.0  \\
            -1.13  0.0  \\
            -1.12  0.0  \\
            -1.11  0.0  \\
            -1.1  0.0  \\
            -1.09  0.0  \\
            -1.08  0.0  \\
            -1.07  0.0  \\
            -1.06  0.0  \\
            -1.05  0.0  \\
            -1.04  0.0  \\
            -1.03  0.0  \\
            -1.02  0.0  \\
            -1.01  0.0  \\
            -1.0  0.0  \\
            -0.99  0.0033222591362126277  \\
            -0.98  0.006622516556291397  \\
            -0.97  0.00990099009900991  \\
            -0.96  0.013157894736842117  \\
            -0.95  0.016393442622950834  \\
            -0.94  0.01960784313725492  \\
            -0.93  0.022801302931596077  \\
            -0.92  0.025974025974025962  \\
            -0.91  0.029126213592233  \\
            -0.9  0.032258064516129024  \\
            -0.89  0.03536977491961415  \\
            -0.88  0.03846153846153846  \\
            -0.87  0.0415335463258786  \\
            -0.86  0.0445859872611465  \\
            -0.85  0.04761904761904763  \\
            -0.84  0.05063291139240507  \\
            -0.83  0.053627760252365944  \\
            -0.82  0.056603773584905676  \\
            -0.81  0.05956112852664575  \\
            -0.8  0.062499999999999986  \\
            -0.79  0.06542056074766354  \\
            -0.78  0.06832298136645962  \\
            -0.77  0.07120743034055727  \\
            -0.76  0.07407407407407407  \\
            -0.75  0.07692307692307693  \\
            -0.74  0.07975460122699388  \\
            -0.73  0.08256880733944955  \\
            -0.72  0.08536585365853659  \\
            -0.71  0.08814589665653497  \\
            -0.7  0.09090909090909093  \\
            -0.69  0.09365558912386708  \\
            -0.68  0.09638554216867469  \\
            -0.67  0.09909909909909909  \\
            -0.66  0.10179640718562874  \\
            -0.65  0.1044776119402985  \\
            -0.64  0.10714285714285714  \\
            -0.63  0.10979228486646883  \\
            -0.62  0.11242603550295859  \\
            -0.61  0.11504424778761062  \\
            -0.6  0.11764705882352942  \\
            -0.59  0.12023460410557185  \\
            -0.58  0.12280701754385967  \\
            -0.57  0.12536443148688048  \\
            -0.56  0.12790697674418602  \\
            -0.55  0.13043478260869562  \\
            -0.54  0.1329479768786127  \\
            -0.53  0.13544668587896252  \\
            -0.52  0.13793103448275862  \\
            -0.51  0.14040114613180515  \\
            -0.5  0.14285714285714285  \\
            -0.49  0.1452991452991453  \\
            -0.48  0.14772727272727273  \\
            -0.47  0.15014164305949007  \\
            -0.46  0.15254237288135594  \\
            -0.45  0.15492957746478875  \\
            -0.44  0.15730337078651688  \\
            -0.43  0.1596638655462185  \\
            -0.42  0.16201117318435757  \\
            -0.41  0.16434540389972147  \\
            -0.4  0.16666666666666666  \\
            -0.39  0.16897506925207756  \\
            -0.38  0.1712707182320442  \\
            -0.37  0.17355371900826447  \\
            -0.36  0.1758241758241758  \\
            -0.35  0.17808219178082194  \\
            -0.34  0.180327868852459  \\
            -0.33  0.18256130790190733  \\
            -0.32  0.18478260869565216  \\
            -0.31  0.18699186991869918  \\
            -0.3  0.18918918918918917  \\
            -0.29  0.19137466307277629  \\
            -0.28  0.1935483870967742  \\
            -0.27  0.19571045576407506  \\
            -0.26  0.1978609625668449  \\
            -0.25  0.2  \\
            -0.24  0.2021276595744681  \\
            -0.23  0.20424403183023873  \\
            -0.22  0.20634920634920634  \\
            -0.21  0.20844327176781002  \\
            -0.2  0.2105263157894737  \\
            -0.19  0.2125984251968504  \\
            -0.18  0.21465968586387435  \\
            -0.17  0.216710182767624  \\
            -0.16  0.21875  \\
            -0.15  0.22077922077922077  \\
            -0.14  0.2227979274611399  \\
            -0.13  0.22480620155038758  \\
            -0.12  0.2268041237113402  \\
            -0.11  0.2287917737789203  \\
            -0.1  0.23076923076923078  \\
            -0.09  0.23273657289002558  \\
            -0.08  0.23469387755102042  \\
            -0.07  0.2366412213740458  \\
            -0.06  0.23857868020304568  \\
            -0.05  0.24050632911392403  \\
            -0.04  0.24242424242424243  \\
            -0.03  0.2443324937027708  \\
            -0.02  0.24623115577889446  \\
            -0.01  0.24812030075187969  \\
            0.0  0.25  \\
            0.01  0.2531328320802005  \\
            0.02  0.2562814070351759  \\
            0.03  0.2594458438287154  \\
            0.04  0.26262626262626265  \\
            0.05  0.26582278481012656  \\
            0.06  0.2690355329949239  \\
            0.07  0.27226463104325704  \\
            0.08  0.2755102040816327  \\
            0.09  0.27877237851662406  \\
            0.1  0.2820512820512821  \\
            0.11  0.28534704370179953  \\
            0.12  0.288659793814433  \\
            0.13  0.2919896640826873  \\
            0.14  0.2953367875647669  \\
            0.15  0.2987012987012987  \\
            0.16  0.3020833333333333  \\
            0.17  0.3054830287206266  \\
            0.18  0.30890052356020936  \\
            0.19  0.3123359580052493  \\
            0.2  0.3157894736842105  \\
            0.21  0.31926121372031663  \\
            0.22  0.32275132275132273  \\
            0.23  0.32625994694960214  \\
            0.24  0.3297872340425532  \\
            0.25  0.3333333333333333  \\
            0.26  0.3368983957219251  \\
            0.27  0.34048257372654156  \\
            0.28  0.34408602150537637  \\
            0.29  0.3477088948787062  \\
            0.3  0.35135135135135137  \\
            0.31  0.3550135501355014  \\
            0.32  0.3586956521739131  \\
            0.33  0.36239782016348776  \\
            0.34  0.366120218579235  \\
            0.35  0.36986301369863017  \\
            0.36  0.3736263736263736  \\
            0.37  0.37741046831955927  \\
            0.38  0.38121546961325964  \\
            0.39  0.38504155124653744  \\
            0.4  0.38888888888888884  \\
            0.41  0.39275766016713093  \\
            0.42  0.3966480446927374  \\
            0.43  0.4005602240896358  \\
            0.44  0.4044943820224719  \\
            0.45  0.4084507042253521  \\
            0.46  0.4124293785310734  \\
            0.47  0.4164305949008498  \\
            0.48  0.42045454545454547  \\
            0.49  0.42450142450142453  \\
            0.5  0.42857142857142855  \\
            0.51  0.43266475644699137  \\
            0.52  0.4367816091954023  \\
            0.53  0.4409221902017291  \\
            0.54  0.44508670520231214  \\
            0.55  0.4492753623188406  \\
            0.56  0.4534883720930233  \\
            0.57  0.4577259475218658  \\
            0.58  0.4619883040935673  \\
            0.59  0.46627565982404684  \\
            0.6  0.4705882352941177  \\
            0.61  0.47492625368731556  \\
            0.62  0.4792899408284024  \\
            0.63  0.48367952522255186  \\
            0.64  0.48809523809523814  \\
            0.65  0.4925373134328358  \\
            0.66  0.4970059880239522  \\
            0.67  0.5015015015015015  \\
            0.68  0.5060240963855422  \\
            0.69  0.5105740181268882  \\
            0.7  0.5151515151515151  \\
            0.71  0.5197568389057751  \\
            0.72  0.5243902439024389  \\
            0.73  0.5290519877675841  \\
            0.74  0.5337423312883436  \\
            0.75  0.5384615384615384  \\
            0.76  0.5432098765432098  \\
            0.77  0.5479876160990712  \\
            0.78  0.5527950310559007  \\
            0.79  0.5576323987538941  \\
            0.8  0.5625  \\
            0.81  0.5673981191222571  \\
            0.82  0.5723270440251571  \\
            0.83  0.5772870662460569  \\
            0.84  0.5822784810126581  \\
            0.85  0.5873015873015873  \\
            0.86  0.5923566878980892  \\
            0.87  0.5974440894568691  \\
            0.88  0.6025641025641025  \\
            0.89  0.6077170418006431  \\
            0.9  0.6129032258064515  \\
            0.91  0.6181229773462784  \\
            0.92  0.6233766233766234  \\
            0.93  0.6286644951140066  \\
            0.94  0.6339869281045751  \\
            0.95  0.639344262295082  \\
            0.96  0.6447368421052632  \\
            0.97  0.6501650165016502  \\
            0.98  0.6556291390728477  \\
            0.99  0.6611295681063123  \\
            1.0  0.6666666666666666  \\
            1.01  0.6677852348993288  \\
            1.02  0.668918918918919  \\
            1.03  0.6700680272108843  \\
            1.04  0.6712328767123288  \\
            1.05  0.6724137931034483  \\
            1.06  0.6736111111111112  \\
            1.07  0.6748251748251748  \\
            1.08  0.676056338028169  \\
            1.09  0.6773049645390071  \\
            1.1  0.6785714285714286  \\
            1.11  0.6798561151079137  \\
            1.12  0.6811594202898551  \\
            1.13  0.6824817518248175  \\
            1.14  0.6838235294117647  \\
            1.15  0.6851851851851851  \\
            1.16  0.6865671641791045  \\
            1.17  0.6879699248120301  \\
            1.18  0.6893939393939393  \\
            1.19  0.6908396946564885  \\
            1.2  0.6923076923076923  \\
            1.21  0.6937984496124031  \\
            1.22  0.6953125  \\
            1.23  0.6968503937007874  \\
            1.24  0.6984126984126984  \\
            1.25  0.7  \\
            1.26  0.7016129032258065  \\
            1.27  0.7032520325203252  \\
            1.28  0.7049180327868853  \\
            1.29  0.7066115702479339  \\
            1.3  0.7083333333333334  \\
            1.31  0.7100840336134454  \\
            1.32  0.711864406779661  \\
            1.33  0.7136752136752137  \\
            1.34  0.7155172413793104  \\
            1.35  0.717391304347826  \\
            1.36  0.7192982456140351  \\
            1.37  0.7212389380530974  \\
            1.38  0.7232142857142857  \\
            1.39  0.7252252252252253  \\
            1.4  0.7272727272727273  \\
            1.41  0.7293577981651376  \\
            1.42  0.7314814814814815  \\
            1.43  0.7336448598130841  \\
            1.44  0.7358490566037735  \\
            1.45  0.7380952380952381  \\
            1.46  0.7403846153846154  \\
            1.47  0.7427184466019418  \\
            1.48  0.7450980392156863  \\
            1.49  0.7475247524752475  \\
            1.5  0.75  \\
            1.51  0.7525252525252525  \\
            1.52  0.7551020408163265  \\
            1.53  0.7577319587628866  \\
            1.54  0.7604166666666666  \\
            1.55  0.7631578947368421  \\
            1.56  0.7659574468085106  \\
            1.57  0.7688172043010753  \\
            1.58  0.7717391304347826  \\
            1.59  0.7747252747252747  \\
            1.6  0.7777777777777778  \\
            1.61  0.7808988764044944  \\
            1.62  0.7840909090909092  \\
            1.63  0.7873563218390804  \\
            1.64  0.7906976744186046  \\
            1.65  0.7941176470588235  \\
            1.66  0.7976190476190476  \\
            1.67  0.8012048192771084  \\
            1.68  0.8048780487804877  \\
            1.69  0.8086419753086419  \\
            1.7  0.8125  \\
            1.71  0.8164556962025317  \\
            1.72  0.8205128205128205  \\
            1.73  0.8246753246753247  \\
            1.74  0.8289473684210527  \\
            1.75  0.8333333333333334  \\
            1.76  0.8378378378378378  \\
            1.77  0.8424657534246576  \\
            1.78  0.8472222222222222  \\
            1.79  0.852112676056338  \\
            1.8  0.8571428571428572  \\
            1.81  0.8623188405797102  \\
            1.82  0.8676470588235294  \\
            1.83  0.873134328358209  \\
            1.84  0.8787878787878788  \\
            1.85  0.8846153846153847  \\
            1.86  0.8906250000000001  \\
            1.87  0.8968253968253969  \\
            1.88  0.9032258064516129  \\
            1.89  0.9098360655737704  \\
            1.9  0.9166666666666666  \\
            1.91  0.923728813559322  \\
            1.92  0.9310344827586207  \\
            1.93  0.9385964912280701  \\
            1.94  0.9464285714285714  \\
            1.95  0.9545454545454545  \\
            1.96  0.9629629629629629  \\
            1.97  0.9716981132075472  \\
            1.98  0.9807692307692307  \\
            1.99  0.9901960784313726  \\
            2.0  1.0  \\
            2.01  1.0  \\
            2.02  1.0  \\
            2.03  1.0  \\
            2.04  1.0  \\
            2.05  1.0  \\
            2.06  1.0  \\
            2.07  1.0  \\
            2.08  1.0  \\
            2.09  1.0  \\
            2.1  1.0  \\
            2.11  1.0  \\
            2.12  1.0  \\
            2.13  1.0  \\
            2.14  1.0  \\
            2.15  1.0  \\
            2.16  1.0  \\
            2.17  1.0  \\
            2.18  1.0  \\
            2.19  1.0  \\
            2.2  1.0  \\
            2.21  1.0  \\
            2.22  1.0  \\
            2.23  1.0  \\
            2.24  1.0  \\
            2.25  1.0  \\
            2.26  1.0  \\
            2.27  1.0  \\
            2.28  1.0  \\
            2.29  1.0  \\
            2.3  1.0  \\
            2.31  1.0  \\
            2.32  1.0  \\
            2.33  1.0  \\
            2.34  1.0  \\
            2.35  1.0  \\
            2.36  1.0  \\
            2.37  1.0  \\
            2.38  1.0  \\
            2.39  1.0  \\
            2.4  1.0  \\
            2.41  1.0  \\
            2.42  1.0  \\
            2.43  1.0  \\
            2.44  1.0  \\
            2.45  1.0  \\
            2.46  1.0  \\
            2.47  1.0  \\
            2.48  1.0  \\
            2.49  1.0  \\
            2.5  1.0  \\
            2.51  1.0  \\
            2.52  1.0  \\
            2.53  1.0  \\
            2.54  1.0  \\
            2.55  1.0  \\
            2.56  1.0  \\
            2.57  1.0  \\
            2.58  1.0  \\
            2.59  1.0  \\
            2.6  1.0  \\
            2.61  1.0  \\
            2.62  1.0  \\
            2.63  1.0  \\
            2.64  1.0  \\
            2.65  1.0  \\
            2.66  1.0  \\
            2.67  1.0  \\
            2.68  1.0  \\
            2.69  1.0  \\
            2.7  1.0  \\
            2.71  1.0  \\
            2.72  1.0  \\
            2.73  1.0  \\
            2.74  1.0  \\
            2.75  1.0  \\
            2.76  1.0  \\
            2.77  1.0  \\
            2.78  1.0  \\
            2.79  1.0  \\
            2.8  1.0  \\
            2.81  1.0  \\
            2.82  1.0  \\
            2.83  1.0  \\
            2.84  1.0  \\
            2.85  1.0  \\
            2.86  1.0  \\
            2.87  1.0  \\
            2.88  1.0  \\
            2.89  1.0  \\
            2.9  1.0  \\
            2.91  1.0  \\
            2.92  1.0  \\
            2.93  1.0  \\
            2.94  1.0  \\
            2.95  1.0  \\
            2.96  1.0  \\
            2.97  1.0  \\
            2.98  1.0  \\
            2.99  1.0  \\
            3.0  nan  \\
        }
        \closedcycle
        ;
    \nextgroupplot[title={$g_y(X) = \mathbb{P}[Y = y \,|\, \lfloor X \rfloor]$}]
    \addplot+[fill, fill opacity={0.2}, thick]
        table[row sep={\\}]
        {
            \\
            -3.0  1.0  \\
            -2.99  1.0  \\
            -2.98  1.0  \\
            -2.97  1.0  \\
            -2.96  1.0  \\
            -2.95  1.0  \\
            -2.94  1.0  \\
            -2.93  1.0  \\
            -2.92  1.0  \\
            -2.91  1.0  \\
            -2.9  1.0  \\
            -2.89  1.0  \\
            -2.88  1.0  \\
            -2.87  1.0  \\
            -2.86  1.0  \\
            -2.85  1.0  \\
            -2.84  1.0  \\
            -2.83  1.0  \\
            -2.82  1.0  \\
            -2.81  1.0  \\
            -2.8  1.0  \\
            -2.79  1.0  \\
            -2.78  1.0  \\
            -2.77  1.0  \\
            -2.76  1.0  \\
            -2.75  1.0  \\
            -2.74  1.0  \\
            -2.73  1.0  \\
            -2.72  1.0  \\
            -2.71  1.0  \\
            -2.7  1.0  \\
            -2.69  1.0  \\
            -2.68  1.0  \\
            -2.67  1.0  \\
            -2.66  1.0  \\
            -2.65  1.0  \\
            -2.64  1.0  \\
            -2.63  1.0  \\
            -2.62  1.0  \\
            -2.61  1.0  \\
            -2.6  1.0  \\
            -2.59  1.0  \\
            -2.58  1.0  \\
            -2.57  1.0  \\
            -2.56  1.0  \\
            -2.55  1.0  \\
            -2.54  1.0  \\
            -2.53  1.0  \\
            -2.52  1.0  \\
            -2.51  1.0  \\
            -2.5  1.0  \\
            -2.49  1.0  \\
            -2.48  1.0  \\
            -2.47  1.0  \\
            -2.46  1.0  \\
            -2.45  1.0  \\
            -2.44  1.0  \\
            -2.43  1.0  \\
            -2.42  1.0  \\
            -2.41  1.0  \\
            -2.4  1.0  \\
            -2.39  1.0  \\
            -2.38  1.0  \\
            -2.37  1.0  \\
            -2.36  1.0  \\
            -2.35  1.0  \\
            -2.34  1.0  \\
            -2.33  1.0  \\
            -2.32  1.0  \\
            -2.31  1.0  \\
            -2.3  1.0  \\
            -2.29  1.0  \\
            -2.28  1.0  \\
            -2.27  1.0  \\
            -2.26  1.0  \\
            -2.25  1.0  \\
            -2.24  1.0  \\
            -2.23  1.0  \\
            -2.22  1.0  \\
            -2.21  1.0  \\
            -2.2  1.0  \\
            -2.19  1.0  \\
            -2.18  1.0  \\
            -2.17  1.0  \\
            -2.16  1.0  \\
            -2.15  1.0  \\
            -2.14  1.0  \\
            -2.13  1.0  \\
            -2.12  1.0  \\
            -2.11  1.0  \\
            -2.1  1.0  \\
            -2.09  1.0  \\
            -2.08  1.0  \\
            -2.07  1.0  \\
            -2.06  1.0  \\
            -2.05  1.0  \\
            -2.04  1.0  \\
            -2.03  1.0  \\
            -2.02  1.0  \\
            -2.01  1.0  \\
            -2.0  0.75  \\
            -1.99  0.75  \\
            -1.98  0.75  \\
            -1.97  0.75  \\
            -1.96  0.75  \\
            -1.95  0.75  \\
            -1.94  0.75  \\
            -1.93  0.75  \\
            -1.92  0.75  \\
            -1.91  0.75  \\
            -1.9  0.75  \\
            -1.89  0.75  \\
            -1.88  0.75  \\
            -1.87  0.75  \\
            -1.86  0.75  \\
            -1.85  0.75  \\
            -1.84  0.75  \\
            -1.83  0.75  \\
            -1.82  0.75  \\
            -1.81  0.75  \\
            -1.8  0.75  \\
            -1.79  0.75  \\
            -1.78  0.75  \\
            -1.77  0.75  \\
            -1.76  0.75  \\
            -1.75  0.75  \\
            -1.74  0.75  \\
            -1.73  0.75  \\
            -1.72  0.75  \\
            -1.71  0.75  \\
            -1.7  0.75  \\
            -1.69  0.75  \\
            -1.68  0.75  \\
            -1.67  0.75  \\
            -1.66  0.75  \\
            -1.65  0.75  \\
            -1.64  0.75  \\
            -1.63  0.75  \\
            -1.62  0.75  \\
            -1.61  0.75  \\
            -1.6  0.75  \\
            -1.59  0.75  \\
            -1.58  0.75  \\
            -1.57  0.75  \\
            -1.56  0.75  \\
            -1.55  0.75  \\
            -1.54  0.75  \\
            -1.53  0.75  \\
            -1.52  0.75  \\
            -1.51  0.75  \\
            -1.5  0.75  \\
            -1.49  0.75  \\
            -1.48  0.75  \\
            -1.47  0.75  \\
            -1.46  0.75  \\
            -1.45  0.75  \\
            -1.44  0.75  \\
            -1.43  0.75  \\
            -1.42  0.75  \\
            -1.41  0.75  \\
            -1.4  0.75  \\
            -1.39  0.75  \\
            -1.38  0.75  \\
            -1.37  0.75  \\
            -1.36  0.75  \\
            -1.35  0.75  \\
            -1.34  0.75  \\
            -1.33  0.75  \\
            -1.32  0.75  \\
            -1.31  0.75  \\
            -1.3  0.75  \\
            -1.29  0.75  \\
            -1.28  0.75  \\
            -1.27  0.75  \\
            -1.26  0.75  \\
            -1.25  0.75  \\
            -1.24  0.75  \\
            -1.23  0.75  \\
            -1.22  0.75  \\
            -1.21  0.75  \\
            -1.2  0.75  \\
            -1.19  0.75  \\
            -1.18  0.75  \\
            -1.17  0.75  \\
            -1.16  0.75  \\
            -1.15  0.75  \\
            -1.14  0.75  \\
            -1.13  0.75  \\
            -1.12  0.75  \\
            -1.11  0.75  \\
            -1.1  0.75  \\
            -1.09  0.75  \\
            -1.08  0.75  \\
            -1.07  0.75  \\
            -1.06  0.75  \\
            -1.05  0.75  \\
            -1.04  0.75  \\
            -1.03  0.75  \\
            -1.02  0.75  \\
            -1.01  0.75  \\
            -1.0  0.42857142857142855  \\
            -0.99  0.42857142857142855  \\
            -0.98  0.42857142857142855  \\
            -0.97  0.42857142857142855  \\
            -0.96  0.42857142857142855  \\
            -0.95  0.42857142857142855  \\
            -0.94  0.42857142857142855  \\
            -0.93  0.42857142857142855  \\
            -0.92  0.42857142857142855  \\
            -0.91  0.42857142857142855  \\
            -0.9  0.42857142857142855  \\
            -0.89  0.42857142857142855  \\
            -0.88  0.42857142857142855  \\
            -0.87  0.42857142857142855  \\
            -0.86  0.42857142857142855  \\
            -0.85  0.42857142857142855  \\
            -0.84  0.42857142857142855  \\
            -0.83  0.42857142857142855  \\
            -0.82  0.42857142857142855  \\
            -0.81  0.42857142857142855  \\
            -0.8  0.42857142857142855  \\
            -0.79  0.42857142857142855  \\
            -0.78  0.42857142857142855  \\
            -0.77  0.42857142857142855  \\
            -0.76  0.42857142857142855  \\
            -0.75  0.42857142857142855  \\
            -0.74  0.42857142857142855  \\
            -0.73  0.42857142857142855  \\
            -0.72  0.42857142857142855  \\
            -0.71  0.42857142857142855  \\
            -0.7  0.42857142857142855  \\
            -0.69  0.42857142857142855  \\
            -0.68  0.42857142857142855  \\
            -0.67  0.42857142857142855  \\
            -0.66  0.42857142857142855  \\
            -0.65  0.42857142857142855  \\
            -0.64  0.42857142857142855  \\
            -0.63  0.42857142857142855  \\
            -0.62  0.42857142857142855  \\
            -0.61  0.42857142857142855  \\
            -0.6  0.42857142857142855  \\
            -0.59  0.42857142857142855  \\
            -0.58  0.42857142857142855  \\
            -0.57  0.42857142857142855  \\
            -0.56  0.42857142857142855  \\
            -0.55  0.42857142857142855  \\
            -0.54  0.42857142857142855  \\
            -0.53  0.42857142857142855  \\
            -0.52  0.42857142857142855  \\
            -0.51  0.42857142857142855  \\
            -0.5  0.42857142857142855  \\
            -0.49  0.42857142857142855  \\
            -0.48  0.42857142857142855  \\
            -0.47  0.42857142857142855  \\
            -0.46  0.42857142857142855  \\
            -0.45  0.42857142857142855  \\
            -0.44  0.42857142857142855  \\
            -0.43  0.42857142857142855  \\
            -0.42  0.42857142857142855  \\
            -0.41  0.42857142857142855  \\
            -0.4  0.42857142857142855  \\
            -0.39  0.42857142857142855  \\
            -0.38  0.42857142857142855  \\
            -0.37  0.42857142857142855  \\
            -0.36  0.42857142857142855  \\
            -0.35  0.42857142857142855  \\
            -0.34  0.42857142857142855  \\
            -0.33  0.42857142857142855  \\
            -0.32  0.42857142857142855  \\
            -0.31  0.42857142857142855  \\
            -0.3  0.42857142857142855  \\
            -0.29  0.42857142857142855  \\
            -0.28  0.42857142857142855  \\
            -0.27  0.42857142857142855  \\
            -0.26  0.42857142857142855  \\
            -0.25  0.42857142857142855  \\
            -0.24  0.42857142857142855  \\
            -0.23  0.42857142857142855  \\
            -0.22  0.42857142857142855  \\
            -0.21  0.42857142857142855  \\
            -0.2  0.42857142857142855  \\
            -0.19  0.42857142857142855  \\
            -0.18  0.42857142857142855  \\
            -0.17  0.42857142857142855  \\
            -0.16  0.42857142857142855  \\
            -0.15  0.42857142857142855  \\
            -0.14  0.42857142857142855  \\
            -0.13  0.42857142857142855  \\
            -0.12  0.42857142857142855  \\
            -0.11  0.42857142857142855  \\
            -0.1  0.42857142857142855  \\
            -0.09  0.42857142857142855  \\
            -0.08  0.42857142857142855  \\
            -0.07  0.42857142857142855  \\
            -0.06  0.42857142857142855  \\
            -0.05  0.42857142857142855  \\
            -0.04  0.42857142857142855  \\
            -0.03  0.42857142857142855  \\
            -0.02  0.42857142857142855  \\
            -0.01  0.42857142857142855  \\
            0.0  0.14285714285714285  \\
            0.01  0.14285714285714285  \\
            0.02  0.14285714285714285  \\
            0.03  0.14285714285714285  \\
            0.04  0.14285714285714285  \\
            0.05  0.14285714285714285  \\
            0.06  0.14285714285714285  \\
            0.07  0.14285714285714285  \\
            0.08  0.14285714285714285  \\
            0.09  0.14285714285714285  \\
            0.1  0.14285714285714285  \\
            0.11  0.14285714285714285  \\
            0.12  0.14285714285714285  \\
            0.13  0.14285714285714285  \\
            0.14  0.14285714285714285  \\
            0.15  0.14285714285714285  \\
            0.16  0.14285714285714285  \\
            0.17  0.14285714285714285  \\
            0.18  0.14285714285714285  \\
            0.19  0.14285714285714285  \\
            0.2  0.14285714285714285  \\
            0.21  0.14285714285714285  \\
            0.22  0.14285714285714285  \\
            0.23  0.14285714285714285  \\
            0.24  0.14285714285714285  \\
            0.25  0.14285714285714285  \\
            0.26  0.14285714285714285  \\
            0.27  0.14285714285714285  \\
            0.28  0.14285714285714285  \\
            0.29  0.14285714285714285  \\
            0.3  0.14285714285714285  \\
            0.31  0.14285714285714285  \\
            0.32  0.14285714285714285  \\
            0.33  0.14285714285714285  \\
            0.34  0.14285714285714285  \\
            0.35  0.14285714285714285  \\
            0.36  0.14285714285714285  \\
            0.37  0.14285714285714285  \\
            0.38  0.14285714285714285  \\
            0.39  0.14285714285714285  \\
            0.4  0.14285714285714285  \\
            0.41  0.14285714285714285  \\
            0.42  0.14285714285714285  \\
            0.43  0.14285714285714285  \\
            0.44  0.14285714285714285  \\
            0.45  0.14285714285714285  \\
            0.46  0.14285714285714285  \\
            0.47  0.14285714285714285  \\
            0.48  0.14285714285714285  \\
            0.49  0.14285714285714285  \\
            0.5  0.14285714285714285  \\
            0.51  0.14285714285714285  \\
            0.52  0.14285714285714285  \\
            0.53  0.14285714285714285  \\
            0.54  0.14285714285714285  \\
            0.55  0.14285714285714285  \\
            0.56  0.14285714285714285  \\
            0.57  0.14285714285714285  \\
            0.58  0.14285714285714285  \\
            0.59  0.14285714285714285  \\
            0.6  0.14285714285714285  \\
            0.61  0.14285714285714285  \\
            0.62  0.14285714285714285  \\
            0.63  0.14285714285714285  \\
            0.64  0.14285714285714285  \\
            0.65  0.14285714285714285  \\
            0.66  0.14285714285714285  \\
            0.67  0.14285714285714285  \\
            0.68  0.14285714285714285  \\
            0.69  0.14285714285714285  \\
            0.7  0.14285714285714285  \\
            0.71  0.14285714285714285  \\
            0.72  0.14285714285714285  \\
            0.73  0.14285714285714285  \\
            0.74  0.14285714285714285  \\
            0.75  0.14285714285714285  \\
            0.76  0.14285714285714285  \\
            0.77  0.14285714285714285  \\
            0.78  0.14285714285714285  \\
            0.79  0.14285714285714285  \\
            0.8  0.14285714285714285  \\
            0.81  0.14285714285714285  \\
            0.82  0.14285714285714285  \\
            0.83  0.14285714285714285  \\
            0.84  0.14285714285714285  \\
            0.85  0.14285714285714285  \\
            0.86  0.14285714285714285  \\
            0.87  0.14285714285714285  \\
            0.88  0.14285714285714285  \\
            0.89  0.14285714285714285  \\
            0.9  0.14285714285714285  \\
            0.91  0.14285714285714285  \\
            0.92  0.14285714285714285  \\
            0.93  0.14285714285714285  \\
            0.94  0.14285714285714285  \\
            0.95  0.14285714285714285  \\
            0.96  0.14285714285714285  \\
            0.97  0.14285714285714285  \\
            0.98  0.14285714285714285  \\
            0.99  0.14285714285714285  \\
            1.0  0.0  \\
            1.01  0.0  \\
            1.02  0.0  \\
            1.03  0.0  \\
            1.04  0.0  \\
            1.05  0.0  \\
            1.06  0.0  \\
            1.07  0.0  \\
            1.08  0.0  \\
            1.09  0.0  \\
            1.1  0.0  \\
            1.11  0.0  \\
            1.12  0.0  \\
            1.13  0.0  \\
            1.14  0.0  \\
            1.15  0.0  \\
            1.16  0.0  \\
            1.17  0.0  \\
            1.18  0.0  \\
            1.19  0.0  \\
            1.2  0.0  \\
            1.21  0.0  \\
            1.22  0.0  \\
            1.23  0.0  \\
            1.24  0.0  \\
            1.25  0.0  \\
            1.26  0.0  \\
            1.27  0.0  \\
            1.28  0.0  \\
            1.29  0.0  \\
            1.3  0.0  \\
            1.31  0.0  \\
            1.32  0.0  \\
            1.33  0.0  \\
            1.34  0.0  \\
            1.35  0.0  \\
            1.36  0.0  \\
            1.37  0.0  \\
            1.38  0.0  \\
            1.39  0.0  \\
            1.4  0.0  \\
            1.41  0.0  \\
            1.42  0.0  \\
            1.43  0.0  \\
            1.44  0.0  \\
            1.45  0.0  \\
            1.46  0.0  \\
            1.47  0.0  \\
            1.48  0.0  \\
            1.49  0.0  \\
            1.5  0.0  \\
            1.51  0.0  \\
            1.52  0.0  \\
            1.53  0.0  \\
            1.54  0.0  \\
            1.55  0.0  \\
            1.56  0.0  \\
            1.57  0.0  \\
            1.58  0.0  \\
            1.59  0.0  \\
            1.6  0.0  \\
            1.61  0.0  \\
            1.62  0.0  \\
            1.63  0.0  \\
            1.64  0.0  \\
            1.65  0.0  \\
            1.66  0.0  \\
            1.67  0.0  \\
            1.68  0.0  \\
            1.69  0.0  \\
            1.7  0.0  \\
            1.71  0.0  \\
            1.72  0.0  \\
            1.73  0.0  \\
            1.74  0.0  \\
            1.75  0.0  \\
            1.76  0.0  \\
            1.77  0.0  \\
            1.78  0.0  \\
            1.79  0.0  \\
            1.8  0.0  \\
            1.81  0.0  \\
            1.82  0.0  \\
            1.83  0.0  \\
            1.84  0.0  \\
            1.85  0.0  \\
            1.86  0.0  \\
            1.87  0.0  \\
            1.88  0.0  \\
            1.89  0.0  \\
            1.9  0.0  \\
            1.91  0.0  \\
            1.92  0.0  \\
            1.93  0.0  \\
            1.94  0.0  \\
            1.95  0.0  \\
            1.96  0.0  \\
            1.97  0.0  \\
            1.98  0.0  \\
            1.99  0.0  \\
            2.0  0.0  \\
            2.01  0.0  \\
            2.02  0.0  \\
            2.03  0.0  \\
            2.04  0.0  \\
            2.05  0.0  \\
            2.06  0.0  \\
            2.07  0.0  \\
            2.08  0.0  \\
            2.09  0.0  \\
            2.1  0.0  \\
            2.11  0.0  \\
            2.12  0.0  \\
            2.13  0.0  \\
            2.14  0.0  \\
            2.15  0.0  \\
            2.16  0.0  \\
            2.17  0.0  \\
            2.18  0.0  \\
            2.19  0.0  \\
            2.2  0.0  \\
            2.21  0.0  \\
            2.22  0.0  \\
            2.23  0.0  \\
            2.24  0.0  \\
            2.25  0.0  \\
            2.26  0.0  \\
            2.27  0.0  \\
            2.28  0.0  \\
            2.29  0.0  \\
            2.3  0.0  \\
            2.31  0.0  \\
            2.32  0.0  \\
            2.33  0.0  \\
            2.34  0.0  \\
            2.35  0.0  \\
            2.36  0.0  \\
            2.37  0.0  \\
            2.38  0.0  \\
            2.39  0.0  \\
            2.4  0.0  \\
            2.41  0.0  \\
            2.42  0.0  \\
            2.43  0.0  \\
            2.44  0.0  \\
            2.45  0.0  \\
            2.46  0.0  \\
            2.47  0.0  \\
            2.48  0.0  \\
            2.49  0.0  \\
            2.5  0.0  \\
            2.51  0.0  \\
            2.52  0.0  \\
            2.53  0.0  \\
            2.54  0.0  \\
            2.55  0.0  \\
            2.56  0.0  \\
            2.57  0.0  \\
            2.58  0.0  \\
            2.59  0.0  \\
            2.6  0.0  \\
            2.61  0.0  \\
            2.62  0.0  \\
            2.63  0.0  \\
            2.64  0.0  \\
            2.65  0.0  \\
            2.66  0.0  \\
            2.67  0.0  \\
            2.68  0.0  \\
            2.69  0.0  \\
            2.7  0.0  \\
            2.71  0.0  \\
            2.72  0.0  \\
            2.73  0.0  \\
            2.74  0.0  \\
            2.75  0.0  \\
            2.76  0.0  \\
            2.77  0.0  \\
            2.78  0.0  \\
            2.79  0.0  \\
            2.8  0.0  \\
            2.81  0.0  \\
            2.82  0.0  \\
            2.83  0.0  \\
            2.84  0.0  \\
            2.85  0.0  \\
            2.86  0.0  \\
            2.87  0.0  \\
            2.88  0.0  \\
            2.89  0.0  \\
            2.9  0.0  \\
            2.91  0.0  \\
            2.92  0.0  \\
            2.93  0.0  \\
            2.94  0.0  \\
            2.95  0.0  \\
            2.96  0.0  \\
            2.97  0.0  \\
            2.98  0.0  \\
            2.99  0.0  \\
            3.0  nan  \\
        }
        \closedcycle
        ;
    \addplot+[fill, fill opacity={0.2}, thick]
        table[row sep={\\}]
        {
            \\
            -3.0  0.0  \\
            -2.99  0.0  \\
            -2.98  0.0  \\
            -2.97  0.0  \\
            -2.96  0.0  \\
            -2.95  0.0  \\
            -2.94  0.0  \\
            -2.93  0.0  \\
            -2.92  0.0  \\
            -2.91  0.0  \\
            -2.9  0.0  \\
            -2.89  0.0  \\
            -2.88  0.0  \\
            -2.87  0.0  \\
            -2.86  0.0  \\
            -2.85  0.0  \\
            -2.84  0.0  \\
            -2.83  0.0  \\
            -2.82  0.0  \\
            -2.81  0.0  \\
            -2.8  0.0  \\
            -2.79  0.0  \\
            -2.78  0.0  \\
            -2.77  0.0  \\
            -2.76  0.0  \\
            -2.75  0.0  \\
            -2.74  0.0  \\
            -2.73  0.0  \\
            -2.72  0.0  \\
            -2.71  0.0  \\
            -2.7  0.0  \\
            -2.69  0.0  \\
            -2.68  0.0  \\
            -2.67  0.0  \\
            -2.66  0.0  \\
            -2.65  0.0  \\
            -2.64  0.0  \\
            -2.63  0.0  \\
            -2.62  0.0  \\
            -2.61  0.0  \\
            -2.6  0.0  \\
            -2.59  0.0  \\
            -2.58  0.0  \\
            -2.57  0.0  \\
            -2.56  0.0  \\
            -2.55  0.0  \\
            -2.54  0.0  \\
            -2.53  0.0  \\
            -2.52  0.0  \\
            -2.51  0.0  \\
            -2.5  0.0  \\
            -2.49  0.0  \\
            -2.48  0.0  \\
            -2.47  0.0  \\
            -2.46  0.0  \\
            -2.45  0.0  \\
            -2.44  0.0  \\
            -2.43  0.0  \\
            -2.42  0.0  \\
            -2.41  0.0  \\
            -2.4  0.0  \\
            -2.39  0.0  \\
            -2.38  0.0  \\
            -2.37  0.0  \\
            -2.36  0.0  \\
            -2.35  0.0  \\
            -2.34  0.0  \\
            -2.33  0.0  \\
            -2.32  0.0  \\
            -2.31  0.0  \\
            -2.3  0.0  \\
            -2.29  0.0  \\
            -2.28  0.0  \\
            -2.27  0.0  \\
            -2.26  0.0  \\
            -2.25  0.0  \\
            -2.24  0.0  \\
            -2.23  0.0  \\
            -2.22  0.0  \\
            -2.21  0.0  \\
            -2.2  0.0  \\
            -2.19  0.0  \\
            -2.18  0.0  \\
            -2.17  0.0  \\
            -2.16  0.0  \\
            -2.15  0.0  \\
            -2.14  0.0  \\
            -2.13  0.0  \\
            -2.12  0.0  \\
            -2.11  0.0  \\
            -2.1  0.0  \\
            -2.09  0.0  \\
            -2.08  0.0  \\
            -2.07  0.0  \\
            -2.06  0.0  \\
            -2.05  0.0  \\
            -2.04  0.0  \\
            -2.03  0.0  \\
            -2.02  0.0  \\
            -2.01  0.0  \\
            -2.0  0.25  \\
            -1.99  0.25  \\
            -1.98  0.25  \\
            -1.97  0.25  \\
            -1.96  0.25  \\
            -1.95  0.25  \\
            -1.94  0.25  \\
            -1.93  0.25  \\
            -1.92  0.25  \\
            -1.91  0.25  \\
            -1.9  0.25  \\
            -1.89  0.25  \\
            -1.88  0.25  \\
            -1.87  0.25  \\
            -1.86  0.25  \\
            -1.85  0.25  \\
            -1.84  0.25  \\
            -1.83  0.25  \\
            -1.82  0.25  \\
            -1.81  0.25  \\
            -1.8  0.25  \\
            -1.79  0.25  \\
            -1.78  0.25  \\
            -1.77  0.25  \\
            -1.76  0.25  \\
            -1.75  0.25  \\
            -1.74  0.25  \\
            -1.73  0.25  \\
            -1.72  0.25  \\
            -1.71  0.25  \\
            -1.7  0.25  \\
            -1.69  0.25  \\
            -1.68  0.25  \\
            -1.67  0.25  \\
            -1.66  0.25  \\
            -1.65  0.25  \\
            -1.64  0.25  \\
            -1.63  0.25  \\
            -1.62  0.25  \\
            -1.61  0.25  \\
            -1.6  0.25  \\
            -1.59  0.25  \\
            -1.58  0.25  \\
            -1.57  0.25  \\
            -1.56  0.25  \\
            -1.55  0.25  \\
            -1.54  0.25  \\
            -1.53  0.25  \\
            -1.52  0.25  \\
            -1.51  0.25  \\
            -1.5  0.25  \\
            -1.49  0.25  \\
            -1.48  0.25  \\
            -1.47  0.25  \\
            -1.46  0.25  \\
            -1.45  0.25  \\
            -1.44  0.25  \\
            -1.43  0.25  \\
            -1.42  0.25  \\
            -1.41  0.25  \\
            -1.4  0.25  \\
            -1.39  0.25  \\
            -1.38  0.25  \\
            -1.37  0.25  \\
            -1.36  0.25  \\
            -1.35  0.25  \\
            -1.34  0.25  \\
            -1.33  0.25  \\
            -1.32  0.25  \\
            -1.31  0.25  \\
            -1.3  0.25  \\
            -1.29  0.25  \\
            -1.28  0.25  \\
            -1.27  0.25  \\
            -1.26  0.25  \\
            -1.25  0.25  \\
            -1.24  0.25  \\
            -1.23  0.25  \\
            -1.22  0.25  \\
            -1.21  0.25  \\
            -1.2  0.25  \\
            -1.19  0.25  \\
            -1.18  0.25  \\
            -1.17  0.25  \\
            -1.16  0.25  \\
            -1.15  0.25  \\
            -1.14  0.25  \\
            -1.13  0.25  \\
            -1.12  0.25  \\
            -1.11  0.25  \\
            -1.1  0.25  \\
            -1.09  0.25  \\
            -1.08  0.25  \\
            -1.07  0.25  \\
            -1.06  0.25  \\
            -1.05  0.25  \\
            -1.04  0.25  \\
            -1.03  0.25  \\
            -1.02  0.25  \\
            -1.01  0.25  \\
            -1.0  0.42857142857142855  \\
            -0.99  0.42857142857142855  \\
            -0.98  0.42857142857142855  \\
            -0.97  0.42857142857142855  \\
            -0.96  0.42857142857142855  \\
            -0.95  0.42857142857142855  \\
            -0.94  0.42857142857142855  \\
            -0.93  0.42857142857142855  \\
            -0.92  0.42857142857142855  \\
            -0.91  0.42857142857142855  \\
            -0.9  0.42857142857142855  \\
            -0.89  0.42857142857142855  \\
            -0.88  0.42857142857142855  \\
            -0.87  0.42857142857142855  \\
            -0.86  0.42857142857142855  \\
            -0.85  0.42857142857142855  \\
            -0.84  0.42857142857142855  \\
            -0.83  0.42857142857142855  \\
            -0.82  0.42857142857142855  \\
            -0.81  0.42857142857142855  \\
            -0.8  0.42857142857142855  \\
            -0.79  0.42857142857142855  \\
            -0.78  0.42857142857142855  \\
            -0.77  0.42857142857142855  \\
            -0.76  0.42857142857142855  \\
            -0.75  0.42857142857142855  \\
            -0.74  0.42857142857142855  \\
            -0.73  0.42857142857142855  \\
            -0.72  0.42857142857142855  \\
            -0.71  0.42857142857142855  \\
            -0.7  0.42857142857142855  \\
            -0.69  0.42857142857142855  \\
            -0.68  0.42857142857142855  \\
            -0.67  0.42857142857142855  \\
            -0.66  0.42857142857142855  \\
            -0.65  0.42857142857142855  \\
            -0.64  0.42857142857142855  \\
            -0.63  0.42857142857142855  \\
            -0.62  0.42857142857142855  \\
            -0.61  0.42857142857142855  \\
            -0.6  0.42857142857142855  \\
            -0.59  0.42857142857142855  \\
            -0.58  0.42857142857142855  \\
            -0.57  0.42857142857142855  \\
            -0.56  0.42857142857142855  \\
            -0.55  0.42857142857142855  \\
            -0.54  0.42857142857142855  \\
            -0.53  0.42857142857142855  \\
            -0.52  0.42857142857142855  \\
            -0.51  0.42857142857142855  \\
            -0.5  0.42857142857142855  \\
            -0.49  0.42857142857142855  \\
            -0.48  0.42857142857142855  \\
            -0.47  0.42857142857142855  \\
            -0.46  0.42857142857142855  \\
            -0.45  0.42857142857142855  \\
            -0.44  0.42857142857142855  \\
            -0.43  0.42857142857142855  \\
            -0.42  0.42857142857142855  \\
            -0.41  0.42857142857142855  \\
            -0.4  0.42857142857142855  \\
            -0.39  0.42857142857142855  \\
            -0.38  0.42857142857142855  \\
            -0.37  0.42857142857142855  \\
            -0.36  0.42857142857142855  \\
            -0.35  0.42857142857142855  \\
            -0.34  0.42857142857142855  \\
            -0.33  0.42857142857142855  \\
            -0.32  0.42857142857142855  \\
            -0.31  0.42857142857142855  \\
            -0.3  0.42857142857142855  \\
            -0.29  0.42857142857142855  \\
            -0.28  0.42857142857142855  \\
            -0.27  0.42857142857142855  \\
            -0.26  0.42857142857142855  \\
            -0.25  0.42857142857142855  \\
            -0.24  0.42857142857142855  \\
            -0.23  0.42857142857142855  \\
            -0.22  0.42857142857142855  \\
            -0.21  0.42857142857142855  \\
            -0.2  0.42857142857142855  \\
            -0.19  0.42857142857142855  \\
            -0.18  0.42857142857142855  \\
            -0.17  0.42857142857142855  \\
            -0.16  0.42857142857142855  \\
            -0.15  0.42857142857142855  \\
            -0.14  0.42857142857142855  \\
            -0.13  0.42857142857142855  \\
            -0.12  0.42857142857142855  \\
            -0.11  0.42857142857142855  \\
            -0.1  0.42857142857142855  \\
            -0.09  0.42857142857142855  \\
            -0.08  0.42857142857142855  \\
            -0.07  0.42857142857142855  \\
            -0.06  0.42857142857142855  \\
            -0.05  0.42857142857142855  \\
            -0.04  0.42857142857142855  \\
            -0.03  0.42857142857142855  \\
            -0.02  0.42857142857142855  \\
            -0.01  0.42857142857142855  \\
            0.0  0.42857142857142855  \\
            0.01  0.42857142857142855  \\
            0.02  0.42857142857142855  \\
            0.03  0.42857142857142855  \\
            0.04  0.42857142857142855  \\
            0.05  0.42857142857142855  \\
            0.06  0.42857142857142855  \\
            0.07  0.42857142857142855  \\
            0.08  0.42857142857142855  \\
            0.09  0.42857142857142855  \\
            0.1  0.42857142857142855  \\
            0.11  0.42857142857142855  \\
            0.12  0.42857142857142855  \\
            0.13  0.42857142857142855  \\
            0.14  0.42857142857142855  \\
            0.15  0.42857142857142855  \\
            0.16  0.42857142857142855  \\
            0.17  0.42857142857142855  \\
            0.18  0.42857142857142855  \\
            0.19  0.42857142857142855  \\
            0.2  0.42857142857142855  \\
            0.21  0.42857142857142855  \\
            0.22  0.42857142857142855  \\
            0.23  0.42857142857142855  \\
            0.24  0.42857142857142855  \\
            0.25  0.42857142857142855  \\
            0.26  0.42857142857142855  \\
            0.27  0.42857142857142855  \\
            0.28  0.42857142857142855  \\
            0.29  0.42857142857142855  \\
            0.3  0.42857142857142855  \\
            0.31  0.42857142857142855  \\
            0.32  0.42857142857142855  \\
            0.33  0.42857142857142855  \\
            0.34  0.42857142857142855  \\
            0.35  0.42857142857142855  \\
            0.36  0.42857142857142855  \\
            0.37  0.42857142857142855  \\
            0.38  0.42857142857142855  \\
            0.39  0.42857142857142855  \\
            0.4  0.42857142857142855  \\
            0.41  0.42857142857142855  \\
            0.42  0.42857142857142855  \\
            0.43  0.42857142857142855  \\
            0.44  0.42857142857142855  \\
            0.45  0.42857142857142855  \\
            0.46  0.42857142857142855  \\
            0.47  0.42857142857142855  \\
            0.48  0.42857142857142855  \\
            0.49  0.42857142857142855  \\
            0.5  0.42857142857142855  \\
            0.51  0.42857142857142855  \\
            0.52  0.42857142857142855  \\
            0.53  0.42857142857142855  \\
            0.54  0.42857142857142855  \\
            0.55  0.42857142857142855  \\
            0.56  0.42857142857142855  \\
            0.57  0.42857142857142855  \\
            0.58  0.42857142857142855  \\
            0.59  0.42857142857142855  \\
            0.6  0.42857142857142855  \\
            0.61  0.42857142857142855  \\
            0.62  0.42857142857142855  \\
            0.63  0.42857142857142855  \\
            0.64  0.42857142857142855  \\
            0.65  0.42857142857142855  \\
            0.66  0.42857142857142855  \\
            0.67  0.42857142857142855  \\
            0.68  0.42857142857142855  \\
            0.69  0.42857142857142855  \\
            0.7  0.42857142857142855  \\
            0.71  0.42857142857142855  \\
            0.72  0.42857142857142855  \\
            0.73  0.42857142857142855  \\
            0.74  0.42857142857142855  \\
            0.75  0.42857142857142855  \\
            0.76  0.42857142857142855  \\
            0.77  0.42857142857142855  \\
            0.78  0.42857142857142855  \\
            0.79  0.42857142857142855  \\
            0.8  0.42857142857142855  \\
            0.81  0.42857142857142855  \\
            0.82  0.42857142857142855  \\
            0.83  0.42857142857142855  \\
            0.84  0.42857142857142855  \\
            0.85  0.42857142857142855  \\
            0.86  0.42857142857142855  \\
            0.87  0.42857142857142855  \\
            0.88  0.42857142857142855  \\
            0.89  0.42857142857142855  \\
            0.9  0.42857142857142855  \\
            0.91  0.42857142857142855  \\
            0.92  0.42857142857142855  \\
            0.93  0.42857142857142855  \\
            0.94  0.42857142857142855  \\
            0.95  0.42857142857142855  \\
            0.96  0.42857142857142855  \\
            0.97  0.42857142857142855  \\
            0.98  0.42857142857142855  \\
            0.99  0.42857142857142855  \\
            1.0  0.25  \\
            1.01  0.25  \\
            1.02  0.25  \\
            1.03  0.25  \\
            1.04  0.25  \\
            1.05  0.25  \\
            1.06  0.25  \\
            1.07  0.25  \\
            1.08  0.25  \\
            1.09  0.25  \\
            1.1  0.25  \\
            1.11  0.25  \\
            1.12  0.25  \\
            1.13  0.25  \\
            1.14  0.25  \\
            1.15  0.25  \\
            1.16  0.25  \\
            1.17  0.25  \\
            1.18  0.25  \\
            1.19  0.25  \\
            1.2  0.25  \\
            1.21  0.25  \\
            1.22  0.25  \\
            1.23  0.25  \\
            1.24  0.25  \\
            1.25  0.25  \\
            1.26  0.25  \\
            1.27  0.25  \\
            1.28  0.25  \\
            1.29  0.25  \\
            1.3  0.25  \\
            1.31  0.25  \\
            1.32  0.25  \\
            1.33  0.25  \\
            1.34  0.25  \\
            1.35  0.25  \\
            1.36  0.25  \\
            1.37  0.25  \\
            1.38  0.25  \\
            1.39  0.25  \\
            1.4  0.25  \\
            1.41  0.25  \\
            1.42  0.25  \\
            1.43  0.25  \\
            1.44  0.25  \\
            1.45  0.25  \\
            1.46  0.25  \\
            1.47  0.25  \\
            1.48  0.25  \\
            1.49  0.25  \\
            1.5  0.25  \\
            1.51  0.25  \\
            1.52  0.25  \\
            1.53  0.25  \\
            1.54  0.25  \\
            1.55  0.25  \\
            1.56  0.25  \\
            1.57  0.25  \\
            1.58  0.25  \\
            1.59  0.25  \\
            1.6  0.25  \\
            1.61  0.25  \\
            1.62  0.25  \\
            1.63  0.25  \\
            1.64  0.25  \\
            1.65  0.25  \\
            1.66  0.25  \\
            1.67  0.25  \\
            1.68  0.25  \\
            1.69  0.25  \\
            1.7  0.25  \\
            1.71  0.25  \\
            1.72  0.25  \\
            1.73  0.25  \\
            1.74  0.25  \\
            1.75  0.25  \\
            1.76  0.25  \\
            1.77  0.25  \\
            1.78  0.25  \\
            1.79  0.25  \\
            1.8  0.25  \\
            1.81  0.25  \\
            1.82  0.25  \\
            1.83  0.25  \\
            1.84  0.25  \\
            1.85  0.25  \\
            1.86  0.25  \\
            1.87  0.25  \\
            1.88  0.25  \\
            1.89  0.25  \\
            1.9  0.25  \\
            1.91  0.25  \\
            1.92  0.25  \\
            1.93  0.25  \\
            1.94  0.25  \\
            1.95  0.25  \\
            1.96  0.25  \\
            1.97  0.25  \\
            1.98  0.25  \\
            1.99  0.25  \\
            2.0  0.0  \\
            2.01  0.0  \\
            2.02  0.0  \\
            2.03  0.0  \\
            2.04  0.0  \\
            2.05  0.0  \\
            2.06  0.0  \\
            2.07  0.0  \\
            2.08  0.0  \\
            2.09  0.0  \\
            2.1  0.0  \\
            2.11  0.0  \\
            2.12  0.0  \\
            2.13  0.0  \\
            2.14  0.0  \\
            2.15  0.0  \\
            2.16  0.0  \\
            2.17  0.0  \\
            2.18  0.0  \\
            2.19  0.0  \\
            2.2  0.0  \\
            2.21  0.0  \\
            2.22  0.0  \\
            2.23  0.0  \\
            2.24  0.0  \\
            2.25  0.0  \\
            2.26  0.0  \\
            2.27  0.0  \\
            2.28  0.0  \\
            2.29  0.0  \\
            2.3  0.0  \\
            2.31  0.0  \\
            2.32  0.0  \\
            2.33  0.0  \\
            2.34  0.0  \\
            2.35  0.0  \\
            2.36  0.0  \\
            2.37  0.0  \\
            2.38  0.0  \\
            2.39  0.0  \\
            2.4  0.0  \\
            2.41  0.0  \\
            2.42  0.0  \\
            2.43  0.0  \\
            2.44  0.0  \\
            2.45  0.0  \\
            2.46  0.0  \\
            2.47  0.0  \\
            2.48  0.0  \\
            2.49  0.0  \\
            2.5  0.0  \\
            2.51  0.0  \\
            2.52  0.0  \\
            2.53  0.0  \\
            2.54  0.0  \\
            2.55  0.0  \\
            2.56  0.0  \\
            2.57  0.0  \\
            2.58  0.0  \\
            2.59  0.0  \\
            2.6  0.0  \\
            2.61  0.0  \\
            2.62  0.0  \\
            2.63  0.0  \\
            2.64  0.0  \\
            2.65  0.0  \\
            2.66  0.0  \\
            2.67  0.0  \\
            2.68  0.0  \\
            2.69  0.0  \\
            2.7  0.0  \\
            2.71  0.0  \\
            2.72  0.0  \\
            2.73  0.0  \\
            2.74  0.0  \\
            2.75  0.0  \\
            2.76  0.0  \\
            2.77  0.0  \\
            2.78  0.0  \\
            2.79  0.0  \\
            2.8  0.0  \\
            2.81  0.0  \\
            2.82  0.0  \\
            2.83  0.0  \\
            2.84  0.0  \\
            2.85  0.0  \\
            2.86  0.0  \\
            2.87  0.0  \\
            2.88  0.0  \\
            2.89  0.0  \\
            2.9  0.0  \\
            2.91  0.0  \\
            2.92  0.0  \\
            2.93  0.0  \\
            2.94  0.0  \\
            2.95  0.0  \\
            2.96  0.0  \\
            2.97  0.0  \\
            2.98  0.0  \\
            2.99  0.0  \\
            3.0  nan  \\
        }
        \closedcycle
        ;
    \addplot+[fill, fill opacity={0.2}, thick]
        table[row sep={\\}]
        {
            \\
            -3.0  0.0  \\
            -2.99  0.0  \\
            -2.98  0.0  \\
            -2.97  0.0  \\
            -2.96  0.0  \\
            -2.95  0.0  \\
            -2.94  0.0  \\
            -2.93  0.0  \\
            -2.92  0.0  \\
            -2.91  0.0  \\
            -2.9  0.0  \\
            -2.89  0.0  \\
            -2.88  0.0  \\
            -2.87  0.0  \\
            -2.86  0.0  \\
            -2.85  0.0  \\
            -2.84  0.0  \\
            -2.83  0.0  \\
            -2.82  0.0  \\
            -2.81  0.0  \\
            -2.8  0.0  \\
            -2.79  0.0  \\
            -2.78  0.0  \\
            -2.77  0.0  \\
            -2.76  0.0  \\
            -2.75  0.0  \\
            -2.74  0.0  \\
            -2.73  0.0  \\
            -2.72  0.0  \\
            -2.71  0.0  \\
            -2.7  0.0  \\
            -2.69  0.0  \\
            -2.68  0.0  \\
            -2.67  0.0  \\
            -2.66  0.0  \\
            -2.65  0.0  \\
            -2.64  0.0  \\
            -2.63  0.0  \\
            -2.62  0.0  \\
            -2.61  0.0  \\
            -2.6  0.0  \\
            -2.59  0.0  \\
            -2.58  0.0  \\
            -2.57  0.0  \\
            -2.56  0.0  \\
            -2.55  0.0  \\
            -2.54  0.0  \\
            -2.53  0.0  \\
            -2.52  0.0  \\
            -2.51  0.0  \\
            -2.5  0.0  \\
            -2.49  0.0  \\
            -2.48  0.0  \\
            -2.47  0.0  \\
            -2.46  0.0  \\
            -2.45  0.0  \\
            -2.44  0.0  \\
            -2.43  0.0  \\
            -2.42  0.0  \\
            -2.41  0.0  \\
            -2.4  0.0  \\
            -2.39  0.0  \\
            -2.38  0.0  \\
            -2.37  0.0  \\
            -2.36  0.0  \\
            -2.35  0.0  \\
            -2.34  0.0  \\
            -2.33  0.0  \\
            -2.32  0.0  \\
            -2.31  0.0  \\
            -2.3  0.0  \\
            -2.29  0.0  \\
            -2.28  0.0  \\
            -2.27  0.0  \\
            -2.26  0.0  \\
            -2.25  0.0  \\
            -2.24  0.0  \\
            -2.23  0.0  \\
            -2.22  0.0  \\
            -2.21  0.0  \\
            -2.2  0.0  \\
            -2.19  0.0  \\
            -2.18  0.0  \\
            -2.17  0.0  \\
            -2.16  0.0  \\
            -2.15  0.0  \\
            -2.14  0.0  \\
            -2.13  0.0  \\
            -2.12  0.0  \\
            -2.11  0.0  \\
            -2.1  0.0  \\
            -2.09  0.0  \\
            -2.08  0.0  \\
            -2.07  0.0  \\
            -2.06  0.0  \\
            -2.05  0.0  \\
            -2.04  0.0  \\
            -2.03  0.0  \\
            -2.02  0.0  \\
            -2.01  0.0  \\
            -2.0  0.0  \\
            -1.99  0.0  \\
            -1.98  0.0  \\
            -1.97  0.0  \\
            -1.96  0.0  \\
            -1.95  0.0  \\
            -1.94  0.0  \\
            -1.93  0.0  \\
            -1.92  0.0  \\
            -1.91  0.0  \\
            -1.9  0.0  \\
            -1.89  0.0  \\
            -1.88  0.0  \\
            -1.87  0.0  \\
            -1.86  0.0  \\
            -1.85  0.0  \\
            -1.84  0.0  \\
            -1.83  0.0  \\
            -1.82  0.0  \\
            -1.81  0.0  \\
            -1.8  0.0  \\
            -1.79  0.0  \\
            -1.78  0.0  \\
            -1.77  0.0  \\
            -1.76  0.0  \\
            -1.75  0.0  \\
            -1.74  0.0  \\
            -1.73  0.0  \\
            -1.72  0.0  \\
            -1.71  0.0  \\
            -1.7  0.0  \\
            -1.69  0.0  \\
            -1.68  0.0  \\
            -1.67  0.0  \\
            -1.66  0.0  \\
            -1.65  0.0  \\
            -1.64  0.0  \\
            -1.63  0.0  \\
            -1.62  0.0  \\
            -1.61  0.0  \\
            -1.6  0.0  \\
            -1.59  0.0  \\
            -1.58  0.0  \\
            -1.57  0.0  \\
            -1.56  0.0  \\
            -1.55  0.0  \\
            -1.54  0.0  \\
            -1.53  0.0  \\
            -1.52  0.0  \\
            -1.51  0.0  \\
            -1.5  0.0  \\
            -1.49  0.0  \\
            -1.48  0.0  \\
            -1.47  0.0  \\
            -1.46  0.0  \\
            -1.45  0.0  \\
            -1.44  0.0  \\
            -1.43  0.0  \\
            -1.42  0.0  \\
            -1.41  0.0  \\
            -1.4  0.0  \\
            -1.39  0.0  \\
            -1.38  0.0  \\
            -1.37  0.0  \\
            -1.36  0.0  \\
            -1.35  0.0  \\
            -1.34  0.0  \\
            -1.33  0.0  \\
            -1.32  0.0  \\
            -1.31  0.0  \\
            -1.3  0.0  \\
            -1.29  0.0  \\
            -1.28  0.0  \\
            -1.27  0.0  \\
            -1.26  0.0  \\
            -1.25  0.0  \\
            -1.24  0.0  \\
            -1.23  0.0  \\
            -1.22  0.0  \\
            -1.21  0.0  \\
            -1.2  0.0  \\
            -1.19  0.0  \\
            -1.18  0.0  \\
            -1.17  0.0  \\
            -1.16  0.0  \\
            -1.15  0.0  \\
            -1.14  0.0  \\
            -1.13  0.0  \\
            -1.12  0.0  \\
            -1.11  0.0  \\
            -1.1  0.0  \\
            -1.09  0.0  \\
            -1.08  0.0  \\
            -1.07  0.0  \\
            -1.06  0.0  \\
            -1.05  0.0  \\
            -1.04  0.0  \\
            -1.03  0.0  \\
            -1.02  0.0  \\
            -1.01  0.0  \\
            -1.0  0.14285714285714285  \\
            -0.99  0.14285714285714285  \\
            -0.98  0.14285714285714285  \\
            -0.97  0.14285714285714285  \\
            -0.96  0.14285714285714285  \\
            -0.95  0.14285714285714285  \\
            -0.94  0.14285714285714285  \\
            -0.93  0.14285714285714285  \\
            -0.92  0.14285714285714285  \\
            -0.91  0.14285714285714285  \\
            -0.9  0.14285714285714285  \\
            -0.89  0.14285714285714285  \\
            -0.88  0.14285714285714285  \\
            -0.87  0.14285714285714285  \\
            -0.86  0.14285714285714285  \\
            -0.85  0.14285714285714285  \\
            -0.84  0.14285714285714285  \\
            -0.83  0.14285714285714285  \\
            -0.82  0.14285714285714285  \\
            -0.81  0.14285714285714285  \\
            -0.8  0.14285714285714285  \\
            -0.79  0.14285714285714285  \\
            -0.78  0.14285714285714285  \\
            -0.77  0.14285714285714285  \\
            -0.76  0.14285714285714285  \\
            -0.75  0.14285714285714285  \\
            -0.74  0.14285714285714285  \\
            -0.73  0.14285714285714285  \\
            -0.72  0.14285714285714285  \\
            -0.71  0.14285714285714285  \\
            -0.7  0.14285714285714285  \\
            -0.69  0.14285714285714285  \\
            -0.68  0.14285714285714285  \\
            -0.67  0.14285714285714285  \\
            -0.66  0.14285714285714285  \\
            -0.65  0.14285714285714285  \\
            -0.64  0.14285714285714285  \\
            -0.63  0.14285714285714285  \\
            -0.62  0.14285714285714285  \\
            -0.61  0.14285714285714285  \\
            -0.6  0.14285714285714285  \\
            -0.59  0.14285714285714285  \\
            -0.58  0.14285714285714285  \\
            -0.57  0.14285714285714285  \\
            -0.56  0.14285714285714285  \\
            -0.55  0.14285714285714285  \\
            -0.54  0.14285714285714285  \\
            -0.53  0.14285714285714285  \\
            -0.52  0.14285714285714285  \\
            -0.51  0.14285714285714285  \\
            -0.5  0.14285714285714285  \\
            -0.49  0.14285714285714285  \\
            -0.48  0.14285714285714285  \\
            -0.47  0.14285714285714285  \\
            -0.46  0.14285714285714285  \\
            -0.45  0.14285714285714285  \\
            -0.44  0.14285714285714285  \\
            -0.43  0.14285714285714285  \\
            -0.42  0.14285714285714285  \\
            -0.41  0.14285714285714285  \\
            -0.4  0.14285714285714285  \\
            -0.39  0.14285714285714285  \\
            -0.38  0.14285714285714285  \\
            -0.37  0.14285714285714285  \\
            -0.36  0.14285714285714285  \\
            -0.35  0.14285714285714285  \\
            -0.34  0.14285714285714285  \\
            -0.33  0.14285714285714285  \\
            -0.32  0.14285714285714285  \\
            -0.31  0.14285714285714285  \\
            -0.3  0.14285714285714285  \\
            -0.29  0.14285714285714285  \\
            -0.28  0.14285714285714285  \\
            -0.27  0.14285714285714285  \\
            -0.26  0.14285714285714285  \\
            -0.25  0.14285714285714285  \\
            -0.24  0.14285714285714285  \\
            -0.23  0.14285714285714285  \\
            -0.22  0.14285714285714285  \\
            -0.21  0.14285714285714285  \\
            -0.2  0.14285714285714285  \\
            -0.19  0.14285714285714285  \\
            -0.18  0.14285714285714285  \\
            -0.17  0.14285714285714285  \\
            -0.16  0.14285714285714285  \\
            -0.15  0.14285714285714285  \\
            -0.14  0.14285714285714285  \\
            -0.13  0.14285714285714285  \\
            -0.12  0.14285714285714285  \\
            -0.11  0.14285714285714285  \\
            -0.1  0.14285714285714285  \\
            -0.09  0.14285714285714285  \\
            -0.08  0.14285714285714285  \\
            -0.07  0.14285714285714285  \\
            -0.06  0.14285714285714285  \\
            -0.05  0.14285714285714285  \\
            -0.04  0.14285714285714285  \\
            -0.03  0.14285714285714285  \\
            -0.02  0.14285714285714285  \\
            -0.01  0.14285714285714285  \\
            0.0  0.42857142857142855  \\
            0.01  0.42857142857142855  \\
            0.02  0.42857142857142855  \\
            0.03  0.42857142857142855  \\
            0.04  0.42857142857142855  \\
            0.05  0.42857142857142855  \\
            0.06  0.42857142857142855  \\
            0.07  0.42857142857142855  \\
            0.08  0.42857142857142855  \\
            0.09  0.42857142857142855  \\
            0.1  0.42857142857142855  \\
            0.11  0.42857142857142855  \\
            0.12  0.42857142857142855  \\
            0.13  0.42857142857142855  \\
            0.14  0.42857142857142855  \\
            0.15  0.42857142857142855  \\
            0.16  0.42857142857142855  \\
            0.17  0.42857142857142855  \\
            0.18  0.42857142857142855  \\
            0.19  0.42857142857142855  \\
            0.2  0.42857142857142855  \\
            0.21  0.42857142857142855  \\
            0.22  0.42857142857142855  \\
            0.23  0.42857142857142855  \\
            0.24  0.42857142857142855  \\
            0.25  0.42857142857142855  \\
            0.26  0.42857142857142855  \\
            0.27  0.42857142857142855  \\
            0.28  0.42857142857142855  \\
            0.29  0.42857142857142855  \\
            0.3  0.42857142857142855  \\
            0.31  0.42857142857142855  \\
            0.32  0.42857142857142855  \\
            0.33  0.42857142857142855  \\
            0.34  0.42857142857142855  \\
            0.35  0.42857142857142855  \\
            0.36  0.42857142857142855  \\
            0.37  0.42857142857142855  \\
            0.38  0.42857142857142855  \\
            0.39  0.42857142857142855  \\
            0.4  0.42857142857142855  \\
            0.41  0.42857142857142855  \\
            0.42  0.42857142857142855  \\
            0.43  0.42857142857142855  \\
            0.44  0.42857142857142855  \\
            0.45  0.42857142857142855  \\
            0.46  0.42857142857142855  \\
            0.47  0.42857142857142855  \\
            0.48  0.42857142857142855  \\
            0.49  0.42857142857142855  \\
            0.5  0.42857142857142855  \\
            0.51  0.42857142857142855  \\
            0.52  0.42857142857142855  \\
            0.53  0.42857142857142855  \\
            0.54  0.42857142857142855  \\
            0.55  0.42857142857142855  \\
            0.56  0.42857142857142855  \\
            0.57  0.42857142857142855  \\
            0.58  0.42857142857142855  \\
            0.59  0.42857142857142855  \\
            0.6  0.42857142857142855  \\
            0.61  0.42857142857142855  \\
            0.62  0.42857142857142855  \\
            0.63  0.42857142857142855  \\
            0.64  0.42857142857142855  \\
            0.65  0.42857142857142855  \\
            0.66  0.42857142857142855  \\
            0.67  0.42857142857142855  \\
            0.68  0.42857142857142855  \\
            0.69  0.42857142857142855  \\
            0.7  0.42857142857142855  \\
            0.71  0.42857142857142855  \\
            0.72  0.42857142857142855  \\
            0.73  0.42857142857142855  \\
            0.74  0.42857142857142855  \\
            0.75  0.42857142857142855  \\
            0.76  0.42857142857142855  \\
            0.77  0.42857142857142855  \\
            0.78  0.42857142857142855  \\
            0.79  0.42857142857142855  \\
            0.8  0.42857142857142855  \\
            0.81  0.42857142857142855  \\
            0.82  0.42857142857142855  \\
            0.83  0.42857142857142855  \\
            0.84  0.42857142857142855  \\
            0.85  0.42857142857142855  \\
            0.86  0.42857142857142855  \\
            0.87  0.42857142857142855  \\
            0.88  0.42857142857142855  \\
            0.89  0.42857142857142855  \\
            0.9  0.42857142857142855  \\
            0.91  0.42857142857142855  \\
            0.92  0.42857142857142855  \\
            0.93  0.42857142857142855  \\
            0.94  0.42857142857142855  \\
            0.95  0.42857142857142855  \\
            0.96  0.42857142857142855  \\
            0.97  0.42857142857142855  \\
            0.98  0.42857142857142855  \\
            0.99  0.42857142857142855  \\
            1.0  0.75  \\
            1.01  0.75  \\
            1.02  0.75  \\
            1.03  0.75  \\
            1.04  0.75  \\
            1.05  0.75  \\
            1.06  0.75  \\
            1.07  0.75  \\
            1.08  0.75  \\
            1.09  0.75  \\
            1.1  0.75  \\
            1.11  0.75  \\
            1.12  0.75  \\
            1.13  0.75  \\
            1.14  0.75  \\
            1.15  0.75  \\
            1.16  0.75  \\
            1.17  0.75  \\
            1.18  0.75  \\
            1.19  0.75  \\
            1.2  0.75  \\
            1.21  0.75  \\
            1.22  0.75  \\
            1.23  0.75  \\
            1.24  0.75  \\
            1.25  0.75  \\
            1.26  0.75  \\
            1.27  0.75  \\
            1.28  0.75  \\
            1.29  0.75  \\
            1.3  0.75  \\
            1.31  0.75  \\
            1.32  0.75  \\
            1.33  0.75  \\
            1.34  0.75  \\
            1.35  0.75  \\
            1.36  0.75  \\
            1.37  0.75  \\
            1.38  0.75  \\
            1.39  0.75  \\
            1.4  0.75  \\
            1.41  0.75  \\
            1.42  0.75  \\
            1.43  0.75  \\
            1.44  0.75  \\
            1.45  0.75  \\
            1.46  0.75  \\
            1.47  0.75  \\
            1.48  0.75  \\
            1.49  0.75  \\
            1.5  0.75  \\
            1.51  0.75  \\
            1.52  0.75  \\
            1.53  0.75  \\
            1.54  0.75  \\
            1.55  0.75  \\
            1.56  0.75  \\
            1.57  0.75  \\
            1.58  0.75  \\
            1.59  0.75  \\
            1.6  0.75  \\
            1.61  0.75  \\
            1.62  0.75  \\
            1.63  0.75  \\
            1.64  0.75  \\
            1.65  0.75  \\
            1.66  0.75  \\
            1.67  0.75  \\
            1.68  0.75  \\
            1.69  0.75  \\
            1.7  0.75  \\
            1.71  0.75  \\
            1.72  0.75  \\
            1.73  0.75  \\
            1.74  0.75  \\
            1.75  0.75  \\
            1.76  0.75  \\
            1.77  0.75  \\
            1.78  0.75  \\
            1.79  0.75  \\
            1.8  0.75  \\
            1.81  0.75  \\
            1.82  0.75  \\
            1.83  0.75  \\
            1.84  0.75  \\
            1.85  0.75  \\
            1.86  0.75  \\
            1.87  0.75  \\
            1.88  0.75  \\
            1.89  0.75  \\
            1.9  0.75  \\
            1.91  0.75  \\
            1.92  0.75  \\
            1.93  0.75  \\
            1.94  0.75  \\
            1.95  0.75  \\
            1.96  0.75  \\
            1.97  0.75  \\
            1.98  0.75  \\
            1.99  0.75  \\
            2.0  1.0  \\
            2.01  1.0  \\
            2.02  1.0  \\
            2.03  1.0  \\
            2.04  1.0  \\
            2.05  1.0  \\
            2.06  1.0  \\
            2.07  1.0  \\
            2.08  1.0  \\
            2.09  1.0  \\
            2.1  1.0  \\
            2.11  1.0  \\
            2.12  1.0  \\
            2.13  1.0  \\
            2.14  1.0  \\
            2.15  1.0  \\
            2.16  1.0  \\
            2.17  1.0  \\
            2.18  1.0  \\
            2.19  1.0  \\
            2.2  1.0  \\
            2.21  1.0  \\
            2.22  1.0  \\
            2.23  1.0  \\
            2.24  1.0  \\
            2.25  1.0  \\
            2.26  1.0  \\
            2.27  1.0  \\
            2.28  1.0  \\
            2.29  1.0  \\
            2.3  1.0  \\
            2.31  1.0  \\
            2.32  1.0  \\
            2.33  1.0  \\
            2.34  1.0  \\
            2.35  1.0  \\
            2.36  1.0  \\
            2.37  1.0  \\
            2.38  1.0  \\
            2.39  1.0  \\
            2.4  1.0  \\
            2.41  1.0  \\
            2.42  1.0  \\
            2.43  1.0  \\
            2.44  1.0  \\
            2.45  1.0  \\
            2.46  1.0  \\
            2.47  1.0  \\
            2.48  1.0  \\
            2.49  1.0  \\
            2.5  1.0  \\
            2.51  1.0  \\
            2.52  1.0  \\
            2.53  1.0  \\
            2.54  1.0  \\
            2.55  1.0  \\
            2.56  1.0  \\
            2.57  1.0  \\
            2.58  1.0  \\
            2.59  1.0  \\
            2.6  1.0  \\
            2.61  1.0  \\
            2.62  1.0  \\
            2.63  1.0  \\
            2.64  1.0  \\
            2.65  1.0  \\
            2.66  1.0  \\
            2.67  1.0  \\
            2.68  1.0  \\
            2.69  1.0  \\
            2.7  1.0  \\
            2.71  1.0  \\
            2.72  1.0  \\
            2.73  1.0  \\
            2.74  1.0  \\
            2.75  1.0  \\
            2.76  1.0  \\
            2.77  1.0  \\
            2.78  1.0  \\
            2.79  1.0  \\
            2.8  1.0  \\
            2.81  1.0  \\
            2.82  1.0  \\
            2.83  1.0  \\
            2.84  1.0  \\
            2.85  1.0  \\
            2.86  1.0  \\
            2.87  1.0  \\
            2.88  1.0  \\
            2.89  1.0  \\
            2.9  1.0  \\
            2.91  1.0  \\
            2.92  1.0  \\
            2.93  1.0  \\
            2.94  1.0  \\
            2.95  1.0  \\
            2.96  1.0  \\
            2.97  1.0  \\
            2.98  1.0  \\
            2.99  1.0  \\
            3.0  nan  \\
        }
        \closedcycle
        ;
    \nextgroupplot[title={$g_y(X) = \mathbb{P}[Y = y \,|\, |X|]$}]
    \addplot+[fill, fill opacity={0.2}, thick]
        table[row sep={\\}]
        {
            \\
            -3.0  nan  \\
            -2.99  0.5  \\
            -2.98  0.5  \\
            -2.97  0.5  \\
            -2.96  0.5  \\
            -2.95  0.5  \\
            -2.94  0.5  \\
            -2.93  0.5  \\
            -2.92  0.5  \\
            -2.91  0.5  \\
            -2.9  0.5  \\
            -2.89  0.5  \\
            -2.88  0.5  \\
            -2.87  0.5  \\
            -2.86  0.5  \\
            -2.85  0.5  \\
            -2.84  0.5  \\
            -2.83  0.5  \\
            -2.82  0.5  \\
            -2.81  0.5  \\
            -2.8  0.5  \\
            -2.79  0.5  \\
            -2.78  0.5  \\
            -2.77  0.5  \\
            -2.76  0.5  \\
            -2.75  0.5  \\
            -2.74  0.5  \\
            -2.73  0.5  \\
            -2.72  0.5  \\
            -2.71  0.5  \\
            -2.7  0.5  \\
            -2.69  0.5  \\
            -2.68  0.5  \\
            -2.67  0.5  \\
            -2.66  0.5  \\
            -2.65  0.5  \\
            -2.64  0.5  \\
            -2.63  0.5  \\
            -2.62  0.5  \\
            -2.61  0.5  \\
            -2.6  0.5  \\
            -2.59  0.5  \\
            -2.58  0.5  \\
            -2.57  0.5  \\
            -2.56  0.5  \\
            -2.55  0.5  \\
            -2.54  0.5  \\
            -2.53  0.5  \\
            -2.52  0.5  \\
            -2.51  0.5  \\
            -2.5  0.5  \\
            -2.49  0.5  \\
            -2.48  0.5  \\
            -2.47  0.5  \\
            -2.46  0.5  \\
            -2.45  0.5  \\
            -2.44  0.5  \\
            -2.43  0.5  \\
            -2.42  0.5  \\
            -2.41  0.5  \\
            -2.4  0.5  \\
            -2.39  0.5  \\
            -2.38  0.5  \\
            -2.37  0.5  \\
            -2.36  0.5  \\
            -2.35  0.5  \\
            -2.34  0.5  \\
            -2.33  0.5  \\
            -2.32  0.5  \\
            -2.31  0.5  \\
            -2.3  0.5  \\
            -2.29  0.5  \\
            -2.28  0.5  \\
            -2.27  0.5  \\
            -2.26  0.5  \\
            -2.25  0.5  \\
            -2.24  0.5  \\
            -2.23  0.5  \\
            -2.22  0.5  \\
            -2.21  0.5  \\
            -2.2  0.5  \\
            -2.19  0.5  \\
            -2.18  0.5  \\
            -2.17  0.5  \\
            -2.16  0.5  \\
            -2.15  0.5  \\
            -2.14  0.5  \\
            -2.13  0.5  \\
            -2.12  0.5  \\
            -2.11  0.5  \\
            -2.1  0.5  \\
            -2.09  0.5  \\
            -2.08  0.5  \\
            -2.07  0.5  \\
            -2.06  0.5  \\
            -2.05  0.5  \\
            -2.04  0.5  \\
            -2.03  0.5  \\
            -2.02  0.5  \\
            -2.01  0.5  \\
            -2.0  0.5  \\
            -1.99  0.4950980392156863  \\
            -1.98  0.49038461538461536  \\
            -1.97  0.4858490566037736  \\
            -1.96  0.48148148148148145  \\
            -1.95  0.47727272727272724  \\
            -1.94  0.4732142857142857  \\
            -1.93  0.46929824561403505  \\
            -1.92  0.46551724137931033  \\
            -1.91  0.461864406779661  \\
            -1.9  0.4583333333333333  \\
            -1.89  0.4549180327868852  \\
            -1.88  0.45161290322580644  \\
            -1.87  0.44841269841269843  \\
            -1.86  0.44531250000000006  \\
            -1.85  0.44230769230769235  \\
            -1.84  0.4393939393939394  \\
            -1.83  0.4365671641791045  \\
            -1.82  0.4338235294117647  \\
            -1.81  0.4311594202898551  \\
            -1.8  0.4285714285714286  \\
            -1.79  0.426056338028169  \\
            -1.78  0.4236111111111111  \\
            -1.77  0.4212328767123288  \\
            -1.76  0.4189189189189189  \\
            -1.75  0.4166666666666667  \\
            -1.74  0.4144736842105263  \\
            -1.73  0.41233766233766234  \\
            -1.72  0.41025641025641024  \\
            -1.71  0.40822784810126583  \\
            -1.7  0.40625  \\
            -1.69  0.40432098765432095  \\
            -1.68  0.40243902439024387  \\
            -1.67  0.4006024096385542  \\
            -1.66  0.3988095238095238  \\
            -1.65  0.39705882352941174  \\
            -1.64  0.3953488372093023  \\
            -1.63  0.3936781609195402  \\
            -1.62  0.3920454545454546  \\
            -1.61  0.3904494382022472  \\
            -1.6  0.3888888888888889  \\
            -1.59  0.3873626373626374  \\
            -1.58  0.3858695652173913  \\
            -1.57  0.3844086021505376  \\
            -1.56  0.3829787234042553  \\
            -1.55  0.3815789473684211  \\
            -1.54  0.3802083333333333  \\
            -1.53  0.3788659793814433  \\
            -1.52  0.37755102040816324  \\
            -1.51  0.37626262626262624  \\
            -1.5  0.375  \\
            -1.49  0.37376237623762376  \\
            -1.48  0.37254901960784315  \\
            -1.47  0.3713592233009709  \\
            -1.46  0.3701923076923077  \\
            -1.45  0.36904761904761907  \\
            -1.44  0.36792452830188677  \\
            -1.43  0.36682242990654207  \\
            -1.42  0.36574074074074076  \\
            -1.41  0.3646788990825688  \\
            -1.4  0.36363636363636365  \\
            -1.39  0.36261261261261263  \\
            -1.38  0.36160714285714285  \\
            -1.37  0.3606194690265487  \\
            -1.36  0.35964912280701755  \\
            -1.35  0.358695652173913  \\
            -1.34  0.3577586206896552  \\
            -1.33  0.35683760683760685  \\
            -1.32  0.3559322033898305  \\
            -1.31  0.3550420168067227  \\
            -1.3  0.3541666666666667  \\
            -1.29  0.35330578512396693  \\
            -1.28  0.3524590163934426  \\
            -1.27  0.3516260162601626  \\
            -1.26  0.35080645161290325  \\
            -1.25  0.35  \\
            -1.24  0.3492063492063492  \\
            -1.23  0.3484251968503937  \\
            -1.22  0.34765625  \\
            -1.21  0.34689922480620156  \\
            -1.2  0.34615384615384615  \\
            -1.19  0.34541984732824427  \\
            -1.18  0.34469696969696967  \\
            -1.17  0.34398496240601506  \\
            -1.16  0.34328358208955223  \\
            -1.15  0.34259259259259256  \\
            -1.14  0.34191176470588236  \\
            -1.13  0.34124087591240876  \\
            -1.12  0.34057971014492755  \\
            -1.11  0.33992805755395683  \\
            -1.1  0.3392857142857143  \\
            -1.09  0.33865248226950356  \\
            -1.08  0.3380281690140845  \\
            -1.07  0.3374125874125874  \\
            -1.06  0.3368055555555556  \\
            -1.05  0.33620689655172414  \\
            -1.04  0.3356164383561644  \\
            -1.03  0.33503401360544216  \\
            -1.02  0.3344594594594595  \\
            -1.01  0.3338926174496644  \\
            -1.0  0.3333333333333333  \\
            -0.99  0.33222591362126247  \\
            -0.98  0.33112582781456956  \\
            -0.97  0.33003300330033003  \\
            -0.96  0.32894736842105265  \\
            -0.95  0.3278688524590164  \\
            -0.94  0.32679738562091504  \\
            -0.93  0.32573289902280134  \\
            -0.92  0.3246753246753247  \\
            -0.91  0.3236245954692557  \\
            -0.9  0.3225806451612903  \\
            -0.89  0.3215434083601286  \\
            -0.88  0.3205128205128205  \\
            -0.87  0.3194888178913738  \\
            -0.86  0.3184713375796178  \\
            -0.85  0.31746031746031744  \\
            -0.84  0.3164556962025316  \\
            -0.83  0.31545741324921134  \\
            -0.82  0.31446540880503143  \\
            -0.81  0.31347962382445144  \\
            -0.8  0.3125  \\
            -0.79  0.3115264797507788  \\
            -0.78  0.31055900621118016  \\
            -0.77  0.30959752321981426  \\
            -0.76  0.30864197530864196  \\
            -0.75  0.3076923076923077  \\
            -0.74  0.3067484662576687  \\
            -0.73  0.3058103975535168  \\
            -0.72  0.3048780487804878  \\
            -0.71  0.303951367781155  \\
            -0.7  0.30303030303030304  \\
            -0.69  0.3021148036253776  \\
            -0.68  0.30120481927710846  \\
            -0.67  0.3003003003003003  \\
            -0.66  0.29940119760479045  \\
            -0.65  0.29850746268656714  \\
            -0.64  0.2976190476190476  \\
            -0.63  0.29673590504451036  \\
            -0.62  0.2958579881656805  \\
            -0.61  0.2949852507374631  \\
            -0.6  0.29411764705882354  \\
            -0.59  0.29325513196480935  \\
            -0.58  0.29239766081871343  \\
            -0.57  0.29154518950437314  \\
            -0.56  0.29069767441860467  \\
            -0.55  0.2898550724637681  \\
            -0.54  0.28901734104046245  \\
            -0.53  0.2881844380403458  \\
            -0.52  0.28735632183908044  \\
            -0.51  0.28653295128939826  \\
            -0.5  0.2857142857142857  \\
            -0.49  0.2849002849002849  \\
            -0.48  0.2840909090909091  \\
            -0.47  0.28328611898017  \\
            -0.46  0.2824858757062147  \\
            -0.45  0.28169014084507044  \\
            -0.44  0.2808988764044944  \\
            -0.43  0.2801120448179272  \\
            -0.42  0.27932960893854747  \\
            -0.41  0.2785515320334262  \\
            -0.4  0.2777777777777778  \\
            -0.39  0.2770083102493075  \\
            -0.38  0.27624309392265195  \\
            -0.37  0.27548209366391185  \\
            -0.36  0.2747252747252747  \\
            -0.35  0.273972602739726  \\
            -0.34  0.27322404371584696  \\
            -0.33  0.2724795640326976  \\
            -0.32  0.27173913043478265  \\
            -0.31  0.2710027100271003  \\
            -0.3  0.27027027027027023  \\
            -0.29  0.2695417789757412  \\
            -0.28  0.2688172043010753  \\
            -0.27  0.2680965147453083  \\
            -0.26  0.267379679144385  \\
            -0.25  0.26666666666666666  \\
            -0.24  0.26595744680851063  \\
            -0.23  0.26525198938992045  \\
            -0.22  0.26455026455026454  \\
            -0.21  0.2638522427440633  \\
            -0.2  0.2631578947368421  \\
            -0.19  0.26246719160104987  \\
            -0.18  0.2617801047120419  \\
            -0.17  0.2610966057441253  \\
            -0.16  0.2604166666666667  \\
            -0.15  0.2597402597402597  \\
            -0.14  0.2590673575129534  \\
            -0.13  0.25839793281653745  \\
            -0.12  0.2577319587628866  \\
            -0.11  0.25706940874035994  \\
            -0.1  0.25641025641025644  \\
            -0.09  0.2557544757033248  \\
            -0.08  0.25510204081632654  \\
            -0.07  0.2544529262086514  \\
            -0.06  0.25380710659898476  \\
            -0.05  0.2531645569620253  \\
            -0.04  0.25252525252525254  \\
            -0.03  0.2518891687657431  \\
            -0.02  0.25125628140703515  \\
            -0.01  0.2506265664160401  \\
            0.0  0.25  \\
            0.01  0.2506265664160401  \\
            0.02  0.25125628140703515  \\
            0.03  0.2518891687657431  \\
            0.04  0.25252525252525254  \\
            0.05  0.2531645569620253  \\
            0.06  0.25380710659898476  \\
            0.07  0.2544529262086514  \\
            0.08  0.25510204081632654  \\
            0.09  0.2557544757033248  \\
            0.1  0.25641025641025644  \\
            0.11  0.25706940874035994  \\
            0.12  0.2577319587628866  \\
            0.13  0.25839793281653745  \\
            0.14  0.2590673575129534  \\
            0.15  0.2597402597402597  \\
            0.16  0.2604166666666667  \\
            0.17  0.2610966057441253  \\
            0.18  0.2617801047120419  \\
            0.19  0.26246719160104987  \\
            0.2  0.2631578947368421  \\
            0.21  0.2638522427440633  \\
            0.22  0.26455026455026454  \\
            0.23  0.26525198938992045  \\
            0.24  0.26595744680851063  \\
            0.25  0.26666666666666666  \\
            0.26  0.267379679144385  \\
            0.27  0.2680965147453083  \\
            0.28  0.2688172043010753  \\
            0.29  0.2695417789757412  \\
            0.3  0.27027027027027023  \\
            0.31  0.2710027100271003  \\
            0.32  0.27173913043478265  \\
            0.33  0.2724795640326976  \\
            0.34  0.27322404371584696  \\
            0.35  0.273972602739726  \\
            0.36  0.2747252747252747  \\
            0.37  0.27548209366391185  \\
            0.38  0.27624309392265195  \\
            0.39  0.2770083102493075  \\
            0.4  0.2777777777777778  \\
            0.41  0.2785515320334262  \\
            0.42  0.27932960893854747  \\
            0.43  0.2801120448179272  \\
            0.44  0.2808988764044944  \\
            0.45  0.28169014084507044  \\
            0.46  0.2824858757062147  \\
            0.47  0.28328611898017  \\
            0.48  0.2840909090909091  \\
            0.49  0.2849002849002849  \\
            0.5  0.2857142857142857  \\
            0.51  0.28653295128939826  \\
            0.52  0.28735632183908044  \\
            0.53  0.2881844380403458  \\
            0.54  0.28901734104046245  \\
            0.55  0.2898550724637681  \\
            0.56  0.29069767441860467  \\
            0.57  0.29154518950437314  \\
            0.58  0.29239766081871343  \\
            0.59  0.29325513196480935  \\
            0.6  0.29411764705882354  \\
            0.61  0.2949852507374631  \\
            0.62  0.2958579881656805  \\
            0.63  0.29673590504451036  \\
            0.64  0.2976190476190476  \\
            0.65  0.29850746268656714  \\
            0.66  0.29940119760479045  \\
            0.67  0.3003003003003003  \\
            0.68  0.30120481927710846  \\
            0.69  0.3021148036253776  \\
            0.7  0.30303030303030304  \\
            0.71  0.303951367781155  \\
            0.72  0.3048780487804878  \\
            0.73  0.3058103975535168  \\
            0.74  0.3067484662576687  \\
            0.75  0.3076923076923077  \\
            0.76  0.30864197530864196  \\
            0.77  0.30959752321981426  \\
            0.78  0.31055900621118016  \\
            0.79  0.3115264797507788  \\
            0.8  0.3125  \\
            0.81  0.31347962382445144  \\
            0.82  0.31446540880503143  \\
            0.83  0.31545741324921134  \\
            0.84  0.3164556962025316  \\
            0.85  0.31746031746031744  \\
            0.86  0.3184713375796178  \\
            0.87  0.3194888178913738  \\
            0.88  0.3205128205128205  \\
            0.89  0.3215434083601286  \\
            0.9  0.3225806451612903  \\
            0.91  0.3236245954692557  \\
            0.92  0.3246753246753247  \\
            0.93  0.32573289902280134  \\
            0.94  0.32679738562091504  \\
            0.95  0.3278688524590164  \\
            0.96  0.32894736842105265  \\
            0.97  0.33003300330033003  \\
            0.98  0.33112582781456956  \\
            0.99  0.33222591362126247  \\
            1.0  0.3333333333333333  \\
            1.01  0.3338926174496644  \\
            1.02  0.3344594594594595  \\
            1.03  0.33503401360544216  \\
            1.04  0.3356164383561644  \\
            1.05  0.33620689655172414  \\
            1.06  0.3368055555555556  \\
            1.07  0.3374125874125874  \\
            1.08  0.3380281690140845  \\
            1.09  0.33865248226950356  \\
            1.1  0.3392857142857143  \\
            1.11  0.33992805755395683  \\
            1.12  0.34057971014492755  \\
            1.13  0.34124087591240876  \\
            1.14  0.34191176470588236  \\
            1.15  0.34259259259259256  \\
            1.16  0.34328358208955223  \\
            1.17  0.34398496240601506  \\
            1.18  0.34469696969696967  \\
            1.19  0.34541984732824427  \\
            1.2  0.34615384615384615  \\
            1.21  0.34689922480620156  \\
            1.22  0.34765625  \\
            1.23  0.3484251968503937  \\
            1.24  0.3492063492063492  \\
            1.25  0.35  \\
            1.26  0.35080645161290325  \\
            1.27  0.3516260162601626  \\
            1.28  0.3524590163934426  \\
            1.29  0.35330578512396693  \\
            1.3  0.3541666666666667  \\
            1.31  0.3550420168067227  \\
            1.32  0.3559322033898305  \\
            1.33  0.35683760683760685  \\
            1.34  0.3577586206896552  \\
            1.35  0.358695652173913  \\
            1.36  0.35964912280701755  \\
            1.37  0.3606194690265487  \\
            1.38  0.36160714285714285  \\
            1.39  0.36261261261261263  \\
            1.4  0.36363636363636365  \\
            1.41  0.3646788990825688  \\
            1.42  0.36574074074074076  \\
            1.43  0.36682242990654207  \\
            1.44  0.36792452830188677  \\
            1.45  0.36904761904761907  \\
            1.46  0.3701923076923077  \\
            1.47  0.3713592233009709  \\
            1.48  0.37254901960784315  \\
            1.49  0.37376237623762376  \\
            1.5  0.375  \\
            1.51  0.37626262626262624  \\
            1.52  0.37755102040816324  \\
            1.53  0.3788659793814433  \\
            1.54  0.3802083333333333  \\
            1.55  0.3815789473684211  \\
            1.56  0.3829787234042553  \\
            1.57  0.3844086021505376  \\
            1.58  0.3858695652173913  \\
            1.59  0.3873626373626374  \\
            1.6  0.3888888888888889  \\
            1.61  0.3904494382022472  \\
            1.62  0.3920454545454546  \\
            1.63  0.3936781609195402  \\
            1.64  0.3953488372093023  \\
            1.65  0.39705882352941174  \\
            1.66  0.3988095238095238  \\
            1.67  0.4006024096385542  \\
            1.68  0.40243902439024387  \\
            1.69  0.40432098765432095  \\
            1.7  0.40625  \\
            1.71  0.40822784810126583  \\
            1.72  0.41025641025641024  \\
            1.73  0.41233766233766234  \\
            1.74  0.4144736842105263  \\
            1.75  0.4166666666666667  \\
            1.76  0.4189189189189189  \\
            1.77  0.4212328767123288  \\
            1.78  0.4236111111111111  \\
            1.79  0.426056338028169  \\
            1.8  0.4285714285714286  \\
            1.81  0.4311594202898551  \\
            1.82  0.4338235294117647  \\
            1.83  0.4365671641791045  \\
            1.84  0.4393939393939394  \\
            1.85  0.44230769230769235  \\
            1.86  0.44531250000000006  \\
            1.87  0.44841269841269843  \\
            1.88  0.45161290322580644  \\
            1.89  0.4549180327868852  \\
            1.9  0.4583333333333333  \\
            1.91  0.461864406779661  \\
            1.92  0.46551724137931033  \\
            1.93  0.46929824561403505  \\
            1.94  0.4732142857142857  \\
            1.95  0.47727272727272724  \\
            1.96  0.48148148148148145  \\
            1.97  0.4858490566037736  \\
            1.98  0.49038461538461536  \\
            1.99  0.4950980392156863  \\
            2.0  0.5  \\
            2.01  0.5  \\
            2.02  0.5  \\
            2.03  0.5  \\
            2.04  0.5  \\
            2.05  0.5  \\
            2.06  0.5  \\
            2.07  0.5  \\
            2.08  0.5  \\
            2.09  0.5  \\
            2.1  0.5  \\
            2.11  0.5  \\
            2.12  0.5  \\
            2.13  0.5  \\
            2.14  0.5  \\
            2.15  0.5  \\
            2.16  0.5  \\
            2.17  0.5  \\
            2.18  0.5  \\
            2.19  0.5  \\
            2.2  0.5  \\
            2.21  0.5  \\
            2.22  0.5  \\
            2.23  0.5  \\
            2.24  0.5  \\
            2.25  0.5  \\
            2.26  0.5  \\
            2.27  0.5  \\
            2.28  0.5  \\
            2.29  0.5  \\
            2.3  0.5  \\
            2.31  0.5  \\
            2.32  0.5  \\
            2.33  0.5  \\
            2.34  0.5  \\
            2.35  0.5  \\
            2.36  0.5  \\
            2.37  0.5  \\
            2.38  0.5  \\
            2.39  0.5  \\
            2.4  0.5  \\
            2.41  0.5  \\
            2.42  0.5  \\
            2.43  0.5  \\
            2.44  0.5  \\
            2.45  0.5  \\
            2.46  0.5  \\
            2.47  0.5  \\
            2.48  0.5  \\
            2.49  0.5  \\
            2.5  0.5  \\
            2.51  0.5  \\
            2.52  0.5  \\
            2.53  0.5  \\
            2.54  0.5  \\
            2.55  0.5  \\
            2.56  0.5  \\
            2.57  0.5  \\
            2.58  0.5  \\
            2.59  0.5  \\
            2.6  0.5  \\
            2.61  0.5  \\
            2.62  0.5  \\
            2.63  0.5  \\
            2.64  0.5  \\
            2.65  0.5  \\
            2.66  0.5  \\
            2.67  0.5  \\
            2.68  0.5  \\
            2.69  0.5  \\
            2.7  0.5  \\
            2.71  0.5  \\
            2.72  0.5  \\
            2.73  0.5  \\
            2.74  0.5  \\
            2.75  0.5  \\
            2.76  0.5  \\
            2.77  0.5  \\
            2.78  0.5  \\
            2.79  0.5  \\
            2.8  0.5  \\
            2.81  0.5  \\
            2.82  0.5  \\
            2.83  0.5  \\
            2.84  0.5  \\
            2.85  0.5  \\
            2.86  0.5  \\
            2.87  0.5  \\
            2.88  0.5  \\
            2.89  0.5  \\
            2.9  0.5  \\
            2.91  0.5  \\
            2.92  0.5  \\
            2.93  0.5  \\
            2.94  0.5  \\
            2.95  0.5  \\
            2.96  0.5  \\
            2.97  0.5  \\
            2.98  0.5  \\
            2.99  0.5  \\
            3.0  nan  \\
        }
        \closedcycle
        ;
    \addplot+[fill, fill opacity={0.2}, thick]
        table[row sep={\\}]
        {
            \\
            -3.0  nan  \\
            -2.99  0.0  \\
            -2.98  0.0  \\
            -2.97  0.0  \\
            -2.96  0.0  \\
            -2.95  0.0  \\
            -2.94  0.0  \\
            -2.93  0.0  \\
            -2.92  0.0  \\
            -2.91  0.0  \\
            -2.9  0.0  \\
            -2.89  0.0  \\
            -2.88  0.0  \\
            -2.87  0.0  \\
            -2.86  0.0  \\
            -2.85  0.0  \\
            -2.84  0.0  \\
            -2.83  0.0  \\
            -2.82  0.0  \\
            -2.81  0.0  \\
            -2.8  0.0  \\
            -2.79  0.0  \\
            -2.78  0.0  \\
            -2.77  0.0  \\
            -2.76  0.0  \\
            -2.75  0.0  \\
            -2.74  0.0  \\
            -2.73  0.0  \\
            -2.72  0.0  \\
            -2.71  0.0  \\
            -2.7  0.0  \\
            -2.69  0.0  \\
            -2.68  0.0  \\
            -2.67  0.0  \\
            -2.66  0.0  \\
            -2.65  0.0  \\
            -2.64  0.0  \\
            -2.63  0.0  \\
            -2.62  0.0  \\
            -2.61  0.0  \\
            -2.6  0.0  \\
            -2.59  0.0  \\
            -2.58  0.0  \\
            -2.57  0.0  \\
            -2.56  0.0  \\
            -2.55  0.0  \\
            -2.54  0.0  \\
            -2.53  0.0  \\
            -2.52  0.0  \\
            -2.51  0.0  \\
            -2.5  0.0  \\
            -2.49  0.0  \\
            -2.48  0.0  \\
            -2.47  0.0  \\
            -2.46  0.0  \\
            -2.45  0.0  \\
            -2.44  0.0  \\
            -2.43  0.0  \\
            -2.42  0.0  \\
            -2.41  0.0  \\
            -2.4  0.0  \\
            -2.39  0.0  \\
            -2.38  0.0  \\
            -2.37  0.0  \\
            -2.36  0.0  \\
            -2.35  0.0  \\
            -2.34  0.0  \\
            -2.33  0.0  \\
            -2.32  0.0  \\
            -2.31  0.0  \\
            -2.3  0.0  \\
            -2.29  0.0  \\
            -2.28  0.0  \\
            -2.27  0.0  \\
            -2.26  0.0  \\
            -2.25  0.0  \\
            -2.24  0.0  \\
            -2.23  0.0  \\
            -2.22  0.0  \\
            -2.21  0.0  \\
            -2.2  0.0  \\
            -2.19  0.0  \\
            -2.18  0.0  \\
            -2.17  0.0  \\
            -2.16  0.0  \\
            -2.15  0.0  \\
            -2.14  0.0  \\
            -2.13  0.0  \\
            -2.12  0.0  \\
            -2.11  0.0  \\
            -2.1  0.0  \\
            -2.09  0.0  \\
            -2.08  0.0  \\
            -2.07  0.0  \\
            -2.06  0.0  \\
            -2.05  0.0  \\
            -2.04  0.0  \\
            -2.03  0.0  \\
            -2.02  0.0  \\
            -2.01  0.0  \\
            -2.0  0.0  \\
            -1.99  0.00980392156862746  \\
            -1.98  0.019230769230769246  \\
            -1.97  0.028301886792452855  \\
            -1.96  0.03703703703703707  \\
            -1.95  0.04545454545454549  \\
            -1.94  0.05357142857142862  \\
            -1.93  0.06140350877192987  \\
            -1.92  0.06896551724137936  \\
            -1.91  0.07627118644067803  \\
            -1.9  0.0833333333333334  \\
            -1.89  0.09016393442622958  \\
            -1.88  0.09677419354838716  \\
            -1.87  0.10317460317460311  \\
            -1.86  0.10937499999999994  \\
            -1.85  0.11538461538461534  \\
            -1.84  0.12121212121212116  \\
            -1.83  0.126865671641791  \\
            -1.82  0.13235294117647056  \\
            -1.81  0.13768115942028983  \\
            -1.8  0.14285714285714282  \\
            -1.79  0.14788732394366194  \\
            -1.78  0.15277777777777776  \\
            -1.77  0.15753424657534246  \\
            -1.76  0.16216216216216217  \\
            -1.75  0.16666666666666666  \\
            -1.74  0.17105263157894737  \\
            -1.73  0.17532467532467533  \\
            -1.72  0.1794871794871795  \\
            -1.71  0.18354430379746836  \\
            -1.7  0.18750000000000003  \\
            -1.69  0.19135802469135804  \\
            -1.68  0.19512195121951223  \\
            -1.67  0.1987951807228916  \\
            -1.66  0.2023809523809524  \\
            -1.65  0.2058823529411765  \\
            -1.64  0.2093023255813954  \\
            -1.63  0.21264367816091959  \\
            -1.62  0.21590909090909088  \\
            -1.61  0.2191011235955056  \\
            -1.6  0.22222222222222218  \\
            -1.59  0.22527472527472525  \\
            -1.58  0.22826086956521738  \\
            -1.57  0.23118279569892472  \\
            -1.56  0.23404255319148934  \\
            -1.55  0.23684210526315788  \\
            -1.54  0.23958333333333331  \\
            -1.53  0.2422680412371134  \\
            -1.52  0.24489795918367346  \\
            -1.51  0.2474747474747475  \\
            -1.5  0.25  \\
            -1.49  0.2524752475247525  \\
            -1.48  0.2549019607843137  \\
            -1.47  0.25728155339805825  \\
            -1.46  0.25961538461538464  \\
            -1.45  0.2619047619047619  \\
            -1.44  0.2641509433962264  \\
            -1.43  0.2663551401869159  \\
            -1.42  0.26851851851851855  \\
            -1.41  0.27064220183486243  \\
            -1.4  0.27272727272727276  \\
            -1.39  0.2747747747747748  \\
            -1.38  0.2767857142857143  \\
            -1.37  0.27876106194690264  \\
            -1.36  0.2807017543859649  \\
            -1.35  0.2826086956521739  \\
            -1.34  0.2844827586206896  \\
            -1.33  0.2863247863247863  \\
            -1.32  0.288135593220339  \\
            -1.31  0.2899159663865546  \\
            -1.3  0.2916666666666667  \\
            -1.29  0.2933884297520661  \\
            -1.28  0.29508196721311475  \\
            -1.27  0.2967479674796748  \\
            -1.26  0.29838709677419356  \\
            -1.25  0.3  \\
            -1.24  0.30158730158730157  \\
            -1.23  0.3031496062992126  \\
            -1.22  0.3046875  \\
            -1.21  0.3062015503875969  \\
            -1.2  0.3076923076923077  \\
            -1.19  0.30916030534351147  \\
            -1.18  0.3106060606060606  \\
            -1.17  0.31203007518796994  \\
            -1.16  0.31343283582089554  \\
            -1.15  0.3148148148148148  \\
            -1.14  0.3161764705882353  \\
            -1.13  0.3175182481751825  \\
            -1.12  0.3188405797101449  \\
            -1.11  0.32014388489208634  \\
            -1.1  0.3214285714285714  \\
            -1.09  0.3226950354609929  \\
            -1.08  0.323943661971831  \\
            -1.07  0.32517482517482516  \\
            -1.06  0.3263888888888889  \\
            -1.05  0.3275862068965517  \\
            -1.04  0.3287671232876712  \\
            -1.03  0.3299319727891156  \\
            -1.02  0.3310810810810811  \\
            -1.01  0.33221476510067116  \\
            -1.0  0.3333333333333333  \\
            -0.99  0.3355481727574751  \\
            -0.98  0.33774834437086093  \\
            -0.97  0.33993399339933994  \\
            -0.96  0.34210526315789475  \\
            -0.95  0.34426229508196726  \\
            -0.94  0.3464052287581699  \\
            -0.93  0.3485342019543974  \\
            -0.92  0.35064935064935066  \\
            -0.91  0.35275080906148865  \\
            -0.9  0.3548387096774194  \\
            -0.89  0.35691318327974275  \\
            -0.88  0.358974358974359  \\
            -0.87  0.36102236421725237  \\
            -0.86  0.36305732484076436  \\
            -0.85  0.36507936507936506  \\
            -0.84  0.3670886075949367  \\
            -0.83  0.36908517350157727  \\
            -0.82  0.37106918238993714  \\
            -0.81  0.3730407523510972  \\
            -0.8  0.37499999999999994  \\
            -0.79  0.37694704049844235  \\
            -0.78  0.37888198757763975  \\
            -0.77  0.38080495356037153  \\
            -0.76  0.38271604938271603  \\
            -0.75  0.38461538461538464  \\
            -0.74  0.3865030674846626  \\
            -0.73  0.38837920489296635  \\
            -0.72  0.39024390243902435  \\
            -0.71  0.39209726443769  \\
            -0.7  0.393939393939394  \\
            -0.69  0.39577039274924475  \\
            -0.68  0.3975903614457831  \\
            -0.67  0.3993993993993994  \\
            -0.66  0.40119760479041916  \\
            -0.65  0.40298507462686567  \\
            -0.64  0.40476190476190477  \\
            -0.63  0.4065281899109792  \\
            -0.62  0.40828402366863903  \\
            -0.61  0.41002949852507375  \\
            -0.6  0.4117647058823529  \\
            -0.59  0.41348973607038125  \\
            -0.58  0.4152046783625731  \\
            -0.57  0.41690962099125367  \\
            -0.56  0.41860465116279066  \\
            -0.55  0.42028985507246375  \\
            -0.54  0.42196531791907516  \\
            -0.53  0.4236311239193084  \\
            -0.52  0.42528735632183906  \\
            -0.51  0.4269340974212034  \\
            -0.5  0.42857142857142855  \\
            -0.49  0.43019943019943024  \\
            -0.48  0.4318181818181818  \\
            -0.47  0.43342776203966005  \\
            -0.46  0.4350282485875706  \\
            -0.45  0.4366197183098592  \\
            -0.44  0.43820224719101125  \\
            -0.43  0.43977591036414565  \\
            -0.42  0.44134078212290506  \\
            -0.41  0.44289693593314766  \\
            -0.4  0.4444444444444445  \\
            -0.39  0.445983379501385  \\
            -0.38  0.44751381215469616  \\
            -0.37  0.4490358126721763  \\
            -0.36  0.45054945054945056  \\
            -0.35  0.4520547945205479  \\
            -0.34  0.45355191256830596  \\
            -0.33  0.4550408719346049  \\
            -0.32  0.4565217391304348  \\
            -0.31  0.45799457994579945  \\
            -0.3  0.45945945945945943  \\
            -0.29  0.4609164420485175  \\
            -0.28  0.4623655913978495  \\
            -0.27  0.46380697050938335  \\
            -0.26  0.4652406417112299  \\
            -0.25  0.4666666666666667  \\
            -0.24  0.46808510638297873  \\
            -0.23  0.46949602122015915  \\
            -0.22  0.4708994708994709  \\
            -0.21  0.47229551451187335  \\
            -0.2  0.4736842105263158  \\
            -0.19  0.47506561679790027  \\
            -0.18  0.47643979057591623  \\
            -0.17  0.47780678851174935  \\
            -0.16  0.4791666666666667  \\
            -0.15  0.4805194805194805  \\
            -0.14  0.48186528497409326  \\
            -0.13  0.48320413436692505  \\
            -0.12  0.4845360824742268  \\
            -0.11  0.48586118251928023  \\
            -0.1  0.48717948717948717  \\
            -0.09  0.48849104859335035  \\
            -0.08  0.4897959183673469  \\
            -0.07  0.4910941475826972  \\
            -0.06  0.49238578680203043  \\
            -0.05  0.49367088607594933  \\
            -0.04  0.494949494949495  \\
            -0.03  0.4962216624685139  \\
            -0.02  0.49748743718592964  \\
            -0.01  0.49874686716791977  \\
            0.0  0.5  \\
            0.01  0.49874686716791977  \\
            0.02  0.49748743718592964  \\
            0.03  0.4962216624685139  \\
            0.04  0.494949494949495  \\
            0.05  0.49367088607594933  \\
            0.06  0.49238578680203043  \\
            0.07  0.4910941475826972  \\
            0.08  0.4897959183673469  \\
            0.09  0.48849104859335035  \\
            0.1  0.48717948717948717  \\
            0.11  0.48586118251928023  \\
            0.12  0.4845360824742268  \\
            0.13  0.48320413436692505  \\
            0.14  0.48186528497409326  \\
            0.15  0.4805194805194805  \\
            0.16  0.4791666666666667  \\
            0.17  0.47780678851174935  \\
            0.18  0.47643979057591623  \\
            0.19  0.47506561679790027  \\
            0.2  0.4736842105263158  \\
            0.21  0.47229551451187335  \\
            0.22  0.4708994708994709  \\
            0.23  0.46949602122015915  \\
            0.24  0.46808510638297873  \\
            0.25  0.4666666666666667  \\
            0.26  0.4652406417112299  \\
            0.27  0.46380697050938335  \\
            0.28  0.4623655913978495  \\
            0.29  0.4609164420485175  \\
            0.3  0.45945945945945943  \\
            0.31  0.45799457994579945  \\
            0.32  0.4565217391304348  \\
            0.33  0.4550408719346049  \\
            0.34  0.45355191256830596  \\
            0.35  0.4520547945205479  \\
            0.36  0.45054945054945056  \\
            0.37  0.4490358126721763  \\
            0.38  0.44751381215469616  \\
            0.39  0.445983379501385  \\
            0.4  0.4444444444444445  \\
            0.41  0.44289693593314766  \\
            0.42  0.44134078212290506  \\
            0.43  0.43977591036414565  \\
            0.44  0.43820224719101125  \\
            0.45  0.4366197183098592  \\
            0.46  0.4350282485875706  \\
            0.47  0.43342776203966005  \\
            0.48  0.4318181818181818  \\
            0.49  0.43019943019943024  \\
            0.5  0.42857142857142855  \\
            0.51  0.4269340974212034  \\
            0.52  0.42528735632183906  \\
            0.53  0.4236311239193084  \\
            0.54  0.42196531791907516  \\
            0.55  0.42028985507246375  \\
            0.56  0.41860465116279066  \\
            0.57  0.41690962099125367  \\
            0.58  0.4152046783625731  \\
            0.59  0.41348973607038125  \\
            0.6  0.4117647058823529  \\
            0.61  0.41002949852507375  \\
            0.62  0.40828402366863903  \\
            0.63  0.4065281899109792  \\
            0.64  0.40476190476190477  \\
            0.65  0.40298507462686567  \\
            0.66  0.40119760479041916  \\
            0.67  0.3993993993993994  \\
            0.68  0.3975903614457831  \\
            0.69  0.39577039274924475  \\
            0.7  0.393939393939394  \\
            0.71  0.39209726443769  \\
            0.72  0.39024390243902435  \\
            0.73  0.38837920489296635  \\
            0.74  0.3865030674846626  \\
            0.75  0.38461538461538464  \\
            0.76  0.38271604938271603  \\
            0.77  0.38080495356037153  \\
            0.78  0.37888198757763975  \\
            0.79  0.37694704049844235  \\
            0.8  0.37499999999999994  \\
            0.81  0.3730407523510972  \\
            0.82  0.37106918238993714  \\
            0.83  0.36908517350157727  \\
            0.84  0.3670886075949367  \\
            0.85  0.36507936507936506  \\
            0.86  0.36305732484076436  \\
            0.87  0.36102236421725237  \\
            0.88  0.358974358974359  \\
            0.89  0.35691318327974275  \\
            0.9  0.3548387096774194  \\
            0.91  0.35275080906148865  \\
            0.92  0.35064935064935066  \\
            0.93  0.3485342019543974  \\
            0.94  0.3464052287581699  \\
            0.95  0.34426229508196726  \\
            0.96  0.34210526315789475  \\
            0.97  0.33993399339933994  \\
            0.98  0.33774834437086093  \\
            0.99  0.3355481727574751  \\
            1.0  0.3333333333333333  \\
            1.01  0.33221476510067116  \\
            1.02  0.3310810810810811  \\
            1.03  0.3299319727891156  \\
            1.04  0.3287671232876712  \\
            1.05  0.3275862068965517  \\
            1.06  0.3263888888888889  \\
            1.07  0.32517482517482516  \\
            1.08  0.323943661971831  \\
            1.09  0.3226950354609929  \\
            1.1  0.3214285714285714  \\
            1.11  0.32014388489208634  \\
            1.12  0.3188405797101449  \\
            1.13  0.3175182481751825  \\
            1.14  0.3161764705882353  \\
            1.15  0.3148148148148148  \\
            1.16  0.31343283582089554  \\
            1.17  0.31203007518796994  \\
            1.18  0.3106060606060606  \\
            1.19  0.30916030534351147  \\
            1.2  0.3076923076923077  \\
            1.21  0.3062015503875969  \\
            1.22  0.3046875  \\
            1.23  0.3031496062992126  \\
            1.24  0.30158730158730157  \\
            1.25  0.3  \\
            1.26  0.29838709677419356  \\
            1.27  0.2967479674796748  \\
            1.28  0.29508196721311475  \\
            1.29  0.2933884297520661  \\
            1.3  0.2916666666666667  \\
            1.31  0.2899159663865546  \\
            1.32  0.288135593220339  \\
            1.33  0.2863247863247863  \\
            1.34  0.2844827586206896  \\
            1.35  0.2826086956521739  \\
            1.36  0.2807017543859649  \\
            1.37  0.27876106194690264  \\
            1.38  0.2767857142857143  \\
            1.39  0.2747747747747748  \\
            1.4  0.27272727272727276  \\
            1.41  0.27064220183486243  \\
            1.42  0.26851851851851855  \\
            1.43  0.2663551401869159  \\
            1.44  0.2641509433962264  \\
            1.45  0.2619047619047619  \\
            1.46  0.25961538461538464  \\
            1.47  0.25728155339805825  \\
            1.48  0.2549019607843137  \\
            1.49  0.2524752475247525  \\
            1.5  0.25  \\
            1.51  0.2474747474747475  \\
            1.52  0.24489795918367346  \\
            1.53  0.2422680412371134  \\
            1.54  0.23958333333333331  \\
            1.55  0.23684210526315788  \\
            1.56  0.23404255319148934  \\
            1.57  0.23118279569892472  \\
            1.58  0.22826086956521738  \\
            1.59  0.22527472527472525  \\
            1.6  0.22222222222222218  \\
            1.61  0.2191011235955056  \\
            1.62  0.21590909090909088  \\
            1.63  0.21264367816091959  \\
            1.64  0.2093023255813954  \\
            1.65  0.2058823529411765  \\
            1.66  0.2023809523809524  \\
            1.67  0.1987951807228916  \\
            1.68  0.19512195121951223  \\
            1.69  0.19135802469135804  \\
            1.7  0.18750000000000003  \\
            1.71  0.18354430379746836  \\
            1.72  0.1794871794871795  \\
            1.73  0.17532467532467533  \\
            1.74  0.17105263157894737  \\
            1.75  0.16666666666666666  \\
            1.76  0.16216216216216217  \\
            1.77  0.15753424657534246  \\
            1.78  0.15277777777777776  \\
            1.79  0.14788732394366194  \\
            1.8  0.14285714285714282  \\
            1.81  0.13768115942028983  \\
            1.82  0.13235294117647056  \\
            1.83  0.126865671641791  \\
            1.84  0.12121212121212116  \\
            1.85  0.11538461538461534  \\
            1.86  0.10937499999999994  \\
            1.87  0.10317460317460311  \\
            1.88  0.09677419354838716  \\
            1.89  0.09016393442622958  \\
            1.9  0.0833333333333334  \\
            1.91  0.07627118644067803  \\
            1.92  0.06896551724137936  \\
            1.93  0.06140350877192987  \\
            1.94  0.05357142857142862  \\
            1.95  0.04545454545454549  \\
            1.96  0.03703703703703707  \\
            1.97  0.028301886792452855  \\
            1.98  0.019230769230769246  \\
            1.99  0.00980392156862746  \\
            2.0  0.0  \\
            2.01  0.0  \\
            2.02  0.0  \\
            2.03  0.0  \\
            2.04  0.0  \\
            2.05  0.0  \\
            2.06  0.0  \\
            2.07  0.0  \\
            2.08  0.0  \\
            2.09  0.0  \\
            2.1  0.0  \\
            2.11  0.0  \\
            2.12  0.0  \\
            2.13  0.0  \\
            2.14  0.0  \\
            2.15  0.0  \\
            2.16  0.0  \\
            2.17  0.0  \\
            2.18  0.0  \\
            2.19  0.0  \\
            2.2  0.0  \\
            2.21  0.0  \\
            2.22  0.0  \\
            2.23  0.0  \\
            2.24  0.0  \\
            2.25  0.0  \\
            2.26  0.0  \\
            2.27  0.0  \\
            2.28  0.0  \\
            2.29  0.0  \\
            2.3  0.0  \\
            2.31  0.0  \\
            2.32  0.0  \\
            2.33  0.0  \\
            2.34  0.0  \\
            2.35  0.0  \\
            2.36  0.0  \\
            2.37  0.0  \\
            2.38  0.0  \\
            2.39  0.0  \\
            2.4  0.0  \\
            2.41  0.0  \\
            2.42  0.0  \\
            2.43  0.0  \\
            2.44  0.0  \\
            2.45  0.0  \\
            2.46  0.0  \\
            2.47  0.0  \\
            2.48  0.0  \\
            2.49  0.0  \\
            2.5  0.0  \\
            2.51  0.0  \\
            2.52  0.0  \\
            2.53  0.0  \\
            2.54  0.0  \\
            2.55  0.0  \\
            2.56  0.0  \\
            2.57  0.0  \\
            2.58  0.0  \\
            2.59  0.0  \\
            2.6  0.0  \\
            2.61  0.0  \\
            2.62  0.0  \\
            2.63  0.0  \\
            2.64  0.0  \\
            2.65  0.0  \\
            2.66  0.0  \\
            2.67  0.0  \\
            2.68  0.0  \\
            2.69  0.0  \\
            2.7  0.0  \\
            2.71  0.0  \\
            2.72  0.0  \\
            2.73  0.0  \\
            2.74  0.0  \\
            2.75  0.0  \\
            2.76  0.0  \\
            2.77  0.0  \\
            2.78  0.0  \\
            2.79  0.0  \\
            2.8  0.0  \\
            2.81  0.0  \\
            2.82  0.0  \\
            2.83  0.0  \\
            2.84  0.0  \\
            2.85  0.0  \\
            2.86  0.0  \\
            2.87  0.0  \\
            2.88  0.0  \\
            2.89  0.0  \\
            2.9  0.0  \\
            2.91  0.0  \\
            2.92  0.0  \\
            2.93  0.0  \\
            2.94  0.0  \\
            2.95  0.0  \\
            2.96  0.0  \\
            2.97  0.0  \\
            2.98  0.0  \\
            2.99  0.0  \\
            3.0  nan  \\
        }
        \closedcycle
        ;
    \addplot+[fill, fill opacity={0.2}, thick]
        table[row sep={\\}]
        {
            \\
            -3.0  nan  \\
            -2.99  0.5  \\
            -2.98  0.5  \\
            -2.97  0.5  \\
            -2.96  0.5  \\
            -2.95  0.5  \\
            -2.94  0.5  \\
            -2.93  0.5  \\
            -2.92  0.5  \\
            -2.91  0.5  \\
            -2.9  0.5  \\
            -2.89  0.5  \\
            -2.88  0.5  \\
            -2.87  0.5  \\
            -2.86  0.5  \\
            -2.85  0.5  \\
            -2.84  0.5  \\
            -2.83  0.5  \\
            -2.82  0.5  \\
            -2.81  0.5  \\
            -2.8  0.5  \\
            -2.79  0.5  \\
            -2.78  0.5  \\
            -2.77  0.5  \\
            -2.76  0.5  \\
            -2.75  0.5  \\
            -2.74  0.5  \\
            -2.73  0.5  \\
            -2.72  0.5  \\
            -2.71  0.5  \\
            -2.7  0.5  \\
            -2.69  0.5  \\
            -2.68  0.5  \\
            -2.67  0.5  \\
            -2.66  0.5  \\
            -2.65  0.5  \\
            -2.64  0.5  \\
            -2.63  0.5  \\
            -2.62  0.5  \\
            -2.61  0.5  \\
            -2.6  0.5  \\
            -2.59  0.5  \\
            -2.58  0.5  \\
            -2.57  0.5  \\
            -2.56  0.5  \\
            -2.55  0.5  \\
            -2.54  0.5  \\
            -2.53  0.5  \\
            -2.52  0.5  \\
            -2.51  0.5  \\
            -2.5  0.5  \\
            -2.49  0.5  \\
            -2.48  0.5  \\
            -2.47  0.5  \\
            -2.46  0.5  \\
            -2.45  0.5  \\
            -2.44  0.5  \\
            -2.43  0.5  \\
            -2.42  0.5  \\
            -2.41  0.5  \\
            -2.4  0.5  \\
            -2.39  0.5  \\
            -2.38  0.5  \\
            -2.37  0.5  \\
            -2.36  0.5  \\
            -2.35  0.5  \\
            -2.34  0.5  \\
            -2.33  0.5  \\
            -2.32  0.5  \\
            -2.31  0.5  \\
            -2.3  0.5  \\
            -2.29  0.5  \\
            -2.28  0.5  \\
            -2.27  0.5  \\
            -2.26  0.5  \\
            -2.25  0.5  \\
            -2.24  0.5  \\
            -2.23  0.5  \\
            -2.22  0.5  \\
            -2.21  0.5  \\
            -2.2  0.5  \\
            -2.19  0.5  \\
            -2.18  0.5  \\
            -2.17  0.5  \\
            -2.16  0.5  \\
            -2.15  0.5  \\
            -2.14  0.5  \\
            -2.13  0.5  \\
            -2.12  0.5  \\
            -2.11  0.5  \\
            -2.1  0.5  \\
            -2.09  0.5  \\
            -2.08  0.5  \\
            -2.07  0.5  \\
            -2.06  0.5  \\
            -2.05  0.5  \\
            -2.04  0.5  \\
            -2.03  0.5  \\
            -2.02  0.5  \\
            -2.01  0.5  \\
            -2.0  0.5  \\
            -1.99  0.4950980392156863  \\
            -1.98  0.49038461538461536  \\
            -1.97  0.4858490566037736  \\
            -1.96  0.48148148148148145  \\
            -1.95  0.47727272727272724  \\
            -1.94  0.4732142857142857  \\
            -1.93  0.46929824561403505  \\
            -1.92  0.46551724137931033  \\
            -1.91  0.461864406779661  \\
            -1.9  0.4583333333333333  \\
            -1.89  0.4549180327868852  \\
            -1.88  0.45161290322580644  \\
            -1.87  0.44841269841269843  \\
            -1.86  0.44531250000000006  \\
            -1.85  0.44230769230769235  \\
            -1.84  0.4393939393939394  \\
            -1.83  0.4365671641791045  \\
            -1.82  0.4338235294117647  \\
            -1.81  0.4311594202898551  \\
            -1.8  0.4285714285714286  \\
            -1.79  0.426056338028169  \\
            -1.78  0.4236111111111111  \\
            -1.77  0.4212328767123288  \\
            -1.76  0.4189189189189189  \\
            -1.75  0.4166666666666667  \\
            -1.74  0.4144736842105263  \\
            -1.73  0.41233766233766234  \\
            -1.72  0.41025641025641024  \\
            -1.71  0.40822784810126583  \\
            -1.7  0.40625  \\
            -1.69  0.40432098765432095  \\
            -1.68  0.40243902439024387  \\
            -1.67  0.4006024096385542  \\
            -1.66  0.3988095238095238  \\
            -1.65  0.39705882352941174  \\
            -1.64  0.3953488372093023  \\
            -1.63  0.3936781609195402  \\
            -1.62  0.3920454545454546  \\
            -1.61  0.3904494382022472  \\
            -1.6  0.3888888888888889  \\
            -1.59  0.3873626373626374  \\
            -1.58  0.3858695652173913  \\
            -1.57  0.3844086021505376  \\
            -1.56  0.3829787234042553  \\
            -1.55  0.3815789473684211  \\
            -1.54  0.3802083333333333  \\
            -1.53  0.3788659793814433  \\
            -1.52  0.37755102040816324  \\
            -1.51  0.37626262626262624  \\
            -1.5  0.375  \\
            -1.49  0.37376237623762376  \\
            -1.48  0.37254901960784315  \\
            -1.47  0.3713592233009709  \\
            -1.46  0.3701923076923077  \\
            -1.45  0.36904761904761907  \\
            -1.44  0.36792452830188677  \\
            -1.43  0.36682242990654207  \\
            -1.42  0.36574074074074076  \\
            -1.41  0.3646788990825688  \\
            -1.4  0.36363636363636365  \\
            -1.39  0.36261261261261263  \\
            -1.38  0.36160714285714285  \\
            -1.37  0.3606194690265487  \\
            -1.36  0.35964912280701755  \\
            -1.35  0.358695652173913  \\
            -1.34  0.3577586206896552  \\
            -1.33  0.35683760683760685  \\
            -1.32  0.3559322033898305  \\
            -1.31  0.3550420168067227  \\
            -1.3  0.3541666666666667  \\
            -1.29  0.35330578512396693  \\
            -1.28  0.3524590163934426  \\
            -1.27  0.3516260162601626  \\
            -1.26  0.35080645161290325  \\
            -1.25  0.35  \\
            -1.24  0.3492063492063492  \\
            -1.23  0.3484251968503937  \\
            -1.22  0.34765625  \\
            -1.21  0.34689922480620156  \\
            -1.2  0.34615384615384615  \\
            -1.19  0.34541984732824427  \\
            -1.18  0.34469696969696967  \\
            -1.17  0.34398496240601506  \\
            -1.16  0.34328358208955223  \\
            -1.15  0.34259259259259256  \\
            -1.14  0.34191176470588236  \\
            -1.13  0.34124087591240876  \\
            -1.12  0.34057971014492755  \\
            -1.11  0.33992805755395683  \\
            -1.1  0.3392857142857143  \\
            -1.09  0.33865248226950356  \\
            -1.08  0.3380281690140845  \\
            -1.07  0.3374125874125874  \\
            -1.06  0.3368055555555556  \\
            -1.05  0.33620689655172414  \\
            -1.04  0.3356164383561644  \\
            -1.03  0.33503401360544216  \\
            -1.02  0.3344594594594595  \\
            -1.01  0.3338926174496644  \\
            -1.0  0.3333333333333333  \\
            -0.99  0.33222591362126247  \\
            -0.98  0.33112582781456956  \\
            -0.97  0.33003300330033003  \\
            -0.96  0.32894736842105265  \\
            -0.95  0.3278688524590164  \\
            -0.94  0.32679738562091504  \\
            -0.93  0.32573289902280134  \\
            -0.92  0.3246753246753247  \\
            -0.91  0.3236245954692557  \\
            -0.9  0.3225806451612903  \\
            -0.89  0.3215434083601286  \\
            -0.88  0.3205128205128205  \\
            -0.87  0.3194888178913738  \\
            -0.86  0.3184713375796178  \\
            -0.85  0.31746031746031744  \\
            -0.84  0.3164556962025316  \\
            -0.83  0.31545741324921134  \\
            -0.82  0.31446540880503143  \\
            -0.81  0.31347962382445144  \\
            -0.8  0.3125  \\
            -0.79  0.3115264797507788  \\
            -0.78  0.31055900621118016  \\
            -0.77  0.30959752321981426  \\
            -0.76  0.30864197530864196  \\
            -0.75  0.3076923076923077  \\
            -0.74  0.3067484662576687  \\
            -0.73  0.3058103975535168  \\
            -0.72  0.3048780487804878  \\
            -0.71  0.303951367781155  \\
            -0.7  0.30303030303030304  \\
            -0.69  0.3021148036253776  \\
            -0.68  0.30120481927710846  \\
            -0.67  0.3003003003003003  \\
            -0.66  0.29940119760479045  \\
            -0.65  0.29850746268656714  \\
            -0.64  0.2976190476190476  \\
            -0.63  0.29673590504451036  \\
            -0.62  0.2958579881656805  \\
            -0.61  0.2949852507374631  \\
            -0.6  0.29411764705882354  \\
            -0.59  0.29325513196480935  \\
            -0.58  0.29239766081871343  \\
            -0.57  0.29154518950437314  \\
            -0.56  0.29069767441860467  \\
            -0.55  0.2898550724637681  \\
            -0.54  0.28901734104046245  \\
            -0.53  0.2881844380403458  \\
            -0.52  0.28735632183908044  \\
            -0.51  0.28653295128939826  \\
            -0.5  0.2857142857142857  \\
            -0.49  0.2849002849002849  \\
            -0.48  0.2840909090909091  \\
            -0.47  0.28328611898017  \\
            -0.46  0.2824858757062147  \\
            -0.45  0.28169014084507044  \\
            -0.44  0.2808988764044944  \\
            -0.43  0.2801120448179272  \\
            -0.42  0.27932960893854747  \\
            -0.41  0.2785515320334262  \\
            -0.4  0.2777777777777778  \\
            -0.39  0.2770083102493075  \\
            -0.38  0.27624309392265195  \\
            -0.37  0.27548209366391185  \\
            -0.36  0.2747252747252747  \\
            -0.35  0.273972602739726  \\
            -0.34  0.27322404371584696  \\
            -0.33  0.2724795640326976  \\
            -0.32  0.27173913043478265  \\
            -0.31  0.2710027100271003  \\
            -0.3  0.27027027027027023  \\
            -0.29  0.2695417789757412  \\
            -0.28  0.2688172043010753  \\
            -0.27  0.2680965147453083  \\
            -0.26  0.267379679144385  \\
            -0.25  0.26666666666666666  \\
            -0.24  0.26595744680851063  \\
            -0.23  0.26525198938992045  \\
            -0.22  0.26455026455026454  \\
            -0.21  0.2638522427440633  \\
            -0.2  0.2631578947368421  \\
            -0.19  0.26246719160104987  \\
            -0.18  0.2617801047120419  \\
            -0.17  0.2610966057441253  \\
            -0.16  0.2604166666666667  \\
            -0.15  0.2597402597402597  \\
            -0.14  0.2590673575129534  \\
            -0.13  0.25839793281653745  \\
            -0.12  0.2577319587628866  \\
            -0.11  0.25706940874035994  \\
            -0.1  0.25641025641025644  \\
            -0.09  0.2557544757033248  \\
            -0.08  0.25510204081632654  \\
            -0.07  0.2544529262086514  \\
            -0.06  0.25380710659898476  \\
            -0.05  0.2531645569620253  \\
            -0.04  0.25252525252525254  \\
            -0.03  0.2518891687657431  \\
            -0.02  0.25125628140703515  \\
            -0.01  0.2506265664160401  \\
            0.0  0.25  \\
            0.01  0.2506265664160401  \\
            0.02  0.25125628140703515  \\
            0.03  0.2518891687657431  \\
            0.04  0.25252525252525254  \\
            0.05  0.2531645569620253  \\
            0.06  0.25380710659898476  \\
            0.07  0.2544529262086514  \\
            0.08  0.25510204081632654  \\
            0.09  0.2557544757033248  \\
            0.1  0.25641025641025644  \\
            0.11  0.25706940874035994  \\
            0.12  0.2577319587628866  \\
            0.13  0.25839793281653745  \\
            0.14  0.2590673575129534  \\
            0.15  0.2597402597402597  \\
            0.16  0.2604166666666667  \\
            0.17  0.2610966057441253  \\
            0.18  0.2617801047120419  \\
            0.19  0.26246719160104987  \\
            0.2  0.2631578947368421  \\
            0.21  0.2638522427440633  \\
            0.22  0.26455026455026454  \\
            0.23  0.26525198938992045  \\
            0.24  0.26595744680851063  \\
            0.25  0.26666666666666666  \\
            0.26  0.267379679144385  \\
            0.27  0.2680965147453083  \\
            0.28  0.2688172043010753  \\
            0.29  0.2695417789757412  \\
            0.3  0.27027027027027023  \\
            0.31  0.2710027100271003  \\
            0.32  0.27173913043478265  \\
            0.33  0.2724795640326976  \\
            0.34  0.27322404371584696  \\
            0.35  0.273972602739726  \\
            0.36  0.2747252747252747  \\
            0.37  0.27548209366391185  \\
            0.38  0.27624309392265195  \\
            0.39  0.2770083102493075  \\
            0.4  0.2777777777777778  \\
            0.41  0.2785515320334262  \\
            0.42  0.27932960893854747  \\
            0.43  0.2801120448179272  \\
            0.44  0.2808988764044944  \\
            0.45  0.28169014084507044  \\
            0.46  0.2824858757062147  \\
            0.47  0.28328611898017  \\
            0.48  0.2840909090909091  \\
            0.49  0.2849002849002849  \\
            0.5  0.2857142857142857  \\
            0.51  0.28653295128939826  \\
            0.52  0.28735632183908044  \\
            0.53  0.2881844380403458  \\
            0.54  0.28901734104046245  \\
            0.55  0.2898550724637681  \\
            0.56  0.29069767441860467  \\
            0.57  0.29154518950437314  \\
            0.58  0.29239766081871343  \\
            0.59  0.29325513196480935  \\
            0.6  0.29411764705882354  \\
            0.61  0.2949852507374631  \\
            0.62  0.2958579881656805  \\
            0.63  0.29673590504451036  \\
            0.64  0.2976190476190476  \\
            0.65  0.29850746268656714  \\
            0.66  0.29940119760479045  \\
            0.67  0.3003003003003003  \\
            0.68  0.30120481927710846  \\
            0.69  0.3021148036253776  \\
            0.7  0.30303030303030304  \\
            0.71  0.303951367781155  \\
            0.72  0.3048780487804878  \\
            0.73  0.3058103975535168  \\
            0.74  0.3067484662576687  \\
            0.75  0.3076923076923077  \\
            0.76  0.30864197530864196  \\
            0.77  0.30959752321981426  \\
            0.78  0.31055900621118016  \\
            0.79  0.3115264797507788  \\
            0.8  0.3125  \\
            0.81  0.31347962382445144  \\
            0.82  0.31446540880503143  \\
            0.83  0.31545741324921134  \\
            0.84  0.3164556962025316  \\
            0.85  0.31746031746031744  \\
            0.86  0.3184713375796178  \\
            0.87  0.3194888178913738  \\
            0.88  0.3205128205128205  \\
            0.89  0.3215434083601286  \\
            0.9  0.3225806451612903  \\
            0.91  0.3236245954692557  \\
            0.92  0.3246753246753247  \\
            0.93  0.32573289902280134  \\
            0.94  0.32679738562091504  \\
            0.95  0.3278688524590164  \\
            0.96  0.32894736842105265  \\
            0.97  0.33003300330033003  \\
            0.98  0.33112582781456956  \\
            0.99  0.33222591362126247  \\
            1.0  0.3333333333333333  \\
            1.01  0.3338926174496644  \\
            1.02  0.3344594594594595  \\
            1.03  0.33503401360544216  \\
            1.04  0.3356164383561644  \\
            1.05  0.33620689655172414  \\
            1.06  0.3368055555555556  \\
            1.07  0.3374125874125874  \\
            1.08  0.3380281690140845  \\
            1.09  0.33865248226950356  \\
            1.1  0.3392857142857143  \\
            1.11  0.33992805755395683  \\
            1.12  0.34057971014492755  \\
            1.13  0.34124087591240876  \\
            1.14  0.34191176470588236  \\
            1.15  0.34259259259259256  \\
            1.16  0.34328358208955223  \\
            1.17  0.34398496240601506  \\
            1.18  0.34469696969696967  \\
            1.19  0.34541984732824427  \\
            1.2  0.34615384615384615  \\
            1.21  0.34689922480620156  \\
            1.22  0.34765625  \\
            1.23  0.3484251968503937  \\
            1.24  0.3492063492063492  \\
            1.25  0.35  \\
            1.26  0.35080645161290325  \\
            1.27  0.3516260162601626  \\
            1.28  0.3524590163934426  \\
            1.29  0.35330578512396693  \\
            1.3  0.3541666666666667  \\
            1.31  0.3550420168067227  \\
            1.32  0.3559322033898305  \\
            1.33  0.35683760683760685  \\
            1.34  0.3577586206896552  \\
            1.35  0.358695652173913  \\
            1.36  0.35964912280701755  \\
            1.37  0.3606194690265487  \\
            1.38  0.36160714285714285  \\
            1.39  0.36261261261261263  \\
            1.4  0.36363636363636365  \\
            1.41  0.3646788990825688  \\
            1.42  0.36574074074074076  \\
            1.43  0.36682242990654207  \\
            1.44  0.36792452830188677  \\
            1.45  0.36904761904761907  \\
            1.46  0.3701923076923077  \\
            1.47  0.3713592233009709  \\
            1.48  0.37254901960784315  \\
            1.49  0.37376237623762376  \\
            1.5  0.375  \\
            1.51  0.37626262626262624  \\
            1.52  0.37755102040816324  \\
            1.53  0.3788659793814433  \\
            1.54  0.3802083333333333  \\
            1.55  0.3815789473684211  \\
            1.56  0.3829787234042553  \\
            1.57  0.3844086021505376  \\
            1.58  0.3858695652173913  \\
            1.59  0.3873626373626374  \\
            1.6  0.3888888888888889  \\
            1.61  0.3904494382022472  \\
            1.62  0.3920454545454546  \\
            1.63  0.3936781609195402  \\
            1.64  0.3953488372093023  \\
            1.65  0.39705882352941174  \\
            1.66  0.3988095238095238  \\
            1.67  0.4006024096385542  \\
            1.68  0.40243902439024387  \\
            1.69  0.40432098765432095  \\
            1.7  0.40625  \\
            1.71  0.40822784810126583  \\
            1.72  0.41025641025641024  \\
            1.73  0.41233766233766234  \\
            1.74  0.4144736842105263  \\
            1.75  0.4166666666666667  \\
            1.76  0.4189189189189189  \\
            1.77  0.4212328767123288  \\
            1.78  0.4236111111111111  \\
            1.79  0.426056338028169  \\
            1.8  0.4285714285714286  \\
            1.81  0.4311594202898551  \\
            1.82  0.4338235294117647  \\
            1.83  0.4365671641791045  \\
            1.84  0.4393939393939394  \\
            1.85  0.44230769230769235  \\
            1.86  0.44531250000000006  \\
            1.87  0.44841269841269843  \\
            1.88  0.45161290322580644  \\
            1.89  0.4549180327868852  \\
            1.9  0.4583333333333333  \\
            1.91  0.461864406779661  \\
            1.92  0.46551724137931033  \\
            1.93  0.46929824561403505  \\
            1.94  0.4732142857142857  \\
            1.95  0.47727272727272724  \\
            1.96  0.48148148148148145  \\
            1.97  0.4858490566037736  \\
            1.98  0.49038461538461536  \\
            1.99  0.4950980392156863  \\
            2.0  0.5  \\
            2.01  0.5  \\
            2.02  0.5  \\
            2.03  0.5  \\
            2.04  0.5  \\
            2.05  0.5  \\
            2.06  0.5  \\
            2.07  0.5  \\
            2.08  0.5  \\
            2.09  0.5  \\
            2.1  0.5  \\
            2.11  0.5  \\
            2.12  0.5  \\
            2.13  0.5  \\
            2.14  0.5  \\
            2.15  0.5  \\
            2.16  0.5  \\
            2.17  0.5  \\
            2.18  0.5  \\
            2.19  0.5  \\
            2.2  0.5  \\
            2.21  0.5  \\
            2.22  0.5  \\
            2.23  0.5  \\
            2.24  0.5  \\
            2.25  0.5  \\
            2.26  0.5  \\
            2.27  0.5  \\
            2.28  0.5  \\
            2.29  0.5  \\
            2.3  0.5  \\
            2.31  0.5  \\
            2.32  0.5  \\
            2.33  0.5  \\
            2.34  0.5  \\
            2.35  0.5  \\
            2.36  0.5  \\
            2.37  0.5  \\
            2.38  0.5  \\
            2.39  0.5  \\
            2.4  0.5  \\
            2.41  0.5  \\
            2.42  0.5  \\
            2.43  0.5  \\
            2.44  0.5  \\
            2.45  0.5  \\
            2.46  0.5  \\
            2.47  0.5  \\
            2.48  0.5  \\
            2.49  0.5  \\
            2.5  0.5  \\
            2.51  0.5  \\
            2.52  0.5  \\
            2.53  0.5  \\
            2.54  0.5  \\
            2.55  0.5  \\
            2.56  0.5  \\
            2.57  0.5  \\
            2.58  0.5  \\
            2.59  0.5  \\
            2.6  0.5  \\
            2.61  0.5  \\
            2.62  0.5  \\
            2.63  0.5  \\
            2.64  0.5  \\
            2.65  0.5  \\
            2.66  0.5  \\
            2.67  0.5  \\
            2.68  0.5  \\
            2.69  0.5  \\
            2.7  0.5  \\
            2.71  0.5  \\
            2.72  0.5  \\
            2.73  0.5  \\
            2.74  0.5  \\
            2.75  0.5  \\
            2.76  0.5  \\
            2.77  0.5  \\
            2.78  0.5  \\
            2.79  0.5  \\
            2.8  0.5  \\
            2.81  0.5  \\
            2.82  0.5  \\
            2.83  0.5  \\
            2.84  0.5  \\
            2.85  0.5  \\
            2.86  0.5  \\
            2.87  0.5  \\
            2.88  0.5  \\
            2.89  0.5  \\
            2.9  0.5  \\
            2.91  0.5  \\
            2.92  0.5  \\
            2.93  0.5  \\
            2.94  0.5  \\
            2.95  0.5  \\
            2.96  0.5  \\
            2.97  0.5  \\
            2.98  0.5  \\
            2.99  0.5  \\
            3.0  nan  \\
        }
        \closedcycle
        ;
    \nextgroupplot[title={$g_y(X) = \mathbb{P}[Y = y]$}]
    \addplot+[fill, fill opacity={0.2}, thick]
        coordinates {
            (-3.0,0.3333333333333333)
            (3.0,0.3333333333333333)
        }
        \closedcycle
        ;
    \addplot+[fill, fill opacity={0.2}, thick]
        coordinates {
            (-3.0,0.3333333333333333)
            (3.0,0.3333333333333333)
        }
        \closedcycle
        ;
    \addplot+[fill, fill opacity={0.2}, thick]
        coordinates {
            (-3.0,0.3333333333333333)
            (3.0,0.3333333333333333)
        }
        \closedcycle
        ;
\end{groupplot}
\end{tikzpicture}

        \end{center}
      \end{tcolorbox}
    \end{itemize}
  }

  \posterbox[adjusted title={Calibration error}]{name=error,column=1,below=calibration}{
    We propose the following general measure of miscalibration which arises naturally from the definition of calibration in \cref{eq:calibration}.

    \begin{tcolorbox}[colback=sandstark]
      The \hl{calibration error}~($\measure$) of model $g$ w.r.t.\ a class $\mathcal{F}$ of functions $f \colon \Delta^m \to \mathbb{R}^m$ is
      \begin{equation*}
        \measure[\mathcal{F}, g]  \coloneqq \sup_{f \in \mathcal{F}} \Expect\left[{(r(g(X)) - g(X))}^\intercal f(g(X)) \right].
      \end{equation*}
    \end{tcolorbox}

    \vspace{-\topsep}
    \begin{itemize}
    \item By design, if model $g$ is calibrated then the $\measure$ is zero, regardless of $\mathcal{F}$.
    \item The $\measure$ is equal to the common expected calibration error ($\ECE$)
      \begin{equation}\label{eq:ece}
        \ECE[d, g] = \Expect[d(r(g(X)), g(X))]
      \end{equation}
      for certain choices of $\mathcal{F}$ and distances $d$ such as the city block distance, the total variation distance, and the squared Euclidean distance.
    \item The $\measure$ captures also the maximum mean calibration error \parencite{kumar18_train_calib_measur_neural_networ}.
    \end{itemize}
  }

  \posterbox[adjusted title={Estimators of the calibration error}]{name=estimator,column=2,below=title}{
    Consider the task of estimating the $\measure$ of model $g$ using a validation set $\{(X_i, Y_i)\}_{i=1}^n$ of i.i.d.\ random pairs of inputs and labels that are distributed according to $(X,Y)$.

    \begin{itemize}
    \item Standard estimators of the $\ECE$ are inconsistent and biased in many cases \parencite{vaicenavicius19_evaluat} and can scale poorly to large $m$.

      \begin{tcolorbox}[colback=blondsvag]
        The main difficulty is the estimation of the function $r$ in \cref{eq:ece}. \Cref{eq:kce} shows that for $\kernelmeasure$ there is no explicit dependence on $r$!
      \end{tcolorbox}

    \item For $i,j \in \{1,\ldots,n\}$ define
      \begin{equation*}
        h_{i,j} \coloneqq {(e_{Y_i} - g(X_i))}^\intercal k(g(X_i), g(X_j)) (e_{Y_j} - g(X_j)).
      \end{equation*}

      \begin{tcolorbox}[colback=sandstark]
        If $\mathbb{E}[\|k(g(X),g(X))\|] < \infty$, then the following estimators are \hl{consistent estimators of the squared kernel calibration error} $\squaredkernelmeasure[k, g] \coloneqq \kernelmeasure[k,g]^2$.
        \begin{center}
          \begin{tabular}{llll} \toprule 
            Notation & Definition & Properties & Complexity\\ \midrule
            $\biasedskce$ & $n^{-2} \sum_{i,j=1}^n h_{i,j}$ & biased & $O(n^2)$ \\
            $\unbiasedskce$ & $ {\binom{n}{2}}^{-1} \sum_{1 \leq i < j \leq n} h_{i,j}$ & unbiased & $O(n^2)$ \\
            $\linearskce$ & $ {\lfloor n/2\rfloor}^{-1} \sum_{i = 1}^{\lfloor n / 2\rfloor} h_{2i-1,2i}$ & unbiased & $O(n)$ \\ \bottomrule
          \end{tabular}
        \end{center}
      \end{tcolorbox}
    \end{itemize}
  }

  \posterbox[adjusted title={Viewing estimators as test statistics}]{name=statistics,column=2,below=estimator}{
    In general, the $\measure$ does not have a meaningful unit or scale. This renders it difficult to interpret an estimated non-zero error and to compare different models.

    \begin{tcolorbox}[colback=sandstark]
      For the consistent and unbiased estimators of the $\squaredkernelmeasure$ we derive \hl{bounds and approximations of the probability of false rejection} of a calibrated model.
    \end{tcolorbox}

    \vspace{-\topsep}
    \begin{itemize}
    \item These results allow us to test the hypothesis that model $g$ is calibrated.
    \item The bounds enable us to transfer unintuitive calibration error estimates to an \hl{intuitive and interpretable} probabilistic setting.
    \end{itemize}
  }

  \posterbox[adjusted title={Kernel calibration error}]{name=kce,column=1,between=error and references}{
    Let $\mathcal{F}$ be the unit ball in a reproducing kernel Hilbert space with matrix-valued kernel $k \colon \Delta^m \times \Delta^m \to \mathbb{R}^{m \times m}$. Then we define the \hl{kernel calibration error} ($\kernelmeasure$) with respect to kernel $k$ as $\kernelmeasure[k, g] \coloneqq \measure[\mathcal{F}, g]$.

    \begin{tcolorbox}[colback=blondsvag]
      An example of a matrix-valued kernel is $k(a, b) = M \tilde{k}(a, b)$, where $M \in \mathbb{R}^{m\times m}$ is a positive semi-definite matrix and $\tilde{k} \colon \Delta^m \times \Delta^m \to \mathbb{R}$ is a real-valued kernel.
    \end{tcolorbox}

    If $k$ is a universal kernel, then the $\kernelmeasure$ is zero if and only if model $g$ is calibrated.

    \begin{tcolorbox}[colback=sandstark]
      If $\Expect[\|k(g(X),g(X))\|] < \infty$, then
      \begin{equation}\label{eq:kce}
        \kernelmeasure[k,g] = {\bigg(\Expect[{(e_Y - g(X))}^{\intercal} k(g(X), g(X')) {(e_{Y'} - g(X'))}]\bigg)}^{1/2},
      \end{equation}
      where $(X', Y')$ is an independent copy of $(X,Y)$ and $e_i$ denotes the $i$th unit vector.
    \end{tcolorbox}
  }

  \posterbox[adjusted title={Experiments}]{name=experiment,column=2,between=statistics and references}{
    We construct data sets $\{g(X_i), Y_i\}_{i=1}^{250}$ of three models with $10$ classes by sampling predictions $g(X_i) \sim \Dir(0.1, \dots, 0.1)$ and labels $Y_i$ conditionally on $g(X_i)$ from
    \begin{align*}
      \text{\textbf{M1: }} &\Categorical(g(X_i)), &
                                                    \text{\textbf{M2: }} &0.5\Categorical(g(X_i)) + 0.5\Categorical(1,0,\dots,0), &
                                                                                                                                    \text{\textbf{M3: }} &\Categorical(0.1, \dots, 0.1).
    \end{align*}
    Model \textbf{M1} is calibrated, and models \textbf{M2} and \textbf{M3} are uncalibrated.

    \begin{center}
      \begin{tikzpicture}
\begin{groupplot}[group style={group size={3 by 4}, xlabels at={edge bottom}, ylabels at={edge left}, horizontal sep={0.1\linewidth}, vertical sep={0.04\linewidth}}, no markers, tick label style={font={\footnotesize}}, grid={major}, title style={align={center}}, width={0.22\linewidth}, height={0.1\linewidth}, every x tick scale label/.style={at={{(1,0)}}, anchor={west}}, ylabel style={font={\small}}]
    \nextgroupplot[title={\textbf{M1}}, ylabel={$\ECE$}]
    \addplot+[ybar interval, fill={blue!25}]
        table[row sep={\\}]
        {
            \\
            0.17  7.0  \\
            0.18  55.0  \\
            0.19  229.0  \\
            0.2  796.0  \\
            0.21  1782.0  \\
            0.22  2550.0  \\
            0.23  2307.0  \\
            0.24  1389.0  \\
            0.25  667.0  \\
            0.26  171.0  \\
            0.27  42.0  \\
            0.28  5.0  \\
            0.29  0.0  \\
        }
        ;
    \draw[solid, black, thick] (0.22876019647666532,\pgfkeysvalueof{/pgfplots/ymin})--(0.22876019647666532,\pgfkeysvalueof{/pgfplots/ymax});
    \draw[dashed, red, thick] (0.0,\pgfkeysvalueof{/pgfplots/ymin})--(0.0,\pgfkeysvalueof{/pgfplots/ymax});
    \addplot+[draw={none}]
        coordinates {
            (0.0,0)
        }
        ;
    \nextgroupplot[title={\textbf{M2}}]
    \addplot+[ybar interval, fill={blue!25}]
        table[row sep={\\}]
        {
            \\
            0.38  1.0  \\
            0.4  3.0  \\
            0.42  30.0  \\
            0.44  208.0  \\
            0.46  827.0  \\
            0.48  2107.0  \\
            0.5  2889.0  \\
            0.52  2406.0  \\
            0.54  1115.0  \\
            0.56  335.0  \\
            0.58  69.0  \\
            0.6  9.0  \\
            0.62  1.0  \\
            0.64  0.0  \\
        }
        ;
    \draw[solid, black, thick] (0.5130312782216148,\pgfkeysvalueof{/pgfplots/ymin})--(0.5130312782216148,\pgfkeysvalueof{/pgfplots/ymax});
    \draw[dashed, red, thick] (0.45,\pgfkeysvalueof{/pgfplots/ymin})--(0.45,\pgfkeysvalueof{/pgfplots/ymax});
    \addplot+[draw={none}]
        coordinates {
            (0.45,0)
        }
        ;
    \nextgroupplot[title={\textbf{M3}}]
    \addplot+[ybar interval, fill={blue!25}]
        table[row sep={\\}]
        {
            \\
            0.3  8.0  \\
            0.32  154.0  \\
            0.34  821.0  \\
            0.36  2308.0  \\
            0.38  3259.0  \\
            0.4  2318.0  \\
            0.42  928.0  \\
            0.44  180.0  \\
            0.46  23.0  \\
            0.48  1.0  \\
            0.5  0.0  \\
        }
        ;
    \draw[solid, black, thick] (0.3907003650959387,\pgfkeysvalueof{/pgfplots/ymin})--(0.3907003650959387,\pgfkeysvalueof{/pgfplots/ymax});
    \draw[dashed, red, thick] (0.7106418012290426,\pgfkeysvalueof{/pgfplots/ymin})--(0.7106418012290426,\pgfkeysvalueof{/pgfplots/ymax});
    \addplot+[draw={none}]
        coordinates {
            (0.7106418012290426,0)
        }
        ;
    \nextgroupplot[ylabel={$\biasedskce$}]
    \addplot+[ybar interval, fill={blue!25}]
        table[row sep={\\}]
        {
            \\
            0.0005  61.0  \\
            0.001  2724.0  \\
            0.0015  4376.0  \\
            0.002  2111.0  \\
            0.0025  597.0  \\
            0.003  108.0  \\
            0.0035  20.0  \\
            0.004  3.0  \\
            0.0045  0.0  \\
        }
        ;
    \draw[solid, black, thick] (0.001791065804877401,\pgfkeysvalueof{/pgfplots/ymin})--(0.001791065804877401,\pgfkeysvalueof{/pgfplots/ymax});
    \draw[dashed, red, thick] (-8.624308735909018e-6,\pgfkeysvalueof{/pgfplots/ymin})--(-8.624308735909018e-6,\pgfkeysvalueof{/pgfplots/ymax});
    \addplot+[draw={none}]
        coordinates {
            (-8.624308735909018e-6,0)
        }
        ;
    \nextgroupplot[]
    \addplot+[ybar interval, fill={blue!25}]
        table[row sep={\\}]
        {
            \\
            0.05  3.0  \\
            0.06  86.0  \\
            0.07  572.0  \\
            0.08  1791.0  \\
            0.09  2834.0  \\
            0.1  2563.0  \\
            0.11  1424.0  \\
            0.12  567.0  \\
            0.13  133.0  \\
            0.14  21.0  \\
            0.15  5.0  \\
            0.16  1.0  \\
            0.17  0.0  \\
        }
        ;
    \draw[solid, black, thick] (0.09955803307414735,\pgfkeysvalueof{/pgfplots/ymin})--(0.09955803307414735,\pgfkeysvalueof{/pgfplots/ymax});
    \draw[dashed, red, thick] (0.09634393682193111,\pgfkeysvalueof{/pgfplots/ymin})--(0.09634393682193111,\pgfkeysvalueof{/pgfplots/ymax});
    \addplot+[draw={none}]
        coordinates {
            (0.09634393682193111,0)
        }
        ;
    \nextgroupplot[]
    \addplot+[ybar interval, fill={blue!25}]
        table[row sep={\\}]
        {
            \\
            0.015  15.0  \\
            0.016  97.0  \\
            0.017  450.0  \\
            0.018  1267.0  \\
            0.019  2139.0  \\
            0.02  2390.0  \\
            0.021  1782.0  \\
            0.022  1061.0  \\
            0.023  493.0  \\
            0.024  216.0  \\
            0.025  61.0  \\
            0.026  19.0  \\
            0.027  8.0  \\
            0.028  2.0  \\
            0.029  0.0  \\
        }
        ;
    \draw[solid, black, thick] (0.020517506214089484,\pgfkeysvalueof{/pgfplots/ymin})--(0.020517506214089484,\pgfkeysvalueof{/pgfplots/ymax});
    \draw[dashed, red, thick] (0.015176641534493648,\pgfkeysvalueof{/pgfplots/ymin})--(0.015176641534493648,\pgfkeysvalueof{/pgfplots/ymax});
    \addplot+[draw={none}]
        coordinates {
            (0.015176641534493648,0)
        }
        ;
    \nextgroupplot[ylabel={$\unbiasedskce$}]
    \addplot+[ybar interval, fill={blue!25}]
        table[row sep={\\}]
        {
            \\
            -0.001  37.0  \\
            -0.0008  438.0  \\
            -0.0006  1286.0  \\
            -0.0004  1968.0  \\
            -0.0002  1919.0  \\
            0.0  1607.0  \\
            0.0002  1132.0  \\
            0.0004  704.0  \\
            0.0006  423.0  \\
            0.0008  242.0  \\
            0.001  122.0  \\
            0.0012  72.0  \\
            0.0014  22.0  \\
            0.0016  20.0  \\
            0.0018  4.0  \\
            0.002  4.0  \\
            0.0022  0.0  \\
        }
        ;
    \draw[solid, black, thick] (-8.624308735909018e-6,\pgfkeysvalueof{/pgfplots/ymin})--(-8.624308735909018e-6,\pgfkeysvalueof{/pgfplots/ymax});
    \draw[dashed, red, thick] (-8.624308735909018e-6,\pgfkeysvalueof{/pgfplots/ymin})--(-8.624308735909018e-6,\pgfkeysvalueof{/pgfplots/ymax});
    \addplot+[draw={none}]
        coordinates {
            (-8.624308735909018e-6,0)
        }
        ;
    \nextgroupplot[]
    \addplot+[ybar interval, fill={blue!25}]
        table[row sep={\\}]
        {
            \\
            0.04  1.0  \\
            0.05  11.0  \\
            0.06  172.0  \\
            0.07  904.0  \\
            0.08  2195.0  \\
            0.09  2942.0  \\
            0.1  2228.0  \\
            0.11  1080.0  \\
            0.12  371.0  \\
            0.13  84.0  \\
            0.14  9.0  \\
            0.15  2.0  \\
            0.16  1.0  \\
            0.17  0.0  \\
        }
        ;
    \draw[solid, black, thick] (0.09634393682193111,\pgfkeysvalueof{/pgfplots/ymin})--(0.09634393682193111,\pgfkeysvalueof{/pgfplots/ymax});
    \draw[dashed, red, thick] (0.09634393682193111,\pgfkeysvalueof{/pgfplots/ymin})--(0.09634393682193111,\pgfkeysvalueof{/pgfplots/ymax});
    \addplot+[draw={none}]
        coordinates {
            (0.09634393682193111,0)
        }
        ;
    \nextgroupplot[]
    \addplot+[ybar interval, fill={blue!25}]
        table[row sep={\\}]
        {
            \\
            0.01  17.0  \\
            0.011  144.0  \\
            0.012  628.0  \\
            0.013  1606.0  \\
            0.014  2478.0  \\
            0.015  2241.0  \\
            0.016  1566.0  \\
            0.017  810.0  \\
            0.018  323.0  \\
            0.019  137.0  \\
            0.02  33.0  \\
            0.021  14.0  \\
            0.022  2.0  \\
            0.023  1.0  \\
            0.024  0.0  \\
        }
        ;
    \draw[solid, black, thick] (0.015176641534493648,\pgfkeysvalueof{/pgfplots/ymin})--(0.015176641534493648,\pgfkeysvalueof{/pgfplots/ymax});
    \draw[dashed, red, thick] (0.015176641534493648,\pgfkeysvalueof{/pgfplots/ymin})--(0.015176641534493648,\pgfkeysvalueof{/pgfplots/ymax});
    \addplot+[draw={none}]
        coordinates {
            (0.015176641534493648,0)
        }
        ;
    \nextgroupplot[ylabel={$\linearskce$}]
    \addplot+[ybar interval, fill={blue!25}]
        table[row sep={\\}]
        {
            \\
            -0.03  1.0  \\
            -0.025  13.0  \\
            -0.02  117.0  \\
            -0.015  555.0  \\
            -0.01  1580.0  \\
            -0.005  2580.0  \\
            0.0  2825.0  \\
            0.005  1594.0  \\
            0.01  564.0  \\
            0.015  145.0  \\
            0.02  21.0  \\
            0.025  5.0  \\
            0.03  0.0  \\
        }
        ;
    \draw[solid, black, thick] (0.00016041013856191272,\pgfkeysvalueof{/pgfplots/ymin})--(0.00016041013856191272,\pgfkeysvalueof{/pgfplots/ymax});
    \draw[dashed, red, thick] (-8.624308735909018e-6,\pgfkeysvalueof{/pgfplots/ymin})--(-8.624308735909018e-6,\pgfkeysvalueof{/pgfplots/ymax});
    \addplot+[draw={none}]
        coordinates {
            (-8.624308735909018e-6,0)
        }
        ;
    \nextgroupplot[]
    \addplot+[ybar interval, fill={blue!25}]
        table[row sep={\\}]
        {
            \\
            0.02  2.0  \\
            0.03  12.0  \\
            0.04  44.0  \\
            0.05  238.0  \\
            0.06  598.0  \\
            0.07  1199.0  \\
            0.08  1766.0  \\
            0.09  1992.0  \\
            0.1  1748.0  \\
            0.11  1195.0  \\
            0.12  706.0  \\
            0.13  304.0  \\
            0.14  132.0  \\
            0.15  48.0  \\
            0.16  13.0  \\
            0.17  3.0  \\
            0.18  0.0  \\
        }
        ;
    \draw[solid, black, thick] (0.09631220152950964,\pgfkeysvalueof{/pgfplots/ymin})--(0.09631220152950964,\pgfkeysvalueof{/pgfplots/ymax});
    \draw[dashed, red, thick] (0.09634393682193111,\pgfkeysvalueof{/pgfplots/ymin})--(0.09634393682193111,\pgfkeysvalueof{/pgfplots/ymax});
    \addplot+[draw={none}]
        coordinates {
            (0.09634393682193111,0)
        }
        ;
    \nextgroupplot[]
    \addplot+[ybar interval, fill={blue!25}]
        table[row sep={\\}]
        {
            \\
            -0.06  2.0  \\
            -0.05  7.0  \\
            -0.04  35.0  \\
            -0.03  216.0  \\
            -0.02  607.0  \\
            -0.01  1213.0  \\
            0.0  1919.0  \\
            0.01  2057.0  \\
            0.02  1844.0  \\
            0.03  1212.0  \\
            0.04  573.0  \\
            0.05  226.0  \\
            0.06  72.0  \\
            0.07  11.0  \\
            0.08  6.0  \\
            0.09  0.0  \\
        }
        ;
    \draw[solid, black, thick] (0.015097861091196648,\pgfkeysvalueof{/pgfplots/ymin})--(0.015097861091196648,\pgfkeysvalueof{/pgfplots/ymax});
    \draw[dashed, red, thick] (0.015176641534493648,\pgfkeysvalueof{/pgfplots/ymin})--(0.015176641534493648,\pgfkeysvalueof{/pgfplots/ymax});
    \addplot+[draw={none}]
        coordinates {
            (0.015176641534493648,0)
        }
        ;
\end{groupplot}
\node[anchor=north] at ($(group c1r4.west |- group c1r4.outer south)!0.5!(group c3r4.east |- group c3r4.outer south)$){estimate};
\node[anchor=south, rotate=90] at ($(group c1r1.north -| group c1r1.outer west)!0.5!(group c1r4.south -| group c1r4.outer west)$){\# runs};
\end{tikzpicture}

      \captionof{figure}{Calibration error estimates of $10^4$ randomly sampled data sets. The solid black line indicates the mean of the calibration error estimates, and the dashed red line displays the true calibration error of the model.}
    \end{center}

    \begin{center}
      \begin{tikzpicture}
\begin{groupplot}[group style={group size={3 by 6}, xlabels at={edge bottom}, ylabels at={edge left}, horizontal sep={0.1\linewidth}, vertical sep={0.02\linewidth}, xticklabels at={edge bottom}}, no markers, tick label style={font={\footnotesize}}, grid={major}, title style={align={center}}, width={0.22\linewidth}, height={0.1\linewidth}, every x tick scale label/.style={at={{(1,0)}}, anchor={west}}, ylabel style={font={\small}}]
    \nextgroupplot[xmin={0}, xmax={1}, ymin={0}, title={\textbf{M1}}, ylabel={$\ECE$}]
    \addplot+[]
        table[row sep={\\}]
        {
            \\
            0.0  0.0091  \\
            0.01  0.0728  \\
            0.02  0.1216  \\
            0.03  0.1663  \\
            0.04  0.2053  \\
            0.05  0.2447  \\
            0.06  0.2751  \\
            0.07  0.303  \\
            0.08  0.3305  \\
            0.09  0.3567  \\
            0.1  0.3827  \\
            0.11  0.4063  \\
            0.12  0.4305  \\
            0.13  0.4501  \\
            0.14  0.473  \\
            0.15  0.493  \\
            0.16  0.5103  \\
            0.17  0.5279  \\
            0.18  0.546  \\
            0.19  0.5617  \\
            0.2  0.5806  \\
            0.21  0.5952  \\
            0.22  0.6095  \\
            0.23  0.6236  \\
            0.24  0.6349  \\
            0.25  0.6497  \\
            0.26  0.6626  \\
            0.27  0.6755  \\
            0.28  0.6873  \\
            0.29  0.6995  \\
            0.3  0.7102  \\
            0.31  0.7221  \\
            0.32  0.7329  \\
            0.33  0.7437  \\
            0.34  0.7533  \\
            0.35  0.7613  \\
            0.36  0.7706  \\
            0.37  0.7793  \\
            0.38  0.7883  \\
            0.39  0.7972  \\
            0.4  0.8049  \\
            0.41  0.8123  \\
            0.42  0.8203  \\
            0.43  0.8282  \\
            0.44  0.8353  \\
            0.45  0.8417  \\
            0.46  0.8471  \\
            0.47  0.8544  \\
            0.48  0.8602  \\
            0.49  0.8664  \\
            0.5  0.8722  \\
            0.51  0.8783  \\
            0.52  0.8837  \\
            0.53  0.8887  \\
            0.54  0.8948  \\
            0.55  0.9007  \\
            0.56  0.9048  \\
            0.57  0.9085  \\
            0.58  0.913  \\
            0.59  0.9167  \\
            0.6  0.9212  \\
            0.61  0.9254  \\
            0.62  0.9295  \\
            0.63  0.9328  \\
            0.64  0.936  \\
            0.65  0.9407  \\
            0.66  0.9449  \\
            0.67  0.9487  \\
            0.68  0.9519  \\
            0.69  0.9553  \\
            0.7  0.9581  \\
            0.71  0.961  \\
            0.72  0.9636  \\
            0.73  0.9671  \\
            0.74  0.9707  \\
            0.75  0.9726  \\
            0.76  0.9747  \\
            0.77  0.9759  \\
            0.78  0.9777  \\
            0.79  0.9798  \\
            0.8  0.9819  \\
            0.81  0.9839  \\
            0.82  0.9851  \\
            0.83  0.9866  \\
            0.84  0.9881  \\
            0.85  0.9895  \\
            0.86  0.991  \\
            0.87  0.9923  \\
            0.88  0.9935  \\
            0.89  0.9942  \\
            0.9  0.995  \\
            0.91  0.9957  \\
            0.92  0.9969  \\
            0.93  0.9974  \\
            0.94  0.9982  \\
            0.95  0.999  \\
            0.96  0.9993  \\
            0.97  0.9995  \\
            0.98  0.9998  \\
            0.99  0.9998  \\
            1.0  1.0  \\
        }
        ;
    \addplot+[dotted]
        coordinates {
            (0,0)
            (1,1)
        }
        ;
    \nextgroupplot[xmin={0}, xmax={1}, ymin={0}, title={\textbf{M2}}]
    \addplot+[]
        coordinates {
            (0.0,0.0)
            (0.01,0.0)
            (0.02,0.0)
            (0.03,0.0)
            (0.04,0.0)
            (0.05,0.0)
            (0.06,0.0)
            (0.07,0.0)
            (0.08,0.0)
            (0.09,0.0)
            (0.1,0.0)
            (0.11,0.0)
            (0.12,0.0)
            (0.13,0.0)
            (0.14,0.0)
            (0.15,0.0)
            (0.16,0.0)
            (0.17,0.0)
            (0.18,0.0)
            (0.19,0.0)
            (0.2,0.0)
            (0.21,0.0)
            (0.22,0.0)
            (0.23,0.0)
            (0.24,0.0)
            (0.25,0.0)
            (0.26,0.0)
            (0.27,0.0)
            (0.28,0.0)
            (0.29,0.0)
            (0.3,0.0)
            (0.31,0.0)
            (0.32,0.0)
            (0.33,0.0)
            (0.34,0.0)
            (0.35,0.0)
            (0.36,0.0)
            (0.37,0.0)
            (0.38,0.0)
            (0.39,0.0)
            (0.4,0.0)
            (0.41,0.0)
            (0.42,0.0)
            (0.43,0.0)
            (0.44,0.0)
            (0.45,0.0)
            (0.46,0.0)
            (0.47,0.0)
            (0.48,0.0)
            (0.49,0.0)
            (0.5,0.0)
            (0.51,0.0)
            (0.52,0.0)
            (0.53,0.0)
            (0.54,0.0)
            (0.55,0.0)
            (0.56,0.0)
            (0.57,0.0)
            (0.58,0.0)
            (0.59,0.0)
            (0.6,0.0)
            (0.61,0.0)
            (0.62,0.0)
            (0.63,0.0)
            (0.64,0.0)
            (0.65,0.0)
            (0.66,0.0)
            (0.67,0.0)
            (0.68,0.0)
            (0.69,0.0)
            (0.7,0.0)
            (0.71,0.0)
            (0.72,0.0)
            (0.73,0.0)
            (0.74,0.0)
            (0.75,0.0)
            (0.76,0.0)
            (0.77,0.0)
            (0.78,0.0)
            (0.79,0.0)
            (0.8,0.0)
            (0.81,0.0)
            (0.82,0.0)
            (0.83,0.0)
            (0.84,0.0)
            (0.85,0.0)
            (0.86,0.0)
            (0.87,0.0)
            (0.88,0.0)
            (0.89,0.0)
            (0.9,0.0)
            (0.91,0.0)
            (0.92,0.0)
            (0.93,0.0)
            (0.94,0.0)
            (0.95,0.0)
            (0.96,0.0)
            (0.97,0.0)
            (0.98,0.0)
            (0.99,0.0)
            (1.0,0.0)
        }
        ;
    \nextgroupplot[xmin={0}, xmax={1}, ymin={0}, title={\textbf{M3}}]
    \addplot+[]
        coordinates {
            (0.0,0.0)
            (0.01,0.0)
            (0.02,0.0)
            (0.03,0.0)
            (0.04,0.0)
            (0.05,0.0)
            (0.06,0.0)
            (0.07,0.0)
            (0.08,0.0)
            (0.09,0.0)
            (0.1,0.0)
            (0.11,0.0)
            (0.12,0.0)
            (0.13,0.0)
            (0.14,0.0)
            (0.15,0.0)
            (0.16,0.0)
            (0.17,0.0)
            (0.18,0.0)
            (0.19,0.0)
            (0.2,0.0)
            (0.21,0.0)
            (0.22,0.0)
            (0.23,0.0)
            (0.24,0.0)
            (0.25,0.0)
            (0.26,0.0)
            (0.27,0.0)
            (0.28,0.0)
            (0.29,0.0)
            (0.3,0.0)
            (0.31,0.0)
            (0.32,0.0)
            (0.33,0.0)
            (0.34,0.0)
            (0.35,0.0)
            (0.36,0.0)
            (0.37,0.0)
            (0.38,0.0)
            (0.39,0.0)
            (0.4,0.0)
            (0.41,0.0)
            (0.42,0.0)
            (0.43,0.0)
            (0.44,0.0)
            (0.45,0.0)
            (0.46,0.0)
            (0.47,0.0)
            (0.48,0.0)
            (0.49,0.0)
            (0.5,0.0)
            (0.51,0.0)
            (0.52,0.0)
            (0.53,0.0)
            (0.54,0.0)
            (0.55,0.0)
            (0.56,0.0)
            (0.57,0.0)
            (0.58,0.0)
            (0.59,0.0)
            (0.6,0.0)
            (0.61,0.0)
            (0.62,0.0)
            (0.63,0.0)
            (0.64,0.0)
            (0.65,0.0)
            (0.66,0.0)
            (0.67,0.0)
            (0.68,0.0)
            (0.69,0.0)
            (0.7,0.0)
            (0.71,0.0)
            (0.72,0.0)
            (0.73,0.0)
            (0.74,0.0)
            (0.75,0.0)
            (0.76,0.0)
            (0.77,0.0)
            (0.78,0.0)
            (0.79,0.0)
            (0.8,0.0)
            (0.81,0.0)
            (0.82,0.0)
            (0.83,0.0)
            (0.84,0.0)
            (0.85,0.0)
            (0.86,0.0)
            (0.87,0.0)
            (0.88,0.0)
            (0.89,0.0)
            (0.9,0.0)
            (0.91,0.0)
            (0.92,0.0)
            (0.93,0.0)
            (0.94,0.0)
            (0.95,0.0)
            (0.96,0.0)
            (0.97,0.0)
            (0.98,0.0)
            (0.99,0.0)
            (1.0,0.0)
        }
        ;
    \nextgroupplot[xmin={0}, xmax={1}, ymin={0}, ylabel={$\biasedskce$}]
    \addplot+[]
        table[row sep={\\}]
        {
            \\
            0.0  0.0  \\
            0.01  0.0  \\
            0.02  0.0  \\
            0.03  0.0  \\
            0.04  0.0  \\
            0.05  0.0  \\
            0.06  0.0  \\
            0.07  0.0  \\
            0.08  0.0  \\
            0.09  0.0  \\
            0.1  0.0  \\
            0.11  0.0  \\
            0.12  0.0  \\
            0.13  0.0  \\
            0.14  0.0  \\
            0.15  0.0  \\
            0.16  0.0  \\
            0.17  0.0  \\
            0.18  0.0  \\
            0.19  0.0  \\
            0.2  0.0  \\
            0.21  0.0  \\
            0.22  0.0  \\
            0.23  0.0  \\
            0.24  0.0  \\
            0.25  0.0  \\
            0.26  0.0  \\
            0.27  0.0  \\
            0.28  0.0  \\
            0.29  0.0  \\
            0.3  0.0  \\
            0.31  0.0  \\
            0.32  0.0  \\
            0.33  0.0  \\
            0.34  0.0  \\
            0.35  0.0  \\
            0.36  0.0  \\
            0.37  0.0  \\
            0.38  0.0  \\
            0.39  0.0  \\
            0.4  0.0  \\
            0.41  0.0  \\
            0.42  0.0  \\
            0.43  0.0  \\
            0.44  0.0  \\
            0.45  0.0  \\
            0.46  0.0  \\
            0.47  0.0  \\
            0.48  0.0  \\
            0.49  0.0  \\
            0.5  0.0  \\
            0.51  0.0  \\
            0.52  0.0  \\
            0.53  0.0  \\
            0.54  0.0  \\
            0.55  0.0  \\
            0.56  0.0  \\
            0.57  0.0  \\
            0.58  0.0  \\
            0.59  0.0  \\
            0.6  0.0  \\
            0.61  0.0  \\
            0.62  0.0  \\
            0.63  0.0  \\
            0.64  0.0  \\
            0.65  0.0  \\
            0.66  0.0  \\
            0.67  0.0  \\
            0.68  0.0  \\
            0.69  0.0  \\
            0.7  0.0  \\
            0.71  0.0  \\
            0.72  0.0  \\
            0.73  0.0  \\
            0.74  0.0  \\
            0.75  0.0  \\
            0.76  0.0  \\
            0.77  0.0  \\
            0.78  0.0  \\
            0.79  0.0  \\
            0.8  0.0  \\
            0.81  0.0  \\
            0.82  0.0  \\
            0.83  0.0  \\
            0.84  0.0  \\
            0.85  0.0  \\
            0.86  0.0  \\
            0.87  0.0  \\
            0.88  0.0  \\
            0.89  0.0  \\
            0.9  0.0  \\
            0.91  0.0  \\
            0.92  0.0  \\
            0.93  0.0  \\
            0.94  0.0  \\
            0.95  0.0  \\
            0.96  0.0  \\
            0.97  0.0  \\
            0.98  0.0  \\
            0.99  0.0  \\
            1.0  1.0  \\
        }
        ;
    \addplot+[dotted]
        coordinates {
            (0,0)
            (1,1)
        }
        ;
    \nextgroupplot[xmin={0}, xmax={1}, ymin={0}]
    \addplot+[]
        coordinates {
            (0.0,1.0)
            (0.01,0.9844)
            (0.02,0.8762)
            (0.03,0.7087)
            (0.04,0.531)
            (0.05,0.38029999999999997)
            (0.06,0.2711)
            (0.07,0.1886)
            (0.08,0.12760000000000005)
            (0.09,0.08589999999999998)
            (0.1,0.05830000000000002)
            (0.11,0.04059999999999997)
            (0.12,0.027800000000000047)
            (0.13,0.017800000000000038)
            (0.14,0.012900000000000023)
            (0.15,0.008099999999999996)
            (0.16,0.005099999999999993)
            (0.17,0.0040000000000000036)
            (0.18,0.0026000000000000467)
            (0.19,0.0013999999999999568)
            (0.2,0.0010000000000000009)
            (0.21,0.0008000000000000229)
            (0.22,0.00029999999999996696)
            (0.23,0.00019999999999997797)
            (0.24,9.999999999998899e-5)
            (0.25,9.999999999998899e-5)
            (0.26,9.999999999998899e-5)
            (0.27,9.999999999998899e-5)
            (0.28,9.999999999998899e-5)
            (0.29,9.999999999998899e-5)
            (0.3,9.999999999998899e-5)
            (0.31,9.999999999998899e-5)
            (0.32,0.0)
            (0.33,0.0)
            (0.34,0.0)
            (0.35,0.0)
            (0.36,0.0)
            (0.37,0.0)
            (0.38,0.0)
            (0.39,0.0)
            (0.4,0.0)
            (0.41,0.0)
            (0.42,0.0)
            (0.43,0.0)
            (0.44,0.0)
            (0.45,0.0)
            (0.46,0.0)
            (0.47,0.0)
            (0.48,0.0)
            (0.49,0.0)
            (0.5,0.0)
            (0.51,0.0)
            (0.52,0.0)
            (0.53,0.0)
            (0.54,0.0)
            (0.55,0.0)
            (0.56,0.0)
            (0.57,0.0)
            (0.58,0.0)
            (0.59,0.0)
            (0.6,0.0)
            (0.61,0.0)
            (0.62,0.0)
            (0.63,0.0)
            (0.64,0.0)
            (0.65,0.0)
            (0.66,0.0)
            (0.67,0.0)
            (0.68,0.0)
            (0.69,0.0)
            (0.7,0.0)
            (0.71,0.0)
            (0.72,0.0)
            (0.73,0.0)
            (0.74,0.0)
            (0.75,0.0)
            (0.76,0.0)
            (0.77,0.0)
            (0.78,0.0)
            (0.79,0.0)
            (0.8,0.0)
            (0.81,0.0)
            (0.82,0.0)
            (0.83,0.0)
            (0.84,0.0)
            (0.85,0.0)
            (0.86,0.0)
            (0.87,0.0)
            (0.88,0.0)
            (0.89,0.0)
            (0.9,0.0)
            (0.91,0.0)
            (0.92,0.0)
            (0.93,0.0)
            (0.94,0.0)
            (0.95,0.0)
            (0.96,0.0)
            (0.97,0.0)
            (0.98,0.0)
            (0.99,0.0)
            (1.0,0.0)
        }
        ;
    \nextgroupplot[xmin={0}, xmax={1}, ymin={0}]
    \addplot+[]
        coordinates {
            (0.0,1.0)
            (0.01,1.0)
            (0.02,1.0)
            (0.03,1.0)
            (0.04,1.0)
            (0.05,1.0)
            (0.06,1.0)
            (0.07,1.0)
            (0.08,1.0)
            (0.09,1.0)
            (0.1,1.0)
            (0.11,1.0)
            (0.12,1.0)
            (0.13,1.0)
            (0.14,1.0)
            (0.15,1.0)
            (0.16,1.0)
            (0.17,1.0)
            (0.18,1.0)
            (0.19,1.0)
            (0.2,1.0)
            (0.21,1.0)
            (0.22,1.0)
            (0.23,1.0)
            (0.24,1.0)
            (0.25,1.0)
            (0.26,1.0)
            (0.27,1.0)
            (0.28,1.0)
            (0.29,1.0)
            (0.3,1.0)
            (0.31,1.0)
            (0.32,1.0)
            (0.33,1.0)
            (0.34,1.0)
            (0.35,1.0)
            (0.36,1.0)
            (0.37,1.0)
            (0.38,1.0)
            (0.39,1.0)
            (0.4,1.0)
            (0.41,1.0)
            (0.42,1.0)
            (0.43,1.0)
            (0.44,1.0)
            (0.45,1.0)
            (0.46,1.0)
            (0.47,1.0)
            (0.48,1.0)
            (0.49,1.0)
            (0.5,1.0)
            (0.51,1.0)
            (0.52,1.0)
            (0.53,1.0)
            (0.54,1.0)
            (0.55,1.0)
            (0.56,1.0)
            (0.57,1.0)
            (0.58,1.0)
            (0.59,1.0)
            (0.6,1.0)
            (0.61,1.0)
            (0.62,1.0)
            (0.63,1.0)
            (0.64,1.0)
            (0.65,1.0)
            (0.66,1.0)
            (0.67,0.9999)
            (0.68,0.9999)
            (0.69,0.9997)
            (0.7,0.9996)
            (0.71,0.9986)
            (0.72,0.9978)
            (0.73,0.9965)
            (0.74,0.9936)
            (0.75,0.9877)
            (0.76,0.9753000000000001)
            (0.77,0.961)
            (0.78,0.9373)
            (0.79,0.9013)
            (0.8,0.8486)
            (0.81,0.7737)
            (0.82,0.6861999999999999)
            (0.83,0.583)
            (0.84,0.45930000000000004)
            (0.85,0.33530000000000004)
            (0.86,0.22250000000000003)
            (0.87,0.13280000000000003)
            (0.88,0.07050000000000001)
            (0.89,0.031100000000000017)
            (0.9,0.01200000000000001)
            (0.91,0.0034999999999999476)
            (0.92,0.0007000000000000339)
            (0.93,9.999999999998899e-5)
            (0.94,0.0)
            (0.95,0.0)
            (0.96,0.0)
            (0.97,0.0)
            (0.98,0.0)
            (0.99,0.0)
            (1.0,0.0)
        }
        ;
    \nextgroupplot[xmin={0}, xmax={1}, ymin={0}, ylabel={$\unbiasedskce$}]
    \addplot+[]
        table[row sep={\\}]
        {
            \\
            0.0  0.0  \\
            0.01  0.0  \\
            0.02  0.0  \\
            0.03  0.0  \\
            0.04  0.0  \\
            0.05  0.0  \\
            0.06  0.0  \\
            0.07  0.0  \\
            0.08  0.0  \\
            0.09  0.0  \\
            0.1  0.0  \\
            0.11  0.0  \\
            0.12  0.0  \\
            0.13  0.0  \\
            0.14  0.0  \\
            0.15  0.0  \\
            0.16  0.0  \\
            0.17  0.0  \\
            0.18  0.0  \\
            0.19  0.0  \\
            0.2  0.0  \\
            0.21  0.0  \\
            0.22  0.0  \\
            0.23  0.0  \\
            0.24  0.0  \\
            0.25  0.0  \\
            0.26  0.0  \\
            0.27  0.0  \\
            0.28  0.0  \\
            0.29  0.0  \\
            0.3  0.0  \\
            0.31  0.0  \\
            0.32  0.0  \\
            0.33  0.0  \\
            0.34  0.0  \\
            0.35  0.0  \\
            0.36  0.0  \\
            0.37  0.0  \\
            0.38  0.0  \\
            0.39  0.0  \\
            0.4  0.0  \\
            0.41  0.0  \\
            0.42  0.0  \\
            0.43  0.0  \\
            0.44  0.0  \\
            0.45  0.0  \\
            0.46  0.0  \\
            0.47  0.0  \\
            0.48  0.0  \\
            0.49  0.0  \\
            0.5  0.0  \\
            0.51  0.0  \\
            0.52  0.0  \\
            0.53  0.0  \\
            0.54  0.0  \\
            0.55  0.0  \\
            0.56  0.0  \\
            0.57  0.0  \\
            0.58  0.0  \\
            0.59  0.0  \\
            0.6  0.0  \\
            0.61  0.0  \\
            0.62  0.0  \\
            0.63  0.0  \\
            0.64  0.0  \\
            0.65  0.0  \\
            0.66  0.0  \\
            0.67  0.0  \\
            0.68  0.0  \\
            0.69  0.0  \\
            0.7  0.0  \\
            0.71  0.0  \\
            0.72  0.0  \\
            0.73  0.0  \\
            0.74  0.0  \\
            0.75  0.0  \\
            0.76  0.0  \\
            0.77  0.0  \\
            0.78  0.0  \\
            0.79  0.0  \\
            0.8  0.0  \\
            0.81  0.0  \\
            0.82  0.0  \\
            0.83  0.0  \\
            0.84  0.0  \\
            0.85  0.0  \\
            0.86  0.0  \\
            0.87  0.0  \\
            0.88  0.0  \\
            0.89  0.0  \\
            0.9  0.0  \\
            0.91  0.0  \\
            0.92  0.0  \\
            0.93  0.0  \\
            0.94  0.0  \\
            0.95  0.0  \\
            0.96  0.0  \\
            0.97  0.0  \\
            0.98  0.0  \\
            0.99  0.0  \\
            1.0  1.0  \\
        }
        ;
    \addplot+[dotted]
        coordinates {
            (0,0)
            (1,1)
        }
        ;
    \nextgroupplot[xmin={0}, xmax={1}, ymin={0}]
    \addplot+[]
        coordinates {
            (0.0,1.0)
            (0.01,1.0)
            (0.02,1.0)
            (0.03,1.0)
            (0.04,1.0)
            (0.05,1.0)
            (0.06,1.0)
            (0.07,1.0)
            (0.08,1.0)
            (0.09,1.0)
            (0.1,1.0)
            (0.11,1.0)
            (0.12,1.0)
            (0.13,1.0)
            (0.14,1.0)
            (0.15,1.0)
            (0.16,1.0)
            (0.17,1.0)
            (0.18,1.0)
            (0.19,1.0)
            (0.2,1.0)
            (0.21,1.0)
            (0.22,1.0)
            (0.23,1.0)
            (0.24,1.0)
            (0.25,1.0)
            (0.26,1.0)
            (0.27,1.0)
            (0.28,1.0)
            (0.29,1.0)
            (0.3,1.0)
            (0.31,1.0)
            (0.32,1.0)
            (0.33,1.0)
            (0.34,1.0)
            (0.35,1.0)
            (0.36,1.0)
            (0.37,1.0)
            (0.38,1.0)
            (0.39,1.0)
            (0.4,1.0)
            (0.41,1.0)
            (0.42,1.0)
            (0.43,1.0)
            (0.44,1.0)
            (0.45,1.0)
            (0.46,1.0)
            (0.47,1.0)
            (0.48,1.0)
            (0.49,1.0)
            (0.5,1.0)
            (0.51,1.0)
            (0.52,1.0)
            (0.53,1.0)
            (0.54,1.0)
            (0.55,1.0)
            (0.56,1.0)
            (0.57,1.0)
            (0.58,1.0)
            (0.59,1.0)
            (0.6,1.0)
            (0.61,1.0)
            (0.62,1.0)
            (0.63,1.0)
            (0.64,1.0)
            (0.65,1.0)
            (0.66,1.0)
            (0.67,0.9999)
            (0.68,0.9999)
            (0.69,0.9998)
            (0.7,0.9997)
            (0.71,0.9996)
            (0.72,0.9994)
            (0.73,0.9991)
            (0.74,0.9983)
            (0.75,0.9967)
            (0.76,0.9937)
            (0.77,0.9892)
            (0.78,0.983)
            (0.79,0.9687)
            (0.8,0.9499)
            (0.81,0.9216)
            (0.82,0.8843)
            (0.83,0.8325)
            (0.84,0.7622)
            (0.85,0.6780999999999999)
            (0.86,0.5712999999999999)
            (0.87,0.4536)
            (0.88,0.3416)
            (0.89,0.23360000000000003)
            (0.9,0.14449999999999996)
            (0.91,0.07509999999999994)
            (0.92,0.0353)
            (0.93,0.012700000000000045)
            (0.94,0.0032999999999999696)
            (0.95,0.00029999999999996696)
            (0.96,9.999999999998899e-5)
            (0.97,0.0)
            (0.98,0.0)
            (0.99,0.0)
            (1.0,0.0)
        }
        ;
    \nextgroupplot[xmin={0}, xmax={1}, ymin={0}]
    \addplot+[]
        coordinates {
            (0.0,1.0)
            (0.01,1.0)
            (0.02,1.0)
            (0.03,1.0)
            (0.04,1.0)
            (0.05,1.0)
            (0.06,1.0)
            (0.07,1.0)
            (0.08,1.0)
            (0.09,1.0)
            (0.1,1.0)
            (0.11,1.0)
            (0.12,1.0)
            (0.13,1.0)
            (0.14,1.0)
            (0.15,1.0)
            (0.16,1.0)
            (0.17,1.0)
            (0.18,1.0)
            (0.19,1.0)
            (0.2,1.0)
            (0.21,1.0)
            (0.22,1.0)
            (0.23,1.0)
            (0.24,1.0)
            (0.25,1.0)
            (0.26,1.0)
            (0.27,1.0)
            (0.28,1.0)
            (0.29,1.0)
            (0.3,1.0)
            (0.31,1.0)
            (0.32,1.0)
            (0.33,1.0)
            (0.34,1.0)
            (0.35,1.0)
            (0.36,1.0)
            (0.37,1.0)
            (0.38,1.0)
            (0.39,1.0)
            (0.4,1.0)
            (0.41,1.0)
            (0.42,1.0)
            (0.43,1.0)
            (0.44,1.0)
            (0.45,1.0)
            (0.46,1.0)
            (0.47,1.0)
            (0.48,1.0)
            (0.49,1.0)
            (0.5,1.0)
            (0.51,1.0)
            (0.52,1.0)
            (0.53,1.0)
            (0.54,1.0)
            (0.55,1.0)
            (0.56,1.0)
            (0.57,1.0)
            (0.58,1.0)
            (0.59,1.0)
            (0.6,1.0)
            (0.61,1.0)
            (0.62,1.0)
            (0.63,1.0)
            (0.64,1.0)
            (0.65,1.0)
            (0.66,1.0)
            (0.67,1.0)
            (0.68,1.0)
            (0.69,1.0)
            (0.7,1.0)
            (0.71,1.0)
            (0.72,1.0)
            (0.73,1.0)
            (0.74,1.0)
            (0.75,1.0)
            (0.76,1.0)
            (0.77,1.0)
            (0.78,1.0)
            (0.79,1.0)
            (0.8,1.0)
            (0.81,1.0)
            (0.82,1.0)
            (0.83,1.0)
            (0.84,1.0)
            (0.85,1.0)
            (0.86,1.0)
            (0.87,1.0)
            (0.88,1.0)
            (0.89,1.0)
            (0.9,1.0)
            (0.91,1.0)
            (0.92,1.0)
            (0.93,1.0)
            (0.94,1.0)
            (0.95,1.0)
            (0.96,1.0)
            (0.97,1.0)
            (0.98,1.0)
            (0.99,1.0)
            (1.0,0.0)
        }
        ;
    \nextgroupplot[xmin={0}, xmax={1}, ymin={0}, ylabel={$\linearskce$}]
    \addplot+[]
        table[row sep={\\}]
        {
            \\
            0.0  0.0  \\
            0.01  0.0  \\
            0.02  0.0  \\
            0.03  0.0  \\
            0.04  0.0  \\
            0.05  0.0  \\
            0.06  0.0  \\
            0.07  0.0  \\
            0.08  0.0  \\
            0.09  0.0  \\
            0.1  0.0  \\
            0.11  0.0  \\
            0.12  0.0  \\
            0.13  0.0  \\
            0.14  0.0  \\
            0.15  0.0  \\
            0.16  0.0  \\
            0.17  0.0  \\
            0.18  0.0  \\
            0.19  0.0  \\
            0.2  0.0  \\
            0.21  0.0  \\
            0.22  0.0  \\
            0.23  0.0  \\
            0.24  0.0  \\
            0.25  0.0  \\
            0.26  0.0  \\
            0.27  0.0  \\
            0.28  0.0  \\
            0.29  0.0  \\
            0.3  0.0  \\
            0.31  0.0  \\
            0.32  0.0  \\
            0.33  0.0  \\
            0.34  0.0  \\
            0.35  0.0  \\
            0.36  0.0  \\
            0.37  0.0  \\
            0.38  0.0  \\
            0.39  0.0  \\
            0.4  0.0  \\
            0.41  0.0  \\
            0.42  0.0  \\
            0.43  0.0  \\
            0.44  0.0  \\
            0.45  0.0  \\
            0.46  0.0  \\
            0.47  0.0  \\
            0.48  0.0  \\
            0.49  0.0  \\
            0.5  0.0  \\
            0.51  0.0  \\
            0.52  0.0  \\
            0.53  0.0  \\
            0.54  0.0  \\
            0.55  0.0  \\
            0.56  0.0  \\
            0.57  0.0  \\
            0.58  0.0  \\
            0.59  0.0  \\
            0.6  0.0  \\
            0.61  0.0  \\
            0.62  0.0  \\
            0.63  0.0  \\
            0.64  0.0  \\
            0.65  0.0  \\
            0.66  0.0  \\
            0.67  0.0  \\
            0.68  0.0  \\
            0.69  0.0  \\
            0.7  0.0  \\
            0.71  0.0  \\
            0.72  0.0  \\
            0.73  0.0  \\
            0.74  0.0  \\
            0.75  0.0  \\
            0.76  0.0  \\
            0.77  0.0  \\
            0.78  0.0  \\
            0.79  0.0  \\
            0.8  0.0  \\
            0.81  0.0  \\
            0.82  0.0  \\
            0.83  0.0  \\
            0.84  0.0  \\
            0.85  0.0  \\
            0.86  0.0  \\
            0.87  0.0  \\
            0.88  0.0  \\
            0.89  0.0  \\
            0.9  0.0  \\
            0.91  0.0  \\
            0.92  0.0  \\
            0.93  0.0  \\
            0.94  0.0  \\
            0.95  0.0  \\
            0.96  0.0  \\
            0.97  0.0  \\
            0.98  0.0  \\
            0.99  0.0006  \\
            1.0  1.0  \\
        }
        ;
    \addplot+[dotted]
        coordinates {
            (0,0)
            (1,1)
        }
        ;
    \nextgroupplot[xmin={0}, xmax={1}, ymin={0}]
    \addplot+[]
        coordinates {
            (0.0,1.0)
            (0.01,1.0)
            (0.02,1.0)
            (0.03,1.0)
            (0.04,1.0)
            (0.05,1.0)
            (0.06,1.0)
            (0.07,1.0)
            (0.08,1.0)
            (0.09,1.0)
            (0.1,1.0)
            (0.11,1.0)
            (0.12,1.0)
            (0.13,1.0)
            (0.14,1.0)
            (0.15,1.0)
            (0.16,1.0)
            (0.17,1.0)
            (0.18,1.0)
            (0.19,1.0)
            (0.2,1.0)
            (0.21,1.0)
            (0.22,1.0)
            (0.23,1.0)
            (0.24,1.0)
            (0.25,1.0)
            (0.26,1.0)
            (0.27,1.0)
            (0.28,1.0)
            (0.29,1.0)
            (0.3,1.0)
            (0.31,1.0)
            (0.32,1.0)
            (0.33,1.0)
            (0.34,1.0)
            (0.35,1.0)
            (0.36,1.0)
            (0.37,1.0)
            (0.38,1.0)
            (0.39,1.0)
            (0.4,1.0)
            (0.41,1.0)
            (0.42,1.0)
            (0.43,1.0)
            (0.44,1.0)
            (0.45,1.0)
            (0.46,1.0)
            (0.47,1.0)
            (0.48,1.0)
            (0.49,1.0)
            (0.5,1.0)
            (0.51,1.0)
            (0.52,1.0)
            (0.53,1.0)
            (0.54,1.0)
            (0.55,1.0)
            (0.56,1.0)
            (0.57,1.0)
            (0.58,1.0)
            (0.59,1.0)
            (0.6,1.0)
            (0.61,0.9999)
            (0.62,0.9998)
            (0.63,0.9997)
            (0.64,0.9993)
            (0.65,0.9992)
            (0.66,0.9987)
            (0.67,0.9984)
            (0.68,0.9972)
            (0.69,0.9962)
            (0.7,0.9949)
            (0.71,0.9921)
            (0.72,0.9881)
            (0.73,0.9838)
            (0.74,0.9784)
            (0.75,0.9707)
            (0.76,0.9607)
            (0.77,0.946)
            (0.78,0.9266)
            (0.79,0.9039)
            (0.8,0.8749)
            (0.81,0.842)
            (0.82,0.7997)
            (0.83,0.7465999999999999)
            (0.84,0.6909000000000001)
            (0.85,0.6224000000000001)
            (0.86,0.5509)
            (0.87,0.4729)
            (0.88,0.3944)
            (0.89,0.31720000000000004)
            (0.9,0.245)
            (0.91,0.17720000000000002)
            (0.92,0.12039999999999995)
            (0.93,0.07379999999999998)
            (0.94,0.042100000000000026)
            (0.95,0.01970000000000005)
            (0.96,0.007099999999999995)
            (0.97,0.0031999999999999806)
            (0.98,0.0007000000000000339)
            (0.99,0.0)
            (1.0,0.0)
        }
        ;
    \nextgroupplot[xmin={0}, xmax={1}, ymin={0}]
    \addplot+[]
        coordinates {
            (0.0,1.0)
            (0.01,1.0)
            (0.02,1.0)
            (0.03,1.0)
            (0.04,1.0)
            (0.05,1.0)
            (0.06,1.0)
            (0.07,1.0)
            (0.08,1.0)
            (0.09,1.0)
            (0.1,1.0)
            (0.11,1.0)
            (0.12,1.0)
            (0.13,1.0)
            (0.14,1.0)
            (0.15,1.0)
            (0.16,1.0)
            (0.17,1.0)
            (0.18,1.0)
            (0.19,1.0)
            (0.2,1.0)
            (0.21,1.0)
            (0.22,1.0)
            (0.23,1.0)
            (0.24,1.0)
            (0.25,1.0)
            (0.26,1.0)
            (0.27,1.0)
            (0.28,1.0)
            (0.29,1.0)
            (0.3,1.0)
            (0.31,1.0)
            (0.32,1.0)
            (0.33,1.0)
            (0.34,1.0)
            (0.35,1.0)
            (0.36,1.0)
            (0.37,1.0)
            (0.38,1.0)
            (0.39,1.0)
            (0.4,1.0)
            (0.41,1.0)
            (0.42,1.0)
            (0.43,1.0)
            (0.44,1.0)
            (0.45,1.0)
            (0.46,1.0)
            (0.47,1.0)
            (0.48,1.0)
            (0.49,1.0)
            (0.5,1.0)
            (0.51,1.0)
            (0.52,1.0)
            (0.53,1.0)
            (0.54,1.0)
            (0.55,1.0)
            (0.56,1.0)
            (0.57,1.0)
            (0.58,1.0)
            (0.59,1.0)
            (0.6,1.0)
            (0.61,1.0)
            (0.62,1.0)
            (0.63,1.0)
            (0.64,1.0)
            (0.65,1.0)
            (0.66,1.0)
            (0.67,1.0)
            (0.68,1.0)
            (0.69,1.0)
            (0.7,1.0)
            (0.71,1.0)
            (0.72,1.0)
            (0.73,1.0)
            (0.74,1.0)
            (0.75,1.0)
            (0.76,1.0)
            (0.77,1.0)
            (0.78,1.0)
            (0.79,1.0)
            (0.8,1.0)
            (0.81,1.0)
            (0.82,1.0)
            (0.83,1.0)
            (0.84,1.0)
            (0.85,1.0)
            (0.86,1.0)
            (0.87,1.0)
            (0.88,1.0)
            (0.89,0.9998)
            (0.9,0.9996)
            (0.91,0.9992)
            (0.92,0.9987)
            (0.93,0.9976)
            (0.94,0.9945)
            (0.95,0.9873)
            (0.96,0.9723)
            (0.97,0.9391)
            (0.98,0.8703)
            (0.99,0.7067)
            (1.0,0.0)
        }
        ;
    \nextgroupplot[xmin={0}, xmax={1}, ymin={0}, ylabel={$\asymplinearskce$}]
    \addplot+[]
        table[row sep={\\}]
        {
            \\
            0.0  0.0  \\
            0.01  0.0077  \\
            0.02  0.0162  \\
            0.03  0.0252  \\
            0.04  0.0353  \\
            0.05  0.0455  \\
            0.06  0.056  \\
            0.07  0.0656  \\
            0.08  0.0773  \\
            0.09  0.0868  \\
            0.1  0.098  \\
            0.11  0.1097  \\
            0.12  0.1186  \\
            0.13  0.1275  \\
            0.14  0.1394  \\
            0.15  0.1491  \\
            0.16  0.1598  \\
            0.17  0.1695  \\
            0.18  0.1798  \\
            0.19  0.191  \\
            0.2  0.2014  \\
            0.21  0.2126  \\
            0.22  0.2227  \\
            0.23  0.2327  \\
            0.24  0.243  \\
            0.25  0.2564  \\
            0.26  0.267  \\
            0.27  0.2759  \\
            0.28  0.2862  \\
            0.29  0.2973  \\
            0.3  0.3076  \\
            0.31  0.3182  \\
            0.32  0.3307  \\
            0.33  0.3402  \\
            0.34  0.3508  \\
            0.35  0.3607  \\
            0.36  0.3718  \\
            0.37  0.3801  \\
            0.38  0.3921  \\
            0.39  0.4033  \\
            0.4  0.4141  \\
            0.41  0.4244  \\
            0.42  0.4334  \\
            0.43  0.4441  \\
            0.44  0.4545  \\
            0.45  0.4647  \\
            0.46  0.4737  \\
            0.47  0.4836  \\
            0.48  0.4943  \\
            0.49  0.5037  \\
            0.5  0.5154  \\
            0.51  0.5236  \\
            0.52  0.5318  \\
            0.53  0.541  \\
            0.54  0.5513  \\
            0.55  0.5606  \\
            0.56  0.5707  \\
            0.57  0.5802  \\
            0.58  0.5899  \\
            0.59  0.5983  \\
            0.6  0.6075  \\
            0.61  0.616  \\
            0.62  0.625  \\
            0.63  0.6337  \\
            0.64  0.6431  \\
            0.65  0.6511  \\
            0.66  0.6614  \\
            0.67  0.6705  \\
            0.68  0.6794  \\
            0.69  0.6871  \\
            0.7  0.6957  \\
            0.71  0.7055  \\
            0.72  0.7152  \\
            0.73  0.7262  \\
            0.74  0.7358  \\
            0.75  0.7433  \\
            0.76  0.7521  \\
            0.77  0.7631  \\
            0.78  0.7724  \\
            0.79  0.781  \\
            0.8  0.7894  \\
            0.81  0.8002  \\
            0.82  0.81  \\
            0.83  0.8193  \\
            0.84  0.8282  \\
            0.85  0.8371  \\
            0.86  0.8452  \\
            0.87  0.8565  \\
            0.88  0.8678  \\
            0.89  0.8781  \\
            0.9  0.8906  \\
            0.91  0.9011  \\
            0.92  0.9121  \\
            0.93  0.9224  \\
            0.94  0.9324  \\
            0.95  0.9425  \\
            0.96  0.9523  \\
            0.97  0.9655  \\
            0.98  0.9774  \\
            0.99  0.9887  \\
            1.0  1.0  \\
        }
        ;
    \addplot+[dotted]
        coordinates {
            (0,0)
            (1,1)
        }
        ;
    \nextgroupplot[xmin={0}, xmax={1}, ymin={0}]
    \addplot+[]
        coordinates {
            (0.0,1.0)
            (0.01,0.0008000000000000229)
            (0.02,9.999999999998899e-5)
            (0.03,9.999999999998899e-5)
            (0.04,0.0)
            (0.05,0.0)
            (0.06,0.0)
            (0.07,0.0)
            (0.08,0.0)
            (0.09,0.0)
            (0.1,0.0)
            (0.11,0.0)
            (0.12,0.0)
            (0.13,0.0)
            (0.14,0.0)
            (0.15,0.0)
            (0.16,0.0)
            (0.17,0.0)
            (0.18,0.0)
            (0.19,0.0)
            (0.2,0.0)
            (0.21,0.0)
            (0.22,0.0)
            (0.23,0.0)
            (0.24,0.0)
            (0.25,0.0)
            (0.26,0.0)
            (0.27,0.0)
            (0.28,0.0)
            (0.29,0.0)
            (0.3,0.0)
            (0.31,0.0)
            (0.32,0.0)
            (0.33,0.0)
            (0.34,0.0)
            (0.35,0.0)
            (0.36,0.0)
            (0.37,0.0)
            (0.38,0.0)
            (0.39,0.0)
            (0.4,0.0)
            (0.41,0.0)
            (0.42,0.0)
            (0.43,0.0)
            (0.44,0.0)
            (0.45,0.0)
            (0.46,0.0)
            (0.47,0.0)
            (0.48,0.0)
            (0.49,0.0)
            (0.5,0.0)
            (0.51,0.0)
            (0.52,0.0)
            (0.53,0.0)
            (0.54,0.0)
            (0.55,0.0)
            (0.56,0.0)
            (0.57,0.0)
            (0.58,0.0)
            (0.59,0.0)
            (0.6,0.0)
            (0.61,0.0)
            (0.62,0.0)
            (0.63,0.0)
            (0.64,0.0)
            (0.65,0.0)
            (0.66,0.0)
            (0.67,0.0)
            (0.68,0.0)
            (0.69,0.0)
            (0.7,0.0)
            (0.71,0.0)
            (0.72,0.0)
            (0.73,0.0)
            (0.74,0.0)
            (0.75,0.0)
            (0.76,0.0)
            (0.77,0.0)
            (0.78,0.0)
            (0.79,0.0)
            (0.8,0.0)
            (0.81,0.0)
            (0.82,0.0)
            (0.83,0.0)
            (0.84,0.0)
            (0.85,0.0)
            (0.86,0.0)
            (0.87,0.0)
            (0.88,0.0)
            (0.89,0.0)
            (0.9,0.0)
            (0.91,0.0)
            (0.92,0.0)
            (0.93,0.0)
            (0.94,0.0)
            (0.95,0.0)
            (0.96,0.0)
            (0.97,0.0)
            (0.98,0.0)
            (0.99,0.0)
            (1.0,0.0)
        }
        ;
    \nextgroupplot[xmin={0}, xmax={1}, ymin={0}]
    \addplot+[]
        coordinates {
            (0.0,1.0)
            (0.01,0.9524)
            (0.02,0.9113)
            (0.03,0.8746)
            (0.04,0.8407)
            (0.05,0.8086)
            (0.06,0.7818)
            (0.07,0.7569)
            (0.08,0.7298)
            (0.09,0.708)
            (0.1,0.6831)
            (0.11,0.6608)
            (0.12,0.6407)
            (0.13,0.6185)
            (0.14,0.6009)
            (0.15,0.5842)
            (0.16,0.567)
            (0.17,0.5481)
            (0.18,0.5327999999999999)
            (0.19,0.5188999999999999)
            (0.2,0.5046999999999999)
            (0.21,0.4908)
            (0.22,0.4777)
            (0.23,0.4666)
            (0.24,0.4554)
            (0.25,0.4437)
            (0.26,0.43079999999999996)
            (0.27,0.41690000000000005)
            (0.28,0.40700000000000003)
            (0.29,0.39570000000000005)
            (0.3,0.38570000000000004)
            (0.31,0.3732)
            (0.32,0.3619)
            (0.33,0.3507)
            (0.34,0.34040000000000004)
            (0.35,0.33130000000000004)
            (0.36,0.3206)
            (0.37,0.31220000000000003)
            (0.38,0.30269999999999997)
            (0.39,0.2914)
            (0.4,0.2833)
            (0.41,0.275)
            (0.42,0.2653)
            (0.43,0.25680000000000003)
            (0.44,0.24950000000000006)
            (0.45,0.24229999999999996)
            (0.46,0.23399999999999999)
            (0.47,0.22560000000000002)
            (0.48,0.21889999999999998)
            (0.49,0.21450000000000002)
            (0.5,0.20799999999999996)
            (0.51,0.20120000000000005)
            (0.52,0.1945)
            (0.53,0.18769999999999998)
            (0.54,0.18069999999999997)
            (0.55,0.17420000000000002)
            (0.56,0.1674)
            (0.57,0.1623)
            (0.58,0.15690000000000004)
            (0.59,0.15159999999999996)
            (0.6,0.14559999999999995)
            (0.61,0.13980000000000004)
            (0.62,0.13490000000000002)
            (0.63,0.13019999999999998)
            (0.64,0.1251)
            (0.65,0.12)
            (0.66,0.11470000000000002)
            (0.67,0.11019999999999996)
            (0.68,0.10719999999999996)
            (0.69,0.10270000000000001)
            (0.7,0.09930000000000005)
            (0.71,0.0948)
            (0.72,0.09089999999999998)
            (0.73,0.08530000000000004)
            (0.74,0.08179999999999998)
            (0.75,0.07709999999999995)
            (0.76,0.07289999999999996)
            (0.77,0.06910000000000005)
            (0.78,0.06559999999999999)
            (0.79,0.060799999999999965)
            (0.8,0.05669999999999997)
            (0.81,0.05259999999999998)
            (0.82,0.04930000000000001)
            (0.83,0.046499999999999986)
            (0.84,0.04310000000000003)
            (0.85,0.03990000000000005)
            (0.86,0.03710000000000002)
            (0.87,0.03400000000000003)
            (0.88,0.03059999999999996)
            (0.89,0.027100000000000013)
            (0.9,0.024399999999999977)
            (0.91,0.022299999999999986)
            (0.92,0.01880000000000004)
            (0.93,0.016199999999999992)
            (0.94,0.0131)
            (0.95,0.01100000000000001)
            (0.96,0.007900000000000018)
            (0.97,0.005600000000000049)
            (0.98,0.0030000000000000027)
            (0.99,0.0019000000000000128)
            (1.0,0.0)
        }
        ;
    \nextgroupplot[xmin={0}, xmax={1}, ymin={0}, ylabel={$\asympunbiasedskce$}]
    \addplot+[]
        table[row sep={\\}]
        {
            \\
            0.0  0.0  \\
            0.01  0.006  \\
            0.02  0.008  \\
            0.03  0.024  \\
            0.04  0.03  \\
            0.05  0.038  \\
            0.06  0.046  \\
            0.07  0.054  \\
            0.08  0.062  \\
            0.09  0.07  \\
            0.1  0.074  \\
            0.11  0.082  \\
            0.12  0.098  \\
            0.13  0.112  \\
            0.14  0.122  \\
            0.15  0.136  \\
            0.16  0.146  \\
            0.17  0.158  \\
            0.18  0.17  \\
            0.19  0.174  \\
            0.2  0.178  \\
            0.21  0.19  \\
            0.22  0.194  \\
            0.23  0.208  \\
            0.24  0.22  \\
            0.25  0.23  \\
            0.26  0.242  \\
            0.27  0.252  \\
            0.28  0.262  \\
            0.29  0.272  \\
            0.3  0.278  \\
            0.31  0.286  \\
            0.32  0.298  \\
            0.33  0.312  \\
            0.34  0.32  \\
            0.35  0.328  \\
            0.36  0.336  \\
            0.37  0.348  \\
            0.38  0.36  \\
            0.39  0.37  \\
            0.4  0.378  \\
            0.41  0.382  \\
            0.42  0.398  \\
            0.43  0.406  \\
            0.44  0.418  \\
            0.45  0.422  \\
            0.46  0.43  \\
            0.47  0.438  \\
            0.48  0.452  \\
            0.49  0.46  \\
            0.5  0.47  \\
            0.51  0.482  \\
            0.52  0.502  \\
            0.53  0.514  \\
            0.54  0.53  \\
            0.55  0.546  \\
            0.56  0.56  \\
            0.57  0.578  \\
            0.58  0.588  \\
            0.59  0.598  \\
            0.6  0.614  \\
            0.61  0.628  \\
            0.62  0.638  \\
            0.63  0.654  \\
            0.64  0.672  \\
            0.65  0.678  \\
            0.66  0.684  \\
            0.67  0.694  \\
            0.68  0.704  \\
            0.69  0.714  \\
            0.7  0.732  \\
            0.71  0.744  \\
            0.72  0.762  \\
            0.73  0.77  \\
            0.74  0.778  \\
            0.75  0.796  \\
            0.76  0.8  \\
            0.77  0.816  \\
            0.78  0.826  \\
            0.79  0.834  \\
            0.8  0.84  \\
            0.81  0.848  \\
            0.82  0.858  \\
            0.83  0.866  \\
            0.84  0.876  \\
            0.85  0.884  \\
            0.86  0.896  \\
            0.87  0.904  \\
            0.88  0.91  \\
            0.89  0.918  \\
            0.9  0.936  \\
            0.91  0.948  \\
            0.92  0.954  \\
            0.93  0.958  \\
            0.94  0.962  \\
            0.95  0.964  \\
            0.96  0.98  \\
            0.97  0.986  \\
            0.98  0.994  \\
            0.99  0.998  \\
            1.0  1.0  \\
        }
        ;
    \addplot+[dotted]
        coordinates {
            (0,0)
            (1,1)
        }
        ;
    \nextgroupplot[xmin={0}, xmax={1}, ymin={0}]
    \addplot+[]
        coordinates {
            (0.0,0.0)
            (0.01,0.0)
            (0.02,0.0)
            (0.03,0.0)
            (0.04,0.0)
            (0.05,0.0)
            (0.06,0.0)
            (0.07,0.0)
            (0.08,0.0)
            (0.09,0.0)
            (0.1,0.0)
            (0.11,0.0)
            (0.12,0.0)
            (0.13,0.0)
            (0.14,0.0)
            (0.15,0.0)
            (0.16,0.0)
            (0.17,0.0)
            (0.18,0.0)
            (0.19,0.0)
            (0.2,0.0)
            (0.21,0.0)
            (0.22,0.0)
            (0.23,0.0)
            (0.24,0.0)
            (0.25,0.0)
            (0.26,0.0)
            (0.27,0.0)
            (0.28,0.0)
            (0.29,0.0)
            (0.3,0.0)
            (0.31,0.0)
            (0.32,0.0)
            (0.33,0.0)
            (0.34,0.0)
            (0.35,0.0)
            (0.36,0.0)
            (0.37,0.0)
            (0.38,0.0)
            (0.39,0.0)
            (0.4,0.0)
            (0.41,0.0)
            (0.42,0.0)
            (0.43,0.0)
            (0.44,0.0)
            (0.45,0.0)
            (0.46,0.0)
            (0.47,0.0)
            (0.48,0.0)
            (0.49,0.0)
            (0.5,0.0)
            (0.51,0.0)
            (0.52,0.0)
            (0.53,0.0)
            (0.54,0.0)
            (0.55,0.0)
            (0.56,0.0)
            (0.57,0.0)
            (0.58,0.0)
            (0.59,0.0)
            (0.6,0.0)
            (0.61,0.0)
            (0.62,0.0)
            (0.63,0.0)
            (0.64,0.0)
            (0.65,0.0)
            (0.66,0.0)
            (0.67,0.0)
            (0.68,0.0)
            (0.69,0.0)
            (0.7,0.0)
            (0.71,0.0)
            (0.72,0.0)
            (0.73,0.0)
            (0.74,0.0)
            (0.75,0.0)
            (0.76,0.0)
            (0.77,0.0)
            (0.78,0.0)
            (0.79,0.0)
            (0.8,0.0)
            (0.81,0.0)
            (0.82,0.0)
            (0.83,0.0)
            (0.84,0.0)
            (0.85,0.0)
            (0.86,0.0)
            (0.87,0.0)
            (0.88,0.0)
            (0.89,0.0)
            (0.9,0.0)
            (0.91,0.0)
            (0.92,0.0)
            (0.93,0.0)
            (0.94,0.0)
            (0.95,0.0)
            (0.96,0.0)
            (0.97,0.0)
            (0.98,0.0)
            (0.99,0.0)
            (1.0,0.0)
        }
        ;
    \nextgroupplot[xmin={0}, xmax={1}, ymin={0}]
    \addplot+[]
        coordinates {
            (0.0,0.0)
            (0.01,0.0)
            (0.02,0.0)
            (0.03,0.0)
            (0.04,0.0)
            (0.05,0.0)
            (0.06,0.0)
            (0.07,0.0)
            (0.08,0.0)
            (0.09,0.0)
            (0.1,0.0)
            (0.11,0.0)
            (0.12,0.0)
            (0.13,0.0)
            (0.14,0.0)
            (0.15,0.0)
            (0.16,0.0)
            (0.17,0.0)
            (0.18,0.0)
            (0.19,0.0)
            (0.2,0.0)
            (0.21,0.0)
            (0.22,0.0)
            (0.23,0.0)
            (0.24,0.0)
            (0.25,0.0)
            (0.26,0.0)
            (0.27,0.0)
            (0.28,0.0)
            (0.29,0.0)
            (0.3,0.0)
            (0.31,0.0)
            (0.32,0.0)
            (0.33,0.0)
            (0.34,0.0)
            (0.35,0.0)
            (0.36,0.0)
            (0.37,0.0)
            (0.38,0.0)
            (0.39,0.0)
            (0.4,0.0)
            (0.41,0.0)
            (0.42,0.0)
            (0.43,0.0)
            (0.44,0.0)
            (0.45,0.0)
            (0.46,0.0)
            (0.47,0.0)
            (0.48,0.0)
            (0.49,0.0)
            (0.5,0.0)
            (0.51,0.0)
            (0.52,0.0)
            (0.53,0.0)
            (0.54,0.0)
            (0.55,0.0)
            (0.56,0.0)
            (0.57,0.0)
            (0.58,0.0)
            (0.59,0.0)
            (0.6,0.0)
            (0.61,0.0)
            (0.62,0.0)
            (0.63,0.0)
            (0.64,0.0)
            (0.65,0.0)
            (0.66,0.0)
            (0.67,0.0)
            (0.68,0.0)
            (0.69,0.0)
            (0.7,0.0)
            (0.71,0.0)
            (0.72,0.0)
            (0.73,0.0)
            (0.74,0.0)
            (0.75,0.0)
            (0.76,0.0)
            (0.77,0.0)
            (0.78,0.0)
            (0.79,0.0)
            (0.8,0.0)
            (0.81,0.0)
            (0.82,0.0)
            (0.83,0.0)
            (0.84,0.0)
            (0.85,0.0)
            (0.86,0.0)
            (0.87,0.0)
            (0.88,0.0)
            (0.89,0.0)
            (0.9,0.0)
            (0.91,0.0)
            (0.92,0.0)
            (0.93,0.0)
            (0.94,0.0)
            (0.95,0.0)
            (0.96,0.0)
            (0.97,0.0)
            (0.98,0.0)
            (0.99,0.0)
            (1.0,0.0)
        }
        ;
\end{groupplot}
\node[anchor=north] at ($(group c1r6.west |- group c1r6.outer south)!0.5!(group c3r6.east |- group c3r6.outer south)$){bound/approximation of probability of false rejection};
\node[anchor=south, rotate=90] at ($(group c1r1.north -| group c1r1.outer west)!0.5!(group c1r6.south -| group c1r6.outer west)$){test error};
\end{tikzpicture}

      \captionof{figure}{Test errors versus bounds/approximations of the probability of false rejection, evaluated on $500$ ($\asympunbiasedskce$) and $10^4$ (all other test statistics) randomly sampled data sets. For model \textbf{M1} the type I error is shown, for both uncalibrated models the type II error is plotted.}
    \end{center}
  }
\end{tcbposter}
\end{document}
