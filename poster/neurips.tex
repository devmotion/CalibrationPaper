% arara: lualatex: { shell: true }
% arara: lualatex: { shell: true }
% arara: lualatex: { shell: true, synctex: true }

% A2  : 420 x 594 mm    |
% 2A0 : 1189 x 1682 mm  > Factor 2.83 => 11pt ~ 31 pt
% 1m  : 1000 x 1414     > Factor 2.38 => 12pt ~ 28 pt

% A3  : 297 x 420 mm    |
% 2A0 : 1189 x 1682 mm  > Factor 4 => 11pt ~ 44 pt
% 1m  : 1000 x 1414     > Factor 3.36 => 10pt ~ 34 pt
\documentclass[10pt]{article}

\usepackage{luatex85}

% layout
\usepackage[a3paper,landscape]{geometry}

% math support
\usepackage{mathtools,amssymb}

% fonts
\RequirePackage[factor=0]{microtype} % no protrusion
\usepackage{unicode-math}
\defaultfontfeatures{Ligatures=TeX}
\IfFileExists{fonts/Berling.otf}{%
  % load fonts of official UU design
  \setmainfont{Berling}[%
  Path=./fonts/,
  Extension=.otf,
  BoldFont=*-Bold,
  ItalicFont=*-Italic,
  BoldItalicFont=*-BoldItalic]
}{%
  \setmainfont{Libertinus Serif}
}
\setsansfont{Libertinus Sans}
\setmonofont{Libertinus Mono}
\setmathfont{Libertinus Math}

\usepackage{bm}

% language support
\usepackage{polyglossia}
\setdefaultlanguage{english}
\usepackage{csquotes}

% better looking tables
\usepackage{booktabs}

% colors
\usepackage[CMYK]{xcolor}
\usepackage{UUcolorPantone}

\newcommand{\hl}[1]{\begingroup\bfseries\boldmath\color{uured}#1\endgroup}

% graphics
\usepackage{graphicx}
\usepackage{svg}
\svgpath{{./figures/}}

% captions
\usepackage{caption,subcaption}
\captionsetup{font=scriptsize}

% fancy lists
\usepackage{enumitem}
\setlist{leftmargin=*,itemsep=0pt}
\setlist[itemize,1]{label={\color{uured}$\blacktriangleright$}}

% hyperlinks
\usepackage{hyperref}

% boxes
\usepackage[poster,xparse,raster]{tcolorbox}

% poster settings
\tcbposterset{
  coverage =
  {
    spread,
    interior style={color=white},
  },
  poster =
  {
    columns=10,
    rows=1,
    showframe=true, % useful for debugging
  },
  boxes =
  {
    enhanced standard jigsaw,
    sharp corners=downhill,
    arc=3pt,
    boxrule=1pt,
    lower separated=false,
    % colors
    coltext=black,
    colback=white,
    colframe=black,
    coltitle=black,
    colbacktitle=uulightgrey,
    % fonts
    fonttitle=\bfseries\large,
    % subtitles
    subtitle style=
    {
      frame empty,
      hbox,
      rounded corners=east,
      arc=8pt,
      coltext=white!50!uulightgrey,
      colback=black!10!uudarkgrey,
    },
  }
}

% plots
\usepackage{pgfplots,pgfplotstable}
\pgfplotsset{compat=1.16}
\usetikzlibrary{positioning,arrows,arrows.meta,calc,decorations.markings,intersections,patterns}

% general settings for plots
\pgfplotsset{grid style=dashed}
\pgfplotsset{enlargelimits=auto}

\usepgfplotslibrary{groupplots,fillbetween,colorbrewer}
\usetikzlibrary{plotmarks,calc}

% plotting options
\pgfplotsset{table/search path={data/}}
\pgfplotsset{max space between ticks=150}
\pgfplotsset{every axis/.append style={axis background style={fill=gray!10}},tick label style={font={\footnotesize}}, label style={font={\small}}}
\pgfplotsset{every axis plot/.append style={thick}}
\pgfplotsset{every axis legend/.append style={font=\small, fill=none}}

% automatic references
\usepackage{cleveref}

% some abbreviations
\newcommand*{\Prob}{\mathbb{P}}
\newcommand*{\E}{\mathbb{E}}
\newcommand*{\transpose}[1]{{#1}^{\mathsf{T}}}
\DeclareMathOperator*{\argmax}{arg\,max}
\DeclareMathOperator{\ECE}{ECE}
\DeclareMathOperator{\biasedskce}{SKCE_b}
\DeclareMathOperator{\unbiasedskce}{SKCE_{uq}}
\DeclareMathOperator{\linearskce}{SKCE_{ul}}
\DeclareMathOperator{\asympunbiasedskce}{aSKCE_{uq}}
\DeclareMathOperator{\asymplinearskce}{aSKCE_{ul}}
\DeclareMathOperator{\measure}{CE}
\DeclareMathOperator{\kernelmeasure}{KCE}
\DeclareMathOperator{\squaredkernelmeasure}{SKCE}
\DeclareMathOperator{\Expect}{\mathbb{E}}
\DeclareMathOperator{\Dir}{Dir}
\DeclareMathOperator{\Categorical}{Cat}

% metadata
\title{Calibration tests in multi-class classification:\\ A unifying framework}
\author{David Widmann$^\star$ Fredrik Lindsten$^\ddagger$ Dave Zachariah$^\star$}
\date{\today}
\makeatletter
\pgfkeys{%
  /my poster/.cd,
  title/.initial=\@title,
  author/.initial=\@author,
  institute/.initial={},
  contact/.initial={},
  date/.initial=\@date,
}
\makeatother

\pgfkeys{%
  /my poster/.cd,
  institute={$^\star$Department of Information Technology, Uppsala University $^\ddagger$Division of Statistics and Machine Learning, Linköping University},
  contact={david.widmann@it.uu.se fredrik.lindsten@liu.se dave.zachariah@it.uu.se},
}

\pagestyle{empty}

\begin{document}
\begin{tcbposter}

  % title
  \posterbox[blankest,interior engine=path,halign=left,valign=center,right=4cm,
  underlay =
  {%
    \node[left,inner sep=0pt,outer sep=0pt,align=center] at (frame.east) {\includegraphics[width=2cm]{figures/logos/UU.pdf}\\[1ex]\includegraphics[width=3cm]{figures/logos/LiU.pdf}};%
  }]{name=title,column=1,span=6,below=top}{%
    \Huge\textbf{\pgfkeysvalueof{/my poster/title}}\\[1ex]
    \large\pgfkeysvalueof{/my poster/author}\\[1ex]
    \normalsize\pgfkeysvalueof{/my poster/institute}%
  }%

  % footline
  \posterbox[blankest,top=2pt,bottom=2pt,valign=center,fontupper=\ttfamily\small,interior engine=path,interior style={color=uumidgrey}%
  ]{name=footline,column=1,span=10,above=bottom}{%
    \pgfkeysvalueof{/my poster/date}\hfill\pgfkeysvalueof{/my poster/contact}%
  }%

  \posterbox[adjusted title={Motivation - what is a calibrated model?}, colback=blondsvag]{name=calibration,column=3,span=4,below=title}{
    \begin{tcolorbox}[colback=blondstark]
      \begin{center}
        A \hl{calibrated model} yields predictions consistent with empirically observed frequencies.
      \end{center}
    \end{tcolorbox}

    \tcbsubtitle{Collision detection system}

    Consider a model that predicts if there is an object, a human, or an animal ahead of a car.

    \begin{minipage}[c]{0.6\linewidth}
        \begin{center}
          \begin{tikzpicture}
            \node[draw, inner sep=2mm] (image) at (0, 0) {\includesvg[height=8mm]{car}};
            \node[above=2mm of image, anchor=base, font=\scriptsize] {Input $X$};

            \node[draw, fill=gronskasvag, right=0.75cm of image, inner sep=2mm] (model)
            {\includesvg[height=8mm]{gear}};
            \node[above=2mm of model, anchor=base, font=\scriptsize] {Model $g$};
            \draw [->] (image) -- (model);

            \node[draw, right=0.75cm of model, minimum height=1.2cm, font=\scriptsize, align=center] (prediction)
            {\begin{tabular}{@{}ccc@{}}
               \includesvg[width=6mm]{barrier} & \includesvg[width=6mm]{pedestrian} & \includesvg[width=6mm]{bear} \\
               80\% & 0\% & 20\% \\
             \end{tabular}};
           \node[above=2mm of prediction, anchor=base, font=\scriptsize] {Prediction $g(X) \in \Delta^m$};
           \draw [->] (model) -- (prediction);
         \end{tikzpicture}
       \end{center}
     \end{minipage}%
     \begin{minipage}[c]{0.4\linewidth}
       We use $m$ for the number of classes, and
       $\Delta^m \coloneqq \{ z \in [0,1]^m \colon \|z\|_1 = 1\}$ for the
       $(m-1)$-dimensional probability simplex.
     \end{minipage}\vspace*{\baselineskip}

     If the model is calibrated we know that for all inputs with this
     prediction there is an object ahead 80\% of the time, a human 0\%
     of the time, and an animal 20\% of the time.

     \begin{center}
       \begin{tikzpicture}
         \node[minimum height=1.2cm, inner sep=2mm] (image) at (0, 0)
         {\begin{tabular}{@{}ccc@{}}
            \includesvg[height=3mm]{car0} & \includesvg[height=3mm]{car1} & \includesvg[height=3mm]{car2} \\
            \includesvg[height=3mm]{car3} & \includesvg[height=3mm]{car4} & $\cdots$ \\
          \end{tabular}};

        \node[draw, fill=gronskasvag, right=0.75cm of image, inner sep=2mm] (model)
        {\includesvg[height=8mm]{gear}};
        \draw [->] (image) -- (model);

        \node[draw, right=0.75cm of model, minimum height=1.2cm, font=\scriptsize, align=center] (prediction)
        {\begin{tabular}{@{}ccc@{}}
           \includesvg[width=6mm]{barrier} & \includesvg[width=6mm]{pedestrian} & \includesvg[width=6mm]{bear} \\
           80\% & 0\% & 20\% \\
         \end{tabular}};
        \draw [->] (model) -- (prediction);

        \node[right=1cm of prediction] (empirical)
        {\begin{tabular}{@{}cccccc@{}} \toprule
           \multicolumn{4}{c}{\includesvg[width=3mm]{barrier}} & \includesvg[width=3mm]{pedestrian} & \includesvg[width=3mm]{bear} \\ \midrule
           \includesvg[height=3mm]{car0} & \includesvg[height=3mm]{car2} & \includesvg[height=3mm]{car3} & \includesvg[height=3mm]{car4} & & \includesvg[height=3mm]{car1} \\
           $\vdots$ & $\vdots$ & $\vdots$ & $\vdots$ & & $\vdots$ \\ \bottomrule
         \end{tabular}};
        \node[above=2mm of empirical, anchor=base, font=\scriptsize] (A) {Empirical frequency $r(g(X)) \in \Delta^m$};
        \node[font=\scriptsize] at (prediction |- A) {Prediction $g(X) \in \Delta^m$};

        \path (prediction) -- node [font=\boldmath\Huge, color=uured, align=center, midway] {$=$} (empirical);
      \end{tikzpicture}
    \end{center}
}

  \posterbox[adjusted title={Quantifying calibration - a unifying framework}, colback=gryningsvag]{name=error,column=1,span=3,between=calibration and footline}{
    \begin{tcolorbox}[colback=blondstark]
      We define the \hl{calibration error}~($\measure$) of model $g$ with respect to a class $\mathcal{F}$ of functions $f \colon \Delta^m \to \mathbb{R}^m$ as
      \begin{equation*}
        \measure[\mathcal{F}, g] \coloneqq \sup_{f \in \mathcal{F}} \Expect\left[\transpose{(r(g(X)) - g(X))} f(g(X)) \right].
      \end{equation*}
    \end{tcolorbox}

    By design, if model $g$ is calibrated then the $\measure$ is zero, regardless of $\mathcal{F}$.

    \tcbsubtitle{Kernel calibration error}

    \begin{tcolorbox}[colback=blondstark]
      We define the \hl{kernel calibration error} ($\kernelmeasure$)
      of model $g$ with respect to a matrix-valued kernel
      $k \colon \Delta^m \times \Delta^m \to \mathbb{R}^{m \times m}$ as
      \begin{equation*}
        \kernelmeasure[k, g] \coloneqq \measure[\mathcal{F}, g],
      \end{equation*}
      where $\mathcal{F}$ is the unit ball in the reproducing kernel
      Hilbert space corresponding to $k$.
    \end{tcolorbox}

    If $k$ is a universal kernel, then the $\kernelmeasure$ is zero if
    and only if model $g$ is calibrated.

    \tcbsubtitle{Relation to existing measures}
    \begin{itemize}
    \item For common distances $d$ the expected calibration error ($\ECE$)
      \begin{equation}\label{eq:ece}
        \ECE[d, g] = \Expect[d(r(g(X)), g(X))]
      \end{equation}
      can be formulated as a $\measure$.

    \item The framework captures the maximum mean calibration error as well.
    \end{itemize}
  }

  \posterbox[adjusted title=The paper in 30 seconds, colback=blondmellan]{name=summary,column=1,span=2,between=title and error}{
    \begin{itemize}
    \item We propose a \hl{unifying framework} of calibration errors
      that allows us to derive a new \hl{kernel calibration error} with
      \hl{unbiased and consistent estimators}.
    \item Calibration error estimates are not interpretable. Instead we
      can conduct hypothesis tests of calibration.
    \item In contrast to existing approaches, the KCE enables
      well-founded bounds and approximations of the p-value for
      calibration tests.
    \end{itemize}

    \tcbsubtitle{Take with you}
    \begin{itemize}
    \item Kernel calibration error (KCE) with unbiased and consistent estimators
    \item Calibration errors have no meaningful unit or scale
    \item Reliable calibrations tests with the KCE
    \end{itemize}
  }

  \posterbox[adjusted title={Estimating the calibration error}, colback=skymningsvag]{name=estimation,column=4,span=3,below=calibration}{
    We want to estimate the $\measure$ of model $g$ using a validation
    data set $\{(X_i, Y_i)\}_{i=1}^n$ of i.i.d.\ pairs of inputs and labels.

    \tcbsubtitle{Kernel calibration error}

    For $i,j \in \{1,\ldots,n\}$, let
    $h_{i,j} \coloneqq \transpose{(e_{Y_i} - g(X_i))} k(g(X_i), g(X_j)) (e_{Y_j} - g(X_j))$,
    where $e_i \in \Delta^m$ denotes the $i$th unit vector.

    \begin{tcolorbox}[colback=blondstark]
      If $\mathbb{E}[\|k(g(X),g(X))\|] < \infty$, then \hl{consistent estimators}
      of the squared kernel calibration error
      $\squaredkernelmeasure[k, g] \coloneqq \kernelmeasure[k,g]^2$ are:
      \begin{center}
        \begin{tabular}{llll} \toprule
          Notation & Definition & Properties & Complexity\\ \midrule
          $\biasedskce$ & $n^{-2} \sum_{i,j=1}^n h_{i,j}$ & biased & $O(n^2)$ \\
          $\unbiasedskce$ & $ {\binom{n}{2}}^{-1} \sum_{1 \leq i < j \leq n} h_{i,j}$ & unbiased & $O(n^2)$ \\
          $\linearskce$ & $ {\lfloor n/2\rfloor}^{-1} \sum_{i = 1}^{\lfloor n / 2\rfloor} h_{2i-1,2i}$ & unbiased & $O(n)$ \\ \bottomrule
        \end{tabular}
      \end{center}
    \end{tcolorbox}

    \tcbsubtitle{Relation to the expected calibration error}

    Standard estimators of the $\ECE$ are usually biased and inconsistent.
    The main difficulty is the estimation of the empirical frequencies
    $r(g(X))$ in \cref{eq:ece}. For the $\kernelmeasure$ there is no need
    to estimate them!
  }

  \posterbox[adjusted title={Example: A simple matrix-valued kernel}]{name=kernel,column=4,span=3,between=estimation and footline}{
    If $\tilde{k} \colon \Delta^m \times \Delta^m \to \mathbb{R}$ is a
    real-valued kernel and $M \in \mathbb{R}^{m \times m}$ is positive semi-definite,
    then $k = M \tilde{k}$ is a matrix-valued kernel.
  }

  \posterbox[adjusted title={Calibration tests - is my model calibrated?}]{name=statistics,column=7,span=4,below=top}{
    In general, the $\measure$ does not have a meaningful unit or scale.
    This renders it difficult to interpret an estimated non-zero error.

    \begin{center}
      \begin{tikzpicture}[
        declare function={normal(\m,\s)=1/(2*\s*sqrt(pi))*exp(-(x-\m)^2/(2*\s^2));},
        declare function={binormal(\ma,\sa,\mb,\sb,\p)=(\p*normal(\ma,\sa)+(1-\p)*normal(\mb,\sb));}
        ]

        \begin{axis}[
          domain = -0.1:0.2,
          no marks,
          xlabel = calibration error estimate,
          ylabel = density,
          grid=major,
          ymin = 0,
          tick label style={font=\tiny},
          label style={font=\small},
          width = 0.73\textwidth,
          height = 0.25\textwidth,
          legend pos=outer north east,
          legend cell align=left,
          legend style=
          {
            fill=none,
            draw=none,
            inner sep={0pt},
            font=\small,
          }
          ]

          \draw [Dark2-A, thick] (0.07,\pgfkeysvalueof{/pgfplots/ymin}) -- (0.07,\pgfkeysvalueof{/pgfplots/ymax}) node [at end, above, anchor=south east, sloped, font=\small] {observed};

          \draw[Dark2-B, thick] (0,\pgfkeysvalueof{/pgfplots/ymin}) -- (0,\pgfkeysvalueof{/pgfplots/ymax}) node [at end, above, anchor=south east, sloped, font=\small] {calibrated};

          % mixture model of normal distributions
          \addplot+ [color=Dark2-B, dashed, thick, samples=31, smooth, name path=A] {binormal(-0.05,0.01,0.05,0.03,0.5)};
          \addlegendentry{distribution under $H_0$};

          % indicate p-value
          \path [name path=B] (\pgfkeysvalueof{/pgfplots/xmin},0) -- (\pgfkeysvalueof{/pgfplots/xmax},0);
          \addplot+ [draw=Dark2-C, pattern color=Dark2-C, pattern={north east lines}] fill between [of=A and B, soft clip={domain=0.07:0.2}];
          \addlegendentry{p-value};

          % add comment
          \node[anchor=west, align=left, text=Dark2-C, font=\small] (annotation) at (0.1, 10) {reject $H_0$ if the \\p-value is small};
          \draw[->, >=stealth, thick, Dark2-C] (annotation) -- (0.08, 1);
        \end{axis}
      \end{tikzpicture}
    \end{center}

    \begin{tcolorbox}[colback=blondstark]
      We derive \hl{well-founded bounds and approximations} of the p-value
      based on the estimators of the $\squaredkernelmeasure$.
    \end{tcolorbox}
  }

  \posterbox[adjusted title={Experiments}]{name=experiment,column=7,span=4,between=statistics and footline}{
    We construct data sets $\{g(X_i), Y_i\}_{i=1}^{250}$ of three models with $10$ classes by sampling predictions $g(X_i) \sim \Dir(0.1, \dots, 0.1)$ and labels $Y_i$ conditionally on $g(X_i)$ from
    \begin{align*}
      \text{\textbf{M1: }} &\Categorical(g(X_i)), &
                                                    \text{\textbf{M2: }} &0.5\Categorical(g(X_i)) + 0.5\Categorical(1,0,\dots,0), &
                                                                                                                                    \text{\textbf{M3: }} &\Categorical(0.1, \dots, 0.1).
    \end{align*}
    Model \textbf{M1} is calibrated, and models \textbf{M2} and \textbf{M3} are uncalibrated.

    \begin{center}
      \begin{tikzpicture}
\begin{groupplot}[group style={group size={3 by 4}, xlabels at={edge bottom}, ylabels at={edge left}, horizontal sep={0.1\linewidth}, vertical sep={0.04\linewidth}}, no markers, tick label style={font={\footnotesize}}, grid={major}, title style={align={center}}, width={0.22\linewidth}, height={0.1\linewidth}, every x tick scale label/.style={at={{(1,0)}}, anchor={west}}, ylabel style={font={\small}}]
    \nextgroupplot[title={\textbf{M1}}, ylabel={$\ECE$}]
    \addplot+[ybar interval, fill={blue!25}]
        table[row sep={\\}]
        {
            \\
            0.17  7.0  \\
            0.18  55.0  \\
            0.19  229.0  \\
            0.2  796.0  \\
            0.21  1782.0  \\
            0.22  2550.0  \\
            0.23  2307.0  \\
            0.24  1389.0  \\
            0.25  667.0  \\
            0.26  171.0  \\
            0.27  42.0  \\
            0.28  5.0  \\
            0.29  0.0  \\
        }
        ;
    \draw[solid, black, thick] (0.22876019647666532,\pgfkeysvalueof{/pgfplots/ymin})--(0.22876019647666532,\pgfkeysvalueof{/pgfplots/ymax});
    \draw[dashed, red, thick] (0.0,\pgfkeysvalueof{/pgfplots/ymin})--(0.0,\pgfkeysvalueof{/pgfplots/ymax});
    \addplot+[draw={none}]
        coordinates {
            (0.0,0)
        }
        ;
    \nextgroupplot[title={\textbf{M2}}]
    \addplot+[ybar interval, fill={blue!25}]
        table[row sep={\\}]
        {
            \\
            0.38  1.0  \\
            0.4  3.0  \\
            0.42  30.0  \\
            0.44  208.0  \\
            0.46  827.0  \\
            0.48  2107.0  \\
            0.5  2889.0  \\
            0.52  2406.0  \\
            0.54  1115.0  \\
            0.56  335.0  \\
            0.58  69.0  \\
            0.6  9.0  \\
            0.62  1.0  \\
            0.64  0.0  \\
        }
        ;
    \draw[solid, black, thick] (0.5130312782216148,\pgfkeysvalueof{/pgfplots/ymin})--(0.5130312782216148,\pgfkeysvalueof{/pgfplots/ymax});
    \draw[dashed, red, thick] (0.45,\pgfkeysvalueof{/pgfplots/ymin})--(0.45,\pgfkeysvalueof{/pgfplots/ymax});
    \addplot+[draw={none}]
        coordinates {
            (0.45,0)
        }
        ;
    \nextgroupplot[title={\textbf{M3}}]
    \addplot+[ybar interval, fill={blue!25}]
        table[row sep={\\}]
        {
            \\
            0.3  8.0  \\
            0.32  154.0  \\
            0.34  821.0  \\
            0.36  2308.0  \\
            0.38  3259.0  \\
            0.4  2318.0  \\
            0.42  928.0  \\
            0.44  180.0  \\
            0.46  23.0  \\
            0.48  1.0  \\
            0.5  0.0  \\
        }
        ;
    \draw[solid, black, thick] (0.3907003650959387,\pgfkeysvalueof{/pgfplots/ymin})--(0.3907003650959387,\pgfkeysvalueof{/pgfplots/ymax});
    \draw[dashed, red, thick] (0.7106418012290426,\pgfkeysvalueof{/pgfplots/ymin})--(0.7106418012290426,\pgfkeysvalueof{/pgfplots/ymax});
    \addplot+[draw={none}]
        coordinates {
            (0.7106418012290426,0)
        }
        ;
    \nextgroupplot[ylabel={$\biasedskce$}]
    \addplot+[ybar interval, fill={blue!25}]
        table[row sep={\\}]
        {
            \\
            0.0005  61.0  \\
            0.001  2724.0  \\
            0.0015  4376.0  \\
            0.002  2111.0  \\
            0.0025  597.0  \\
            0.003  108.0  \\
            0.0035  20.0  \\
            0.004  3.0  \\
            0.0045  0.0  \\
        }
        ;
    \draw[solid, black, thick] (0.001791065804877401,\pgfkeysvalueof{/pgfplots/ymin})--(0.001791065804877401,\pgfkeysvalueof{/pgfplots/ymax});
    \draw[dashed, red, thick] (-8.624308735909018e-6,\pgfkeysvalueof{/pgfplots/ymin})--(-8.624308735909018e-6,\pgfkeysvalueof{/pgfplots/ymax});
    \addplot+[draw={none}]
        coordinates {
            (-8.624308735909018e-6,0)
        }
        ;
    \nextgroupplot[]
    \addplot+[ybar interval, fill={blue!25}]
        table[row sep={\\}]
        {
            \\
            0.05  3.0  \\
            0.06  86.0  \\
            0.07  572.0  \\
            0.08  1791.0  \\
            0.09  2834.0  \\
            0.1  2563.0  \\
            0.11  1424.0  \\
            0.12  567.0  \\
            0.13  133.0  \\
            0.14  21.0  \\
            0.15  5.0  \\
            0.16  1.0  \\
            0.17  0.0  \\
        }
        ;
    \draw[solid, black, thick] (0.09955803307414735,\pgfkeysvalueof{/pgfplots/ymin})--(0.09955803307414735,\pgfkeysvalueof{/pgfplots/ymax});
    \draw[dashed, red, thick] (0.09634393682193111,\pgfkeysvalueof{/pgfplots/ymin})--(0.09634393682193111,\pgfkeysvalueof{/pgfplots/ymax});
    \addplot+[draw={none}]
        coordinates {
            (0.09634393682193111,0)
        }
        ;
    \nextgroupplot[]
    \addplot+[ybar interval, fill={blue!25}]
        table[row sep={\\}]
        {
            \\
            0.015  15.0  \\
            0.016  97.0  \\
            0.017  450.0  \\
            0.018  1267.0  \\
            0.019  2139.0  \\
            0.02  2390.0  \\
            0.021  1782.0  \\
            0.022  1061.0  \\
            0.023  493.0  \\
            0.024  216.0  \\
            0.025  61.0  \\
            0.026  19.0  \\
            0.027  8.0  \\
            0.028  2.0  \\
            0.029  0.0  \\
        }
        ;
    \draw[solid, black, thick] (0.020517506214089484,\pgfkeysvalueof{/pgfplots/ymin})--(0.020517506214089484,\pgfkeysvalueof{/pgfplots/ymax});
    \draw[dashed, red, thick] (0.015176641534493648,\pgfkeysvalueof{/pgfplots/ymin})--(0.015176641534493648,\pgfkeysvalueof{/pgfplots/ymax});
    \addplot+[draw={none}]
        coordinates {
            (0.015176641534493648,0)
        }
        ;
    \nextgroupplot[ylabel={$\unbiasedskce$}]
    \addplot+[ybar interval, fill={blue!25}]
        table[row sep={\\}]
        {
            \\
            -0.001  37.0  \\
            -0.0008  438.0  \\
            -0.0006  1286.0  \\
            -0.0004  1968.0  \\
            -0.0002  1919.0  \\
            0.0  1607.0  \\
            0.0002  1132.0  \\
            0.0004  704.0  \\
            0.0006  423.0  \\
            0.0008  242.0  \\
            0.001  122.0  \\
            0.0012  72.0  \\
            0.0014  22.0  \\
            0.0016  20.0  \\
            0.0018  4.0  \\
            0.002  4.0  \\
            0.0022  0.0  \\
        }
        ;
    \draw[solid, black, thick] (-8.624308735909018e-6,\pgfkeysvalueof{/pgfplots/ymin})--(-8.624308735909018e-6,\pgfkeysvalueof{/pgfplots/ymax});
    \draw[dashed, red, thick] (-8.624308735909018e-6,\pgfkeysvalueof{/pgfplots/ymin})--(-8.624308735909018e-6,\pgfkeysvalueof{/pgfplots/ymax});
    \addplot+[draw={none}]
        coordinates {
            (-8.624308735909018e-6,0)
        }
        ;
    \nextgroupplot[]
    \addplot+[ybar interval, fill={blue!25}]
        table[row sep={\\}]
        {
            \\
            0.04  1.0  \\
            0.05  11.0  \\
            0.06  172.0  \\
            0.07  904.0  \\
            0.08  2195.0  \\
            0.09  2942.0  \\
            0.1  2228.0  \\
            0.11  1080.0  \\
            0.12  371.0  \\
            0.13  84.0  \\
            0.14  9.0  \\
            0.15  2.0  \\
            0.16  1.0  \\
            0.17  0.0  \\
        }
        ;
    \draw[solid, black, thick] (0.09634393682193111,\pgfkeysvalueof{/pgfplots/ymin})--(0.09634393682193111,\pgfkeysvalueof{/pgfplots/ymax});
    \draw[dashed, red, thick] (0.09634393682193111,\pgfkeysvalueof{/pgfplots/ymin})--(0.09634393682193111,\pgfkeysvalueof{/pgfplots/ymax});
    \addplot+[draw={none}]
        coordinates {
            (0.09634393682193111,0)
        }
        ;
    \nextgroupplot[]
    \addplot+[ybar interval, fill={blue!25}]
        table[row sep={\\}]
        {
            \\
            0.01  17.0  \\
            0.011  144.0  \\
            0.012  628.0  \\
            0.013  1606.0  \\
            0.014  2478.0  \\
            0.015  2241.0  \\
            0.016  1566.0  \\
            0.017  810.0  \\
            0.018  323.0  \\
            0.019  137.0  \\
            0.02  33.0  \\
            0.021  14.0  \\
            0.022  2.0  \\
            0.023  1.0  \\
            0.024  0.0  \\
        }
        ;
    \draw[solid, black, thick] (0.015176641534493648,\pgfkeysvalueof{/pgfplots/ymin})--(0.015176641534493648,\pgfkeysvalueof{/pgfplots/ymax});
    \draw[dashed, red, thick] (0.015176641534493648,\pgfkeysvalueof{/pgfplots/ymin})--(0.015176641534493648,\pgfkeysvalueof{/pgfplots/ymax});
    \addplot+[draw={none}]
        coordinates {
            (0.015176641534493648,0)
        }
        ;
    \nextgroupplot[ylabel={$\linearskce$}]
    \addplot+[ybar interval, fill={blue!25}]
        table[row sep={\\}]
        {
            \\
            -0.03  1.0  \\
            -0.025  13.0  \\
            -0.02  117.0  \\
            -0.015  555.0  \\
            -0.01  1580.0  \\
            -0.005  2580.0  \\
            0.0  2825.0  \\
            0.005  1594.0  \\
            0.01  564.0  \\
            0.015  145.0  \\
            0.02  21.0  \\
            0.025  5.0  \\
            0.03  0.0  \\
        }
        ;
    \draw[solid, black, thick] (0.00016041013856191272,\pgfkeysvalueof{/pgfplots/ymin})--(0.00016041013856191272,\pgfkeysvalueof{/pgfplots/ymax});
    \draw[dashed, red, thick] (-8.624308735909018e-6,\pgfkeysvalueof{/pgfplots/ymin})--(-8.624308735909018e-6,\pgfkeysvalueof{/pgfplots/ymax});
    \addplot+[draw={none}]
        coordinates {
            (-8.624308735909018e-6,0)
        }
        ;
    \nextgroupplot[]
    \addplot+[ybar interval, fill={blue!25}]
        table[row sep={\\}]
        {
            \\
            0.02  2.0  \\
            0.03  12.0  \\
            0.04  44.0  \\
            0.05  238.0  \\
            0.06  598.0  \\
            0.07  1199.0  \\
            0.08  1766.0  \\
            0.09  1992.0  \\
            0.1  1748.0  \\
            0.11  1195.0  \\
            0.12  706.0  \\
            0.13  304.0  \\
            0.14  132.0  \\
            0.15  48.0  \\
            0.16  13.0  \\
            0.17  3.0  \\
            0.18  0.0  \\
        }
        ;
    \draw[solid, black, thick] (0.09631220152950964,\pgfkeysvalueof{/pgfplots/ymin})--(0.09631220152950964,\pgfkeysvalueof{/pgfplots/ymax});
    \draw[dashed, red, thick] (0.09634393682193111,\pgfkeysvalueof{/pgfplots/ymin})--(0.09634393682193111,\pgfkeysvalueof{/pgfplots/ymax});
    \addplot+[draw={none}]
        coordinates {
            (0.09634393682193111,0)
        }
        ;
    \nextgroupplot[]
    \addplot+[ybar interval, fill={blue!25}]
        table[row sep={\\}]
        {
            \\
            -0.06  2.0  \\
            -0.05  7.0  \\
            -0.04  35.0  \\
            -0.03  216.0  \\
            -0.02  607.0  \\
            -0.01  1213.0  \\
            0.0  1919.0  \\
            0.01  2057.0  \\
            0.02  1844.0  \\
            0.03  1212.0  \\
            0.04  573.0  \\
            0.05  226.0  \\
            0.06  72.0  \\
            0.07  11.0  \\
            0.08  6.0  \\
            0.09  0.0  \\
        }
        ;
    \draw[solid, black, thick] (0.015097861091196648,\pgfkeysvalueof{/pgfplots/ymin})--(0.015097861091196648,\pgfkeysvalueof{/pgfplots/ymax});
    \draw[dashed, red, thick] (0.015176641534493648,\pgfkeysvalueof{/pgfplots/ymin})--(0.015176641534493648,\pgfkeysvalueof{/pgfplots/ymax});
    \addplot+[draw={none}]
        coordinates {
            (0.015176641534493648,0)
        }
        ;
\end{groupplot}
\node[anchor=north] at ($(group c1r4.west |- group c1r4.outer south)!0.5!(group c3r4.east |- group c3r4.outer south)$){estimate};
\node[anchor=south, rotate=90] at ($(group c1r1.north -| group c1r1.outer west)!0.5!(group c1r4.south -| group c1r4.outer west)$){\# runs};
\end{tikzpicture}

      \captionof{figure}{Calibration error estimates of $10^4$ randomly sampled data sets. The solid black line indicates the mean of the calibration error estimates, and the dashed red line displays the true calibration error of the model.}
    \end{center}

    \begin{center}
      \begin{tikzpicture}
\begin{groupplot}[group style={group size={3 by 6}, xlabels at={edge bottom}, ylabels at={edge left}, horizontal sep={0.1\linewidth}, vertical sep={0.02\linewidth}, xticklabels at={edge bottom}}, no markers, tick label style={font={\footnotesize}}, grid={major}, title style={align={center}}, width={0.22\linewidth}, height={0.1\linewidth}, every x tick scale label/.style={at={{(1,0)}}, anchor={west}}, ylabel style={font={\small}}]
    \nextgroupplot[xmin={0}, xmax={1}, ymin={0}, title={\textbf{M1}}, ylabel={$\ECE$}]
    \addplot+[]
        table[row sep={\\}]
        {
            \\
            0.0  0.0091  \\
            0.01  0.0728  \\
            0.02  0.1216  \\
            0.03  0.1663  \\
            0.04  0.2053  \\
            0.05  0.2447  \\
            0.06  0.2751  \\
            0.07  0.303  \\
            0.08  0.3305  \\
            0.09  0.3567  \\
            0.1  0.3827  \\
            0.11  0.4063  \\
            0.12  0.4305  \\
            0.13  0.4501  \\
            0.14  0.473  \\
            0.15  0.493  \\
            0.16  0.5103  \\
            0.17  0.5279  \\
            0.18  0.546  \\
            0.19  0.5617  \\
            0.2  0.5806  \\
            0.21  0.5952  \\
            0.22  0.6095  \\
            0.23  0.6236  \\
            0.24  0.6349  \\
            0.25  0.6497  \\
            0.26  0.6626  \\
            0.27  0.6755  \\
            0.28  0.6873  \\
            0.29  0.6995  \\
            0.3  0.7102  \\
            0.31  0.7221  \\
            0.32  0.7329  \\
            0.33  0.7437  \\
            0.34  0.7533  \\
            0.35  0.7613  \\
            0.36  0.7706  \\
            0.37  0.7793  \\
            0.38  0.7883  \\
            0.39  0.7972  \\
            0.4  0.8049  \\
            0.41  0.8123  \\
            0.42  0.8203  \\
            0.43  0.8282  \\
            0.44  0.8353  \\
            0.45  0.8417  \\
            0.46  0.8471  \\
            0.47  0.8544  \\
            0.48  0.8602  \\
            0.49  0.8664  \\
            0.5  0.8722  \\
            0.51  0.8783  \\
            0.52  0.8837  \\
            0.53  0.8887  \\
            0.54  0.8948  \\
            0.55  0.9007  \\
            0.56  0.9048  \\
            0.57  0.9085  \\
            0.58  0.913  \\
            0.59  0.9167  \\
            0.6  0.9212  \\
            0.61  0.9254  \\
            0.62  0.9295  \\
            0.63  0.9328  \\
            0.64  0.936  \\
            0.65  0.9407  \\
            0.66  0.9449  \\
            0.67  0.9487  \\
            0.68  0.9519  \\
            0.69  0.9553  \\
            0.7  0.9581  \\
            0.71  0.961  \\
            0.72  0.9636  \\
            0.73  0.9671  \\
            0.74  0.9707  \\
            0.75  0.9726  \\
            0.76  0.9747  \\
            0.77  0.9759  \\
            0.78  0.9777  \\
            0.79  0.9798  \\
            0.8  0.9819  \\
            0.81  0.9839  \\
            0.82  0.9851  \\
            0.83  0.9866  \\
            0.84  0.9881  \\
            0.85  0.9895  \\
            0.86  0.991  \\
            0.87  0.9923  \\
            0.88  0.9935  \\
            0.89  0.9942  \\
            0.9  0.995  \\
            0.91  0.9957  \\
            0.92  0.9969  \\
            0.93  0.9974  \\
            0.94  0.9982  \\
            0.95  0.999  \\
            0.96  0.9993  \\
            0.97  0.9995  \\
            0.98  0.9998  \\
            0.99  0.9998  \\
            1.0  1.0  \\
        }
        ;
    \addplot+[dotted]
        coordinates {
            (0,0)
            (1,1)
        }
        ;
    \nextgroupplot[xmin={0}, xmax={1}, ymin={0}, title={\textbf{M2}}]
    \addplot+[]
        coordinates {
            (0.0,0.0)
            (0.01,0.0)
            (0.02,0.0)
            (0.03,0.0)
            (0.04,0.0)
            (0.05,0.0)
            (0.06,0.0)
            (0.07,0.0)
            (0.08,0.0)
            (0.09,0.0)
            (0.1,0.0)
            (0.11,0.0)
            (0.12,0.0)
            (0.13,0.0)
            (0.14,0.0)
            (0.15,0.0)
            (0.16,0.0)
            (0.17,0.0)
            (0.18,0.0)
            (0.19,0.0)
            (0.2,0.0)
            (0.21,0.0)
            (0.22,0.0)
            (0.23,0.0)
            (0.24,0.0)
            (0.25,0.0)
            (0.26,0.0)
            (0.27,0.0)
            (0.28,0.0)
            (0.29,0.0)
            (0.3,0.0)
            (0.31,0.0)
            (0.32,0.0)
            (0.33,0.0)
            (0.34,0.0)
            (0.35,0.0)
            (0.36,0.0)
            (0.37,0.0)
            (0.38,0.0)
            (0.39,0.0)
            (0.4,0.0)
            (0.41,0.0)
            (0.42,0.0)
            (0.43,0.0)
            (0.44,0.0)
            (0.45,0.0)
            (0.46,0.0)
            (0.47,0.0)
            (0.48,0.0)
            (0.49,0.0)
            (0.5,0.0)
            (0.51,0.0)
            (0.52,0.0)
            (0.53,0.0)
            (0.54,0.0)
            (0.55,0.0)
            (0.56,0.0)
            (0.57,0.0)
            (0.58,0.0)
            (0.59,0.0)
            (0.6,0.0)
            (0.61,0.0)
            (0.62,0.0)
            (0.63,0.0)
            (0.64,0.0)
            (0.65,0.0)
            (0.66,0.0)
            (0.67,0.0)
            (0.68,0.0)
            (0.69,0.0)
            (0.7,0.0)
            (0.71,0.0)
            (0.72,0.0)
            (0.73,0.0)
            (0.74,0.0)
            (0.75,0.0)
            (0.76,0.0)
            (0.77,0.0)
            (0.78,0.0)
            (0.79,0.0)
            (0.8,0.0)
            (0.81,0.0)
            (0.82,0.0)
            (0.83,0.0)
            (0.84,0.0)
            (0.85,0.0)
            (0.86,0.0)
            (0.87,0.0)
            (0.88,0.0)
            (0.89,0.0)
            (0.9,0.0)
            (0.91,0.0)
            (0.92,0.0)
            (0.93,0.0)
            (0.94,0.0)
            (0.95,0.0)
            (0.96,0.0)
            (0.97,0.0)
            (0.98,0.0)
            (0.99,0.0)
            (1.0,0.0)
        }
        ;
    \nextgroupplot[xmin={0}, xmax={1}, ymin={0}, title={\textbf{M3}}]
    \addplot+[]
        coordinates {
            (0.0,0.0)
            (0.01,0.0)
            (0.02,0.0)
            (0.03,0.0)
            (0.04,0.0)
            (0.05,0.0)
            (0.06,0.0)
            (0.07,0.0)
            (0.08,0.0)
            (0.09,0.0)
            (0.1,0.0)
            (0.11,0.0)
            (0.12,0.0)
            (0.13,0.0)
            (0.14,0.0)
            (0.15,0.0)
            (0.16,0.0)
            (0.17,0.0)
            (0.18,0.0)
            (0.19,0.0)
            (0.2,0.0)
            (0.21,0.0)
            (0.22,0.0)
            (0.23,0.0)
            (0.24,0.0)
            (0.25,0.0)
            (0.26,0.0)
            (0.27,0.0)
            (0.28,0.0)
            (0.29,0.0)
            (0.3,0.0)
            (0.31,0.0)
            (0.32,0.0)
            (0.33,0.0)
            (0.34,0.0)
            (0.35,0.0)
            (0.36,0.0)
            (0.37,0.0)
            (0.38,0.0)
            (0.39,0.0)
            (0.4,0.0)
            (0.41,0.0)
            (0.42,0.0)
            (0.43,0.0)
            (0.44,0.0)
            (0.45,0.0)
            (0.46,0.0)
            (0.47,0.0)
            (0.48,0.0)
            (0.49,0.0)
            (0.5,0.0)
            (0.51,0.0)
            (0.52,0.0)
            (0.53,0.0)
            (0.54,0.0)
            (0.55,0.0)
            (0.56,0.0)
            (0.57,0.0)
            (0.58,0.0)
            (0.59,0.0)
            (0.6,0.0)
            (0.61,0.0)
            (0.62,0.0)
            (0.63,0.0)
            (0.64,0.0)
            (0.65,0.0)
            (0.66,0.0)
            (0.67,0.0)
            (0.68,0.0)
            (0.69,0.0)
            (0.7,0.0)
            (0.71,0.0)
            (0.72,0.0)
            (0.73,0.0)
            (0.74,0.0)
            (0.75,0.0)
            (0.76,0.0)
            (0.77,0.0)
            (0.78,0.0)
            (0.79,0.0)
            (0.8,0.0)
            (0.81,0.0)
            (0.82,0.0)
            (0.83,0.0)
            (0.84,0.0)
            (0.85,0.0)
            (0.86,0.0)
            (0.87,0.0)
            (0.88,0.0)
            (0.89,0.0)
            (0.9,0.0)
            (0.91,0.0)
            (0.92,0.0)
            (0.93,0.0)
            (0.94,0.0)
            (0.95,0.0)
            (0.96,0.0)
            (0.97,0.0)
            (0.98,0.0)
            (0.99,0.0)
            (1.0,0.0)
        }
        ;
    \nextgroupplot[xmin={0}, xmax={1}, ymin={0}, ylabel={$\biasedskce$}]
    \addplot+[]
        table[row sep={\\}]
        {
            \\
            0.0  0.0  \\
            0.01  0.0  \\
            0.02  0.0  \\
            0.03  0.0  \\
            0.04  0.0  \\
            0.05  0.0  \\
            0.06  0.0  \\
            0.07  0.0  \\
            0.08  0.0  \\
            0.09  0.0  \\
            0.1  0.0  \\
            0.11  0.0  \\
            0.12  0.0  \\
            0.13  0.0  \\
            0.14  0.0  \\
            0.15  0.0  \\
            0.16  0.0  \\
            0.17  0.0  \\
            0.18  0.0  \\
            0.19  0.0  \\
            0.2  0.0  \\
            0.21  0.0  \\
            0.22  0.0  \\
            0.23  0.0  \\
            0.24  0.0  \\
            0.25  0.0  \\
            0.26  0.0  \\
            0.27  0.0  \\
            0.28  0.0  \\
            0.29  0.0  \\
            0.3  0.0  \\
            0.31  0.0  \\
            0.32  0.0  \\
            0.33  0.0  \\
            0.34  0.0  \\
            0.35  0.0  \\
            0.36  0.0  \\
            0.37  0.0  \\
            0.38  0.0  \\
            0.39  0.0  \\
            0.4  0.0  \\
            0.41  0.0  \\
            0.42  0.0  \\
            0.43  0.0  \\
            0.44  0.0  \\
            0.45  0.0  \\
            0.46  0.0  \\
            0.47  0.0  \\
            0.48  0.0  \\
            0.49  0.0  \\
            0.5  0.0  \\
            0.51  0.0  \\
            0.52  0.0  \\
            0.53  0.0  \\
            0.54  0.0  \\
            0.55  0.0  \\
            0.56  0.0  \\
            0.57  0.0  \\
            0.58  0.0  \\
            0.59  0.0  \\
            0.6  0.0  \\
            0.61  0.0  \\
            0.62  0.0  \\
            0.63  0.0  \\
            0.64  0.0  \\
            0.65  0.0  \\
            0.66  0.0  \\
            0.67  0.0  \\
            0.68  0.0  \\
            0.69  0.0  \\
            0.7  0.0  \\
            0.71  0.0  \\
            0.72  0.0  \\
            0.73  0.0  \\
            0.74  0.0  \\
            0.75  0.0  \\
            0.76  0.0  \\
            0.77  0.0  \\
            0.78  0.0  \\
            0.79  0.0  \\
            0.8  0.0  \\
            0.81  0.0  \\
            0.82  0.0  \\
            0.83  0.0  \\
            0.84  0.0  \\
            0.85  0.0  \\
            0.86  0.0  \\
            0.87  0.0  \\
            0.88  0.0  \\
            0.89  0.0  \\
            0.9  0.0  \\
            0.91  0.0  \\
            0.92  0.0  \\
            0.93  0.0  \\
            0.94  0.0  \\
            0.95  0.0  \\
            0.96  0.0  \\
            0.97  0.0  \\
            0.98  0.0  \\
            0.99  0.0  \\
            1.0  1.0  \\
        }
        ;
    \addplot+[dotted]
        coordinates {
            (0,0)
            (1,1)
        }
        ;
    \nextgroupplot[xmin={0}, xmax={1}, ymin={0}]
    \addplot+[]
        coordinates {
            (0.0,1.0)
            (0.01,0.9844)
            (0.02,0.8762)
            (0.03,0.7087)
            (0.04,0.531)
            (0.05,0.38029999999999997)
            (0.06,0.2711)
            (0.07,0.1886)
            (0.08,0.12760000000000005)
            (0.09,0.08589999999999998)
            (0.1,0.05830000000000002)
            (0.11,0.04059999999999997)
            (0.12,0.027800000000000047)
            (0.13,0.017800000000000038)
            (0.14,0.012900000000000023)
            (0.15,0.008099999999999996)
            (0.16,0.005099999999999993)
            (0.17,0.0040000000000000036)
            (0.18,0.0026000000000000467)
            (0.19,0.0013999999999999568)
            (0.2,0.0010000000000000009)
            (0.21,0.0008000000000000229)
            (0.22,0.00029999999999996696)
            (0.23,0.00019999999999997797)
            (0.24,9.999999999998899e-5)
            (0.25,9.999999999998899e-5)
            (0.26,9.999999999998899e-5)
            (0.27,9.999999999998899e-5)
            (0.28,9.999999999998899e-5)
            (0.29,9.999999999998899e-5)
            (0.3,9.999999999998899e-5)
            (0.31,9.999999999998899e-5)
            (0.32,0.0)
            (0.33,0.0)
            (0.34,0.0)
            (0.35,0.0)
            (0.36,0.0)
            (0.37,0.0)
            (0.38,0.0)
            (0.39,0.0)
            (0.4,0.0)
            (0.41,0.0)
            (0.42,0.0)
            (0.43,0.0)
            (0.44,0.0)
            (0.45,0.0)
            (0.46,0.0)
            (0.47,0.0)
            (0.48,0.0)
            (0.49,0.0)
            (0.5,0.0)
            (0.51,0.0)
            (0.52,0.0)
            (0.53,0.0)
            (0.54,0.0)
            (0.55,0.0)
            (0.56,0.0)
            (0.57,0.0)
            (0.58,0.0)
            (0.59,0.0)
            (0.6,0.0)
            (0.61,0.0)
            (0.62,0.0)
            (0.63,0.0)
            (0.64,0.0)
            (0.65,0.0)
            (0.66,0.0)
            (0.67,0.0)
            (0.68,0.0)
            (0.69,0.0)
            (0.7,0.0)
            (0.71,0.0)
            (0.72,0.0)
            (0.73,0.0)
            (0.74,0.0)
            (0.75,0.0)
            (0.76,0.0)
            (0.77,0.0)
            (0.78,0.0)
            (0.79,0.0)
            (0.8,0.0)
            (0.81,0.0)
            (0.82,0.0)
            (0.83,0.0)
            (0.84,0.0)
            (0.85,0.0)
            (0.86,0.0)
            (0.87,0.0)
            (0.88,0.0)
            (0.89,0.0)
            (0.9,0.0)
            (0.91,0.0)
            (0.92,0.0)
            (0.93,0.0)
            (0.94,0.0)
            (0.95,0.0)
            (0.96,0.0)
            (0.97,0.0)
            (0.98,0.0)
            (0.99,0.0)
            (1.0,0.0)
        }
        ;
    \nextgroupplot[xmin={0}, xmax={1}, ymin={0}]
    \addplot+[]
        coordinates {
            (0.0,1.0)
            (0.01,1.0)
            (0.02,1.0)
            (0.03,1.0)
            (0.04,1.0)
            (0.05,1.0)
            (0.06,1.0)
            (0.07,1.0)
            (0.08,1.0)
            (0.09,1.0)
            (0.1,1.0)
            (0.11,1.0)
            (0.12,1.0)
            (0.13,1.0)
            (0.14,1.0)
            (0.15,1.0)
            (0.16,1.0)
            (0.17,1.0)
            (0.18,1.0)
            (0.19,1.0)
            (0.2,1.0)
            (0.21,1.0)
            (0.22,1.0)
            (0.23,1.0)
            (0.24,1.0)
            (0.25,1.0)
            (0.26,1.0)
            (0.27,1.0)
            (0.28,1.0)
            (0.29,1.0)
            (0.3,1.0)
            (0.31,1.0)
            (0.32,1.0)
            (0.33,1.0)
            (0.34,1.0)
            (0.35,1.0)
            (0.36,1.0)
            (0.37,1.0)
            (0.38,1.0)
            (0.39,1.0)
            (0.4,1.0)
            (0.41,1.0)
            (0.42,1.0)
            (0.43,1.0)
            (0.44,1.0)
            (0.45,1.0)
            (0.46,1.0)
            (0.47,1.0)
            (0.48,1.0)
            (0.49,1.0)
            (0.5,1.0)
            (0.51,1.0)
            (0.52,1.0)
            (0.53,1.0)
            (0.54,1.0)
            (0.55,1.0)
            (0.56,1.0)
            (0.57,1.0)
            (0.58,1.0)
            (0.59,1.0)
            (0.6,1.0)
            (0.61,1.0)
            (0.62,1.0)
            (0.63,1.0)
            (0.64,1.0)
            (0.65,1.0)
            (0.66,1.0)
            (0.67,0.9999)
            (0.68,0.9999)
            (0.69,0.9997)
            (0.7,0.9996)
            (0.71,0.9986)
            (0.72,0.9978)
            (0.73,0.9965)
            (0.74,0.9936)
            (0.75,0.9877)
            (0.76,0.9753000000000001)
            (0.77,0.961)
            (0.78,0.9373)
            (0.79,0.9013)
            (0.8,0.8486)
            (0.81,0.7737)
            (0.82,0.6861999999999999)
            (0.83,0.583)
            (0.84,0.45930000000000004)
            (0.85,0.33530000000000004)
            (0.86,0.22250000000000003)
            (0.87,0.13280000000000003)
            (0.88,0.07050000000000001)
            (0.89,0.031100000000000017)
            (0.9,0.01200000000000001)
            (0.91,0.0034999999999999476)
            (0.92,0.0007000000000000339)
            (0.93,9.999999999998899e-5)
            (0.94,0.0)
            (0.95,0.0)
            (0.96,0.0)
            (0.97,0.0)
            (0.98,0.0)
            (0.99,0.0)
            (1.0,0.0)
        }
        ;
    \nextgroupplot[xmin={0}, xmax={1}, ymin={0}, ylabel={$\unbiasedskce$}]
    \addplot+[]
        table[row sep={\\}]
        {
            \\
            0.0  0.0  \\
            0.01  0.0  \\
            0.02  0.0  \\
            0.03  0.0  \\
            0.04  0.0  \\
            0.05  0.0  \\
            0.06  0.0  \\
            0.07  0.0  \\
            0.08  0.0  \\
            0.09  0.0  \\
            0.1  0.0  \\
            0.11  0.0  \\
            0.12  0.0  \\
            0.13  0.0  \\
            0.14  0.0  \\
            0.15  0.0  \\
            0.16  0.0  \\
            0.17  0.0  \\
            0.18  0.0  \\
            0.19  0.0  \\
            0.2  0.0  \\
            0.21  0.0  \\
            0.22  0.0  \\
            0.23  0.0  \\
            0.24  0.0  \\
            0.25  0.0  \\
            0.26  0.0  \\
            0.27  0.0  \\
            0.28  0.0  \\
            0.29  0.0  \\
            0.3  0.0  \\
            0.31  0.0  \\
            0.32  0.0  \\
            0.33  0.0  \\
            0.34  0.0  \\
            0.35  0.0  \\
            0.36  0.0  \\
            0.37  0.0  \\
            0.38  0.0  \\
            0.39  0.0  \\
            0.4  0.0  \\
            0.41  0.0  \\
            0.42  0.0  \\
            0.43  0.0  \\
            0.44  0.0  \\
            0.45  0.0  \\
            0.46  0.0  \\
            0.47  0.0  \\
            0.48  0.0  \\
            0.49  0.0  \\
            0.5  0.0  \\
            0.51  0.0  \\
            0.52  0.0  \\
            0.53  0.0  \\
            0.54  0.0  \\
            0.55  0.0  \\
            0.56  0.0  \\
            0.57  0.0  \\
            0.58  0.0  \\
            0.59  0.0  \\
            0.6  0.0  \\
            0.61  0.0  \\
            0.62  0.0  \\
            0.63  0.0  \\
            0.64  0.0  \\
            0.65  0.0  \\
            0.66  0.0  \\
            0.67  0.0  \\
            0.68  0.0  \\
            0.69  0.0  \\
            0.7  0.0  \\
            0.71  0.0  \\
            0.72  0.0  \\
            0.73  0.0  \\
            0.74  0.0  \\
            0.75  0.0  \\
            0.76  0.0  \\
            0.77  0.0  \\
            0.78  0.0  \\
            0.79  0.0  \\
            0.8  0.0  \\
            0.81  0.0  \\
            0.82  0.0  \\
            0.83  0.0  \\
            0.84  0.0  \\
            0.85  0.0  \\
            0.86  0.0  \\
            0.87  0.0  \\
            0.88  0.0  \\
            0.89  0.0  \\
            0.9  0.0  \\
            0.91  0.0  \\
            0.92  0.0  \\
            0.93  0.0  \\
            0.94  0.0  \\
            0.95  0.0  \\
            0.96  0.0  \\
            0.97  0.0  \\
            0.98  0.0  \\
            0.99  0.0  \\
            1.0  1.0  \\
        }
        ;
    \addplot+[dotted]
        coordinates {
            (0,0)
            (1,1)
        }
        ;
    \nextgroupplot[xmin={0}, xmax={1}, ymin={0}]
    \addplot+[]
        coordinates {
            (0.0,1.0)
            (0.01,1.0)
            (0.02,1.0)
            (0.03,1.0)
            (0.04,1.0)
            (0.05,1.0)
            (0.06,1.0)
            (0.07,1.0)
            (0.08,1.0)
            (0.09,1.0)
            (0.1,1.0)
            (0.11,1.0)
            (0.12,1.0)
            (0.13,1.0)
            (0.14,1.0)
            (0.15,1.0)
            (0.16,1.0)
            (0.17,1.0)
            (0.18,1.0)
            (0.19,1.0)
            (0.2,1.0)
            (0.21,1.0)
            (0.22,1.0)
            (0.23,1.0)
            (0.24,1.0)
            (0.25,1.0)
            (0.26,1.0)
            (0.27,1.0)
            (0.28,1.0)
            (0.29,1.0)
            (0.3,1.0)
            (0.31,1.0)
            (0.32,1.0)
            (0.33,1.0)
            (0.34,1.0)
            (0.35,1.0)
            (0.36,1.0)
            (0.37,1.0)
            (0.38,1.0)
            (0.39,1.0)
            (0.4,1.0)
            (0.41,1.0)
            (0.42,1.0)
            (0.43,1.0)
            (0.44,1.0)
            (0.45,1.0)
            (0.46,1.0)
            (0.47,1.0)
            (0.48,1.0)
            (0.49,1.0)
            (0.5,1.0)
            (0.51,1.0)
            (0.52,1.0)
            (0.53,1.0)
            (0.54,1.0)
            (0.55,1.0)
            (0.56,1.0)
            (0.57,1.0)
            (0.58,1.0)
            (0.59,1.0)
            (0.6,1.0)
            (0.61,1.0)
            (0.62,1.0)
            (0.63,1.0)
            (0.64,1.0)
            (0.65,1.0)
            (0.66,1.0)
            (0.67,0.9999)
            (0.68,0.9999)
            (0.69,0.9998)
            (0.7,0.9997)
            (0.71,0.9996)
            (0.72,0.9994)
            (0.73,0.9991)
            (0.74,0.9983)
            (0.75,0.9967)
            (0.76,0.9937)
            (0.77,0.9892)
            (0.78,0.983)
            (0.79,0.9687)
            (0.8,0.9499)
            (0.81,0.9216)
            (0.82,0.8843)
            (0.83,0.8325)
            (0.84,0.7622)
            (0.85,0.6780999999999999)
            (0.86,0.5712999999999999)
            (0.87,0.4536)
            (0.88,0.3416)
            (0.89,0.23360000000000003)
            (0.9,0.14449999999999996)
            (0.91,0.07509999999999994)
            (0.92,0.0353)
            (0.93,0.012700000000000045)
            (0.94,0.0032999999999999696)
            (0.95,0.00029999999999996696)
            (0.96,9.999999999998899e-5)
            (0.97,0.0)
            (0.98,0.0)
            (0.99,0.0)
            (1.0,0.0)
        }
        ;
    \nextgroupplot[xmin={0}, xmax={1}, ymin={0}]
    \addplot+[]
        coordinates {
            (0.0,1.0)
            (0.01,1.0)
            (0.02,1.0)
            (0.03,1.0)
            (0.04,1.0)
            (0.05,1.0)
            (0.06,1.0)
            (0.07,1.0)
            (0.08,1.0)
            (0.09,1.0)
            (0.1,1.0)
            (0.11,1.0)
            (0.12,1.0)
            (0.13,1.0)
            (0.14,1.0)
            (0.15,1.0)
            (0.16,1.0)
            (0.17,1.0)
            (0.18,1.0)
            (0.19,1.0)
            (0.2,1.0)
            (0.21,1.0)
            (0.22,1.0)
            (0.23,1.0)
            (0.24,1.0)
            (0.25,1.0)
            (0.26,1.0)
            (0.27,1.0)
            (0.28,1.0)
            (0.29,1.0)
            (0.3,1.0)
            (0.31,1.0)
            (0.32,1.0)
            (0.33,1.0)
            (0.34,1.0)
            (0.35,1.0)
            (0.36,1.0)
            (0.37,1.0)
            (0.38,1.0)
            (0.39,1.0)
            (0.4,1.0)
            (0.41,1.0)
            (0.42,1.0)
            (0.43,1.0)
            (0.44,1.0)
            (0.45,1.0)
            (0.46,1.0)
            (0.47,1.0)
            (0.48,1.0)
            (0.49,1.0)
            (0.5,1.0)
            (0.51,1.0)
            (0.52,1.0)
            (0.53,1.0)
            (0.54,1.0)
            (0.55,1.0)
            (0.56,1.0)
            (0.57,1.0)
            (0.58,1.0)
            (0.59,1.0)
            (0.6,1.0)
            (0.61,1.0)
            (0.62,1.0)
            (0.63,1.0)
            (0.64,1.0)
            (0.65,1.0)
            (0.66,1.0)
            (0.67,1.0)
            (0.68,1.0)
            (0.69,1.0)
            (0.7,1.0)
            (0.71,1.0)
            (0.72,1.0)
            (0.73,1.0)
            (0.74,1.0)
            (0.75,1.0)
            (0.76,1.0)
            (0.77,1.0)
            (0.78,1.0)
            (0.79,1.0)
            (0.8,1.0)
            (0.81,1.0)
            (0.82,1.0)
            (0.83,1.0)
            (0.84,1.0)
            (0.85,1.0)
            (0.86,1.0)
            (0.87,1.0)
            (0.88,1.0)
            (0.89,1.0)
            (0.9,1.0)
            (0.91,1.0)
            (0.92,1.0)
            (0.93,1.0)
            (0.94,1.0)
            (0.95,1.0)
            (0.96,1.0)
            (0.97,1.0)
            (0.98,1.0)
            (0.99,1.0)
            (1.0,0.0)
        }
        ;
    \nextgroupplot[xmin={0}, xmax={1}, ymin={0}, ylabel={$\linearskce$}]
    \addplot+[]
        table[row sep={\\}]
        {
            \\
            0.0  0.0  \\
            0.01  0.0  \\
            0.02  0.0  \\
            0.03  0.0  \\
            0.04  0.0  \\
            0.05  0.0  \\
            0.06  0.0  \\
            0.07  0.0  \\
            0.08  0.0  \\
            0.09  0.0  \\
            0.1  0.0  \\
            0.11  0.0  \\
            0.12  0.0  \\
            0.13  0.0  \\
            0.14  0.0  \\
            0.15  0.0  \\
            0.16  0.0  \\
            0.17  0.0  \\
            0.18  0.0  \\
            0.19  0.0  \\
            0.2  0.0  \\
            0.21  0.0  \\
            0.22  0.0  \\
            0.23  0.0  \\
            0.24  0.0  \\
            0.25  0.0  \\
            0.26  0.0  \\
            0.27  0.0  \\
            0.28  0.0  \\
            0.29  0.0  \\
            0.3  0.0  \\
            0.31  0.0  \\
            0.32  0.0  \\
            0.33  0.0  \\
            0.34  0.0  \\
            0.35  0.0  \\
            0.36  0.0  \\
            0.37  0.0  \\
            0.38  0.0  \\
            0.39  0.0  \\
            0.4  0.0  \\
            0.41  0.0  \\
            0.42  0.0  \\
            0.43  0.0  \\
            0.44  0.0  \\
            0.45  0.0  \\
            0.46  0.0  \\
            0.47  0.0  \\
            0.48  0.0  \\
            0.49  0.0  \\
            0.5  0.0  \\
            0.51  0.0  \\
            0.52  0.0  \\
            0.53  0.0  \\
            0.54  0.0  \\
            0.55  0.0  \\
            0.56  0.0  \\
            0.57  0.0  \\
            0.58  0.0  \\
            0.59  0.0  \\
            0.6  0.0  \\
            0.61  0.0  \\
            0.62  0.0  \\
            0.63  0.0  \\
            0.64  0.0  \\
            0.65  0.0  \\
            0.66  0.0  \\
            0.67  0.0  \\
            0.68  0.0  \\
            0.69  0.0  \\
            0.7  0.0  \\
            0.71  0.0  \\
            0.72  0.0  \\
            0.73  0.0  \\
            0.74  0.0  \\
            0.75  0.0  \\
            0.76  0.0  \\
            0.77  0.0  \\
            0.78  0.0  \\
            0.79  0.0  \\
            0.8  0.0  \\
            0.81  0.0  \\
            0.82  0.0  \\
            0.83  0.0  \\
            0.84  0.0  \\
            0.85  0.0  \\
            0.86  0.0  \\
            0.87  0.0  \\
            0.88  0.0  \\
            0.89  0.0  \\
            0.9  0.0  \\
            0.91  0.0  \\
            0.92  0.0  \\
            0.93  0.0  \\
            0.94  0.0  \\
            0.95  0.0  \\
            0.96  0.0  \\
            0.97  0.0  \\
            0.98  0.0  \\
            0.99  0.0006  \\
            1.0  1.0  \\
        }
        ;
    \addplot+[dotted]
        coordinates {
            (0,0)
            (1,1)
        }
        ;
    \nextgroupplot[xmin={0}, xmax={1}, ymin={0}]
    \addplot+[]
        coordinates {
            (0.0,1.0)
            (0.01,1.0)
            (0.02,1.0)
            (0.03,1.0)
            (0.04,1.0)
            (0.05,1.0)
            (0.06,1.0)
            (0.07,1.0)
            (0.08,1.0)
            (0.09,1.0)
            (0.1,1.0)
            (0.11,1.0)
            (0.12,1.0)
            (0.13,1.0)
            (0.14,1.0)
            (0.15,1.0)
            (0.16,1.0)
            (0.17,1.0)
            (0.18,1.0)
            (0.19,1.0)
            (0.2,1.0)
            (0.21,1.0)
            (0.22,1.0)
            (0.23,1.0)
            (0.24,1.0)
            (0.25,1.0)
            (0.26,1.0)
            (0.27,1.0)
            (0.28,1.0)
            (0.29,1.0)
            (0.3,1.0)
            (0.31,1.0)
            (0.32,1.0)
            (0.33,1.0)
            (0.34,1.0)
            (0.35,1.0)
            (0.36,1.0)
            (0.37,1.0)
            (0.38,1.0)
            (0.39,1.0)
            (0.4,1.0)
            (0.41,1.0)
            (0.42,1.0)
            (0.43,1.0)
            (0.44,1.0)
            (0.45,1.0)
            (0.46,1.0)
            (0.47,1.0)
            (0.48,1.0)
            (0.49,1.0)
            (0.5,1.0)
            (0.51,1.0)
            (0.52,1.0)
            (0.53,1.0)
            (0.54,1.0)
            (0.55,1.0)
            (0.56,1.0)
            (0.57,1.0)
            (0.58,1.0)
            (0.59,1.0)
            (0.6,1.0)
            (0.61,0.9999)
            (0.62,0.9998)
            (0.63,0.9997)
            (0.64,0.9993)
            (0.65,0.9992)
            (0.66,0.9987)
            (0.67,0.9984)
            (0.68,0.9972)
            (0.69,0.9962)
            (0.7,0.9949)
            (0.71,0.9921)
            (0.72,0.9881)
            (0.73,0.9838)
            (0.74,0.9784)
            (0.75,0.9707)
            (0.76,0.9607)
            (0.77,0.946)
            (0.78,0.9266)
            (0.79,0.9039)
            (0.8,0.8749)
            (0.81,0.842)
            (0.82,0.7997)
            (0.83,0.7465999999999999)
            (0.84,0.6909000000000001)
            (0.85,0.6224000000000001)
            (0.86,0.5509)
            (0.87,0.4729)
            (0.88,0.3944)
            (0.89,0.31720000000000004)
            (0.9,0.245)
            (0.91,0.17720000000000002)
            (0.92,0.12039999999999995)
            (0.93,0.07379999999999998)
            (0.94,0.042100000000000026)
            (0.95,0.01970000000000005)
            (0.96,0.007099999999999995)
            (0.97,0.0031999999999999806)
            (0.98,0.0007000000000000339)
            (0.99,0.0)
            (1.0,0.0)
        }
        ;
    \nextgroupplot[xmin={0}, xmax={1}, ymin={0}]
    \addplot+[]
        coordinates {
            (0.0,1.0)
            (0.01,1.0)
            (0.02,1.0)
            (0.03,1.0)
            (0.04,1.0)
            (0.05,1.0)
            (0.06,1.0)
            (0.07,1.0)
            (0.08,1.0)
            (0.09,1.0)
            (0.1,1.0)
            (0.11,1.0)
            (0.12,1.0)
            (0.13,1.0)
            (0.14,1.0)
            (0.15,1.0)
            (0.16,1.0)
            (0.17,1.0)
            (0.18,1.0)
            (0.19,1.0)
            (0.2,1.0)
            (0.21,1.0)
            (0.22,1.0)
            (0.23,1.0)
            (0.24,1.0)
            (0.25,1.0)
            (0.26,1.0)
            (0.27,1.0)
            (0.28,1.0)
            (0.29,1.0)
            (0.3,1.0)
            (0.31,1.0)
            (0.32,1.0)
            (0.33,1.0)
            (0.34,1.0)
            (0.35,1.0)
            (0.36,1.0)
            (0.37,1.0)
            (0.38,1.0)
            (0.39,1.0)
            (0.4,1.0)
            (0.41,1.0)
            (0.42,1.0)
            (0.43,1.0)
            (0.44,1.0)
            (0.45,1.0)
            (0.46,1.0)
            (0.47,1.0)
            (0.48,1.0)
            (0.49,1.0)
            (0.5,1.0)
            (0.51,1.0)
            (0.52,1.0)
            (0.53,1.0)
            (0.54,1.0)
            (0.55,1.0)
            (0.56,1.0)
            (0.57,1.0)
            (0.58,1.0)
            (0.59,1.0)
            (0.6,1.0)
            (0.61,1.0)
            (0.62,1.0)
            (0.63,1.0)
            (0.64,1.0)
            (0.65,1.0)
            (0.66,1.0)
            (0.67,1.0)
            (0.68,1.0)
            (0.69,1.0)
            (0.7,1.0)
            (0.71,1.0)
            (0.72,1.0)
            (0.73,1.0)
            (0.74,1.0)
            (0.75,1.0)
            (0.76,1.0)
            (0.77,1.0)
            (0.78,1.0)
            (0.79,1.0)
            (0.8,1.0)
            (0.81,1.0)
            (0.82,1.0)
            (0.83,1.0)
            (0.84,1.0)
            (0.85,1.0)
            (0.86,1.0)
            (0.87,1.0)
            (0.88,1.0)
            (0.89,0.9998)
            (0.9,0.9996)
            (0.91,0.9992)
            (0.92,0.9987)
            (0.93,0.9976)
            (0.94,0.9945)
            (0.95,0.9873)
            (0.96,0.9723)
            (0.97,0.9391)
            (0.98,0.8703)
            (0.99,0.7067)
            (1.0,0.0)
        }
        ;
    \nextgroupplot[xmin={0}, xmax={1}, ymin={0}, ylabel={$\asymplinearskce$}]
    \addplot+[]
        table[row sep={\\}]
        {
            \\
            0.0  0.0  \\
            0.01  0.0077  \\
            0.02  0.0162  \\
            0.03  0.0252  \\
            0.04  0.0353  \\
            0.05  0.0455  \\
            0.06  0.056  \\
            0.07  0.0656  \\
            0.08  0.0773  \\
            0.09  0.0868  \\
            0.1  0.098  \\
            0.11  0.1097  \\
            0.12  0.1186  \\
            0.13  0.1275  \\
            0.14  0.1394  \\
            0.15  0.1491  \\
            0.16  0.1598  \\
            0.17  0.1695  \\
            0.18  0.1798  \\
            0.19  0.191  \\
            0.2  0.2014  \\
            0.21  0.2126  \\
            0.22  0.2227  \\
            0.23  0.2327  \\
            0.24  0.243  \\
            0.25  0.2564  \\
            0.26  0.267  \\
            0.27  0.2759  \\
            0.28  0.2862  \\
            0.29  0.2973  \\
            0.3  0.3076  \\
            0.31  0.3182  \\
            0.32  0.3307  \\
            0.33  0.3402  \\
            0.34  0.3508  \\
            0.35  0.3607  \\
            0.36  0.3718  \\
            0.37  0.3801  \\
            0.38  0.3921  \\
            0.39  0.4033  \\
            0.4  0.4141  \\
            0.41  0.4244  \\
            0.42  0.4334  \\
            0.43  0.4441  \\
            0.44  0.4545  \\
            0.45  0.4647  \\
            0.46  0.4737  \\
            0.47  0.4836  \\
            0.48  0.4943  \\
            0.49  0.5037  \\
            0.5  0.5154  \\
            0.51  0.5236  \\
            0.52  0.5318  \\
            0.53  0.541  \\
            0.54  0.5513  \\
            0.55  0.5606  \\
            0.56  0.5707  \\
            0.57  0.5802  \\
            0.58  0.5899  \\
            0.59  0.5983  \\
            0.6  0.6075  \\
            0.61  0.616  \\
            0.62  0.625  \\
            0.63  0.6337  \\
            0.64  0.6431  \\
            0.65  0.6511  \\
            0.66  0.6614  \\
            0.67  0.6705  \\
            0.68  0.6794  \\
            0.69  0.6871  \\
            0.7  0.6957  \\
            0.71  0.7055  \\
            0.72  0.7152  \\
            0.73  0.7262  \\
            0.74  0.7358  \\
            0.75  0.7433  \\
            0.76  0.7521  \\
            0.77  0.7631  \\
            0.78  0.7724  \\
            0.79  0.781  \\
            0.8  0.7894  \\
            0.81  0.8002  \\
            0.82  0.81  \\
            0.83  0.8193  \\
            0.84  0.8282  \\
            0.85  0.8371  \\
            0.86  0.8452  \\
            0.87  0.8565  \\
            0.88  0.8678  \\
            0.89  0.8781  \\
            0.9  0.8906  \\
            0.91  0.9011  \\
            0.92  0.9121  \\
            0.93  0.9224  \\
            0.94  0.9324  \\
            0.95  0.9425  \\
            0.96  0.9523  \\
            0.97  0.9655  \\
            0.98  0.9774  \\
            0.99  0.9887  \\
            1.0  1.0  \\
        }
        ;
    \addplot+[dotted]
        coordinates {
            (0,0)
            (1,1)
        }
        ;
    \nextgroupplot[xmin={0}, xmax={1}, ymin={0}]
    \addplot+[]
        coordinates {
            (0.0,1.0)
            (0.01,0.0008000000000000229)
            (0.02,9.999999999998899e-5)
            (0.03,9.999999999998899e-5)
            (0.04,0.0)
            (0.05,0.0)
            (0.06,0.0)
            (0.07,0.0)
            (0.08,0.0)
            (0.09,0.0)
            (0.1,0.0)
            (0.11,0.0)
            (0.12,0.0)
            (0.13,0.0)
            (0.14,0.0)
            (0.15,0.0)
            (0.16,0.0)
            (0.17,0.0)
            (0.18,0.0)
            (0.19,0.0)
            (0.2,0.0)
            (0.21,0.0)
            (0.22,0.0)
            (0.23,0.0)
            (0.24,0.0)
            (0.25,0.0)
            (0.26,0.0)
            (0.27,0.0)
            (0.28,0.0)
            (0.29,0.0)
            (0.3,0.0)
            (0.31,0.0)
            (0.32,0.0)
            (0.33,0.0)
            (0.34,0.0)
            (0.35,0.0)
            (0.36,0.0)
            (0.37,0.0)
            (0.38,0.0)
            (0.39,0.0)
            (0.4,0.0)
            (0.41,0.0)
            (0.42,0.0)
            (0.43,0.0)
            (0.44,0.0)
            (0.45,0.0)
            (0.46,0.0)
            (0.47,0.0)
            (0.48,0.0)
            (0.49,0.0)
            (0.5,0.0)
            (0.51,0.0)
            (0.52,0.0)
            (0.53,0.0)
            (0.54,0.0)
            (0.55,0.0)
            (0.56,0.0)
            (0.57,0.0)
            (0.58,0.0)
            (0.59,0.0)
            (0.6,0.0)
            (0.61,0.0)
            (0.62,0.0)
            (0.63,0.0)
            (0.64,0.0)
            (0.65,0.0)
            (0.66,0.0)
            (0.67,0.0)
            (0.68,0.0)
            (0.69,0.0)
            (0.7,0.0)
            (0.71,0.0)
            (0.72,0.0)
            (0.73,0.0)
            (0.74,0.0)
            (0.75,0.0)
            (0.76,0.0)
            (0.77,0.0)
            (0.78,0.0)
            (0.79,0.0)
            (0.8,0.0)
            (0.81,0.0)
            (0.82,0.0)
            (0.83,0.0)
            (0.84,0.0)
            (0.85,0.0)
            (0.86,0.0)
            (0.87,0.0)
            (0.88,0.0)
            (0.89,0.0)
            (0.9,0.0)
            (0.91,0.0)
            (0.92,0.0)
            (0.93,0.0)
            (0.94,0.0)
            (0.95,0.0)
            (0.96,0.0)
            (0.97,0.0)
            (0.98,0.0)
            (0.99,0.0)
            (1.0,0.0)
        }
        ;
    \nextgroupplot[xmin={0}, xmax={1}, ymin={0}]
    \addplot+[]
        coordinates {
            (0.0,1.0)
            (0.01,0.9524)
            (0.02,0.9113)
            (0.03,0.8746)
            (0.04,0.8407)
            (0.05,0.8086)
            (0.06,0.7818)
            (0.07,0.7569)
            (0.08,0.7298)
            (0.09,0.708)
            (0.1,0.6831)
            (0.11,0.6608)
            (0.12,0.6407)
            (0.13,0.6185)
            (0.14,0.6009)
            (0.15,0.5842)
            (0.16,0.567)
            (0.17,0.5481)
            (0.18,0.5327999999999999)
            (0.19,0.5188999999999999)
            (0.2,0.5046999999999999)
            (0.21,0.4908)
            (0.22,0.4777)
            (0.23,0.4666)
            (0.24,0.4554)
            (0.25,0.4437)
            (0.26,0.43079999999999996)
            (0.27,0.41690000000000005)
            (0.28,0.40700000000000003)
            (0.29,0.39570000000000005)
            (0.3,0.38570000000000004)
            (0.31,0.3732)
            (0.32,0.3619)
            (0.33,0.3507)
            (0.34,0.34040000000000004)
            (0.35,0.33130000000000004)
            (0.36,0.3206)
            (0.37,0.31220000000000003)
            (0.38,0.30269999999999997)
            (0.39,0.2914)
            (0.4,0.2833)
            (0.41,0.275)
            (0.42,0.2653)
            (0.43,0.25680000000000003)
            (0.44,0.24950000000000006)
            (0.45,0.24229999999999996)
            (0.46,0.23399999999999999)
            (0.47,0.22560000000000002)
            (0.48,0.21889999999999998)
            (0.49,0.21450000000000002)
            (0.5,0.20799999999999996)
            (0.51,0.20120000000000005)
            (0.52,0.1945)
            (0.53,0.18769999999999998)
            (0.54,0.18069999999999997)
            (0.55,0.17420000000000002)
            (0.56,0.1674)
            (0.57,0.1623)
            (0.58,0.15690000000000004)
            (0.59,0.15159999999999996)
            (0.6,0.14559999999999995)
            (0.61,0.13980000000000004)
            (0.62,0.13490000000000002)
            (0.63,0.13019999999999998)
            (0.64,0.1251)
            (0.65,0.12)
            (0.66,0.11470000000000002)
            (0.67,0.11019999999999996)
            (0.68,0.10719999999999996)
            (0.69,0.10270000000000001)
            (0.7,0.09930000000000005)
            (0.71,0.0948)
            (0.72,0.09089999999999998)
            (0.73,0.08530000000000004)
            (0.74,0.08179999999999998)
            (0.75,0.07709999999999995)
            (0.76,0.07289999999999996)
            (0.77,0.06910000000000005)
            (0.78,0.06559999999999999)
            (0.79,0.060799999999999965)
            (0.8,0.05669999999999997)
            (0.81,0.05259999999999998)
            (0.82,0.04930000000000001)
            (0.83,0.046499999999999986)
            (0.84,0.04310000000000003)
            (0.85,0.03990000000000005)
            (0.86,0.03710000000000002)
            (0.87,0.03400000000000003)
            (0.88,0.03059999999999996)
            (0.89,0.027100000000000013)
            (0.9,0.024399999999999977)
            (0.91,0.022299999999999986)
            (0.92,0.01880000000000004)
            (0.93,0.016199999999999992)
            (0.94,0.0131)
            (0.95,0.01100000000000001)
            (0.96,0.007900000000000018)
            (0.97,0.005600000000000049)
            (0.98,0.0030000000000000027)
            (0.99,0.0019000000000000128)
            (1.0,0.0)
        }
        ;
    \nextgroupplot[xmin={0}, xmax={1}, ymin={0}, ylabel={$\asympunbiasedskce$}]
    \addplot+[]
        table[row sep={\\}]
        {
            \\
            0.0  0.0  \\
            0.01  0.006  \\
            0.02  0.008  \\
            0.03  0.024  \\
            0.04  0.03  \\
            0.05  0.038  \\
            0.06  0.046  \\
            0.07  0.054  \\
            0.08  0.062  \\
            0.09  0.07  \\
            0.1  0.074  \\
            0.11  0.082  \\
            0.12  0.098  \\
            0.13  0.112  \\
            0.14  0.122  \\
            0.15  0.136  \\
            0.16  0.146  \\
            0.17  0.158  \\
            0.18  0.17  \\
            0.19  0.174  \\
            0.2  0.178  \\
            0.21  0.19  \\
            0.22  0.194  \\
            0.23  0.208  \\
            0.24  0.22  \\
            0.25  0.23  \\
            0.26  0.242  \\
            0.27  0.252  \\
            0.28  0.262  \\
            0.29  0.272  \\
            0.3  0.278  \\
            0.31  0.286  \\
            0.32  0.298  \\
            0.33  0.312  \\
            0.34  0.32  \\
            0.35  0.328  \\
            0.36  0.336  \\
            0.37  0.348  \\
            0.38  0.36  \\
            0.39  0.37  \\
            0.4  0.378  \\
            0.41  0.382  \\
            0.42  0.398  \\
            0.43  0.406  \\
            0.44  0.418  \\
            0.45  0.422  \\
            0.46  0.43  \\
            0.47  0.438  \\
            0.48  0.452  \\
            0.49  0.46  \\
            0.5  0.47  \\
            0.51  0.482  \\
            0.52  0.502  \\
            0.53  0.514  \\
            0.54  0.53  \\
            0.55  0.546  \\
            0.56  0.56  \\
            0.57  0.578  \\
            0.58  0.588  \\
            0.59  0.598  \\
            0.6  0.614  \\
            0.61  0.628  \\
            0.62  0.638  \\
            0.63  0.654  \\
            0.64  0.672  \\
            0.65  0.678  \\
            0.66  0.684  \\
            0.67  0.694  \\
            0.68  0.704  \\
            0.69  0.714  \\
            0.7  0.732  \\
            0.71  0.744  \\
            0.72  0.762  \\
            0.73  0.77  \\
            0.74  0.778  \\
            0.75  0.796  \\
            0.76  0.8  \\
            0.77  0.816  \\
            0.78  0.826  \\
            0.79  0.834  \\
            0.8  0.84  \\
            0.81  0.848  \\
            0.82  0.858  \\
            0.83  0.866  \\
            0.84  0.876  \\
            0.85  0.884  \\
            0.86  0.896  \\
            0.87  0.904  \\
            0.88  0.91  \\
            0.89  0.918  \\
            0.9  0.936  \\
            0.91  0.948  \\
            0.92  0.954  \\
            0.93  0.958  \\
            0.94  0.962  \\
            0.95  0.964  \\
            0.96  0.98  \\
            0.97  0.986  \\
            0.98  0.994  \\
            0.99  0.998  \\
            1.0  1.0  \\
        }
        ;
    \addplot+[dotted]
        coordinates {
            (0,0)
            (1,1)
        }
        ;
    \nextgroupplot[xmin={0}, xmax={1}, ymin={0}]
    \addplot+[]
        coordinates {
            (0.0,0.0)
            (0.01,0.0)
            (0.02,0.0)
            (0.03,0.0)
            (0.04,0.0)
            (0.05,0.0)
            (0.06,0.0)
            (0.07,0.0)
            (0.08,0.0)
            (0.09,0.0)
            (0.1,0.0)
            (0.11,0.0)
            (0.12,0.0)
            (0.13,0.0)
            (0.14,0.0)
            (0.15,0.0)
            (0.16,0.0)
            (0.17,0.0)
            (0.18,0.0)
            (0.19,0.0)
            (0.2,0.0)
            (0.21,0.0)
            (0.22,0.0)
            (0.23,0.0)
            (0.24,0.0)
            (0.25,0.0)
            (0.26,0.0)
            (0.27,0.0)
            (0.28,0.0)
            (0.29,0.0)
            (0.3,0.0)
            (0.31,0.0)
            (0.32,0.0)
            (0.33,0.0)
            (0.34,0.0)
            (0.35,0.0)
            (0.36,0.0)
            (0.37,0.0)
            (0.38,0.0)
            (0.39,0.0)
            (0.4,0.0)
            (0.41,0.0)
            (0.42,0.0)
            (0.43,0.0)
            (0.44,0.0)
            (0.45,0.0)
            (0.46,0.0)
            (0.47,0.0)
            (0.48,0.0)
            (0.49,0.0)
            (0.5,0.0)
            (0.51,0.0)
            (0.52,0.0)
            (0.53,0.0)
            (0.54,0.0)
            (0.55,0.0)
            (0.56,0.0)
            (0.57,0.0)
            (0.58,0.0)
            (0.59,0.0)
            (0.6,0.0)
            (0.61,0.0)
            (0.62,0.0)
            (0.63,0.0)
            (0.64,0.0)
            (0.65,0.0)
            (0.66,0.0)
            (0.67,0.0)
            (0.68,0.0)
            (0.69,0.0)
            (0.7,0.0)
            (0.71,0.0)
            (0.72,0.0)
            (0.73,0.0)
            (0.74,0.0)
            (0.75,0.0)
            (0.76,0.0)
            (0.77,0.0)
            (0.78,0.0)
            (0.79,0.0)
            (0.8,0.0)
            (0.81,0.0)
            (0.82,0.0)
            (0.83,0.0)
            (0.84,0.0)
            (0.85,0.0)
            (0.86,0.0)
            (0.87,0.0)
            (0.88,0.0)
            (0.89,0.0)
            (0.9,0.0)
            (0.91,0.0)
            (0.92,0.0)
            (0.93,0.0)
            (0.94,0.0)
            (0.95,0.0)
            (0.96,0.0)
            (0.97,0.0)
            (0.98,0.0)
            (0.99,0.0)
            (1.0,0.0)
        }
        ;
    \nextgroupplot[xmin={0}, xmax={1}, ymin={0}]
    \addplot+[]
        coordinates {
            (0.0,0.0)
            (0.01,0.0)
            (0.02,0.0)
            (0.03,0.0)
            (0.04,0.0)
            (0.05,0.0)
            (0.06,0.0)
            (0.07,0.0)
            (0.08,0.0)
            (0.09,0.0)
            (0.1,0.0)
            (0.11,0.0)
            (0.12,0.0)
            (0.13,0.0)
            (0.14,0.0)
            (0.15,0.0)
            (0.16,0.0)
            (0.17,0.0)
            (0.18,0.0)
            (0.19,0.0)
            (0.2,0.0)
            (0.21,0.0)
            (0.22,0.0)
            (0.23,0.0)
            (0.24,0.0)
            (0.25,0.0)
            (0.26,0.0)
            (0.27,0.0)
            (0.28,0.0)
            (0.29,0.0)
            (0.3,0.0)
            (0.31,0.0)
            (0.32,0.0)
            (0.33,0.0)
            (0.34,0.0)
            (0.35,0.0)
            (0.36,0.0)
            (0.37,0.0)
            (0.38,0.0)
            (0.39,0.0)
            (0.4,0.0)
            (0.41,0.0)
            (0.42,0.0)
            (0.43,0.0)
            (0.44,0.0)
            (0.45,0.0)
            (0.46,0.0)
            (0.47,0.0)
            (0.48,0.0)
            (0.49,0.0)
            (0.5,0.0)
            (0.51,0.0)
            (0.52,0.0)
            (0.53,0.0)
            (0.54,0.0)
            (0.55,0.0)
            (0.56,0.0)
            (0.57,0.0)
            (0.58,0.0)
            (0.59,0.0)
            (0.6,0.0)
            (0.61,0.0)
            (0.62,0.0)
            (0.63,0.0)
            (0.64,0.0)
            (0.65,0.0)
            (0.66,0.0)
            (0.67,0.0)
            (0.68,0.0)
            (0.69,0.0)
            (0.7,0.0)
            (0.71,0.0)
            (0.72,0.0)
            (0.73,0.0)
            (0.74,0.0)
            (0.75,0.0)
            (0.76,0.0)
            (0.77,0.0)
            (0.78,0.0)
            (0.79,0.0)
            (0.8,0.0)
            (0.81,0.0)
            (0.82,0.0)
            (0.83,0.0)
            (0.84,0.0)
            (0.85,0.0)
            (0.86,0.0)
            (0.87,0.0)
            (0.88,0.0)
            (0.89,0.0)
            (0.9,0.0)
            (0.91,0.0)
            (0.92,0.0)
            (0.93,0.0)
            (0.94,0.0)
            (0.95,0.0)
            (0.96,0.0)
            (0.97,0.0)
            (0.98,0.0)
            (0.99,0.0)
            (1.0,0.0)
        }
        ;
\end{groupplot}
\node[anchor=north] at ($(group c1r6.west |- group c1r6.outer south)!0.5!(group c3r6.east |- group c3r6.outer south)$){bound/approximation of probability of false rejection};
\node[anchor=south, rotate=90] at ($(group c1r1.north -| group c1r1.outer west)!0.5!(group c1r6.south -| group c1r6.outer west)$){test error};
\end{tikzpicture}

      \captionof{figure}{Test errors versus bounds/approximations of the probability of false rejection, evaluated on $500$ ($\asympunbiasedskce$) and $10^4$ (all other test statistics) randomly sampled data sets. For model \textbf{M1} the type I error is shown, for both uncalibrated models the type II error is plotted.}
    \end{center}
  }
\end{tcbposter}
\end{document}
