% arara: lualatex: { shell: true }
% arara: lualatex: { shell: true }
% arara: lualatex: { shell: true, synctex: true }

% A2  : 420 x 594 mm    |
% 2A0 : 1189 x 1682 mm  > Factor 2.83 => 11pt ~ 31 pt
% 1m  : 1000 x 1414     > Factor 2.38 => 12pt ~ 28 pt

% A3  : 297 x 420 mm    |
% 2A0 : 1189 x 1682 mm  > Factor 4 => 11pt ~ 44 pt
% 1m  : 1000 x 1414     > Factor 3.36 => 10pt ~ 34 pt
\PassOptionsToPackage{force}{filehook} % see https://tex.stackexchange.com/questions/513051/filehook-error-with-memoir-after-update-texlive-2019-in-oct-15
\documentclass[10pt]{article}

\usepackage{luatex85}

% layout
\usepackage[a3paper,landscape]{geometry}

% math support
\usepackage{mathtools,amssymb}

% fonts
\RequirePackage[factor=0]{microtype} % no protrusion
\usepackage{unicode-math}
\defaultfontfeatures{Ligatures=TeX}
\IfFileExists{fonts/Berling.otf}{%
  % load fonts of official UU design
  \setmainfont{Berling}[%
  Path=./fonts/,
  Extension=.otf,
  BoldFont=*-Bold,
  ItalicFont=*-Italic,
  BoldItalicFont=*-BoldItalic]
}{%
  \setmainfont{Libertinus Serif}
}
\setsansfont{Libertinus Sans}
\setmonofont{Libertinus Mono}
\setmathfont{Libertinus Math}

\usepackage{bm}

% language support
\usepackage{polyglossia}
\setdefaultlanguage{english}
\usepackage{csquotes}

% better looking tables
\usepackage{booktabs}

% colors
\usepackage[CMYK]{xcolor}
\usepackage{UUcolorPantone}

\newcommand{\hl}[1]{\begingroup\bfseries\boldmath\color{uured}#1\endgroup}

% graphics
\usepackage{graphicx}
\usepackage{svg}
\svgpath{{./figures/}}

% captions
\usepackage{caption,subcaption}
\captionsetup{font=scriptsize}

% fancy lists
\usepackage{enumitem}
\setlist{leftmargin=*,itemsep=0pt}
\setlist[itemize,1]{label={\color{uured}$\blacktriangleright$}}

% hyperlinks
\usepackage{hyperref}

% boxes
\usepackage[poster,xparse,raster]{tcolorbox}

% poster settings
\tcbposterset{
  coverage =
  {
    spread,
    interior style={color=white},
  },
  poster =
  {
    columns=10,
    rows=1,
  },
  boxes =
  {
    enhanced standard jigsaw,
    sharp corners=downhill,
    arc=3pt,
    boxrule=1pt,
    lower separated=false,
    % colors
    coltext=black,
    colback=white,
    colframe=black,
    coltitle=black,
    colbacktitle=uulightgrey,
    % fonts
    fonttitle=\bfseries\large,
    % subtitles
    subtitle style=
    {
      frame empty,
      hbox,
      rounded corners=east,
      arc=8pt,
      coltext=white!50!uulightgrey,
      colback=black!10!uudarkgrey,
    },
  }
}

% plots
\usepackage{pgfplots,pgfplotstable}
\pgfplotsset{compat=1.16}
\usetikzlibrary{positioning,arrows,arrows.meta,calc,decorations.markings,intersections,patterns}

\usepgfplotslibrary{groupplots,fillbetween}
\usetikzlibrary{plotmarks}

% Use colorbrewer
\usepgfplotslibrary{colorbrewer}
\pgfplotsset{
  % initialize Dark2-8:
  cycle list/Dark2-8,
  % combine it with ’mark list*’:
  cycle multiindex* list={
    mark list*\nextlist
    Dark2-8\nextlist
  },
}

% plotting options
\pgfplotsset{every axis/.append style={axis background style={fill=gray!10}}}

% automatic references
\usepackage{cleveref}

% some abbreviations
\newcommand*{\Prob}{\mathbb{P}}
\newcommand*{\Expect}{\mathbb{E}}
\newcommand*{\transpose}[1]{{#1}^{\mathsf{T}}}
\newcommand*{\ECE}{\mathup{ECE}}
\newcommand*{\measure}{\mathup{CE}}
\newcommand*{\kernelmeasure}{\mathup{KCE}}
\newcommand*{\squaredkernelmeasure}{\mathup{SKCE}}
\newcommand*{\biasedestimator}{\widehat{\mathup{SKCE}}_{\mathup{b}}}
\newcommand*{\unbiasedestimator}{\widehat{\mathup{SKCE}}_{\mathup{uq}}}
\newcommand*{\linearestimator}{\widehat{\mathup{SKCE}}_{\mathup{ul}}}
\newcommand*{\Dir}{\mathup{Dir}}
\newcommand*{\Categorical}{\mathup{Cat}}

% metadata
\title{Calibration tests in multi-class classification:\\ A unifying framework}
\author{David Widmann$^\star$ Fredrik Lindsten$^\ddagger$ Dave Zachariah$^\star$}
\date{\today}
\makeatletter
\pgfkeys{%
  /my poster/.cd,
  title/.initial=\@title,
  author/.initial=\@author,
  institute/.initial={},
  contact/.initial={},
  date/.initial=\@date,
}
\makeatother

\pgfkeys{%
  /my poster/.cd,
  institute={$^\star$Department of Information Technology, Uppsala University $^\ddagger$Division of Statistics and Machine Learning, Linköping University},
  contact={david.widmann@it.uu.se fredrik.lindsten@liu.se dave.zachariah@it.uu.se},
}

\pagestyle{empty}

\begin{document}
\begin{tcbposter}

  % title
  \posterbox[blankest,interior engine=path,halign=left,valign=center,right=4cm,
  underlay =
  {%
    \node[left,inner sep=0pt,outer sep=0pt,align=center] at (frame.east) {\includegraphics[width=2cm]{figures/logos/UU.pdf}\\[1ex]\includegraphics[width=3cm]{figures/logos/LiU.pdf}};%
  }]{name=title,column=1,span=6,below=top}{%
    \Huge\textbf{\pgfkeysvalueof{/my poster/title}}\\[1ex]
    \large\pgfkeysvalueof{/my poster/author}\\[1ex]
    \normalsize\pgfkeysvalueof{/my poster/institute}%
  }%

  % footline
  \posterbox[blankest,top=2pt,bottom=2pt,valign=center,fontupper=\ttfamily\small,interior engine=path,interior style={color=uumidgrey}%
  ]{name=footline,column=1,span=10,above=bottom}{%
    \pgfkeysvalueof{/my poster/date}\hfill\pgfkeysvalueof{/my poster/contact}%
  }%

  \posterbox[adjusted title={Motivation - what is a calibrated model?}, colback=blondsvag]{name=calibration,column=3,span=4,below=title}{
    \begin{tcolorbox}[colback=blondstark]
      \begin{center}
        A \hl{calibrated model} yields predictions consistent with empirically observed frequencies.
      \end{center}
    \end{tcolorbox}

    \tcbsubtitle{Collision detection system}

    Consider a model that predicts if there is an object, a human, or an animal ahead of a car.

    \begin{minipage}[c]{0.6\linewidth}
        \begin{center}
          \begin{tikzpicture}
            \node[draw, inner sep=2mm] (image) at (0, 0) {\includesvg[height=8mm]{car}};
            \node[above=2mm of image, anchor=base, font=\scriptsize] {Input $X$};

            \node[draw, fill=gronskasvag, right=0.75cm of image, inner sep=2mm] (model)
            {\includesvg[height=8mm]{gear}};
            \node[above=2mm of model, anchor=base, font=\scriptsize] {Model $g$};
            \draw [->] (image) -- (model);

            \node[draw, right=0.75cm of model, minimum height=1.2cm, font=\scriptsize, align=center] (prediction)
            {\begin{tabular}{@{}ccc@{}}
               \includesvg[width=6mm]{barrier} & \includesvg[width=6mm]{pedestrian} & \includesvg[width=6mm]{bear} \\
               80\% & 0\% & 20\% \\
             \end{tabular}};
           \node[above=2mm of prediction, anchor=base, font=\scriptsize] {Prediction $g(X) \in \Delta^m$};
           \draw [->] (model) -- (prediction);
         \end{tikzpicture}
       \end{center}
     \end{minipage}%
     \begin{minipage}[c]{0.4\linewidth}
       We use $m$ for the number of classes, and
       $\Delta^m \coloneqq \{ z \in [0,1]^m \colon \|z\|_1 = 1\}$ for the
       $(m-1)$-dimensional probability simplex.
     \end{minipage}\vspace*{\baselineskip}

     If the model is calibrated we know that for all inputs with this
     prediction there is an object ahead 80\% of the time, a human 0\%
     of the time, and an animal 20\% of the time.

     \begin{center}
       \begin{tikzpicture}
         \node[minimum height=1.2cm, inner sep=2mm] (image) at (0, 0)
         {\begin{tabular}{@{}ccc@{}}
            \includesvg[height=3mm]{car0} & \includesvg[height=3mm]{car1} & \includesvg[height=3mm]{car2} \\
            \includesvg[height=3mm]{car3} & \includesvg[height=3mm]{car4} & $\cdots$ \\
          \end{tabular}};

        \node[draw, fill=gronskasvag, right=0.75cm of image, inner sep=2mm] (model)
        {\includesvg[height=8mm]{gear}};
        \draw [->] (image) -- (model);

        \node[draw, right=0.75cm of model, minimum height=1.2cm, font=\scriptsize, align=center] (prediction)
        {\begin{tabular}{@{}ccc@{}}
           \includesvg[width=6mm]{barrier} & \includesvg[width=6mm]{pedestrian} & \includesvg[width=6mm]{bear} \\
           80\% & 0\% & 20\% \\
         \end{tabular}};
        \draw [->] (model) -- (prediction);

        \node[right=1cm of prediction] (empirical)
        {\begin{tabular}{@{}cccccc@{}} \toprule
           \multicolumn{4}{c}{\includesvg[width=3mm]{barrier}} & \includesvg[width=3mm]{pedestrian} & \includesvg[width=3mm]{bear} \\ \midrule
           \includesvg[height=3mm]{car0} & \includesvg[height=3mm]{car2} & \includesvg[height=3mm]{car3} & \includesvg[height=3mm]{car4} & & \includesvg[height=3mm]{car1} \\
           $\vdots$ & $\vdots$ & $\vdots$ & $\vdots$ & & $\vdots$ \\ \bottomrule
         \end{tabular}};
        \node[above=2mm of empirical, anchor=base, font=\scriptsize] (A) {Empirical frequency $r(g(X)) \in \Delta^m$};
        \node[font=\scriptsize] at (prediction |- A) {Prediction $g(X) \in \Delta^m$};

        \path (prediction) -- node [font=\boldmath\Huge, color=uured, align=center, midway] {$=$} (empirical);
      \end{tikzpicture}
    \end{center}
}

  \posterbox[adjusted title={Quantifying calibration - a unifying framework}, colback=gryningmellan]{name=error,column=1,span=3,between=calibration and footline}{
    \begin{tcolorbox}[colback=blondstark]
      We define the \hl{calibration error}~($\measure$) of model $g$ with respect to a class $\mathcal{F}$ of functions $f \colon \Delta^m \to \mathbb{R}^m$ as
      \begin{equation*}
        \measure[\mathcal{F}, g] \coloneqq \sup_{f \in \mathcal{F}} \Expect\left[\transpose{(r(g(X)) - g(X))} f(g(X)) \right].
      \end{equation*}
    \end{tcolorbox}

    By design, if model $g$ is calibrated then the $\measure$ is zero, regardless of $\mathcal{F}$.

    \tcbsubtitle{Kernel calibration error}

    \begin{tcolorbox}[colback=blondstark]
      We define the \hl{kernel calibration error} ($\kernelmeasure$)
      of model $g$ with respect to a matrix-valued kernel
      $k \colon \Delta^m \times \Delta^m \to \mathbb{R}^{m \times m}$ as
      \begin{equation*}
        \kernelmeasure[k, g] \coloneqq \measure[\mathcal{F}, g],
      \end{equation*}
      where $\mathcal{F}$ is the unit ball in the reproducing kernel
      Hilbert space corresponding to $k$.
    \end{tcolorbox}

    If $k$ is a universal kernel, then the $\kernelmeasure$ is zero if
    and only if model $g$ is calibrated.

    \tcbsubtitle{Relation to existing measures}
    \begin{itemize}
    \item For common distances $d$ the expected calibration error ($\ECE$)
      \begin{equation}\label{eq:ece}
        \ECE[d, g] = \Expect[d(r(g(X)), g(X))]
      \end{equation}
      can be formulated as a $\measure$.

    \item The framework captures the maximum mean calibration error as well.
    \end{itemize}
  }

  \posterbox[adjusted title=The paper in 30 seconds, colback=blondmellan]{name=summary,column=1,span=2,between=title and error}{
    \begin{itemize}
    \item We propose a \hl{unifying framework} of calibration errors
      that allows us to derive a new \hl{kernel calibration error} with
      \hl{unbiased and consistent estimators}.
    \item Calibration error estimates are not interpretable. Instead we
      can conduct hypothesis tests of calibration.
    \item In contrast to existing approaches, the KCE enables
      well-founded bounds and approximations of the p-value for
      calibration tests.
    \end{itemize}

    \tcbsubtitle{Take with you}
    \begin{itemize}
    \item Kernel calibration error (KCE) with unbiased and consistent estimators
    \item Calibration errors have no meaningful unit or scale
    \item Reliable calibrations tests with the KCE
    \end{itemize}
  }

  \posterbox[adjusted title={Estimating the calibration error}, colback=gronskasvag]{name=estimation,column=4,span=3,below=calibration}{
    We want to estimate the $\measure$ of model $g$ using a validation
    data set $\{(X_i, Y_i)\}_{i=1}^n$ of i.i.d.\ pairs of inputs and labels.

    \tcbsubtitle{Kernel calibration error}

    For $i,j \in \{1,\ldots,n\}$, let
    $h_{i,j} \coloneqq \transpose{(e_{Y_i} - g(X_i))} k(g(X_i), g(X_j)) (e_{Y_j} - g(X_j))$,
    where $e_i \in \Delta^m$ denotes the $i$th unit vector.

    \begin{tcolorbox}[colback=blondstark]
      If $\mathbb{E}[\|k(g(X),g(X))\|] < \infty$, then \hl{consistent estimators}
      of the squared kernel calibration error
      $\squaredkernelmeasure[k, g] \coloneqq \kernelmeasure^2[k,g]$ are:
      \begin{center}
        \begin{tabular}{llll} \toprule
          Notation & Definition & Properties & Complexity\\ \midrule
          $\biasedestimator$ & $n^{-2} \sum_{i,j=1}^n h_{i,j}$ & biased & $O(n^2)$ \\
          $\unbiasedestimator$ & $ {\binom{n}{2}}^{-1} \sum_{1 \leq i < j \leq n} h_{i,j}$ & unbiased & $O(n^2)$ \\
          $\linearestimator$ & $ {\lfloor n/2\rfloor}^{-1} \sum_{i = 1}^{\lfloor n / 2\rfloor} h_{2i-1,2i}$ & unbiased & $O(n)$ \\ \bottomrule
        \end{tabular}
      \end{center}
    \end{tcolorbox}

    \tcbsubtitle{Relation to the expected calibration error}

    Standard estimators of the $\ECE$ are usually biased and inconsistent.
    The main difficulty is the estimation of the empirical frequencies
    $r(g(X))$ in \cref{eq:ece}. For the $\kernelmeasure$ there is no need
    to estimate them!
  }

  \posterbox[adjusted title={Example: A simple matrix-valued kernel}, colback=sandsvag]{name=kernel,column=4,span=3,between=estimation and footline}{
    If $\tilde{k} \colon \Delta^m \times \Delta^m \to \mathbb{R}$ is a
    kernel and $M \in \mathbb{R}^{m \times m}$ is positive semi-definite,
    then $k = M \tilde{k}$ is a matrix-valued kernel.
    If $\tilde{k}$ is universal (e.g., if $\tilde{k}$ is a Gaussian or
    Laplacian kernel), then $k$ is universal if and only if $M$ is
    positive definite.
  }

  \posterbox[adjusted title={Is my model calibrated?}, colback=sandsvag]{name=statistics,column=7,span=4,below=top}{
    In general, calibration errors have no meaningful unit or scale.
    This renders it difficult to interpret an estimated non-zero error.

    \tcbsubtitle{Calibration tests}
    \begin{minipage}[t]{0.35\linewidth}
      \vspace*{0pt}
      We can use the calibration error estimates to perform a
      statistical test of the null hypothesis
      \begin{equation*}
        H_0 \coloneqq \text{\enquote{the model is calibrated}}.
      \end{equation*}
    \end{minipage}
    \begin{minipage}[t]{0.65\linewidth}
      \vspace*{0pt}
      \begin{center}
        \begin{tikzpicture}[
          declare function={normal(\m,\s)=1/(2*\s*sqrt(pi))*exp(-(x-\m)^2/(2*\s^2));},
          declare function={binormal(\ma,\sa,\mb,\sb,\p)=(\p*normal(\ma,\sa)+(1-\p)*normal(\mb,\sb));}
          ]

          \begin{axis}[
            domain = -0.1:0.2,
            no marks,
            xlabel = calibration error estimate,
            ylabel = density,
            grid=major,
            ymin = 0,
            tick label style={font=\tiny},
            label style={font=\small},
            width = 0.75\linewidth,
            height = 0.33\linewidth,
            legend pos=outer north east,
            legend cell align=left,
            legend style=
            {
              fill=none,
              draw=none,
              inner sep={0pt},
              font=\small,
              align=left,
            }
            ]

            \draw [Dark2-A, thick] (0.07,\pgfkeysvalueof{/pgfplots/ymin}) -- (0.07,\pgfkeysvalueof{/pgfplots/ymax}) node [at end, above, anchor=south east, sloped, font=\small] {observed};

            \draw[Dark2-B, thick] (0,\pgfkeysvalueof{/pgfplots/ymin}) -- (0,\pgfkeysvalueof{/pgfplots/ymax}) node [at end, above, anchor=south east, sloped, font=\small] {calibrated};

            % mixture model of normal distributions
            \addplot+ [color=Dark2-B, dashed, thick, samples=31, smooth, name path=A] {binormal(-0.05,0.01,0.05,0.03,0.5)};
            \addlegendentry{distribution\\ under $H_0$};

            % indicate p-value
            \path [name path=B] (\pgfkeysvalueof{/pgfplots/xmin},0) -- (\pgfkeysvalueof{/pgfplots/xmax},0);
            \addplot+ [draw=Dark2-C, pattern color=Dark2-C, pattern={north east lines}] fill between [of=A and B, soft clip={domain=0.07:0.2}];
            \addlegendentry{p-value};

            % add comment
            \node[anchor=west, align=left, text=Dark2-C, font=\small] (annotation) at (0.075, 10) {reject $H_0$ if the \\p-value is small};
            \draw[->, >=stealth, thick, Dark2-C] (annotation) -- (0.08, 1);
          \end{axis}
        \end{tikzpicture}
      \end{center}
    \end{minipage}

    \begin{tcolorbox}[colback=blondstark]
      We derive \hl{well-founded bounds and approximations} of the p-value
      based on the $\squaredkernelmeasure$.
    \end{tcolorbox}
  }

  \posterbox[adjusted title={Experiments}, colback=gryningmellan]{name=experiment,column=7,span=4,between=statistics and footline}{
    We sample $10^4$ synthetic data sets $\{(g(X_i), Y_i)\}_{i=1}^{250}$
    from three generative models with $10$ classes by sampling
    predictions $g(X_i) \sim \Dir(0.1, \dots, 0.1)$ and labels $Y_i$
    conditionally on $g(X_i)$ from
    \begin{equation*}
      \symbf{M1}\colon \, \Categorical(g(X_i)), \quad
      \symbf{M2}\colon \, 0.5\Categorical(g(X_i)) + 0.5\Categorical(1,0,\dots,0), \quad
      \symbf{M3}\colon \, \Categorical(0.1, \dots, 0.1).
    \end{equation*}
    Model $\symbf{M1}$ is calibrated, and models $\symbf{M2}$ and
    $\symbf{M3}$ are uncalibrated.

    \tcbsubtitle{Calibration error estimates}

    \begin{minipage}[t]{0.35\linewidth}
      \vspace*{0pt}
      We show the distribution of a standard estimator of the $\ECE$,
      denoted by $\widehat{\ECE}$, and of the three proposed estimators
      of the $\squaredkernelmeasure$ with kernel
      \begin{equation*}
        k(x, y) = \exp{(- \|x - y\| / \nu)} \symbf{I}_{10},
      \end{equation*}
      where the kernel bandwidth $\nu > 0$ is chosen by the median
      heuristic.

      \vspace{\baselineskip}

      The solid line indicates the sample mean of the estimates, and the
      dashed line displays the true calibration error.
    \end{minipage}%
    \begin{minipage}[t]{0.65\linewidth}
      \vspace*{0pt}
      \begin{center}
        \begin{tikzpicture}
\begin{groupplot}[group style={group size={3 by 4}, xlabels at={edge bottom}, ylabels at={edge left}, horizontal sep={0.1\linewidth}, vertical sep={0.05\linewidth}}, no markers, tick label style={font={\tiny}}, grid={major}, title style={align={center}}, width={0.23\linewidth}, height={0.155\linewidth}, every x tick scale label/.style={at={{(1,0)}}, anchor={west}}, ylabel style={font={\small}}]
    \nextgroupplot[title={$\symbf{M1}$}, ylabel={$\widehat{\ECE}$}]
    \addplot+[ybar interval, fill={Dark2-A!30!white}]
        table[row sep={\\}]
        {
            \\
            0.17  7.0  \\
            0.18  55.0  \\
            0.19  229.0  \\
            0.2  796.0  \\
            0.21  1782.0  \\
            0.22  2550.0  \\
            0.23  2307.0  \\
            0.24  1389.0  \\
            0.25  667.0  \\
            0.26  171.0  \\
            0.27  42.0  \\
            0.28  5.0  \\
            0.29  0.0  \\
        }
        ;
    \draw[solid, thick, Dark2-B] (0.22876019647666532,\pgfkeysvalueof{/pgfplots/ymin})--(0.22876019647666532,\pgfkeysvalueof{/pgfplots/ymax});
    \draw[dashed, thick, Dark2-C] (0.0,\pgfkeysvalueof{/pgfplots/ymin})--(0.0,\pgfkeysvalueof{/pgfplots/ymax});
    \addplot+[draw={none}]
        coordinates {
            (0.0,0)
        }
        ;
    \nextgroupplot[title={$\symbf{M2}$}]
    \addplot+[ybar interval, fill={Dark2-A!30!white}]
        table[row sep={\\}]
        {
            \\
            0.42  23.0  \\
            0.44  207.0  \\
            0.46  838.0  \\
            0.48  1942.0  \\
            0.5  2982.0  \\
            0.52  2440.0  \\
            0.54  1165.0  \\
            0.56  345.0  \\
            0.58  52.0  \\
            0.6  6.0  \\
            0.62  0.0  \\
        }
        ;
    \draw[solid, thick, Dark2-B] (0.5133844252714233,\pgfkeysvalueof{/pgfplots/ymin})--(0.5133844252714233,\pgfkeysvalueof{/pgfplots/ymax});
    \draw[dashed, thick, Dark2-C] (0.45,\pgfkeysvalueof{/pgfplots/ymin})--(0.45,\pgfkeysvalueof{/pgfplots/ymax});
    \addplot+[draw={none}]
        coordinates {
            (0.45,0)
        }
        ;
    \nextgroupplot[title={$\symbf{M3}$}]
    \addplot+[ybar interval, fill={Dark2-A!30!white}]
        table[row sep={\\}]
        {
            \\
            0.3  15.0  \\
            0.32  118.0  \\
            0.34  804.0  \\
            0.36  2310.0  \\
            0.38  3279.0  \\
            0.4  2413.0  \\
            0.42  863.0  \\
            0.44  185.0  \\
            0.46  13.0  \\
            0.48  0.0  \\
        }
        ;
    \draw[solid, thick, Dark2-B] (0.3908207321571612,\pgfkeysvalueof{/pgfplots/ymin})--(0.3908207321571612,\pgfkeysvalueof{/pgfplots/ymax});
    \draw[dashed, thick, Dark2-C] (0.7106418012290426,\pgfkeysvalueof{/pgfplots/ymin})--(0.7106418012290426,\pgfkeysvalueof{/pgfplots/ymax});
    \addplot+[draw={none}]
        coordinates {
            (0.7106418012290426,0)
        }
        ;
    \nextgroupplot[ylabel={$\biasedestimator$}]
    \addplot+[ybar interval, fill={Dark2-A!30!white}]
        table[row sep={\\}]
        {
            \\
            0.0005  61.0  \\
            0.001  2724.0  \\
            0.0015  4376.0  \\
            0.002  2111.0  \\
            0.0025  597.0  \\
            0.003  108.0  \\
            0.0035  20.0  \\
            0.004  3.0  \\
            0.0045  0.0  \\
        }
        ;
    \draw[solid, thick, Dark2-B] (0.001791065804877401,\pgfkeysvalueof{/pgfplots/ymin})--(0.001791065804877401,\pgfkeysvalueof{/pgfplots/ymax});
    \draw[dashed, thick, Dark2-C] (-8.624308735909016e-6,\pgfkeysvalueof{/pgfplots/ymin})--(-8.624308735909016e-6,\pgfkeysvalueof{/pgfplots/ymax});
    \addplot+[draw={none}]
        coordinates {
            (-8.624308735909016e-6,0)
        }
        ;
    \nextgroupplot
    \addplot+[ybar interval, fill={Dark2-A!30!white}]
        table[row sep={\\}]
        {
            \\
            0.05  3.0  \\
            0.06  89.0  \\
            0.07  558.0  \\
            0.08  1735.0  \\
            0.09  2790.0  \\
            0.1  2643.0  \\
            0.11  1483.0  \\
            0.12  532.0  \\
            0.13  133.0  \\
            0.14  31.0  \\
            0.15  3.0  \\
            0.16  0.0  \\
        }
        ;
    \draw[solid, thick, Dark2-B] (0.09976616408194146,\pgfkeysvalueof{/pgfplots/ymin})--(0.09976616408194146,\pgfkeysvalueof{/pgfplots/ymax});
    \draw[dashed, thick, Dark2-C] (0.09655136747590724,\pgfkeysvalueof{/pgfplots/ymin})--(0.09655136747590724,\pgfkeysvalueof{/pgfplots/ymax});
    \addplot+[draw={none}]
        coordinates {
            (0.09655136747590724,0)
        }
        ;
    \nextgroupplot
    \addplot+[ybar interval, fill={Dark2-A!30!white}]
        table[row sep={\\}]
        {
            \\
            0.014  2.0  \\
            0.015  12.0  \\
            0.016  124.0  \\
            0.017  527.0  \\
            0.018  1314.0  \\
            0.019  2150.0  \\
            0.02  2374.0  \\
            0.021  1705.0  \\
            0.022  986.0  \\
            0.023  518.0  \\
            0.024  209.0  \\
            0.025  57.0  \\
            0.026  14.0  \\
            0.027  6.0  \\
            0.028  2.0  \\
            0.029  0.0  \\
        }
        ;
    \draw[solid, thick, Dark2-B] (0.02045969775710109,\pgfkeysvalueof{/pgfplots/ymin})--(0.02045969775710109,\pgfkeysvalueof{/pgfplots/ymax});
    \draw[dashed, thick, Dark2-C] (0.015122104194196453,\pgfkeysvalueof{/pgfplots/ymin})--(0.015122104194196453,\pgfkeysvalueof{/pgfplots/ymax});
    \addplot+[draw={none}]
        coordinates {
            (0.015122104194196453,0)
        }
        ;
    \nextgroupplot[ylabel={$\unbiasedestimator$}]
    \addplot+[ybar interval, fill={Dark2-A!30!white}]
        table[row sep={\\}]
        {
            \\
            -0.001  37.0  \\
            -0.0008  438.0  \\
            -0.0006  1286.0  \\
            -0.0004  1968.0  \\
            -0.0002  1919.0  \\
            0.0  1607.0  \\
            0.0002  1132.0  \\
            0.0004  704.0  \\
            0.0006  423.0  \\
            0.0008  242.0  \\
            0.001  122.0  \\
            0.0012  72.0  \\
            0.0014  22.0  \\
            0.0016  20.0  \\
            0.0018  4.0  \\
            0.002  4.0  \\
            0.0022  0.0  \\
        }
        ;
    \draw[solid, thick, Dark2-B] (-8.624308735909016e-6,\pgfkeysvalueof{/pgfplots/ymin})--(-8.624308735909016e-6,\pgfkeysvalueof{/pgfplots/ymax});
    \draw[dashed, thick, Dark2-C] (-8.624308735909016e-6,\pgfkeysvalueof{/pgfplots/ymin})--(-8.624308735909016e-6,\pgfkeysvalueof{/pgfplots/ymax});
    \addplot+[draw={none}]
        coordinates {
            (-8.624308735909016e-6,0)
        }
        ;
    \nextgroupplot
    \addplot+[ybar interval, fill={Dark2-A!30!white}]
        table[row sep={\\}]
        {
            \\
            0.05  11.0  \\
            0.06  151.0  \\
            0.07  882.0  \\
            0.08  2126.0  \\
            0.09  2995.0  \\
            0.1  2302.0  \\
            0.11  1076.0  \\
            0.12  352.0  \\
            0.13  92.0  \\
            0.14  12.0  \\
            0.15  1.0  \\
            0.16  0.0  \\
        }
        ;
    \draw[solid, thick, Dark2-B] (0.09655136747590724,\pgfkeysvalueof{/pgfplots/ymin})--(0.09655136747590724,\pgfkeysvalueof{/pgfplots/ymax});
    \draw[dashed, thick, Dark2-C] (0.09655136747590724,\pgfkeysvalueof{/pgfplots/ymin})--(0.09655136747590724,\pgfkeysvalueof{/pgfplots/ymax});
    \addplot+[draw={none}]
        coordinates {
            (0.09655136747590724,0)
        }
        ;
    \nextgroupplot
    \addplot+[ybar interval, fill={Dark2-A!30!white}]
        table[row sep={\\}]
        {
            \\
            0.009  2.0  \\
            0.01  15.0  \\
            0.011  166.0  \\
            0.012  702.0  \\
            0.013  1588.0  \\
            0.014  2501.0  \\
            0.015  2295.0  \\
            0.016  1441.0  \\
            0.017  767.0  \\
            0.018  352.0  \\
            0.019  126.0  \\
            0.02  34.0  \\
            0.021  7.0  \\
            0.022  3.0  \\
            0.023  1.0  \\
            0.024  0.0  \\
        }
        ;
    \draw[solid, thick, Dark2-B] (0.015122104194196453,\pgfkeysvalueof{/pgfplots/ymin})--(0.015122104194196453,\pgfkeysvalueof{/pgfplots/ymax});
    \draw[dashed, thick, Dark2-C] (0.015122104194196453,\pgfkeysvalueof{/pgfplots/ymin})--(0.015122104194196453,\pgfkeysvalueof{/pgfplots/ymax});
    \addplot+[draw={none}]
        coordinates {
            (0.015122104194196453,0)
        }
        ;
    \nextgroupplot[ylabel={$\linearestimator$}]
    \addplot+[ybar interval, fill={Dark2-A!30!white}]
        table[row sep={\\}]
        {
            \\
            -0.03  1.0  \\
            -0.025  13.0  \\
            -0.02  117.0  \\
            -0.015  555.0  \\
            -0.01  1580.0  \\
            -0.005  2580.0  \\
            0.0  2825.0  \\
            0.005  1594.0  \\
            0.01  564.0  \\
            0.015  145.0  \\
            0.02  21.0  \\
            0.025  5.0  \\
            0.03  0.0  \\
        }
        ;
    \draw[solid, thick, Dark2-B] (0.00016041013856191278,\pgfkeysvalueof{/pgfplots/ymin})--(0.00016041013856191278,\pgfkeysvalueof{/pgfplots/ymax});
    \draw[dashed, thick, Dark2-C] (-8.624308735909016e-6,\pgfkeysvalueof{/pgfplots/ymin})--(-8.624308735909016e-6,\pgfkeysvalueof{/pgfplots/ymax});
    \addplot+[draw={none}]
        coordinates {
            (-8.624308735909016e-6,0)
        }
        ;
    \nextgroupplot
    \addplot+[ybar interval, fill={Dark2-A!30!white}]
        table[row sep={\\}]
        {
            \\
            0.03  10.0  \\
            0.04  52.0  \\
            0.05  211.0  \\
            0.06  603.0  \\
            0.07  1159.0  \\
            0.08  1823.0  \\
            0.09  1930.0  \\
            0.1  1743.0  \\
            0.11  1273.0  \\
            0.12  679.0  \\
            0.13  325.0  \\
            0.14  134.0  \\
            0.15  40.0  \\
            0.16  13.0  \\
            0.17  5.0  \\
            0.18  0.0  \\
        }
        ;
    \draw[solid, thick, Dark2-B] (0.09655648558846669,\pgfkeysvalueof{/pgfplots/ymin})--(0.09655648558846669,\pgfkeysvalueof{/pgfplots/ymax});
    \draw[dashed, thick, Dark2-C] (0.09655136747590724,\pgfkeysvalueof{/pgfplots/ymin})--(0.09655136747590724,\pgfkeysvalueof{/pgfplots/ymax});
    \addplot+[draw={none}]
        coordinates {
            (0.09655136747590724,0)
        }
        ;
    \nextgroupplot
    \addplot+[ybar interval, fill={Dark2-A!30!white}]
        table[row sep={\\}]
        {
            \\
            -0.06  2.0  \\
            -0.05  6.0  \\
            -0.04  43.0  \\
            -0.03  194.0  \\
            -0.02  586.0  \\
            -0.01  1248.0  \\
            0.0  1912.0  \\
            0.01  2179.0  \\
            0.02  1794.0  \\
            0.03  1100.0  \\
            0.04  604.0  \\
            0.05  252.0  \\
            0.06  60.0  \\
            0.07  18.0  \\
            0.08  2.0  \\
            0.09  0.0  \\
        }
        ;
    \draw[solid, thick, Dark2-B] (0.015012349168529436,\pgfkeysvalueof{/pgfplots/ymin})--(0.015012349168529436,\pgfkeysvalueof{/pgfplots/ymax});
    \draw[dashed, thick, Dark2-C] (0.015122104194196453,\pgfkeysvalueof{/pgfplots/ymin})--(0.015122104194196453,\pgfkeysvalueof{/pgfplots/ymax});
    \addplot+[draw={none}]
        coordinates {
            (0.015122104194196453,0)
        }
        ;
\end{groupplot}
\node[anchor=north] at ($(group c1r4.west |- group c1r4.outer south)!0.5!(group c3r4.east |- group c3r4.outer south)$){calibration error estimate};
\node[anchor=south, rotate=90, yshift=1ex] at ($(group c1r1.north -| group c1r1.outer west)!0.5!(group c1r4.south -| group c1r4.outer west)$){\# runs};
\end{tikzpicture}

      \end{center}
    \end{minipage}

    \tcbsubtitle{Empirical test errors}

    \begin{minipage}[t]{0.4\linewidth}
      \vspace*{0pt}
      We evaluate the derived bounds $\symbf{D}_{\mathup{b}}$,
      $\symbf{D}_{\mathup{uq}}$, and $\symbf{D}_{\mathup{ul}}$, and
      approximations $\symbf{A}_{\mathup{uq}}$ and
      $\symbf{A}_{\mathup{l}}$ of the p-value based on the
      $\squaredkernelmeasure$. We compare them with a previously
      proposed hypothesis test for the standard $\ECE$ estimator
      ($\symbf{C}$).

      \vspace{\baselineskip}

      For a chosen significance level $\alpha$ we compute from the
      p-value approximations $p_1,\ldots,p_{10^4}$ the empirical test
      error
      \begin{equation*}
        \frac{1}{10^4} \sum_{i=1}^{10^4} \mathbb{1}_{[0, \alpha]}(p_i) \quad \text{ (for } \symbf{M1} \text{)}
      \end{equation*}
      and
      \begin{equation*}
        \frac{1}{10^4} \sum_{i=1}^{10^4} \mathbb{1}_{(\alpha, 1]}(p_i) \quad \text{ (for } \symbf{M2} \text{ and } \symbf{M3} \text{)}.
      \end{equation*}
    \end{minipage}%
    \begin{minipage}[t]{0.6\linewidth}
      \vspace*{0pt}
      \begin{center}
        \begin{tikzpicture}
\begin{groupplot}[group style={group size={3 by 6}, xlabels at={edge bottom}, ylabels at={edge left}, horizontal sep={0.1\linewidth}, vertical sep={0.015\linewidth}, xticklabels at={edge bottom}}, no markers, tick label style={font={\tiny}}, grid={major}, title style={align={center}}, width={0.23\linewidth}, height={0.155\linewidth}, every x tick scale label/.style={at={{(1,0)}}, anchor={west}}, ylabel style={font={\small}}, xmin={0}, xmax={1}, ymin={-0.1}, ymax={1.1}]
    \nextgroupplot[title={$\symbf{M1}$}, ylabel={$\symbf{C}$}]
    \addplot+[thick]
        table[row sep={\\}]
        {
            \\
            0.0  0.0091  \\
            0.01  0.0728  \\
            0.02  0.1216  \\
            0.03  0.1663  \\
            0.04  0.2053  \\
            0.05  0.2447  \\
            0.06  0.2751  \\
            0.07  0.303  \\
            0.08  0.3305  \\
            0.09  0.3567  \\
            0.1  0.3827  \\
            0.11  0.4063  \\
            0.12  0.4305  \\
            0.13  0.4501  \\
            0.14  0.473  \\
            0.15  0.493  \\
            0.16  0.5103  \\
            0.17  0.5279  \\
            0.18  0.546  \\
            0.19  0.5617  \\
            0.2  0.5806  \\
            0.21  0.5952  \\
            0.22  0.6095  \\
            0.23  0.6236  \\
            0.24  0.6349  \\
            0.25  0.6497  \\
            0.26  0.6626  \\
            0.27  0.6755  \\
            0.28  0.6873  \\
            0.29  0.6995  \\
            0.3  0.7102  \\
            0.31  0.7221  \\
            0.32  0.7329  \\
            0.33  0.7437  \\
            0.34  0.7533  \\
            0.35  0.7613  \\
            0.36  0.7706  \\
            0.37  0.7793  \\
            0.38  0.7883  \\
            0.39  0.7972  \\
            0.4  0.8049  \\
            0.41  0.8123  \\
            0.42  0.8203  \\
            0.43  0.8282  \\
            0.44  0.8353  \\
            0.45  0.8417  \\
            0.46  0.8471  \\
            0.47  0.8544  \\
            0.48  0.8602  \\
            0.49  0.8664  \\
            0.5  0.8722  \\
            0.51  0.8783  \\
            0.52  0.8837  \\
            0.53  0.8887  \\
            0.54  0.8948  \\
            0.55  0.9007  \\
            0.56  0.9048  \\
            0.57  0.9085  \\
            0.58  0.913  \\
            0.59  0.9167  \\
            0.6  0.9212  \\
            0.61  0.9254  \\
            0.62  0.9295  \\
            0.63  0.9328  \\
            0.64  0.936  \\
            0.65  0.9407  \\
            0.66  0.9449  \\
            0.67  0.9487  \\
            0.68  0.9519  \\
            0.69  0.9553  \\
            0.7  0.9581  \\
            0.71  0.961  \\
            0.72  0.9636  \\
            0.73  0.9671  \\
            0.74  0.9707  \\
            0.75  0.9726  \\
            0.76  0.9747  \\
            0.77  0.9759  \\
            0.78  0.9777  \\
            0.79  0.9798  \\
            0.8  0.9819  \\
            0.81  0.9839  \\
            0.82  0.9851  \\
            0.83  0.9866  \\
            0.84  0.9881  \\
            0.85  0.9895  \\
            0.86  0.991  \\
            0.87  0.9923  \\
            0.88  0.9935  \\
            0.89  0.9942  \\
            0.9  0.995  \\
            0.91  0.9957  \\
            0.92  0.9969  \\
            0.93  0.9974  \\
            0.94  0.9982  \\
            0.95  0.999  \\
            0.96  0.9993  \\
            0.97  0.9995  \\
            0.98  0.9998  \\
            0.99  0.9998  \\
            1.0  1.0  \\
        }
        ;
    \addplot+[dashed, thick]
        coordinates {
            (0,0)
            (1,1)
        }
        ;
    \nextgroupplot[title={$\symbf{M2}$}]
    \addplot+[thick]
        table[row sep={\\}]
        {
            \\
            0.0  0.0  \\
            0.01  0.0  \\
            0.02  0.0  \\
            0.03  0.0  \\
            0.04  0.0  \\
            0.05  0.0  \\
            0.06  0.0  \\
            0.07  0.0  \\
            0.08  0.0  \\
            0.09  0.0  \\
            0.1  0.0  \\
            0.11  0.0  \\
            0.12  0.0  \\
            0.13  0.0  \\
            0.14  0.0  \\
            0.15  0.0  \\
            0.16  0.0  \\
            0.17  0.0  \\
            0.18  0.0  \\
            0.19  0.0  \\
            0.2  0.0  \\
            0.21  0.0  \\
            0.22  0.0  \\
            0.23  0.0  \\
            0.24  0.0  \\
            0.25  0.0  \\
            0.26  0.0  \\
            0.27  0.0  \\
            0.28  0.0  \\
            0.29  0.0  \\
            0.3  0.0  \\
            0.31  0.0  \\
            0.32  0.0  \\
            0.33  0.0  \\
            0.34  0.0  \\
            0.35  0.0  \\
            0.36  0.0  \\
            0.37  0.0  \\
            0.38  0.0  \\
            0.39  0.0  \\
            0.4  0.0  \\
            0.41  0.0  \\
            0.42  0.0  \\
            0.43  0.0  \\
            0.44  0.0  \\
            0.45  0.0  \\
            0.46  0.0  \\
            0.47  0.0  \\
            0.48  0.0  \\
            0.49  0.0  \\
            0.5  0.0  \\
            0.51  0.0  \\
            0.52  0.0  \\
            0.53  0.0  \\
            0.54  0.0  \\
            0.55  0.0  \\
            0.56  0.0  \\
            0.57  0.0  \\
            0.58  0.0  \\
            0.59  0.0  \\
            0.6  0.0  \\
            0.61  0.0  \\
            0.62  0.0  \\
            0.63  0.0  \\
            0.64  0.0  \\
            0.65  0.0  \\
            0.66  0.0  \\
            0.67  0.0  \\
            0.68  0.0  \\
            0.69  0.0  \\
            0.7  0.0  \\
            0.71  0.0  \\
            0.72  0.0  \\
            0.73  0.0  \\
            0.74  0.0  \\
            0.75  0.0  \\
            0.76  0.0  \\
            0.77  0.0  \\
            0.78  0.0  \\
            0.79  0.0  \\
            0.8  0.0  \\
            0.81  0.0  \\
            0.82  0.0  \\
            0.83  0.0  \\
            0.84  0.0  \\
            0.85  0.0  \\
            0.86  0.0  \\
            0.87  0.0  \\
            0.88  0.0  \\
            0.89  0.0  \\
            0.9  0.0  \\
            0.91  0.0  \\
            0.92  0.0  \\
            0.93  0.0  \\
            0.94  0.0  \\
            0.95  0.0  \\
            0.96  0.0  \\
            0.97  0.0  \\
            0.98  0.0  \\
            0.99  0.0  \\
            1.0  0.0  \\
        }
        ;
    \nextgroupplot[title={$\symbf{M3}$}]
    \addplot+[thick]
        table[row sep={\\}]
        {
            \\
            0.0  0.0  \\
            0.01  0.0  \\
            0.02  0.0  \\
            0.03  0.0  \\
            0.04  0.0  \\
            0.05  0.0  \\
            0.06  0.0  \\
            0.07  0.0  \\
            0.08  0.0  \\
            0.09  0.0  \\
            0.1  0.0  \\
            0.11  0.0  \\
            0.12  0.0  \\
            0.13  0.0  \\
            0.14  0.0  \\
            0.15  0.0  \\
            0.16  0.0  \\
            0.17  0.0  \\
            0.18  0.0  \\
            0.19  0.0  \\
            0.2  0.0  \\
            0.21  0.0  \\
            0.22  0.0  \\
            0.23  0.0  \\
            0.24  0.0  \\
            0.25  0.0  \\
            0.26  0.0  \\
            0.27  0.0  \\
            0.28  0.0  \\
            0.29  0.0  \\
            0.3  0.0  \\
            0.31  0.0  \\
            0.32  0.0  \\
            0.33  0.0  \\
            0.34  0.0  \\
            0.35  0.0  \\
            0.36  0.0  \\
            0.37  0.0  \\
            0.38  0.0  \\
            0.39  0.0  \\
            0.4  0.0  \\
            0.41  0.0  \\
            0.42  0.0  \\
            0.43  0.0  \\
            0.44  0.0  \\
            0.45  0.0  \\
            0.46  0.0  \\
            0.47  0.0  \\
            0.48  0.0  \\
            0.49  0.0  \\
            0.5  0.0  \\
            0.51  0.0  \\
            0.52  0.0  \\
            0.53  0.0  \\
            0.54  0.0  \\
            0.55  0.0  \\
            0.56  0.0  \\
            0.57  0.0  \\
            0.58  0.0  \\
            0.59  0.0  \\
            0.6  0.0  \\
            0.61  0.0  \\
            0.62  0.0  \\
            0.63  0.0  \\
            0.64  0.0  \\
            0.65  0.0  \\
            0.66  0.0  \\
            0.67  0.0  \\
            0.68  0.0  \\
            0.69  0.0  \\
            0.7  0.0  \\
            0.71  0.0  \\
            0.72  0.0  \\
            0.73  0.0  \\
            0.74  0.0  \\
            0.75  0.0  \\
            0.76  0.0  \\
            0.77  0.0  \\
            0.78  0.0  \\
            0.79  0.0  \\
            0.8  0.0  \\
            0.81  0.0  \\
            0.82  0.0  \\
            0.83  0.0  \\
            0.84  0.0  \\
            0.85  0.0  \\
            0.86  0.0  \\
            0.87  0.0  \\
            0.88  0.0  \\
            0.89  0.0  \\
            0.9  0.0  \\
            0.91  0.0  \\
            0.92  0.0  \\
            0.93  0.0  \\
            0.94  0.0  \\
            0.95  0.0  \\
            0.96  0.0  \\
            0.97  0.0  \\
            0.98  0.0  \\
            0.99  0.0  \\
            1.0  0.0  \\
        }
        ;
    \nextgroupplot[ylabel={$\symbf{D}_{\mathrm{b}}$}]
    \addplot+[thick]
        table[row sep={\\}]
        {
            \\
            0.0  0.0  \\
            0.01  0.0  \\
            0.02  0.0  \\
            0.03  0.0  \\
            0.04  0.0  \\
            0.05  0.0  \\
            0.06  0.0  \\
            0.07  0.0  \\
            0.08  0.0  \\
            0.09  0.0  \\
            0.1  0.0  \\
            0.11  0.0  \\
            0.12  0.0  \\
            0.13  0.0  \\
            0.14  0.0  \\
            0.15  0.0  \\
            0.16  0.0  \\
            0.17  0.0  \\
            0.18  0.0  \\
            0.19  0.0  \\
            0.2  0.0  \\
            0.21  0.0  \\
            0.22  0.0  \\
            0.23  0.0  \\
            0.24  0.0  \\
            0.25  0.0  \\
            0.26  0.0  \\
            0.27  0.0  \\
            0.28  0.0  \\
            0.29  0.0  \\
            0.3  0.0  \\
            0.31  0.0  \\
            0.32  0.0  \\
            0.33  0.0  \\
            0.34  0.0  \\
            0.35  0.0  \\
            0.36  0.0  \\
            0.37  0.0  \\
            0.38  0.0  \\
            0.39  0.0  \\
            0.4  0.0  \\
            0.41  0.0  \\
            0.42  0.0  \\
            0.43  0.0  \\
            0.44  0.0  \\
            0.45  0.0  \\
            0.46  0.0  \\
            0.47  0.0  \\
            0.48  0.0  \\
            0.49  0.0  \\
            0.5  0.0  \\
            0.51  0.0  \\
            0.52  0.0  \\
            0.53  0.0  \\
            0.54  0.0  \\
            0.55  0.0  \\
            0.56  0.0  \\
            0.57  0.0  \\
            0.58  0.0  \\
            0.59  0.0  \\
            0.6  0.0  \\
            0.61  0.0  \\
            0.62  0.0  \\
            0.63  0.0  \\
            0.64  0.0  \\
            0.65  0.0  \\
            0.66  0.0  \\
            0.67  0.0  \\
            0.68  0.0  \\
            0.69  0.0  \\
            0.7  0.0  \\
            0.71  0.0  \\
            0.72  0.0  \\
            0.73  0.0  \\
            0.74  0.0  \\
            0.75  0.0  \\
            0.76  0.0  \\
            0.77  0.0  \\
            0.78  0.0  \\
            0.79  0.0  \\
            0.8  0.0  \\
            0.81  0.0  \\
            0.82  0.0  \\
            0.83  0.0  \\
            0.84  0.0  \\
            0.85  0.0  \\
            0.86  0.0  \\
            0.87  0.0  \\
            0.88  0.0  \\
            0.89  0.0  \\
            0.9  0.0  \\
            0.91  0.0  \\
            0.92  0.0  \\
            0.93  0.0  \\
            0.94  0.0  \\
            0.95  0.0  \\
            0.96  0.0  \\
            0.97  0.0  \\
            0.98  0.0  \\
            0.99  0.0  \\
            1.0  1.0  \\
        }
        ;
    \addplot+[dashed, thick]
        coordinates {
            (0,0)
            (1,1)
        }
        ;
    \nextgroupplot
    \addplot+[thick]
        table[row sep={\\}]
        {
            \\
            0.0  1.0  \\
            0.01  0.9843  \\
            0.02  0.8808  \\
            0.03  0.6994  \\
            0.04  0.5196000000000001  \\
            0.05  0.37060000000000004  \\
            0.06  0.25970000000000004  \\
            0.07  0.18200000000000005  \\
            0.08  0.12390000000000001  \\
            0.09  0.08450000000000002  \\
            0.1  0.056599999999999984  \\
            0.11  0.03739999999999999  \\
            0.12  0.025900000000000034  \\
            0.13  0.015700000000000047  \\
            0.14  0.0121  \\
            0.15  0.008099999999999996  \\
            0.16  0.005199999999999982  \\
            0.17  0.0038000000000000256  \\
            0.18  0.0024999999999999467  \\
            0.19  0.0013999999999999568  \\
            0.2  0.0010000000000000009  \\
            0.21  0.0004999999999999449  \\
            0.22  0.00039999999999995595  \\
            0.23  9.999999999998899e-5  \\
            0.24  0.0  \\
            0.25  0.0  \\
            0.26  0.0  \\
            0.27  0.0  \\
            0.28  0.0  \\
            0.29  0.0  \\
            0.3  0.0  \\
            0.31  0.0  \\
            0.32  0.0  \\
            0.33  0.0  \\
            0.34  0.0  \\
            0.35  0.0  \\
            0.36  0.0  \\
            0.37  0.0  \\
            0.38  0.0  \\
            0.39  0.0  \\
            0.4  0.0  \\
            0.41  0.0  \\
            0.42  0.0  \\
            0.43  0.0  \\
            0.44  0.0  \\
            0.45  0.0  \\
            0.46  0.0  \\
            0.47  0.0  \\
            0.48  0.0  \\
            0.49  0.0  \\
            0.5  0.0  \\
            0.51  0.0  \\
            0.52  0.0  \\
            0.53  0.0  \\
            0.54  0.0  \\
            0.55  0.0  \\
            0.56  0.0  \\
            0.57  0.0  \\
            0.58  0.0  \\
            0.59  0.0  \\
            0.6  0.0  \\
            0.61  0.0  \\
            0.62  0.0  \\
            0.63  0.0  \\
            0.64  0.0  \\
            0.65  0.0  \\
            0.66  0.0  \\
            0.67  0.0  \\
            0.68  0.0  \\
            0.69  0.0  \\
            0.7  0.0  \\
            0.71  0.0  \\
            0.72  0.0  \\
            0.73  0.0  \\
            0.74  0.0  \\
            0.75  0.0  \\
            0.76  0.0  \\
            0.77  0.0  \\
            0.78  0.0  \\
            0.79  0.0  \\
            0.8  0.0  \\
            0.81  0.0  \\
            0.82  0.0  \\
            0.83  0.0  \\
            0.84  0.0  \\
            0.85  0.0  \\
            0.86  0.0  \\
            0.87  0.0  \\
            0.88  0.0  \\
            0.89  0.0  \\
            0.9  0.0  \\
            0.91  0.0  \\
            0.92  0.0  \\
            0.93  0.0  \\
            0.94  0.0  \\
            0.95  0.0  \\
            0.96  0.0  \\
            0.97  0.0  \\
            0.98  0.0  \\
            0.99  0.0  \\
            1.0  0.0  \\
        }
        ;
    \nextgroupplot
    \addplot+[thick]
        table[row sep={\\}]
        {
            \\
            0.0  1.0  \\
            0.01  1.0  \\
            0.02  1.0  \\
            0.03  1.0  \\
            0.04  1.0  \\
            0.05  1.0  \\
            0.06  1.0  \\
            0.07  1.0  \\
            0.08  1.0  \\
            0.09  1.0  \\
            0.1  1.0  \\
            0.11  1.0  \\
            0.12  1.0  \\
            0.13  1.0  \\
            0.14  1.0  \\
            0.15  1.0  \\
            0.16  1.0  \\
            0.17  1.0  \\
            0.18  1.0  \\
            0.19  1.0  \\
            0.2  1.0  \\
            0.21  1.0  \\
            0.22  1.0  \\
            0.23  1.0  \\
            0.24  1.0  \\
            0.25  1.0  \\
            0.26  1.0  \\
            0.27  1.0  \\
            0.28  1.0  \\
            0.29  1.0  \\
            0.3  1.0  \\
            0.31  1.0  \\
            0.32  1.0  \\
            0.33  1.0  \\
            0.34  1.0  \\
            0.35  1.0  \\
            0.36  1.0  \\
            0.37  1.0  \\
            0.38  1.0  \\
            0.39  1.0  \\
            0.4  1.0  \\
            0.41  1.0  \\
            0.42  1.0  \\
            0.43  1.0  \\
            0.44  1.0  \\
            0.45  1.0  \\
            0.46  1.0  \\
            0.47  1.0  \\
            0.48  1.0  \\
            0.49  1.0  \\
            0.5  1.0  \\
            0.51  1.0  \\
            0.52  1.0  \\
            0.53  1.0  \\
            0.54  1.0  \\
            0.55  1.0  \\
            0.56  1.0  \\
            0.57  1.0  \\
            0.58  1.0  \\
            0.59  1.0  \\
            0.6  1.0  \\
            0.61  1.0  \\
            0.62  1.0  \\
            0.63  1.0  \\
            0.64  1.0  \\
            0.65  1.0  \\
            0.66  1.0  \\
            0.67  1.0  \\
            0.68  0.9998  \\
            0.69  0.9997  \\
            0.7  0.9993  \\
            0.71  0.999  \\
            0.72  0.9981  \\
            0.73  0.9967  \\
            0.74  0.994  \\
            0.75  0.9886  \\
            0.76  0.9777  \\
            0.77  0.9618  \\
            0.78  0.9367  \\
            0.79  0.901  \\
            0.8  0.8527  \\
            0.81  0.7873  \\
            0.82  0.6995  \\
            0.83  0.5954999999999999  \\
            0.84  0.472  \\
            0.85  0.34709999999999996  \\
            0.86  0.23250000000000004  \\
            0.87  0.14559999999999995  \\
            0.88  0.07789999999999997  \\
            0.89  0.03590000000000004  \\
            0.9  0.01419999999999999  \\
            0.91  0.0048000000000000265  \\
            0.92  0.0010999999999999899  \\
            0.93  0.00039999999999995595  \\
            0.94  9.999999999998899e-5  \\
            0.95  0.0  \\
            0.96  0.0  \\
            0.97  0.0  \\
            0.98  0.0  \\
            0.99  0.0  \\
            1.0  0.0  \\
        }
        ;
    \nextgroupplot[ylabel={$\symbf{D}_{\mathrm{uq}}$}]
    \addplot+[thick]
        table[row sep={\\}]
        {
            \\
            0.0  0.0  \\
            0.01  0.0  \\
            0.02  0.0  \\
            0.03  0.0  \\
            0.04  0.0  \\
            0.05  0.0  \\
            0.06  0.0  \\
            0.07  0.0  \\
            0.08  0.0  \\
            0.09  0.0  \\
            0.1  0.0  \\
            0.11  0.0  \\
            0.12  0.0  \\
            0.13  0.0  \\
            0.14  0.0  \\
            0.15  0.0  \\
            0.16  0.0  \\
            0.17  0.0  \\
            0.18  0.0  \\
            0.19  0.0  \\
            0.2  0.0  \\
            0.21  0.0  \\
            0.22  0.0  \\
            0.23  0.0  \\
            0.24  0.0  \\
            0.25  0.0  \\
            0.26  0.0  \\
            0.27  0.0  \\
            0.28  0.0  \\
            0.29  0.0  \\
            0.3  0.0  \\
            0.31  0.0  \\
            0.32  0.0  \\
            0.33  0.0  \\
            0.34  0.0  \\
            0.35  0.0  \\
            0.36  0.0  \\
            0.37  0.0  \\
            0.38  0.0  \\
            0.39  0.0  \\
            0.4  0.0  \\
            0.41  0.0  \\
            0.42  0.0  \\
            0.43  0.0  \\
            0.44  0.0  \\
            0.45  0.0  \\
            0.46  0.0  \\
            0.47  0.0  \\
            0.48  0.0  \\
            0.49  0.0  \\
            0.5  0.0  \\
            0.51  0.0  \\
            0.52  0.0  \\
            0.53  0.0  \\
            0.54  0.0  \\
            0.55  0.0  \\
            0.56  0.0  \\
            0.57  0.0  \\
            0.58  0.0  \\
            0.59  0.0  \\
            0.6  0.0  \\
            0.61  0.0  \\
            0.62  0.0  \\
            0.63  0.0  \\
            0.64  0.0  \\
            0.65  0.0  \\
            0.66  0.0  \\
            0.67  0.0  \\
            0.68  0.0  \\
            0.69  0.0  \\
            0.7  0.0  \\
            0.71  0.0  \\
            0.72  0.0  \\
            0.73  0.0  \\
            0.74  0.0  \\
            0.75  0.0  \\
            0.76  0.0  \\
            0.77  0.0  \\
            0.78  0.0  \\
            0.79  0.0  \\
            0.8  0.0  \\
            0.81  0.0  \\
            0.82  0.0  \\
            0.83  0.0  \\
            0.84  0.0  \\
            0.85  0.0  \\
            0.86  0.0  \\
            0.87  0.0  \\
            0.88  0.0  \\
            0.89  0.0  \\
            0.9  0.0  \\
            0.91  0.0  \\
            0.92  0.0  \\
            0.93  0.0  \\
            0.94  0.0  \\
            0.95  0.0  \\
            0.96  0.0  \\
            0.97  0.0  \\
            0.98  0.0  \\
            0.99  0.0  \\
            1.0  1.0  \\
        }
        ;
    \addplot+[dashed, thick]
        coordinates {
            (0,0)
            (1,1)
        }
        ;
    \nextgroupplot
    \addplot+[thick]
        table[row sep={\\}]
        {
            \\
            0.0  1.0  \\
            0.01  1.0  \\
            0.02  1.0  \\
            0.03  1.0  \\
            0.04  1.0  \\
            0.05  1.0  \\
            0.06  1.0  \\
            0.07  1.0  \\
            0.08  1.0  \\
            0.09  1.0  \\
            0.1  1.0  \\
            0.11  1.0  \\
            0.12  1.0  \\
            0.13  1.0  \\
            0.14  1.0  \\
            0.15  1.0  \\
            0.16  1.0  \\
            0.17  1.0  \\
            0.18  1.0  \\
            0.19  1.0  \\
            0.2  1.0  \\
            0.21  1.0  \\
            0.22  1.0  \\
            0.23  1.0  \\
            0.24  1.0  \\
            0.25  1.0  \\
            0.26  1.0  \\
            0.27  1.0  \\
            0.28  1.0  \\
            0.29  1.0  \\
            0.3  1.0  \\
            0.31  1.0  \\
            0.32  1.0  \\
            0.33  1.0  \\
            0.34  1.0  \\
            0.35  1.0  \\
            0.36  1.0  \\
            0.37  1.0  \\
            0.38  1.0  \\
            0.39  1.0  \\
            0.4  1.0  \\
            0.41  1.0  \\
            0.42  1.0  \\
            0.43  1.0  \\
            0.44  1.0  \\
            0.45  1.0  \\
            0.46  1.0  \\
            0.47  1.0  \\
            0.48  1.0  \\
            0.49  1.0  \\
            0.5  1.0  \\
            0.51  1.0  \\
            0.52  1.0  \\
            0.53  1.0  \\
            0.54  1.0  \\
            0.55  1.0  \\
            0.56  1.0  \\
            0.57  1.0  \\
            0.58  1.0  \\
            0.59  1.0  \\
            0.6  1.0  \\
            0.61  1.0  \\
            0.62  1.0  \\
            0.63  1.0  \\
            0.64  1.0  \\
            0.65  1.0  \\
            0.66  1.0  \\
            0.67  1.0  \\
            0.68  1.0  \\
            0.69  1.0  \\
            0.7  0.9999  \\
            0.71  0.9998  \\
            0.72  0.9997  \\
            0.73  0.9993  \\
            0.74  0.9982  \\
            0.75  0.9962  \\
            0.76  0.9931  \\
            0.77  0.9877  \\
            0.78  0.9818  \\
            0.79  0.9711  \\
            0.8  0.952  \\
            0.81  0.9261  \\
            0.82  0.8895  \\
            0.83  0.8322  \\
            0.84  0.7575000000000001  \\
            0.85  0.6692  \\
            0.86  0.5622  \\
            0.87  0.4465  \\
            0.88  0.32830000000000004  \\
            0.89  0.2278  \\
            0.9  0.13970000000000005  \\
            0.91  0.07399999999999995  \\
            0.92  0.03149999999999997  \\
            0.93  0.012199999999999989  \\
            0.94  0.0030999999999999917  \\
            0.95  0.00039999999999995595  \\
            0.96  0.0  \\
            0.97  0.0  \\
            0.98  0.0  \\
            0.99  0.0  \\
            1.0  0.0  \\
        }
        ;
    \nextgroupplot
    \addplot+[thick]
        table[row sep={\\}]
        {
            \\
            0.0  1.0  \\
            0.01  1.0  \\
            0.02  1.0  \\
            0.03  1.0  \\
            0.04  1.0  \\
            0.05  1.0  \\
            0.06  1.0  \\
            0.07  1.0  \\
            0.08  1.0  \\
            0.09  1.0  \\
            0.1  1.0  \\
            0.11  1.0  \\
            0.12  1.0  \\
            0.13  1.0  \\
            0.14  1.0  \\
            0.15  1.0  \\
            0.16  1.0  \\
            0.17  1.0  \\
            0.18  1.0  \\
            0.19  1.0  \\
            0.2  1.0  \\
            0.21  1.0  \\
            0.22  1.0  \\
            0.23  1.0  \\
            0.24  1.0  \\
            0.25  1.0  \\
            0.26  1.0  \\
            0.27  1.0  \\
            0.28  1.0  \\
            0.29  1.0  \\
            0.3  1.0  \\
            0.31  1.0  \\
            0.32  1.0  \\
            0.33  1.0  \\
            0.34  1.0  \\
            0.35  1.0  \\
            0.36  1.0  \\
            0.37  1.0  \\
            0.38  1.0  \\
            0.39  1.0  \\
            0.4  1.0  \\
            0.41  1.0  \\
            0.42  1.0  \\
            0.43  1.0  \\
            0.44  1.0  \\
            0.45  1.0  \\
            0.46  1.0  \\
            0.47  1.0  \\
            0.48  1.0  \\
            0.49  1.0  \\
            0.5  1.0  \\
            0.51  1.0  \\
            0.52  1.0  \\
            0.53  1.0  \\
            0.54  1.0  \\
            0.55  1.0  \\
            0.56  1.0  \\
            0.57  1.0  \\
            0.58  1.0  \\
            0.59  1.0  \\
            0.6  1.0  \\
            0.61  1.0  \\
            0.62  1.0  \\
            0.63  1.0  \\
            0.64  1.0  \\
            0.65  1.0  \\
            0.66  1.0  \\
            0.67  1.0  \\
            0.68  1.0  \\
            0.69  1.0  \\
            0.7  1.0  \\
            0.71  1.0  \\
            0.72  1.0  \\
            0.73  1.0  \\
            0.74  1.0  \\
            0.75  1.0  \\
            0.76  1.0  \\
            0.77  1.0  \\
            0.78  1.0  \\
            0.79  1.0  \\
            0.8  1.0  \\
            0.81  1.0  \\
            0.82  1.0  \\
            0.83  1.0  \\
            0.84  1.0  \\
            0.85  1.0  \\
            0.86  1.0  \\
            0.87  1.0  \\
            0.88  1.0  \\
            0.89  1.0  \\
            0.9  1.0  \\
            0.91  1.0  \\
            0.92  1.0  \\
            0.93  1.0  \\
            0.94  1.0  \\
            0.95  1.0  \\
            0.96  1.0  \\
            0.97  1.0  \\
            0.98  1.0  \\
            0.99  1.0  \\
            1.0  0.0  \\
        }
        ;
    \nextgroupplot[ylabel={$\symbf{D}_{\mathrm{l}}$}]
    \addplot+[thick]
        table[row sep={\\}]
        {
            \\
            0.0  0.0  \\
            0.01  0.0  \\
            0.02  0.0  \\
            0.03  0.0  \\
            0.04  0.0  \\
            0.05  0.0  \\
            0.06  0.0  \\
            0.07  0.0  \\
            0.08  0.0  \\
            0.09  0.0  \\
            0.1  0.0  \\
            0.11  0.0  \\
            0.12  0.0  \\
            0.13  0.0  \\
            0.14  0.0  \\
            0.15  0.0  \\
            0.16  0.0  \\
            0.17  0.0  \\
            0.18  0.0  \\
            0.19  0.0  \\
            0.2  0.0  \\
            0.21  0.0  \\
            0.22  0.0  \\
            0.23  0.0  \\
            0.24  0.0  \\
            0.25  0.0  \\
            0.26  0.0  \\
            0.27  0.0  \\
            0.28  0.0  \\
            0.29  0.0  \\
            0.3  0.0  \\
            0.31  0.0  \\
            0.32  0.0  \\
            0.33  0.0  \\
            0.34  0.0  \\
            0.35  0.0  \\
            0.36  0.0  \\
            0.37  0.0  \\
            0.38  0.0  \\
            0.39  0.0  \\
            0.4  0.0  \\
            0.41  0.0  \\
            0.42  0.0  \\
            0.43  0.0  \\
            0.44  0.0  \\
            0.45  0.0  \\
            0.46  0.0  \\
            0.47  0.0  \\
            0.48  0.0  \\
            0.49  0.0  \\
            0.5  0.0  \\
            0.51  0.0  \\
            0.52  0.0  \\
            0.53  0.0  \\
            0.54  0.0  \\
            0.55  0.0  \\
            0.56  0.0  \\
            0.57  0.0  \\
            0.58  0.0  \\
            0.59  0.0  \\
            0.6  0.0  \\
            0.61  0.0  \\
            0.62  0.0  \\
            0.63  0.0  \\
            0.64  0.0  \\
            0.65  0.0  \\
            0.66  0.0  \\
            0.67  0.0  \\
            0.68  0.0  \\
            0.69  0.0  \\
            0.7  0.0  \\
            0.71  0.0  \\
            0.72  0.0  \\
            0.73  0.0  \\
            0.74  0.0  \\
            0.75  0.0  \\
            0.76  0.0  \\
            0.77  0.0  \\
            0.78  0.0  \\
            0.79  0.0  \\
            0.8  0.0  \\
            0.81  0.0  \\
            0.82  0.0  \\
            0.83  0.0  \\
            0.84  0.0  \\
            0.85  0.0  \\
            0.86  0.0  \\
            0.87  0.0  \\
            0.88  0.0  \\
            0.89  0.0  \\
            0.9  0.0  \\
            0.91  0.0  \\
            0.92  0.0  \\
            0.93  0.0  \\
            0.94  0.0  \\
            0.95  0.0  \\
            0.96  0.0  \\
            0.97  0.0  \\
            0.98  0.0  \\
            0.99  0.0006  \\
            1.0  1.0  \\
        }
        ;
    \addplot+[dashed, thick]
        coordinates {
            (0,0)
            (1,1)
        }
        ;
    \nextgroupplot
    \addplot+[thick]
        table[row sep={\\}]
        {
            \\
            0.0  1.0  \\
            0.01  1.0  \\
            0.02  1.0  \\
            0.03  1.0  \\
            0.04  1.0  \\
            0.05  1.0  \\
            0.06  1.0  \\
            0.07  1.0  \\
            0.08  1.0  \\
            0.09  1.0  \\
            0.1  1.0  \\
            0.11  1.0  \\
            0.12  1.0  \\
            0.13  1.0  \\
            0.14  1.0  \\
            0.15  1.0  \\
            0.16  1.0  \\
            0.17  1.0  \\
            0.18  1.0  \\
            0.19  1.0  \\
            0.2  1.0  \\
            0.21  1.0  \\
            0.22  1.0  \\
            0.23  1.0  \\
            0.24  1.0  \\
            0.25  1.0  \\
            0.26  1.0  \\
            0.27  1.0  \\
            0.28  1.0  \\
            0.29  1.0  \\
            0.3  1.0  \\
            0.31  1.0  \\
            0.32  1.0  \\
            0.33  1.0  \\
            0.34  1.0  \\
            0.35  1.0  \\
            0.36  1.0  \\
            0.37  1.0  \\
            0.38  1.0  \\
            0.39  1.0  \\
            0.4  1.0  \\
            0.41  1.0  \\
            0.42  1.0  \\
            0.43  1.0  \\
            0.44  1.0  \\
            0.45  1.0  \\
            0.46  1.0  \\
            0.47  1.0  \\
            0.48  1.0  \\
            0.49  1.0  \\
            0.5  1.0  \\
            0.51  1.0  \\
            0.52  1.0  \\
            0.53  1.0  \\
            0.54  1.0  \\
            0.55  1.0  \\
            0.56  1.0  \\
            0.57  1.0  \\
            0.58  1.0  \\
            0.59  1.0  \\
            0.6  1.0  \\
            0.61  0.9999  \\
            0.62  0.9997  \\
            0.63  0.9995  \\
            0.64  0.9993  \\
            0.65  0.9989  \\
            0.66  0.9987  \\
            0.67  0.9982  \\
            0.68  0.9973  \\
            0.69  0.9961  \\
            0.7  0.9951  \\
            0.71  0.9925  \\
            0.72  0.9888  \\
            0.73  0.985  \\
            0.74  0.9773  \\
            0.75  0.969  \\
            0.76  0.9596  \\
            0.77  0.9449  \\
            0.78  0.9262  \\
            0.79  0.9023  \\
            0.8  0.8748  \\
            0.81  0.836  \\
            0.82  0.7911  \\
            0.83  0.7404999999999999  \\
            0.84  0.6813  \\
            0.85  0.6166  \\
            0.86  0.5448  \\
            0.87  0.47019999999999995  \\
            0.88  0.39339999999999997  \\
            0.89  0.3164  \\
            0.9  0.23939999999999995  \\
            0.91  0.17420000000000002  \\
            0.92  0.11980000000000002  \\
            0.93  0.07279999999999998  \\
            0.94  0.03969999999999996  \\
            0.95  0.01870000000000005  \\
            0.96  0.007099999999999995  \\
            0.97  0.0022999999999999687  \\
            0.98  0.00039999999999995595  \\
            0.99  0.0  \\
            1.0  0.0  \\
        }
        ;
    \nextgroupplot
    \addplot+[thick]
        table[row sep={\\}]
        {
            \\
            0.0  1.0  \\
            0.01  1.0  \\
            0.02  1.0  \\
            0.03  1.0  \\
            0.04  1.0  \\
            0.05  1.0  \\
            0.06  1.0  \\
            0.07  1.0  \\
            0.08  1.0  \\
            0.09  1.0  \\
            0.1  1.0  \\
            0.11  1.0  \\
            0.12  1.0  \\
            0.13  1.0  \\
            0.14  1.0  \\
            0.15  1.0  \\
            0.16  1.0  \\
            0.17  1.0  \\
            0.18  1.0  \\
            0.19  1.0  \\
            0.2  1.0  \\
            0.21  1.0  \\
            0.22  1.0  \\
            0.23  1.0  \\
            0.24  1.0  \\
            0.25  1.0  \\
            0.26  1.0  \\
            0.27  1.0  \\
            0.28  1.0  \\
            0.29  1.0  \\
            0.3  1.0  \\
            0.31  1.0  \\
            0.32  1.0  \\
            0.33  1.0  \\
            0.34  1.0  \\
            0.35  1.0  \\
            0.36  1.0  \\
            0.37  1.0  \\
            0.38  1.0  \\
            0.39  1.0  \\
            0.4  1.0  \\
            0.41  1.0  \\
            0.42  1.0  \\
            0.43  1.0  \\
            0.44  1.0  \\
            0.45  1.0  \\
            0.46  1.0  \\
            0.47  1.0  \\
            0.48  1.0  \\
            0.49  1.0  \\
            0.5  1.0  \\
            0.51  1.0  \\
            0.52  1.0  \\
            0.53  1.0  \\
            0.54  1.0  \\
            0.55  1.0  \\
            0.56  1.0  \\
            0.57  1.0  \\
            0.58  1.0  \\
            0.59  1.0  \\
            0.6  1.0  \\
            0.61  1.0  \\
            0.62  1.0  \\
            0.63  1.0  \\
            0.64  1.0  \\
            0.65  1.0  \\
            0.66  1.0  \\
            0.67  1.0  \\
            0.68  1.0  \\
            0.69  1.0  \\
            0.7  1.0  \\
            0.71  1.0  \\
            0.72  1.0  \\
            0.73  1.0  \\
            0.74  1.0  \\
            0.75  1.0  \\
            0.76  1.0  \\
            0.77  1.0  \\
            0.78  1.0  \\
            0.79  1.0  \\
            0.8  1.0  \\
            0.81  1.0  \\
            0.82  1.0  \\
            0.83  1.0  \\
            0.84  1.0  \\
            0.85  1.0  \\
            0.86  1.0  \\
            0.87  1.0  \\
            0.88  1.0  \\
            0.89  1.0  \\
            0.9  0.9998  \\
            0.91  0.9996  \\
            0.92  0.999  \\
            0.93  0.9977  \\
            0.94  0.9948  \\
            0.95  0.9877  \\
            0.96  0.9723  \\
            0.97  0.9383  \\
            0.98  0.8665  \\
            0.99  0.7092  \\
            1.0  0.0  \\
        }
        ;
    \nextgroupplot[ylabel={$\symbf{A}_{\mathrm{uq}}$}]
    \addplot+[thick]
        table[row sep={\\}]
        {
            \\
            0.0  0.0003  \\
            0.01  0.0057  \\
            0.02  0.0125  \\
            0.03  0.0204  \\
            0.04  0.0285  \\
            0.05  0.0365  \\
            0.06  0.0441  \\
            0.07  0.0533  \\
            0.08  0.0625  \\
            0.09  0.072  \\
            0.1  0.0806  \\
            0.11  0.0891  \\
            0.12  0.1009  \\
            0.13  0.112  \\
            0.14  0.123  \\
            0.15  0.1317  \\
            0.16  0.1433  \\
            0.17  0.1521  \\
            0.18  0.1613  \\
            0.19  0.1713  \\
            0.2  0.1813  \\
            0.21  0.1907  \\
            0.22  0.1996  \\
            0.23  0.2127  \\
            0.24  0.2227  \\
            0.25  0.2332  \\
            0.26  0.243  \\
            0.27  0.2554  \\
            0.28  0.265  \\
            0.29  0.2761  \\
            0.3  0.285  \\
            0.31  0.2971  \\
            0.32  0.307  \\
            0.33  0.3185  \\
            0.34  0.3304  \\
            0.35  0.3422  \\
            0.36  0.3505  \\
            0.37  0.3621  \\
            0.38  0.3728  \\
            0.39  0.3836  \\
            0.4  0.3951  \\
            0.41  0.4049  \\
            0.42  0.4154  \\
            0.43  0.4265  \\
            0.44  0.4373  \\
            0.45  0.4476  \\
            0.46  0.4577  \\
            0.47  0.4686  \\
            0.48  0.4811  \\
            0.49  0.4921  \\
            0.5  0.5017  \\
            0.51  0.5122  \\
            0.52  0.5227  \\
            0.53  0.5345  \\
            0.54  0.546  \\
            0.55  0.5595  \\
            0.56  0.5716  \\
            0.57  0.5828  \\
            0.58  0.5928  \\
            0.59  0.6037  \\
            0.6  0.6165  \\
            0.61  0.6256  \\
            0.62  0.6358  \\
            0.63  0.6472  \\
            0.64  0.6592  \\
            0.65  0.6707  \\
            0.66  0.6829  \\
            0.67  0.6941  \\
            0.68  0.7038  \\
            0.69  0.716  \\
            0.7  0.7265  \\
            0.71  0.7385  \\
            0.72  0.7483  \\
            0.73  0.7608  \\
            0.74  0.7747  \\
            0.75  0.7859  \\
            0.76  0.7962  \\
            0.77  0.8055  \\
            0.78  0.8151  \\
            0.79  0.8261  \\
            0.8  0.8361  \\
            0.81  0.847  \\
            0.82  0.8564  \\
            0.83  0.8695  \\
            0.84  0.8783  \\
            0.85  0.8899  \\
            0.86  0.899  \\
            0.87  0.9069  \\
            0.88  0.9164  \\
            0.89  0.9261  \\
            0.9  0.9356  \\
            0.91  0.9429  \\
            0.92  0.9522  \\
            0.93  0.9599  \\
            0.94  0.9683  \\
            0.95  0.9763  \\
            0.96  0.9837  \\
            0.97  0.9893  \\
            0.98  0.9946  \\
            0.99  0.9984  \\
            1.0  1.0  \\
        }
        ;
    \addplot+[dashed, thick]
        coordinates {
            (0,0)
            (1,1)
        }
        ;
    \nextgroupplot
    \addplot+[thick]
        table[row sep={\\}]
        {
            \\
            0.0  0.0  \\
            0.01  0.0  \\
            0.02  0.0  \\
            0.03  0.0  \\
            0.04  0.0  \\
            0.05  0.0  \\
            0.06  0.0  \\
            0.07  0.0  \\
            0.08  0.0  \\
            0.09  0.0  \\
            0.1  0.0  \\
            0.11  0.0  \\
            0.12  0.0  \\
            0.13  0.0  \\
            0.14  0.0  \\
            0.15  0.0  \\
            0.16  0.0  \\
            0.17  0.0  \\
            0.18  0.0  \\
            0.19  0.0  \\
            0.2  0.0  \\
            0.21  0.0  \\
            0.22  0.0  \\
            0.23  0.0  \\
            0.24  0.0  \\
            0.25  0.0  \\
            0.26  0.0  \\
            0.27  0.0  \\
            0.28  0.0  \\
            0.29  0.0  \\
            0.3  0.0  \\
            0.31  0.0  \\
            0.32  0.0  \\
            0.33  0.0  \\
            0.34  0.0  \\
            0.35  0.0  \\
            0.36  0.0  \\
            0.37  0.0  \\
            0.38  0.0  \\
            0.39  0.0  \\
            0.4  0.0  \\
            0.41  0.0  \\
            0.42  0.0  \\
            0.43  0.0  \\
            0.44  0.0  \\
            0.45  0.0  \\
            0.46  0.0  \\
            0.47  0.0  \\
            0.48  0.0  \\
            0.49  0.0  \\
            0.5  0.0  \\
            0.51  0.0  \\
            0.52  0.0  \\
            0.53  0.0  \\
            0.54  0.0  \\
            0.55  0.0  \\
            0.56  0.0  \\
            0.57  0.0  \\
            0.58  0.0  \\
            0.59  0.0  \\
            0.6  0.0  \\
            0.61  0.0  \\
            0.62  0.0  \\
            0.63  0.0  \\
            0.64  0.0  \\
            0.65  0.0  \\
            0.66  0.0  \\
            0.67  0.0  \\
            0.68  0.0  \\
            0.69  0.0  \\
            0.7  0.0  \\
            0.71  0.0  \\
            0.72  0.0  \\
            0.73  0.0  \\
            0.74  0.0  \\
            0.75  0.0  \\
            0.76  0.0  \\
            0.77  0.0  \\
            0.78  0.0  \\
            0.79  0.0  \\
            0.8  0.0  \\
            0.81  0.0  \\
            0.82  0.0  \\
            0.83  0.0  \\
            0.84  0.0  \\
            0.85  0.0  \\
            0.86  0.0  \\
            0.87  0.0  \\
            0.88  0.0  \\
            0.89  0.0  \\
            0.9  0.0  \\
            0.91  0.0  \\
            0.92  0.0  \\
            0.93  0.0  \\
            0.94  0.0  \\
            0.95  0.0  \\
            0.96  0.0  \\
            0.97  0.0  \\
            0.98  0.0  \\
            0.99  0.0  \\
            1.0  0.0  \\
        }
        ;
    \nextgroupplot
    \addplot+[thick]
        table[row sep={\\}]
        {
            \\
            0.0  0.0  \\
            0.01  0.0  \\
            0.02  0.0  \\
            0.03  0.0  \\
            0.04  0.0  \\
            0.05  0.0  \\
            0.06  0.0  \\
            0.07  0.0  \\
            0.08  0.0  \\
            0.09  0.0  \\
            0.1  0.0  \\
            0.11  0.0  \\
            0.12  0.0  \\
            0.13  0.0  \\
            0.14  0.0  \\
            0.15  0.0  \\
            0.16  0.0  \\
            0.17  0.0  \\
            0.18  0.0  \\
            0.19  0.0  \\
            0.2  0.0  \\
            0.21  0.0  \\
            0.22  0.0  \\
            0.23  0.0  \\
            0.24  0.0  \\
            0.25  0.0  \\
            0.26  0.0  \\
            0.27  0.0  \\
            0.28  0.0  \\
            0.29  0.0  \\
            0.3  0.0  \\
            0.31  0.0  \\
            0.32  0.0  \\
            0.33  0.0  \\
            0.34  0.0  \\
            0.35  0.0  \\
            0.36  0.0  \\
            0.37  0.0  \\
            0.38  0.0  \\
            0.39  0.0  \\
            0.4  0.0  \\
            0.41  0.0  \\
            0.42  0.0  \\
            0.43  0.0  \\
            0.44  0.0  \\
            0.45  0.0  \\
            0.46  0.0  \\
            0.47  0.0  \\
            0.48  0.0  \\
            0.49  0.0  \\
            0.5  0.0  \\
            0.51  0.0  \\
            0.52  0.0  \\
            0.53  0.0  \\
            0.54  0.0  \\
            0.55  0.0  \\
            0.56  0.0  \\
            0.57  0.0  \\
            0.58  0.0  \\
            0.59  0.0  \\
            0.6  0.0  \\
            0.61  0.0  \\
            0.62  0.0  \\
            0.63  0.0  \\
            0.64  0.0  \\
            0.65  0.0  \\
            0.66  0.0  \\
            0.67  0.0  \\
            0.68  0.0  \\
            0.69  0.0  \\
            0.7  0.0  \\
            0.71  0.0  \\
            0.72  0.0  \\
            0.73  0.0  \\
            0.74  0.0  \\
            0.75  0.0  \\
            0.76  0.0  \\
            0.77  0.0  \\
            0.78  0.0  \\
            0.79  0.0  \\
            0.8  0.0  \\
            0.81  0.0  \\
            0.82  0.0  \\
            0.83  0.0  \\
            0.84  0.0  \\
            0.85  0.0  \\
            0.86  0.0  \\
            0.87  0.0  \\
            0.88  0.0  \\
            0.89  0.0  \\
            0.9  0.0  \\
            0.91  0.0  \\
            0.92  0.0  \\
            0.93  0.0  \\
            0.94  0.0  \\
            0.95  0.0  \\
            0.96  0.0  \\
            0.97  0.0  \\
            0.98  0.0  \\
            0.99  0.0  \\
            1.0  0.0  \\
        }
        ;
    \nextgroupplot[ylabel={$\symbf{A}_{\mathrm{l}}$}]
    \addplot+[thick]
        table[row sep={\\}]
        {
            \\
            0.0  0.0  \\
            0.01  0.0077  \\
            0.02  0.0162  \\
            0.03  0.0252  \\
            0.04  0.0353  \\
            0.05  0.0455  \\
            0.06  0.056  \\
            0.07  0.0656  \\
            0.08  0.0773  \\
            0.09  0.0868  \\
            0.1  0.098  \\
            0.11  0.1097  \\
            0.12  0.1186  \\
            0.13  0.1275  \\
            0.14  0.1394  \\
            0.15  0.1491  \\
            0.16  0.1598  \\
            0.17  0.1695  \\
            0.18  0.1798  \\
            0.19  0.191  \\
            0.2  0.2014  \\
            0.21  0.2126  \\
            0.22  0.2227  \\
            0.23  0.2327  \\
            0.24  0.243  \\
            0.25  0.2564  \\
            0.26  0.267  \\
            0.27  0.2759  \\
            0.28  0.2862  \\
            0.29  0.2973  \\
            0.3  0.3076  \\
            0.31  0.3182  \\
            0.32  0.3307  \\
            0.33  0.3402  \\
            0.34  0.3508  \\
            0.35  0.3607  \\
            0.36  0.3718  \\
            0.37  0.3801  \\
            0.38  0.3921  \\
            0.39  0.4033  \\
            0.4  0.4141  \\
            0.41  0.4244  \\
            0.42  0.4334  \\
            0.43  0.4441  \\
            0.44  0.4545  \\
            0.45  0.4647  \\
            0.46  0.4737  \\
            0.47  0.4836  \\
            0.48  0.4943  \\
            0.49  0.5037  \\
            0.5  0.5154  \\
            0.51  0.5236  \\
            0.52  0.5318  \\
            0.53  0.541  \\
            0.54  0.5513  \\
            0.55  0.5606  \\
            0.56  0.5707  \\
            0.57  0.5802  \\
            0.58  0.5899  \\
            0.59  0.5983  \\
            0.6  0.6075  \\
            0.61  0.616  \\
            0.62  0.625  \\
            0.63  0.6337  \\
            0.64  0.6431  \\
            0.65  0.6511  \\
            0.66  0.6614  \\
            0.67  0.6705  \\
            0.68  0.6794  \\
            0.69  0.6871  \\
            0.7  0.6957  \\
            0.71  0.7055  \\
            0.72  0.7152  \\
            0.73  0.7262  \\
            0.74  0.7358  \\
            0.75  0.7433  \\
            0.76  0.7521  \\
            0.77  0.7631  \\
            0.78  0.7724  \\
            0.79  0.781  \\
            0.8  0.7894  \\
            0.81  0.8002  \\
            0.82  0.81  \\
            0.83  0.8193  \\
            0.84  0.8282  \\
            0.85  0.8371  \\
            0.86  0.8452  \\
            0.87  0.8565  \\
            0.88  0.8678  \\
            0.89  0.8781  \\
            0.9  0.8906  \\
            0.91  0.9011  \\
            0.92  0.9121  \\
            0.93  0.9224  \\
            0.94  0.9324  \\
            0.95  0.9425  \\
            0.96  0.9523  \\
            0.97  0.9655  \\
            0.98  0.9774  \\
            0.99  0.9887  \\
            1.0  1.0  \\
        }
        ;
    \addplot+[dashed, thick]
        coordinates {
            (0,0)
            (1,1)
        }
        ;
    \nextgroupplot
    \addplot+[thick]
        table[row sep={\\}]
        {
            \\
            0.0  1.0  \\
            0.01  0.00039999999999995595  \\
            0.02  9.999999999998899e-5  \\
            0.03  0.0  \\
            0.04  0.0  \\
            0.05  0.0  \\
            0.06  0.0  \\
            0.07  0.0  \\
            0.08  0.0  \\
            0.09  0.0  \\
            0.1  0.0  \\
            0.11  0.0  \\
            0.12  0.0  \\
            0.13  0.0  \\
            0.14  0.0  \\
            0.15  0.0  \\
            0.16  0.0  \\
            0.17  0.0  \\
            0.18  0.0  \\
            0.19  0.0  \\
            0.2  0.0  \\
            0.21  0.0  \\
            0.22  0.0  \\
            0.23  0.0  \\
            0.24  0.0  \\
            0.25  0.0  \\
            0.26  0.0  \\
            0.27  0.0  \\
            0.28  0.0  \\
            0.29  0.0  \\
            0.3  0.0  \\
            0.31  0.0  \\
            0.32  0.0  \\
            0.33  0.0  \\
            0.34  0.0  \\
            0.35  0.0  \\
            0.36  0.0  \\
            0.37  0.0  \\
            0.38  0.0  \\
            0.39  0.0  \\
            0.4  0.0  \\
            0.41  0.0  \\
            0.42  0.0  \\
            0.43  0.0  \\
            0.44  0.0  \\
            0.45  0.0  \\
            0.46  0.0  \\
            0.47  0.0  \\
            0.48  0.0  \\
            0.49  0.0  \\
            0.5  0.0  \\
            0.51  0.0  \\
            0.52  0.0  \\
            0.53  0.0  \\
            0.54  0.0  \\
            0.55  0.0  \\
            0.56  0.0  \\
            0.57  0.0  \\
            0.58  0.0  \\
            0.59  0.0  \\
            0.6  0.0  \\
            0.61  0.0  \\
            0.62  0.0  \\
            0.63  0.0  \\
            0.64  0.0  \\
            0.65  0.0  \\
            0.66  0.0  \\
            0.67  0.0  \\
            0.68  0.0  \\
            0.69  0.0  \\
            0.7  0.0  \\
            0.71  0.0  \\
            0.72  0.0  \\
            0.73  0.0  \\
            0.74  0.0  \\
            0.75  0.0  \\
            0.76  0.0  \\
            0.77  0.0  \\
            0.78  0.0  \\
            0.79  0.0  \\
            0.8  0.0  \\
            0.81  0.0  \\
            0.82  0.0  \\
            0.83  0.0  \\
            0.84  0.0  \\
            0.85  0.0  \\
            0.86  0.0  \\
            0.87  0.0  \\
            0.88  0.0  \\
            0.89  0.0  \\
            0.9  0.0  \\
            0.91  0.0  \\
            0.92  0.0  \\
            0.93  0.0  \\
            0.94  0.0  \\
            0.95  0.0  \\
            0.96  0.0  \\
            0.97  0.0  \\
            0.98  0.0  \\
            0.99  0.0  \\
            1.0  0.0  \\
        }
        ;
    \nextgroupplot
    \addplot+[thick]
        table[row sep={\\}]
        {
            \\
            0.0  1.0  \\
            0.01  0.9501  \\
            0.02  0.9077  \\
            0.03  0.872  \\
            0.04  0.8418  \\
            0.05  0.8146  \\
            0.06  0.7867999999999999  \\
            0.07  0.7612  \\
            0.08  0.7364999999999999  \\
            0.09  0.7128  \\
            0.1  0.6935  \\
            0.11  0.6708000000000001  \\
            0.12  0.6504  \\
            0.13  0.6284000000000001  \\
            0.14  0.6113  \\
            0.15  0.5927  \\
            0.16  0.5751999999999999  \\
            0.17  0.5590999999999999  \\
            0.18  0.5437000000000001  \\
            0.19  0.5276000000000001  \\
            0.2  0.5143  \\
            0.21  0.49950000000000006  \\
            0.22  0.48450000000000004  \\
            0.23  0.4726  \\
            0.24  0.4586  \\
            0.25  0.44499999999999995  \\
            0.26  0.43220000000000003  \\
            0.27  0.4196  \\
            0.28  0.406  \\
            0.29  0.39370000000000005  \\
            0.3  0.38270000000000004  \\
            0.31  0.3699  \\
            0.32  0.35950000000000004  \\
            0.33  0.3506  \\
            0.34  0.3419  \\
            0.35  0.33140000000000003  \\
            0.36  0.32189999999999996  \\
            0.37  0.31210000000000004  \\
            0.38  0.3036  \\
            0.39  0.29369999999999996  \\
            0.4  0.2852  \\
            0.41  0.27559999999999996  \\
            0.42  0.2661  \\
            0.43  0.25860000000000005  \\
            0.44  0.24970000000000003  \\
            0.45  0.24280000000000002  \\
            0.46  0.23460000000000003  \\
            0.47  0.22840000000000005  \\
            0.48  0.2217  \\
            0.49  0.21489999999999998  \\
            0.5  0.20789999999999997  \\
            0.51  0.2006  \\
            0.52  0.19420000000000004  \\
            0.53  0.1884  \\
            0.54  0.1834  \\
            0.55  0.17769999999999997  \\
            0.56  0.17220000000000002  \\
            0.57  0.16559999999999997  \\
            0.58  0.15810000000000002  \\
            0.59  0.1513  \\
            0.6  0.14649999999999996  \\
            0.61  0.14059999999999995  \\
            0.62  0.13570000000000004  \\
            0.63  0.12980000000000003  \\
            0.64  0.12429999999999997  \\
            0.65  0.11919999999999997  \\
            0.66  0.11429999999999996  \\
            0.67  0.11009999999999998  \\
            0.68  0.1058  \\
            0.69  0.10089999999999999  \\
            0.7  0.09450000000000003  \\
            0.71  0.08960000000000001  \\
            0.72  0.08520000000000005  \\
            0.73  0.08109999999999995  \\
            0.74  0.07779999999999998  \\
            0.75  0.07340000000000002  \\
            0.76  0.06930000000000003  \\
            0.77  0.06520000000000004  \\
            0.78  0.060799999999999965  \\
            0.79  0.05779999999999996  \\
            0.8  0.05510000000000004  \\
            0.81  0.05120000000000002  \\
            0.82  0.04690000000000005  \\
            0.83  0.04310000000000003  \\
            0.84  0.04049999999999998  \\
            0.85  0.037799999999999945  \\
            0.86  0.0353  \\
            0.87  0.03149999999999997  \\
            0.88  0.027699999999999947  \\
            0.89  0.024900000000000033  \\
            0.9  0.022800000000000042  \\
            0.91  0.020399999999999974  \\
            0.92  0.017800000000000038  \\
            0.93  0.015199999999999991  \\
            0.94  0.013000000000000012  \\
            0.95  0.01090000000000002  \\
            0.96  0.008299999999999974  \\
            0.97  0.006000000000000005  \\
            0.98  0.0044999999999999485  \\
            0.99  0.0027000000000000357  \\
            1.0  0.0  \\
        }
        ;
\end{groupplot}
\node[anchor=north] at ($(group c1r6.west |- group c1r6.outer south)!0.5!(group c3r6.east |- group c3r6.outer south)$){significance level};
\node[anchor=south, rotate=90] at ($(group c1r1.north -| group c1r1.outer west)!0.5!(group c1r6.south -| group c1r6.outer west)$){empirical test error};
\end{tikzpicture}

      \end{center}
    \end{minipage}
  }
\end{tcbposter}
\end{document}
