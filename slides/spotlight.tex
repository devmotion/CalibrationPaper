% arara: lualatex: { shell: true }
% arara: biber
% arara: lualatex: { shell: true }
% arara: lualatex: { shell: true }
\PassOptionsToPackage{force}{filehook} % see https://tex.stackexchange.com/questions/513051/filehook-error-with-memoir-after-update-texlive-2019-in-oct-15
\documentclass[aspectratio=169]{beamer}
% \documentclass[aspectratio=169,handout]{beamer} % for handouts

% Plots
\usepackage{pgfplots}
\pgfplotsset{compat=1.16}
\usepgfplotslibrary{groupplots,fillbetween}
\usetikzlibrary{positioning,arrows,arrows.meta,calc,decorations.markings,intersections,patterns}

% Use colorbrewer
\usepgfplotslibrary{colorbrewer}
\pgfplotsset{
  % initialize Dark2-8:
  cycle list/Dark2-8,
  % combine it with ’mark list*’:
  cycle multiindex* list={
    mark list*\nextlist
    Dark2-8\nextlist
  },
}

\tikzset{
  invisible/.style={opacity=0},
  visible on/.style={alt={#1{}{invisible}}},
  alt/.code args={<#1>#2#3}{%
    \alt<#1>{\pgfkeysalso{#2}}{\pgfkeysalso{#3}} % \pgfkeysalso doesn't change the path
  },
}

% Beamer theme
\usetheme[progressbar=frametitle,numbering=fraction,block=fill]{metropolis}

% Math support
\usepackage{mathtools,amssymb}

% Fonts
% open source alternative to Gill Sans in UU's official layout
\usepackage{gillius2}
\usepackage{microtype}

% Languages
\usepackage{polyglossia}
\setdefaultlanguage{english}
\usepackage{csquotes}

% Graphics
\usepackage{graphicx}
\usepackage{svg}
\svgpath{{./figures/}}

% Tables
\usepackage{booktabs}

% Colors
\def\UseCMYK{true}
\usepackage{UUcolorPantone}
\makeatletter
\setlength{\metropolis@progressonsectionpage@linewidth}{1pt}
\makeatother
\setbeamercolor{progress bar}{fg=uured,bg=uured!50}

% Itemize
\setbeamertemplate{itemize item}{\color{uured}$\blacktriangleright$}

% Boxes
\usepackage[export]{adjustbox}
\usepackage{tcolorbox}
\tcbuselibrary{most}
\tcbset{coltitle=black,fonttitle=\bfseries\large\scshape}

\newenvironment{uuredbox}[1][]%
{\begin{tcolorbox}[colback=uured!15,colframe=uured,#1]}%
{\end{tcolorbox}}
\newenvironment{uugreenbox}[1][]%
{\begin{tcolorbox}[colback=gronskasvag,colframe=gronskastark,#1]}%
{\end{tcolorbox}}
\newenvironment{uuyellowbox}[1][]%
{\begin{tcolorbox}[colback=blondsvag,colframe=blondstark,#1]}%
{\end{tcolorbox}}
\newenvironment{uubluebox}[1][]%
{\begin{tcolorbox}[colback=gryningmellan,colframe=gryningstark,#1]}%
{\end{tcolorbox}}
\newenvironment{uugraybox}[1][]%
{\begin{tcolorbox}[colback=uulightgrey,colframe=black,#1]}%
{\end{tcolorbox}}

% References
\usepackage[style=authortitle-icomp,doi=false,url=false,isbn=false]{biblatex}
\addbibresource{references.bib}

\newenvironment{refitemize}%
{%
  \noindent\rule{\textwidth}{0.4pt}\par%
  \vspace*{-\parskip}%
  \setbeamertemplate{itemize item}{\leavevmode\tiny\includegraphics[raise=-2pt,width=5pt]{beamericonarticle}}%
  \setbeamerfont{itemize/enumerate body}{size=\tiny}%
  \setlength\topsep{0pt}%NEW
  \setlength\partopsep{0pt}%NEW
  \setlength\itemsep{0pt}%NEW
  \settowidth{\leftmargini}{\usebeamertemplate{itemize item}}%
  \addtolength{\leftmargini}{\labelsep}%
  \begin{itemize}%
}{%
  \end{itemize}%
  \par\ignorespacesafterend%
}
\newcommand\refitem[1]{\item \fullcite{#1}}

\newcommand\hl[1]{\begingroup\bfseries\boldmath\color{uured}#1\endgroup}

\title{\texorpdfstring{Calibration tests in multi-class classification:\\A unifying framework}{Calibration tests in multi-class classification: A unifying framework}}
\date{}
\author{\texorpdfstring{David Widmann$^\star$ Fredrik Lindsten$^\ddagger$ Dave Zachariah$^\star$}{David Widmann, Fredrik Lindsten, Dave Zachariah}}
\titlegraphic{\hfill%
	\begin{tikzpicture}
		\node[right,inner sep=0pt,outer sep=0pt] (UU) {\includegraphics[width=1.5cm]{figures/logos/UU.pdf}};
		\node[right,inner sep=0pt,outer sep=0pt, below right=0mm and 2mm of UU.north east]  (LiU) {\includegraphics[width=2.5cm]{figures/logos/LiU.pdf}};
	\end{tikzpicture}%
}
\institute{$^\star$Department of Information Technology, Uppsala University, Sweden\\$^\ddagger$Division of Statistics and Machine Learning, Linköping University, Sweden}

\begin{document}

\pgfplotsset{compat=1.16}

\maketitle

\begin{frame}{Motivation}
  \begin{center}
    \begin{tikzpicture}
      \node[draw, inner sep=0pt] (image) at (0, 0)
      {\includegraphics[height=0.15\textwidth]{figures/malignant/ISIC_0000002.jpg}};
      \node[above=2mm of image, anchor=base] {Skin image};

      \onslide<2->{%
        \node[draw, fill=gronskasvag, right=1cm of image, inner sep=2mm] (model)
        {\includesvg[height=\dimexpr0.15\textwidth-4mm\relax]{gear}};
        \node[above=2mm of model, anchor=base] {Model};
        \draw [->] (image) -- (model);
      }%

      \onslide<3->{%
        \node[draw, fill=blondsvag, right=1cm of model, minimum height=0.15\textwidth, align=center] (prediction)
        {\begin{tabular}{@{}cc@{}}
           \texttt{malignant} & \texttt{benign} \\ \midrule
           80\% & 20\% \\
         \end{tabular}};
       \node[above=2mm of prediction, anchor=base] {Prediction};
       \draw [->] (model) -- (prediction);
     }%
   \end{tikzpicture}
 \end{center}

 \onslide<4->{%
   \begin{uubluebox}
     \begin{center}
       How can we ensure that the predicted confidence scores are \enquote{meaningful}?
     \end{center}
   \end{uubluebox}
 }%
\end{frame}

\begin{frame}{Introduction}
  \begin{center}
    \begin{tikzpicture}
      \node[draw, fill=gronskasvag, inner sep=2mm] (model) at (0, 0)
      {\includesvg[height=\dimexpr0.15\textwidth-4mm\relax]{gear}};

      \visible<2-4>{%
        \node[draw, inner sep=0pt, left=1cm of model] (image)
        {\includegraphics[height=0.15\textwidth]{figures/malignant/ISIC_0000002.jpg}};
      }%
      \visible<5-7>{%
        \node[draw, inner sep=0pt, left=1cm of model] (image)
        {\includegraphics[height=0.15\textwidth]{figures/benign/ISIC_0000000.jpg}};
      }%
      \visible<8-10>{%
        \node[draw, inner sep=0pt, left=1cm of model] (image)
        {\includegraphics[height=0.15\textwidth]{figures/malignant/ISIC_0000013.jpg}};
      }%

      \visible<11>{%
        \node[left=1cm of model, minimum height=0.15\textwidth] (image)
        {\includegraphics[width=0.05\textwidth]{figures/malignant/ISIC_0000022.jpg} \hspace{2mm} \includegraphics[width=0.05\textwidth]{figures/malignant/ISIC_0000026.jpg} \hspace{2mm} $\cdots$};
      }%

      \visible<2->{%
        \draw [->] (image) -- (model);
      }%

      \visible<3-4,6-7,9-11>{%
        \node[draw, fill=blondsvag, right=1cm of model, align=left, minimum height=0.15\textwidth] (prediction)
        {\begin{tabular}{@{}cc@{}}
           \texttt{malignant} & \texttt{benign} \\ \midrule
           80\% & 20\% \\
         \end{tabular}};
       \draw [->] (model) -- (prediction);
     }%
   \end{tikzpicture}
 \end{center}

 \onslide<4->{%
   \begin{uubluebox}[title={Empirical frequency}]
     \begin{center}
       \begin{tabular}{@{}ccccc@{}} \toprule
         \multicolumn{4}{c}{\texttt{malignant}} & \texttt{benign} \\ \midrule
         \includegraphics[width=5mm]{figures/malignant/ISIC_0000002.jpg} & \visible<10->{\includegraphics[width=5mm]{figures/malignant/ISIC_0000013.jpg}} & \visible<11->{\includegraphics[width=5mm]{figures/malignant/ISIC_0000022.jpg}} & \visible<11->{\includegraphics[width=5mm]{figures/malignant/ISIC_0000026.jpg}} & \visible<7->{\includegraphics[width=5mm]{figures/benign/ISIC_0000000.jpg}}
                                                                                                                                                                                                                                                                                                                              \visible<11->{\\ $\vdots$ & $\vdots$ & $\vdots$ & $\vdots$ & $\vdots$} \\ \bottomrule
       \end{tabular}
     \end{center}
   \end{uubluebox}
 }%
\end{frame}

\begin{frame}{Calibrated model}
  \begin{tcbraster}[raster columns=2,raster equal height=rows]
    \begin{uugreenbox}[raster multicolumn=2]
      \begin{center}
        A \hl{calibrated model} reports\\
        \hl{predictions consistent with empirically observed frequencies} of outcomes.
      \end{center}
    \end{uugreenbox}
		\begin{uuyellowbox}[enhanced, title={Prediction}, valign=center, remember as=A]
      \begin{center}
        \begin{tabular}{@{}cc@{}} \toprule
          \texttt{malignant} & \texttt{benign} \\ \midrule
          80\% & 20\% \\ \bottomrule
        \end{tabular}
      \end{center}
    \end{uuyellowbox}%
    \begin{uubluebox}[enhanced, title={Empirical frequency}, valign=center, remember as=B]
      \begin{center}
        \begin{tabular}{@{}ccccc@{}} \toprule
          \multicolumn{4}{c}{\texttt{malignant}} & \texttt{benign} \\ \midrule
          \includegraphics[width=5mm]{figures/malignant/ISIC_0000002.jpg} & \includegraphics[width=5mm]{figures/malignant/ISIC_0000013.jpg} & \includegraphics[width=5mm]{figures/malignant/ISIC_0000022.jpg} & \includegraphics[width=5mm]{figures/malignant/ISIC_0000026.jpg} & \includegraphics[width=5mm]{figures/benign/ISIC_0000000.jpg} \\
          $\vdots$ & $\vdots$ & $\vdots$ & $\vdots$ & $\vdots$ \\ \bottomrule
        \end{tabular}
      \end{center}
    \end{uubluebox}
  \end{tcbraster}

  \begin{tikzpicture}[remember picture, overlay]
    \path (A) -- node [font=\boldmath\Huge, color=uured, align=center, midway] {$\stackrel{?}{=}$} (B);
  \end{tikzpicture}
\end{frame}

\begin{frame}{Multi-class classification: all scores matter!}
  \begin{tcbraster}[raster columns=1]
    \begin{tcolorbox}[blankest]
      \begin{center}
        \begin{tikzpicture}
          \node[minimum height=0.11\textwidth, inner sep=2mm] (image) at (0, 0)
          {\begin{tabular}{@{}ccc@{}}
             \includesvg[height=3mm]{car0} & \includesvg[height=3mm]{car1} & \includesvg[height=3mm]{car2} \\
             \includesvg[height=3mm]{car3} & \includesvg[height=3mm]{car4} & $\cdots$ \\
           \end{tabular}};

          \onslide<2->{%
            \node[draw, fill=gronskasvag, right=1cm of image, inner sep=2mm] (model)
            {\includesvg[height=\dimexpr0.11\textwidth-4mm\relax]{gear}};
            \draw [->] (image) -- (model);
          }%

          \onslide<3->{%
            \node[draw, fill=blondsvag, right=1cm of model, minimum height=0.11\textwidth, align=center] (prediction)
            {\begin{tabular}{@{}ccc@{}}
               \texttt{object} & \texttt{human} & \texttt{animal} \\ \midrule
               80\% & 0\% & 20\% \\
             \end{tabular}};
           \draw [->] (model) -- (prediction);
         }%
       \end{tikzpicture}
     \end{center}
   \end{tcolorbox}
   \onslide<4->{%
     \begin{uubluebox}
       \begin{center}
         Common calibration evaluation techniques consider only the
         most-confident score
       \end{center}
     \end{uubluebox}
   }%
   \onslide<5-> & 0\% & 20\% \\
           \hl{80\%} & 20\% & 0\% \\ \bottomrule
         \end{tabular}
       \end{center}
     \end{uuredbox}
   }%
   \begin{tcolorbox}[blankest]
     \begin{refitemize}
       \refitem{vaicenavicius19_evaluat}
     \end{refitemize}
   \end{tcolorbox}
 \end{tcbraster}
\end{frame}

\begin{frame}{Our contribution: Calibration errors in multi-class classification}
  \begin{tcbraster}[raster columns=1, raster rows=3]
    \begin{uuyellowbox}[title={Unifying framework of calibration errors}, left=0pt]
      \begin{itemize}
      \item Based on the full predictions with all scores
      \item<2-> Encompasses existing measures such as the
        expected calibration error ($\mathrm{ECE}$)
      \item<3-> Enables derivation of a \hl{kernel calibration error ($\mathrm{KCE}$)}
      \end{itemize}
    \end{uuyellowbox}
    \onslide<4->{%
      \begin{uugreenbox}[left=0pt, title={Estimating calibration errors}]
        \begin{itemize}
        \item The standard $\mathrm{ECE}$ estimator is usually biased and inconsistent
        \item<5-> The $\mathrm{KCE}$ yields \hl{unbiased} and \hl{consistent} estimators
        \end{itemize}
      \end{uugreenbox}
    }%
  \end{tcbraster}
\end{frame}

\begin{frame}{Our contribution: Calibration tests in multi-class classification}
  \only<1>{%
    \begin{uuredbox}
      \begin{center}
        Calibration errors have no meaningful unit or scale
      \end{center}
    \end{uuredbox}
  }%

  \only<2->{%
    \begin{tcbraster}[raster columns=1]
      \onslide<3->{%
        \begin{uubluebox}
          \begin{center}
            Test the null hypothesis $H_0 \coloneqq \text{\enquote{model is calibrated}}$
          \end{center}
        \end{uubluebox}
      }%

      \begin{tcolorbox}[blankest]
        \begin{center}
          \begin{tikzpicture}[
            declare function={normal(\m,\s)=1/(2*\s*sqrt(pi))*exp(-(x-\m)^2/(2*\s^2));},
            declare function={binormal(\ma,\sa,\mb,\sb,\p)=(\p*normal(\ma,\sa)+(1-\p)*normal(\mb,\sb));}
            ]

            \begin{axis}[
              domain = -0.1:0.2,
              no marks,
              xlabel = calibration error estimate,
              ylabel = density,
              grid=major,
              ymin = 0,
              tick label style={font=\tiny},
              label style={font=\small},
              width = 0.73\textwidth,
              height = 0.25\textwidth,
              legend pos=outer north east,
              legend cell align=left,
              legend style=
              {
                fill=none,
                draw=none,
                inner sep={0pt},
                font=\small,
              }
              ]

              \draw [Dark2-A, thick] (0.07,\pgfkeysvalueof{/pgfplots/ymin}) -- (0.07,\pgfkeysvalueof{/pgfplots/ymax}) node [at end, above, anchor=south east, sloped, font=\small] {observed};

              \draw[Dark2-B, thick] (0,\pgfkeysvalueof{/pgfplots/ymin}) -- (0,\pgfkeysvalueof{/pgfplots/ymax}) node [at end, above, anchor=south east, sloped, font=\small] {calibrated};

              % mixture model of normal distributions
              \addplot+ [color=Dark2-B, dashed, thick, visible on=<4->, samples=31, smooth, name path=A] {binormal(-0.05,0.01,0.05,0.03,0.5)};
              \addlegendentry[visible on=<4->]{distribution under $H_0$};

              % indicate p-value
              \path [name path=B] (\pgfkeysvalueof{/pgfplots/xmin},0) -- (\pgfkeysvalueof{/pgfplots/xmax},0);
              \addplot+ [draw=Dark2-C, pattern color=Dark2-C, pattern={north east lines}, visible on=<5->] fill between [of=A and B, soft clip={domain=0.07:0.2}];
              \addlegendentry[visible on=<5->]{p-value};

              % add comment
              \node<6->[anchor=west, align=left, text=Dark2-C, font=\small] (annotation) at (0.1, 10) {reject $H_0$ if the \\p-value is small};
              \draw<6->[->, >=stealth, thick, Dark2-C] (annotation) -- (0.08, 1);
            \end{axis}
          \end{tikzpicture}
        \end{center}
      \end{tcolorbox}
      \onslide<7->{%
        \begin{uugreenbox}[left=0pt]
          \begin{itemize}
          \item Existing $\mathrm{ECE}$-based approach seems prone to
            underestimating the p-value
          \item<8-> \hl{Well-founded bounds and approximations} of the p-value for the $\mathrm{KCE}$
          \end{itemize}
        \end{uugreenbox}
      }%
      \onslide<3->{%
        \begin{tcolorbox}[blankest]
          \begin{refitemize}
            \refitem{broecker07_increas_reliab_reliab_diagr}
            \refitem{vaicenavicius19_evaluat}
          \end{refitemize}
        \end{tcolorbox}
      }%
    \end{tcbraster}
  }%
\end{frame}

\begin{frame}[standout]
  \vspace*{3\baselineskip}
	Thank you for listening!

	Come see our poster: \#39
  \vspace*{2\baselineskip}
	\begin{flushleft}
    \normalsize
		Code available at:\\
    \url{https://github.com/devmotion/CalibrationPaper}
	\end{flushleft}
\end{frame}

\end{document}